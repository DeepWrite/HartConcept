\documentclass[12pt,oneside]{book}  % 글씨 크기 13pt

% 언어 및 입력
\usepackage[english]{babel}
\usepackage[T1]{fontenc}
\usepackage[utf8]{inputenc}

% 가독성 좋은 단행본용 서체 (Libertinus 또는 Palatino)
\usepackage{libertinus}  % 또는: \usepackage[sc]{mathpazo}

\linespread{1.25}  % 줄간격: 단행본용

% 여백 설정 (A4, 단행본 스타일)
\usepackage[a4paper, margin=1in]{geometry}

% 수학 및 기호
\usepackage{amsmath, amssymb, amsthm, mathtools}

% 표와 배열
\usepackage{booktabs, tabularx}

% 인용 및 각주
\usepackage[notes, backend=biber]{biblatex-chicago}
\addbibresource{attention.bib}
\usepackage[flushmargin, hang]{footmisc}

% 기타 기능
\usepackage{csquotes}
\usepackage{enumitem}
\usepackage[hidelinks]{hyperref}

% 헤더와 푸터
\usepackage{fancyhdr}
\pagestyle{fancy}
\fancyhf{}
\fancyfoot[C]{\thepage}

% 자동 번호매기기 제거: 섹션/장 번호 숨기기
\makeatletter
\renewcommand{\@seccntformat}[1]{}
\makeatother

% 섹션 번호링 없음: chapter, section, subsection 모두 제목만 표시
\setcounter{secnumdepth}{0}

% 문서 시작
\title{The Concept of Law, Third Edition}
\author{H. L. A. Hart}
\date{}

\begin{document}

\begin{titlepage}
  \centering
  \vspace*{3cm}
  {\Huge\textsc{The Concept of Law}}\\[1.5ex]
  {\Large Third Edition}\\[4ex]
  \textsc{H. L. A. Hart}\\[6ex]
  {\small Oxford University Press\\
  2025 (Reprint Typeset Edition)}
  \vfill
\end{titlepage}


\section{\texorpdfstring{\textbf{I PERSISTENT
QUESTIONS}}{I PERSISTENT QUESTIONS}}\label{i-persistent-questions}

\subsection{\texorpdfstring{\textbf{1. PERPLEXITIES OF LEGAL
THEORY}}{1. PERPLEXITIES OF LEGAL THEORY}}\label{perplexities-of-legal-theory}

Few questions concerning human society have been asked with such
persistence and answered by serious thinkers in so many diverse,
strange, and even paradoxical ways as the question `What is law?' Even
if we confine our attention to the legal theory of the last 150 years
and neglect classical and medieval speculation about the `nature' of
law, we shall find a situation not paralleled in any other subject
systematically studied as a separate academic discipline. No vast
literature is dedicated to answering the questions `What is chemistry?'
or `What is medicine?', as it is to the question `What is law?' A few
lines on the opening page of an elementary textbook is all that the
student of these sciences is asked to consider; and the answers he is
given are of a very different kind from those tendered to the student of
law. No one has thought it illuminating or important to insist that
medicine is `what doctors do about illnesses', or `a prediction of what
doctors will do', or to declare that what is ordinarily recognized as a
characteristic, central part of chemistry, say the study of acids, is
not really part of chemistry at all. Yet, in the case of law, things
which at first sight look as strange as these have often been said, and
not only said but urged with eloquence and passion, as if they were
revelations of truths about law, long obscured by gross
misrepresentations of its essential nature.

`What officials do about disputes is \ldots{} the law
itself';\footnote{Llewellyn, \emph{The Bramble Bush} (2nd edn., 1951),
  p.~9.} `The prophecies of what the courts will do \ldots{} are what I
mean by the law';\footnote{O. W. Holmes, `The Path of the Law' in
  \emph{Collected Papers} (1920), p.~173.} Statutes are `sources of Law
\ldots{} not parts of the Law itself';\footnote{J. C. Gray, \emph{The
  Nature and Sources of the Law} (1902), s. 276.} `Constitutional law is
positive morality merely';\footnote{Austin, \emph{The Province of
  Jurisprudence Determined} (1832), Lecture VI (1954 edn., p.~259).}
`One shall not steal; if somebody steals he shall be punished.
\textbf{(p.~2)} \ldots{} If at all existent, the first norm is contained
in the second norm which is the only genuine norm\ldots{} . Law is the
primary norm which stipulates the sanction'. \footnote{Kelsen,
  \emph{General Theory of Law and State} (1949), p.~61.}

These are only a few of many assertions and denials concerning the
nature of law which at first sight, at least, seem strange and
paradoxical. Some of them seem to conflict with the most firmly rooted
beliefs and to be easily refutable; so that we are tempted to reply,
`Surely statutes \emph{are} law, at least one kind of law even if there
are others': `Surely law cannot just mean what officials do or courts
will do, since it takes a law to make an official or a court'.

Yet these seemingly paradoxical utterances were not made by visionaries
or philosophers professionally concerned to doubt the plainest
deliverances of common sense. They are the outcome of prolonged
reflection on law made by men who were primarily lawyers, concerned
professionally either to teach or practise law, and in some cases to
administer it as judges. Moreover, what they said about law actually did
in their time and place increase our understanding of it. For,
understood in their context, such statements are \emph{both}
illuminating and puzzling: they are more like great exaggerations of
some truths about law unduly neglected, than cool definitions. They
throw a light which makes us see much in law that lay hidden; but the
light is so bright that it blinds us to the remainder and so leaves us
still without a clear view of the whole.

To this unending theoretical debate in books we find a strange contrast
in the ability of most men to cite, with ease and confidence, examples
of law if they are asked to do so. Few Englishmen are unaware that there
is a law forbidding murder, or requiring the payment of income tax, or
specifying what must be done to make a valid will. Virtually everyone
except the child or foreigner coming across the English word `law' for
the first time could easily multiply such examples, and most people
could do more. They could describe, at least in outline, how to find out
whether something is the law in England; they know that there are
experts to consult and courts with a final authoritative voice on all
such questions. \textbf{(p.~3)} Much more than this is quite generally
known. Most educated people have the idea that the laws in England form
some sort of system, and that in France or the United States or Soviet
Russia and, indeed, in almost every part of the world which is thought
of as a separate `country' there are legal systems which are broadly
similar in structure in spite of important differences. Indeed an
education would have seriously failed if it left people in ignorance of
these facts, and we would hardly think it a mark of great sophistication
if those who knew this could also say what are the important points of
similarity between different legal systems. Any educated man might be
expected to be able to identify these salient features in some such
skeleton way as follows. They comprise (i) rules forbidding or enjoining
certain types of behaviour under penalty; (ii) rules requiring people to
compensate those whom they injure in certain ways; (iii) rules
specifying what must be done to make wills, contracts or other
arrangements which confer rights and create obligations; (iv) courts to
determine what the rules are and when they have been broken, and to fix
the punishment or compensation to be paid; (v) a legislature to make new
rules and abolish old ones.

If all this is common knowledge, how is it that the question `What is
law?' has persisted and so many various and extraordinary answers have
been given to it? Is it because, besides the clear standard cases
constituted by the legal systems of modern states, which no one in his
senses doubts are legal systems, there exist also doubtful cases, and
about their `legal quality' not only ordinary educated men but even
lawyers hesitate? Primitive law and international law are the foremost
of such doubtful cases, and it is notorious that many find that there
are reasons, though usually not conclusive ones, for denying the
propriety of the now conventional use of the word `law' in these cases.
The existence of these questionable or challengeable cases has indeed
given rise to a prolonged and somewhat sterile controversy, but surely
they cannot account for the perplexities about the general nature of law
expressed by the persistent question `What is law?' That these cannot be
the root of the difficulty seems plain for two reasons.

First, it is quite obvious why hesitation is felt in these cases.
International law lacks a legislature, states cannot be brought before
international courts without their prior consent, and there is no
centrally organized effective system of sanctions. Certain types of
primitive law, including those out of which some contemporary legal
systems may have gradually evolved, similarly lack these features, and
it is perfectly clear to everyone that it is their deviation in these
respects from the standard case which makes their classification appear
questionable. There is no mystery about this.

Secondly, it is not a peculiarity of complex terms like `law' and `legal
system' that we are forced to recognize both clear standard cases and
challengeable borderline cases. It is now a familiar fact (though once
too little stressed) that this distinction must be made in the case of
almost every general term which we use in classifying features of human
life and of the world in which we live. Sometimes the difference between
the clear, standard case or paradigm for the use of an expression and
the questionable cases is only a matter of degree. A man with a shining
smooth pate is clearly bald; another with a luxuriant mop clearly is
not; but the question whether a third man, with a fringe of hair here
and there, is bald might be indefinitely disputed, if it were thought
worth while or any practical issue turned on it.

Sometimes the deviation from the standard case is not a mere matter of
degree but arises when the standard case is in fact a complex of
normally concomitant but distinct elements, some one or more of which
may be lacking in the cases open to challenge. Is a flying boat a
`vessel'? Is it still `chess' if the game is played without a queen?
Such questions may be instructive because they force us to reflect on,
and make explicit, our conception of the composition of the standard
case; but it is plain that what may be called the borderline aspect of
things is too common to account for the long debate about law. Moreover,
only a relatively small and unimportant part of the most famous and
controversial theories of law is concerned with the propriety of using
the expressions `primitive law' or `international law' to describe the
cases to which they are conventionally applied.

When we reflect on the quite general ability of people to recognize and
cite examples of laws and on how much is generally known about the
standard case of a legal system, it might seem that we could easily put
an end to the persistent question, `What is law?', simply by issuing a
series of reminders of what is already familiar. Why should we not just
repeat the skeleton account of the salient features of a municipal legal
system which, perhaps optimistically, we put (on page 3) into the mouth
of an educated man? We can then simply say, `Such is the standard case
of what is meant by ``law'' and ``legal system''; remember that besides
these standard cases you will also find arrangements in social life
which, while sharing some of these salient features, also lack others of
them. These are disputed cases where there can be no conclusive argument
for or against their classification as law.'

Such a way with the question would be agreeably short. But it would have
nothing else to recommend it. For, in the first place, it is clear that
those who are most perplexed by the question `What is law?' have not
forgotten and need no reminder of the familiar facts which this skeleton
answer offers them. The deep perplexity which has kept alive the
question, is not ignorance or forgetfulness or inability to recognize
the phenomena to which the word `law' commonly refers. Moreover, if we
consider the terms of our skeleton account of a legal system, it is
plain that it does little more than assert that in the standard, normal
case laws of various sorts go together. This is so because both a court
and a legislature, which appear in this short account as typical
elements of a standard legal system, are themselves creatures of law.
Only when there are certain types of laws giving men jurisdiction to try
cases and authority to make laws do they constitute a court or a
legislature.

This short way with the question, which does little more than remind the
questioner of the existing conventions governing the use of the words
`law' and `legal system', is therefore useless. Plainly the best course
is to defer giving any answer to the query `What is law?' until we have
found out what it is about law that has in fact puzzled those who have
asked or attempted to answer it, even though their familiarity with the
law and their ability to recognize examples are beyond question. What
more do they want to know and why do they want to know it? To
\emph{this} question something like a general answer can be given. For
there are certain recurrent main themes which have formed a constant
focus of argument and counterargument about the nature of law, and
provoked exaggerated and paradoxical assertions about law such as those
we have already cited. Speculation about the nature of law has a long
and complicated history; yet in retrospect it is apparent that it has
centred almost continuously upon a few principal issues. These were not
gratuitously chosen or invented for the pleasure of academic discussion;
they concern aspects of law which seem naturally, at all times, to give
rise to misunderstanding, so that confusion and a consequent need for
greater clarity about them may coexist even in the minds of thoughtful
men with a firm mastery and knowledge of the law.

\subsection{\texorpdfstring{\textbf{2. THREE RECURRENT
ISSUES}}{2. THREE RECURRENT ISSUES}}\label{three-recurrent-issues}

We shall distinguish here three such principal recurrent issues, and
show later why they come together in the form of a request for a
\emph{definition} of law or an answer to the question `What is law?', or
in more obscurely framed questions such as `What is the nature (or the
essence) of law?'

Two of these issues arise in the following way. The most prominent
general feature of law at all times and places is that its existence
means that certain kinds of human conduct are no longer optional, but in
\emph{some} sense obligatory. Yet this apparently simple characteristic
of law is not in fact a simple one; for within the sphere of
non-optional obligatory conduct we can distinguish different forms. The
first, simplest sense in which conduct is no longer optional, is when
one man is forced to do what another tells him, not because he is
physically compelled in the sense that his body is pushed or pulled
about, but because the other threatens him with unpleasant consequences
if he refuses. The gunman orders his victim to hand over his purse and
threatens to shoot if he refuses; if the victim complies we refer to the
way in which he was forced to do so by saying that he was \emph{obliged}
to do so. To some it has seemed clear that in this situation where one
person gives another an order backed by threats, and, in this sense of
`oblige', obliges him to comply, we have the essence of law, or at least
`the key to the science of jurisprudence'. \footnote{Austin, op. cit.,
  Lecture I, p.~13. He adds `and morals'.} This is the starting-point of
Austin's analysis by which so much English jurisprudence has been
influenced.

There is of course no doubt that a legal system often presents this
aspect among others. A penal statute declaring certain conduct to be an
offence and specifying the punishment to which the offender is liable,
may appear to be the gunman situation writ large; and the only
difference to be the relatively minor one, that in the case of statutes,
the orders are addressed generally to a group which customarily obeys
such orders. But attractive as this reduction of the complex phenomena
of law to this simple element may seem, it has been found, when examined
closely, to be a distortion and a source of confusion even in the case
of a penal statute where an analysis in these simple terms seems most
plausible. How then do law and legal obligation differ from, and how are
they related to, orders backed by threats? This at all times has been
one cardinal issue latent in the question `What is law?'.

A second such issue arises from a second way in which conduct may be not
optional but obligatory. Moral rules impose obligations and withdraw
certain areas of conduct from the free option of the individual to do as
he likes. Just as a legal system obviously contains elements closely
connected with the simple cases of orders backed by threats, so equally
obviously it contains elements closely connected with certain aspects of
morality. In both cases alike there is a difficulty in identifying
precisely the relationship and a temptation to see in the obviously
close connection an identity. Not only do law and morals share a
vocabulary so that there are both legal and moral obligations, duties,
and rights; but all municipal legal systems reproduce the substance of
certain fundamental moral requirements. Killing and the wanton use of
violence are only the most obvious examples of the coincidence between
the prohibitions of law and morals. Further, there is one idea, that of
justice, which seems to unite both fields: it is both a virtue specially
appropriate to law and the most legal of the virtues. We think and talk
of `justice \emph{according to} law' and yet also of the justice or
injustice \emph{of} the laws.

These facts suggest the view that law is best understood as a `branch'
of morality or justice and that its congruence with the principles of
morality or justice rather than its incorporation of orders and threats
is of its `essence'. This is the doctrine characteristic not only of
scholastic theories of natural law but of some contemporary legal theory
which is critical of the legal `positivism' inherited from Austin. Yet
here again theories that make this close assimilation of law to morality
seem, in the end, often to confuse one kind of obligatory conduct with
another, and to leave insufficient room for differences in kind between
legal and moral rules and for divergences in their requirements. These
are at least as important as the similarity and convergence which we may
also find. So the assertion that `an unjust law is not a law'\footnote{`Nam
  mihi lex esse non videtur quae justa non fuerit': St.~Augustine I,
  \emph{De Libero Arbitrio}; Aquinas, \emph{Summa Theologica}, 1--11,
  Qu. 95 Art. 2; Qu. 96 Art. 4.} has the same ring of exaggeration and
paradox, if not falsity, as `statutes are not laws' or `constitutional
law is not law'. It is characteristic of the oscillation between
extremes which makes up the history of legal theory, that those who have
seen in the close assimilation of law and morals nothing more than a
mistaken inference from the fact that law and morals share a common
vocabulary of rights and duties, should have protested against it in
terms equally exaggerated and paradoxical. `The prophecies of what the
courts will do in fact, and nothing more pretentious, are what I mean by
the law.' \footnote{Holmes, loc. cit.}

The third main issue perennially prompting the question `What is law?'
is a more general one. At first sight it might seem that the statement
that a legal system consists, in general at any rate, of \emph{rules}
could hardly be doubted or found difficult to understand. Both those who
have found the key to the understanding of law in the notion of orders
backed by threats, and those who have found it in its relation to
morality or justice, alike speak of law as containing, if not consisting
largely of, rules. Yet dissatisfaction, confusion, and uncertainty
concerning this seemingly unproblematic notion underlies much of the
perplexity about the nature of law. What \emph{are} rules? What does it
mean to say that a rule \emph{exists?} Do courts really apply rules or
merely pretend to do so? Once the notion is queried, as it has been
especially in the jurisprudence of this century, major divergencies in
opinion appear. These we shall merely outline here.

It is of course true that there are rules of many different types, not
only in the obvious sense that besides legal rules there are rules of
etiquette and of language, rules of games and clubs, but in the less
obvious sense that even within any one of these spheres, what are called
rules may originate in different ways and may have very different
relationships to the conduct with which they are concerned. Thus even
within the law some rules are made by legislation; others are not made
by any such deliberate act. More important, some rules are mandatory in
the sense that they require people to behave in certain ways,
e.g.~abstain from violence or pay taxes, whether they wish to or not;
other rules such as those prescribing the procedures, formalities, and
conditions for the making of marriages, wills, or contracts indicate
what people should do to give effect to the wishes they have. The same
contrast between these two types of rule is also to be seen between
those rules of a game which veto certain types of conduct under penalty
(foul play or abuse of the referee) and those which specify what must be
done to score or to win. But even if we neglect for the moment this
complexity and consider only the first sort of rule (which is typical of
the criminal law) we shall find, even among contemporary writers, the
widest diver gence of view as to the meaning of the assertion that a
rule of this simple mandatory type exists. Some indeed find the notion
utterly mysterious. The account which we are at first perhaps naturally
tempted to give of the apparently simple idea of a mandatory rule has
soon to be abandoned. It is that to say that a rule exists means only
that a group of people, or most of them, behave `as a rule'
i.e.~\emph{generally}, in a specified similar way in certain kinds of
circumstances. So to say that in England there is a rule that a man must
not wear a hat in church or that one must stand up when `God Save the
Queen' is played means, on this account of the matter, only that most
people generally do these things. Plainly this is not enough, even
though it conveys part of what is meant. Mere convergence in behaviour
between members of a social group may exist (all may regularly drink tea
at breakfast or go weekly to the cinema) and yet there may be no rule
\emph{requiring} it. The difference between the two social situations of
mere convergent behaviour and the existence of a social rule shows
itself often linguistically. In describing the latter we may, though we
need not, make use of certain words which would be misleading if we
meant only to assert the former. These are the words `must', `should',
and `ought to', which in spite of differences share certain common
functions in indicating the presence of a rule requiring certain
conduct. There is in England no rule, nor is it true, that everyone must
or ought to or should go to the cinema each week: it is only true that
there is regular resort to the cinema each week. But there \emph{is} a
rule that a man must bare his head in church.

What then is the crucial difference between merely convergent habitual
behaviour in a social group and the existence of a rule of which the
words `must', `should', and `ought to' are often a sign? Here indeed
legal theorists have been divided, especially in our own day when
several things have forced this issue to the front. In the case of legal
rules it is very often held that the crucial difference (the element of
`must' or `ought') consists in the fact that deviations from certain
types of behaviour will probably meet with hostile reaction, and in the
case of legal rules be punished by officials. In the case of what may be
called mere group habits, like that of going weekly to the cinema,
deviations are not met with punishment or even reproof; but wherever
there are rules requiring certain conduct, even non-legal rules like
that requiring men to bare their heads in church, something of this sort
is likely to result from deviation. In the case of legal rules this
predictable consequence is definite and officially organized, whereas in
the non-legal case, though a similar hostile reaction to deviation is
probable, this is not organized or definite in character.

It is obvious that predictability of punishment is one important aspect
of legal rules; but it is not possible to accept this as an exhaustive
account of what is meant by the statement that a social rule exists or
of the element of `must' or `ought' involved in rules. To such a
predictive account there are many objections, but one in particular,
which characterizes a whole school of legal theory in Scandinavia,
deserves careful consideration. It is that if we look closely at the
activity of the judge or official who punishes deviations from legal
rules (or those private persons who reprove or criticize deviations from
non-legal rules), we see that rules are involved in this activity in a
way which this predictive account leaves quite unexplained. For the
judge, in punishing, takes the rule as his \emph{guide} and the breach
of the rule as his \emph{reason} and \emph{justification} for punishing
the offender. He does not look upon the rule as a statement that he and
others are likely to punish deviations, though a spectator might look
upon the rule in just this way. The predictive aspect of the rule
(though real enough) is irrelevant to his purposes, whereas its status
as a guide and justification is essential. The same is true of informal
reproofs administered for the breach of non-legal rules. These too are
not merely predictable reactions to deviations, but something which
existence of the rule guides and is held to justify. So we say that we
reprove or punish a man \emph{because} he has broken the rule: and not
merely that it was probable that we would reprove or punish him.

Yet among critics who have pressed these objections to the predictive
account some confess that there is something obscure here; something
which resists analysis in clear, hard, factual terms. What \emph{can}
there be in a rule apart from regular and hence predictable punishment
or reproof of those who deviate from the usual patterns of conduct,
which distinguishes it from a mere group habit? Can there really be
something over and above these clear ascertainable facts, some extra
element, which guides the judge and justifies or gives him a reason for
punishing? The difficulty of saying what exactly this extra element is
has led these critics of the predictive theory to insist at this point
that all talk of rules, and the corresponding use of words like `must',
`ought', and `should', is fraught with a confusion which perhaps
enhances their importance in men's eyes but has no rational basis. We
merely \emph{think}, so such critics claim, that there is something in
the rule which binds us to do certain things and guides or justifies us
in doing them, but this is an illusion even if it is a useful one. All
that there is, over and above the clear ascertainable facts of group
behaviour and predictable reaction to deviation, are our own powerful
`feelings' of compulsion to behave in accordance with the rule and to
act against those who do not. We do not recognize these feelings for
what they are but imagine that there is something external, some
invisible part of the fabric of the universe guiding and controlling us
in these activities. We are here in the realm of fiction, with which it
is said the law has always been connected. It is only because we adopt
this fiction that we can talk solemnly of the government `of laws not
men'. This type of criticism, whatever the merits of its positive
contentions, at least calls for further elucidation of the distinction
between social rules and mere convergent habits of behaviour. This
distinction is crucial for the understanding of law, and much of the
early chapters of this book is concerned with it.

Scepticism about the character of legal rules has not, however, always
taken the extreme form of condemning the very notion of a binding rule
as confused or fictitious. Instead, the most prevalent form of
scepticism in England and the United States invites us to reconsider the
view that a legal system \emph{wholly}, or even \emph{primarily},
consists of rules. No doubt the courts so frame their judgments as to
give the impression that their decisions are the necessary consequence
of predetermined rules whose meaning is fixed and clear. In very simple
cases this may be so; but in the vast majority of cases that trouble the
courts, neither statutes nor precedents in which the rules are allegedly
contained allow of only one result. In most important cases there is
always a choice. The judge has to choose between alternative meanings to
be given to the words of a statute or between rival interpretations of
what a precedent `amounts to'. It is only the tradition that judges
`find' and do not `make' law that conceals this, and presents their
decisions as if they were deductions smoothly made from clear
pre-existing rules without intrusion of the judge's choice. Legal rules
may have a central core of undisputed meaning, and in some cases it may
be difficult to imagine a dispute as to the meaning of a rule breaking
out. The provision of s. 9 of the Wills Act, 1837, that there must be
two witnesses to a will may not seem likely to raise problems of
interpretation. Yet all rules have a penumbra of uncertainty where the
judge must choose between alternatives. Even the meaning of the
innocent-seeming provision of the Wills Act that the testator must
\emph{sign} the will may prove doubtful in certain circumstances. What
if the testator used a pseudonym? Or if his hand was guided by another?
Or if he wrote his initials only? Or if he put his full, correct, name
unaided, but at the top of the first page instead of at the bottom of
the last? Would all these cases be `signing' within the meaning of the
legal rule?

If so much uncertainty may break out in humble spheres of private law,
how much more shall we find in the magniloquent phrases of a
constitution such as the Fifth and Fourteenth Amendments to the
Constitution of the United States, providing that no person shall be
`deprived of life liberty or property without due process of law'? Of
this one writer \footnote{J. D. March, `Sociological Jurisprudence
  Revisited', 8 \emph{Stanford Law Review} (1956), p.~518.} has said
that the true meaning of this phrase is really quite clear. It means `no
\emph{w} shall be \emph{x} or \emph{y} without \emph{z} where \emph{w,
x, y}, and \emph{z} can assume any values within a wide range'. To cap
the tale sceptics remind us that not only are the rules uncertain, but
the court's interpretation of them may be not only authoritative but
final. In view of all this, is not the conception of law as essentially
a matter of rules a gross exaggeration if not a mistake? Such thoughts
lead to the paradoxical denial which we have already cited: `Statutes
are sources of law, not part of the law itself.'\footnote{Gray, loc.
  cit.}

\subsection{\texorpdfstring{\textbf{3.
DEFINITION}}{3. DEFINITION}}\label{definition}

Here then are the three recurrent issues: How does law differ from and
how is it related to orders backed by threats? How does legal obligation
differ from, and how is it related to, moral obligation? What are rules
and to what extent is law an affair of rules? To dispel doubt and
perplexity on these three issues has been the chief aim of most
speculation about the `nature' of law. It is possible now to see why
this speculation has usually been conceived as a search for the
definition of law, and also why at least the familiar forms of
definition have done so little to resolve the persistent difficulties
and doubts. Definition, as the word suggests, is primarily a matter of
drawing lines or distinguishing between one kind of thing and another,
which language marks off by a separate word. The need for such a drawing
of lines is often felt by those who are perfectly at home with the
day-to-day use of the word in question, but cannot state or explain the
distinctions which, they sense, divide one kind of thing from another.
All of us are sometimes in this predicament: it is fundamentally that of
the man who says, `I can recognize an elephant when I see one but I
cannot define it.' The same predicament was expressed by some famous
words of St Augustine \footnote{\emph{Confessiones}, xiv. 17.} about the
notion of time. `What then is time? If no one asks me I know: if I wish
to explain it to one that asks I know not.' It is in this way that even
skilled lawyers have felt that, though they know the law, there is much
about law and its relations to other things that they cannot explain and
do not fully understand. Like a man who can get from one point to
another in a familiar town but cannot explain or show others how to do
it, those who press for a definition need a map exhibiting clearly the
relationships dimly felt to exist between the law they know and other
things.

Sometimes in such cases a definition of a word can supply such a map: at
one and the same time it may make explicit the latent principle which
guides our use of a word, and may exhibit relationships between the type
of phenomena to which we apply the word and other phenomena. It is
sometimes said that definition is `merely verbal' or `just about words';
but this may be most misleading where the expression defined is one in
current use. Even the definition of a triangle as a `three-sided
rectilinear figure', or the definition of an elephant as a `quadruped
distinguished from others by its possession of a thick skin, tusks, and
trunk', instructs us in a humble way both as to the standard use of
these words and about the things to which the words apply. A definition
of this familiar type does two things at once. It simultaneously
provides a code or formula translating the word into other
well-understood terms and locates for us the kind of thing to which the
word is used to refer, by indicating the features which it shares in
common with a wider family of things and those which mark it off from
others of that same family. In searching for and finding such
definitions we `are looking not merely at words \ldots{} but also at the
realities we use words to talk about. We are using a sharpened awareness
of words to sharpen our perception of the phenomena.' \footnote{J. L.
  Austin, `A Plea for Excuses', \emph{Proceedings of the Aristotelian
  Society}, vol.~57 (1956--7), p.~8.}

This form of definition (\emph{per genus et differentiam}) which we see
in the simple case of the triangle or elephant is the simplest and to
some the most satisfying, because it gives us a form of words which can
always be substituted for the word defined. But it is not always
available nor, when it is available, always illuminating. Its success
depends on conditions which are often not satisfied. Chief among these
is that there should be a wider family of things or \emph{genus}, about
the character of which we are clear, and within which the definition
locates what it defines; for plainly a definition which tells us that
something is a member of a family cannot help us if we have only vague
or confused ideas as to the character of the family. It is this
requirement that in the case of law renders this form of definition
useless, for here there is no familiar well-understood general category
of which law is a member. The most obvious candidate for use in this way
in a definition of law is the general family of \emph{rules of
behaviour}; yet the concept of a rule as we have seen is as perplexing
as that of law itself, so that definitions of law that start by
identifying laws as a species of rule usually advance our understanding
of law no further. For this, something more fundamental is required than
a form of definition which is successfully used to locate some special,
subordinate, kind within some familiar, well-understood, general kind of
thing.

There are, however, further formidable obstacles to the profitable use
of this simple form of definition in the case of law. The supposition
that a general expression can be defined in this way rests on the tacit
assumption that all the instances of what is to be defined as triangles
and elephants have common characteristics which are signified by the
expression defined. Of course, even at a relatively elementary stage,
the existence of borderline cases is forced upon our attention, and this
shows that the assumption that the several instances of a general term
must have the same characteristics may be dogmatic. Very often the
ordinary, or even the technical, usage of a term is quite `open' in that
it does not \emph{forbid} the extension of the term to cases where only
some of the normally concomitant characteristics are present. This, as
we have already observed, is true of international law and of certain
forms of primitive law, so that it is always possible to argue with
plausibility for and against such an extension. What is more important
is that, apart from such borderline cases, the several instances of a
general term are often linked together in quite different ways from that
postulated by the simple form of definition. They may be linked by
analogy as when we speak of the `foot' of a man and also of the `foot'
of a mountain. They may be linked by \emph{different} relationships to a
central element. Such a unifying principle is seen in the application of
the word `healthy' not only to a man but to his complexion and to his
morning exercise; the second being a \emph{sign} and the third a
\emph{cause} of the first central characteristic. Or again---and here
perhaps we have a principle similar to that which unifies the different
types of rules which make up a legal system---the several instances may
be different constituents of some complex activity. The use of the
adjectival expression `railway' not only of a train but also of the
lines, of a station, of a porter, and of a limited company, is governed
by this type of unifying principle.

There are of course many other kinds of definition besides the very
simple traditional form which we have discussed, but it seems clear,
when we recall the character of the three main issues which we have
identified as underlying the recurrent question `What is law?', that
nothing concise enough to be recognized as a definition could provide a
satisfactory answer to it. The underlying issues are too different from
each other and too fundamental to be capable of this sort of resolution.
This the history of attempts to provide concise definitions has shown.
Yet the instinct which has often brought these three questions together
under a single question or request for definition has not been
misguided; for, as we shall show in the course of this book, it is
possible to isolate and characterize a central set of elements which
form a common part of the answer to all three. What these elements are
and why they deserve the important place assigned to them in this book
will best emerge, if we first consider, in detail, the deficiencies of
the theory which has dominated so much English jurisprudence since
Austin expounded it. This is the claim that the key to the understanding
of law is to be found in the simple notion of an order backed by
threats, which Austin himself termed a `command'. The investigation of
the deficiencies of this theory occupies the next three chapters. In
criticizing it first and deferring to the later chapters of this book
consideration of its main rival, we have consciously disregarded the
historical order in which modern legal theory has developed; for the
rival claim that law is best understood through its `necessary'
connection with morality is an older doctrine which Austin, like Bentham
before him, took as a principal object of attack. Our excuse, if one is
needed, for this unhistorical treatment, is that the errors of the
simple imperative theory are a better pointer to the truth than those of
its more complex rivals.

At various points in this book the reader will find discussions of the
borderline cases where legal theorists have felt doubts about the
application of the expression `law' or `legal system', but the suggested
resolution of these doubts, which he will also find here, is only a
secondary concern of the book. For its purpose is not to provide a
definition of law, in the sense of a rule by reference to which the
correctness of the use of the word can be tested; it is to advance legal
theory by providing an improved analysis of the distinctive structure of
a municipal legal system and a better understanding of the resemblances
and differences between law, coercion, and morality, as types of social
phenomena. The set of elements identified in the course of the critical
discussion of the next three chapters and described in detail in
Chapters V and VI serve this purpose in ways which are demonstrated in
the rest of the book. It is for this reason that they are treated as the
central elements in the concept of law and of prime importance in its
elucidation.

\newpage

\section{NOTES}
\addcontentsline{toc}{chapter}{NOTES}

The text of this book is self-contained, and the reader may find it best to read each chapter through before turning to these notes. The footnotes in the text give only the sources of quotations, and references to cases or statutes cited. The following notes are designed to bring to the reader's attention matters of three different kinds, viz.:

\begin{enumerate}[label=(\roman*)]
  \item Further illustrations or examples of general statements made in the text;
  \item Writings in which the views adopted or referred to in the text are further expounded or criticized;
  \item Suggestions for the further investigation of questions raised in the text.
\end{enumerate}

All references to this book are indicated simply by chapter and section numbers, e.g. Chapter~1, s.~1. The following abbreviations are used:

\vspace{1em}

\begin{tabular}{@{}l p{10cm}@{}}
\textbf{Austin, The Province}   & \textit{Austin, The Province of Jurisprudence Determined} (ed. Hart, London, 1954) \\
\textbf{Austin, The Lectures}   & \textit{Austin, Lectures on the Philosophy of Positive Law} \\
\textbf{Kelsen, General Theory} & \textit{Kelsen, General Theory of Law and State} \\
\textbf{BYBIL}                  & \textit{British Year Book of International Law} \\
\textbf{HLR}                    & \textit{Harvard Law Review} \\
\textbf{LQR}                    & \textit{Law Quarterly Review} \\
\textbf{MLR}                    & \textit{Modern Law Review} \\
\textbf{PAS}                    & \textit{Proceedings of the Aristotelian Society} \\
\end{tabular}

\subsection{CHAPTER I NOTES}\label{chapter-i-notes}

\emph{Pages} 1--2. Each of the quotations on these pages from Llewellyn,
Holmes, Gray, Austin, and Kelsen, are paradoxical or exaggerated ways of
emphasizing some aspect of law which, in the author's view, is either
obscured by ordinary legal terminology, or has been unduly neglected by
previous theorists. In the case of any important jurist, it is
frequently profitable to defer consideration of the question whether his
statements about law are literally true or false, and to examine first,
the detailed reasons given by him in support of his statements and
secondly, the conception or theory of law which his statement is
designed to displace.

A similar use of paradoxical or exaggerated assertions, as a method of
emphasizing neglected truths is familiar in philosophy. See J. Wisdom,
`Metaphysics and Verification' in \emph{Philosophy and Psychoanalysis}
(1953); Frank, \emph{Law and the Modern Mind} (London, 1949), Appendix
VII (`Notes on Fictions').

The doctrines asserted or implied in each of the five quotations on
these pages are examined in Chapter VII, ss. 2 and 3 (Holmes, Gray, and
Llewellyn); Chapter IV, ss. 3 and 4 (Austin); and Chapter III, s. 1,
pp.~35--42 (Kelsen).

\emph{Page} 4. \emph{Standard cases and borderline cases.} The feature
of language referred to here is generally discussed under the heading of
`The Open Texture of Law' in Chapter VII, s. 1. It is something to be
kept in mind not only when a definition is expressly sought for general
terms like `law', `state', `crime', \&c., but also when attempts are
made to characterize the reasoning involved in the application of rules,
framed in general terms, to particular cases. Among legal writers who
have stressed the importance of this feature of language are: Austin,
\emph{The Province}, Lecture VI, pp.~202--7, and \emph{Lectures in
Jurisprudence} (5th edn., 1885), p.~997 (`Note on Interpretation');
Glanville Williams, `International Law and the Controversy Concerning
the Word ``Law''\,', \emph{22 BYBIL} (1945), and `Language in the Law'
(five articles), \emph{61} and \emph{62 LQR} (1945--6). On the latter,
however, see comments by J. Wisdom in `Gods' and in `Philosophy,
Metaphysics and Psycho-Analysis', both in \emph{Philosophy and
Psycho-Analysis} (1953).

\emph{Page} 6. \emph{Austin on obligation.} See \emph{The Province,}
Lecture I, pp.~14--18; \emph{The Lectures}, Lectures 22 and 23. The idea
of obligation and the differences between `having an obligation' and
`being obliged' by coercion are examined in detail in Chapter V, s. 2.
On Austin's analysis see notes to Chapter II, below, p.~282.

\emph{Page} 8. \emph{Legal and moral obligation.} The claim that law is
best understood through its connection with morality is examined in
Chapters VIII and IX. It has taken very many different forms. Sometimes,
as in the classical and scholastic theories of Natural Law, this claim
is associated with the assertion that fundamental moral distinctions are
`objective truths' discoverable by human reason; but many other jurists,
equally concerned to stress the interdependence of law and morals, are
not committed to this view of the nature of morality. See notes to
Chapter IX, below, p.~302.

\emph{Page} 10. \emph{Scandinavian legal theory and the idea of a
binding rule.} The most important works of this school, for English
readers, are Hägerström (1868--1939), \emph{Inquiries into the Nature of
Law and Morals} (trans. Broad, 1953), and Olivecrona, \emph{Law as Fact}
(1939). The clearest statement of their views on the character of legal
rules is to be found in Olivecrona, op. cit. His criticism of the
predictive analysis of legal rules favoured by many American jurists
(see op. cit., pp.~85--8, 213--15) should be compared with the similar
criticisms in Kelsen, \emph{General Theory} (pp.~165 ff., `The
Prediction of the Legal Function'). It is worth inquiring why such
different conclusions as to the character of legal rules are drawn by
these two jurists in spite of their agreement on many points. For
criticisms of the Scandinavian School, see Hart, review of Hägerström,
op. cit. in 30 \emph{Philosophy} (1955); `Scandinavian Realism',
\emph{Cambridge Law Journal} (1959); Marshall, `Law in a Cold Climate',
\emph{Juridical Review} (1956).

\emph{Page} 12. \emph{Rule-scepticism in American legal theory.} See
Chapter VII, ss. 1 and 2 on `Formalism and Rule-scepticism', where some
of the principal doctrines which have come to be known as `Legal
Realism' are examined.

\emph{Pages} 12--13. \emph{Doubt as to meaning of common words.} For
cases on the meaning of `sign' or `signature' see 34 Halsbury,
\emph{Laws of England} (2nd edn.), paras. 165--9 and In the Estate of
Cook (1960), 1 AER 689 and cases there cited.

\emph{Page} 13. \emph{Definition.} For a general modern view of the
forms and functions of definition see Robinson, \emph{Definition}
(Oxford, 1952). The inadequacy of the traditional definition \emph{per
genus et differentiam} as a method of elucidating legal terms is
discussed by Bentham, \emph{Fragment on Government} (notes to Chapter V,
s. 6), and Ogden, \emph{Bentham's Theory of Fictions} (pp.~75--104). See
also Hart, `Definition and Theory in Jurisprudence', \emph{70 LQR}
(1954), and Cohen and Hart, `Theory and Definition in Jurisprudence,'
\emph{PAS} Suppl. vol.~xxix (1955).

For the definition of the term `law' see Glanville Williams, op. cit.;
R. Wollheim, `The Nature of Law' in \emph{2 Political Studies} (1954);
and Kantorowicz, \emph{The Definition of Law} (1958), esp.~Chapter 1. On
the general need for, and clarificatory function of, a definition of
terms, though no doubts are felt about their day-to-day use in
particular cases, see Ryle, \emph{Philosophical Arguments} (1945);
Austin, `A Plea for Excuses', \emph{57 PAS} (1956--7), pp.~15 ff.

\emph{Page} 15. \emph{General terms and common qualities.} The
uncritical belief that if a general term (e.g.~`law', `state', `nation',
`crime', `good', `just') is correctly used, then the range of instances
to which it is applied must all share `common qualities' has been the
source of much confusion. Much time and ingenuity has been wasted in
jurisprudence in the vain attempt to discover, for the purposes of
definition, the common qualities which are, on this view, held to be the
\emph{only} respectable reason for using the same word of many different
things (see Glanville Williams, op. cit. It is however important to
notice that this mistaken view of the character of general words does
not always involve the further confusion of `verbal questions' with
questions of fact which this author suggests).

Understanding of the different ways in which the several instances of a
general term may be related is of particular importance in the case of
legal, moral, and political terms. For analogy: see Aristotle,
\emph{Nicomachean Ethics}, i, ch.~6 (where it is suggested that the
different instances of `good' may be so related), Austin, \emph{The
Province}, Lecture V, pp.~119--24. For different relationships to a
central case, e.g.~healthy: see Aristotle, \emph{Categories}, chap.~1
and examples in \emph{Topics}, 1, chap.~15, ii, chap.~9, of `paronyms'.
For the notion of `family resemblance': see Wittgenstein,
\emph{Philosophical Investigations}, i, paras. 66--76. Cf. Chapter VIII,
s. 1 on the structure of the term `just'. Wittgenstein's advice (op.
cit., para. 66) is peculiarly relevant to the analysis of legal and
political terms. Considering the definition of `game' he said, `Don't
say there \emph{must} be something common or they would not be called
'games', but \emph{look} and \emph{see} whether there is anything common
to all. For if you look at them you will not see anything common to
\emph{all} but similarities, relationships, and a whole series at that.'

\subsection{CHAPTER I 3rd ed.~NOTES}\label{chapter-i-3rd-ed.-notes}

Hart's own notes have been left unaltered in this edition. They remain
useful for understanding nuances in his argument, thoughts not
elaborated in the main text, and some comparisons between his own views
and those of others. As a scholarly resource, however, they are often
superseded. What follows are pointers to more recent work in English
that elaborates or criticizes his arguments. No attempt is made to be
comprehensive---the literature is enormous---but only to suggest some
items that will be of particular use to students. Where possible I have
chosen works by Hart's most persistent interlocutors and by others who
directly engage or develop his writings. There are many general works on
Hart's legal philosophy. Two good book-length treatments are Neil
MacCormick, \emph{H. L. A. Hart} (2nd edn., Stanford University Press,
2008; all page references in these notes are to 1st edn., 1981) and
Michael D. Bayles, \emph{Hart's Legal Philosophy: An Examination}
(Kluwer Academic, 1992). A brief overview is Joseph Raz's obituary `H.
L. A. Hart (1907--1992)' (1993) 5 \emph{Utilitas} 145. For Hart's life
and influences, see Nicola Lacey, \emph{A Life of H. L. A. Hart: The
Nightmare and the Noble Dream} (Oxford University Press, 2004).

\emph{Pages} 3--4. \emph{Common knowledge about the law}. For discussion
of the importance of ordinary understandings, and self-understanding, to
legal theory see Joseph Raz, \emph{Between Authority and Interpretation}
(Oxford University Press, 2009), chap.~2.

\emph{Page} 4. \emph{Borderline cases of legal systems}. Hart's point is
that philosophic controversy about law does not usually result from the
existence of borderline cases. Ronald Dworkin agrees: see \emph{Law's
Empire} (Harvard University Press, 1986) 40--3. For doubts about whether
domestic legal systems \emph{are} the central case of law see John
Griffiths, `What is Legal Pluralism?' (1986) 24 \emph{Journal of Legal
Pluralism \& Unofficial Law} 1; William Twining, \emph{General
Jurisprudence: Understanding Law from a Global Perspective} (Cambridge
University Press, 2009), chap.~4; and Keith Culver and Michael Giudice,
\emph{Legality's Borders} (Oxford University Press, 2010).

\emph{Pages} 6--13. \emph{Recurrent issues}. Hart later thought of the
agenda for legal philosophy somewhat differently. In 1967 he added to
the issues treated in this book problems of legal reasoning and problems
in the criticism of law, including the appropriate standards to judge
law by, and the basis for law's moral authority. See `Problems of the
Philosophy of Law', chap.~3 of his \emph{Essays in Jurisprudence and
Philosophy} (Oxford University Press, 1983).

For contrasting views of the agenda for legal theory see Ronald Dworkin,
\emph{Taking Rights Seriously} (rev. edn., Harvard University Press,
1978) 14--16; Ronald Dworkin, \emph{Law's Empire} 1--6; John Finnis,
\emph{Natural Law and Natural Rights} (2nd edn., Oxford University
Press, 2011), chap.~1; and Hugh Collins, \emph{Marxism and Law} (Oxford
University Press, 1984) chap.~1. For an assessment of progress on Hart's
agenda see Leslie Green, `General Jurisprudence: A 25th Anniversary
Essay' (2005) 25 \emph{Oxford Journal of Legal Studies} 565.

\emph{Page} 14. \emph{The relationship between word and object}. See P.
M. S. Hacker, `Hart's Philosophy of Law' in P. M. S. Hacker and J. Raz
eds., \emph{Law, Morality, and Society: Essays in Honour of H. L. A.
Hart} (Oxford University Press, 1977) esp.~2--12; and Neil MacCormick,
\emph{H. L. A. Hart} 12--19. Dworkin interprets Hart as holding `that
lawyers all follow certain linguistic criteria for judging propositions
of law'---this is the basis of the `semantic sting' argument,
\emph{Law's Empire} 45--6. Criticism of this aspect of Hart's
methodology can also be found in Nicos Stavropoulos, `Hart's Semantics'
in Jules Coleman ed., \emph{Hart's Postscript} (Oxford University Press,
2001) and, on different grounds, in Brian Leiter, `Beyond the
Hart/Dworkin Debate: The Methodology Problem in Jurisprudence'; (2003)
48 \emph{American Journal of Jurisprudence} 17, esp.~43--51.

On the semantics of `law' see Jules Coleman and Ori Simchen, `Law'
(2003) 9 \emph{Legal Theory} 1. For doubts about whether jurisprudence
has any stake in semantics see Joseph Raz, \emph{Ethics in the Public
Domain} (rev. edn., Oxford University Press, 1995, chap.~9, esp.~195--8)
and Joseph Raz, \emph{Between Authority and Interpretation}, 49--59. For
general doubts about the linguistic approach to political theory see
David Miller, `Linguistic Philosophy and Political Theory', in David
Miller and Larry Siedentop eds., \emph{The Nature of Political Theory}
(Oxford University Press, 1983).

\emph{Pages} 15--16. \emph{Definition per genus et differentiam}. For a
critique of Hart's position see P. M. S. Hacker, `Definition in
Jurisprudence' (1969) 19 \emph{Philosophical Quarterly} 343.

\emph{Page} 16. \emph{A central set of elements which form a common part
of the answer.} They are presented at 91--9. Whether or not these amount
to a `conceptual analysis' of `law' or `legal system' depends on what
one takes such an analysis to require. Compare Frank Jackson, \emph{From
Metaphysics to Ethics: A Defense of Conceptual Analysis} (Oxford
University Press, 1998) and Colin McGinn, \emph{Truth by Analysis}:
\emph{Games, Names, and Philosophy} (Oxford University Press, 2012),
esp.~chap.~2. On the relationship between a theory of Hart's sort and
sociological theory see H. L. A. Hart, `Analytical Jurisprudence in
Mid-Twentieth Century: A Reply to Professor Bodenheimer' (1956) 105
\emph{University of Pennsylvania Law Review} 953; M. Krygier, `\,``The
Concept of Law'' and Social Theory' (1982) 2 \emph{Oxford Journal of
Legal Studies} 155; B. Z. Tamanaha, `Socio-Legal Positivism and a
General Jurisprudence' (2001) 21 \emph{Oxford Journal of Legal Studies}
1; and Denis Galligan, `Legal Theory and Empirical Research' in Peter
Cane and Herbert Kritzer eds., \emph{Oxford Handbook of Empirical Legal
Research} (Oxford University Press, 2010).

\subsection{FOOTNOTES CHAPTER
I}\label{footnotes-chapter-i}  % Pandoc 본문 삽입 지점

\end{document}
