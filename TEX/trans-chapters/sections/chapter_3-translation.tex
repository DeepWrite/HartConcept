\documentclass[12pt, oneside]{book}  % 13pt는 직접 설정함

% --- Language and Font Settings ---
\usepackage[english]{babel}
\usepackage{fontspec}
\usepackage{kotex}  % Korean typesetting

% --- Font Configuration (Main + Korean) ---
\usepackage{libertinus}
\setmainhangulfont[
  Path = ./,
  UprightFont = *Batang Light.ttf,
  BoldFont    = *Batang Medium.ttf
]{KoPubWorld}

% --- Page Geometry: A4 with wide margins for print readability ---
\usepackage[a4paper, margin=1in]{geometry}

% --- Line Spacing ---
\usepackage{setspace}
\setstretch{1.45}

% --- Section Formatting: Disable numbering ---
\usepackage{titlesec}
\setcounter{secnumdepth}{0}

\titleformat{\chapter}[display]
  {\normalfont\Huge\bfseries}
  {}{0pt}{\Huge}\titleformat{\section}
  {\normalfont\Large\bfseries}
  {}{0pt}{\Large}
\titleformat{\subsection}
  {\normalfont\large\bfseries}
  {}{0pt}{\large}

% --- Math and Tables ---
\usepackage{amsmath, amssymb, amsthm, mathtools}
\usepackage{graphicx}
\usepackage{enumitem}
\usepackage{tabularx, booktabs}
\usepackage{footmisc}

% --- Quotes, Hyperlink, Header/Footer ---
\usepackage{csquotes}
\usepackage[hidelinks]{hyperref}
\usepackage{fancyhdr}
\pagestyle{fancy}
\fancyhf{}
\fancyfoot[C]{\thepage}
\setlength{\headheight}{15pt}

% --- Bibliography (biblatex) ---
\usepackage[style=verbose-note, backend=biber, maxbibnames=99]{biblatex}
\addbibresource{references.bib}
\AtEveryBibitem{\clearfield{pages}}

% --- Title Info ---
\title{\Huge\textsc{The Concept of Law} \\[2ex] \Large Third Edition}
\author{\Large H. L. A. Hart}
\date{}

\begin{document}

% --- Title Page ---
\begin{titlepage}
  \centering
  \vspace*{3cm}
  {\Huge\textsc{The Concept of Law}}\\[1.5ex]
  {\Large Third Edition}\\[4ex]
  \textsc{H. L. A. Hart}\\[6ex]
  {\small 2025 SUMMER 강의용 한국어판\\
  강좌 외의 사용을 불허함}
  \vfill
\end{titlepage}

\section{\texorpdfstring{\textbf{III. 법의
다양성}}{III. 법의 다양성}}\label{iii.-uxbc95uxc758-uxb2e4uxc591uxc131}

현대의 법체계, 예컨대 영국법에서 발견되는 다양한 법의 종류들을 앞 장에서
구성한 강제 명령이라는 단순한 모델과 비교해보면, 무수한 이의들이 즉시
떠오른다. 모든 법이 사람들에게 어떤 일을 하거나 하지 말라고 명령하는
것은 분명 아니다. 유언, 계약, 결혼을 할 수 있는 사적 권한을 개인에게
부여하는 법이나, 판사에게 재판 권한을, 장관에게 규칙 제정 권한을, 지방
자치 단체에게 조례 제정 권한을 부여하는 법을 그렇게 분류하는 것은 오해를
불러일으키는 것 아닐까? 모든 법이 제정된 것도 아니며, 모두가 어떤 사람의
바람을 표현한 것---우리 모델에서처럼 일반 명령의 형태로---도 아니다.
대부분의 법체계에서 관습이 일정한 자리를 차지한다는 점에서 이는 사실과
다르다. 법률은, 그것이 의도적으로 제정된 것일지라도, 반드시
\emph{타인}에게만 적용되는 명령이어야만 하는가? 입법자가 자신에게도
구속력을 갖는 법률을 제정하는 경우는 없지 않은가? 마지막으로, 제정된
법이 법이기 위해서는 반드시 어떤 입법자의 실제 바람이나 의도, 소망을
표현해야만 하는가? 만약 법안이 적절한 절차를 거쳐 통과되었지만 (영국의
재정법 조항 중 많은 경우가 그렇듯) 표결에 참여한 사람들이 그것이 무슨
의미인지조차 몰랐다면, 그것은 법이 아닌가?

이러한 이의들은 가능한 수많은 반론들 중에서 가장 중요한 것들이다. 분명히
원래의 단순한 모델에 일정한 수정을 가해야만 이러한 반론에 대응할 수 있을
것이며, 그 모든 반론을 수용한 뒤에는 위협에 의해 뒷받침된 일반
명령이라는 개념이 원형을 알아볼 수 없을 정도로 변형되어 있을 수도 있다.

우리가 언급한 반론들은 세 가지 주요 부류로 나눌 수 있다. 일부는 법의
\emph{내용}에 관한 것이고, 다른 일부는 법의 \emph{기원 방식}, 그리고
나머지는 법의 \emph{적용 범위}에 관한 것이다. 모든 법체계는---적어도
겉보기에는---이 세 가지 사항 중 하나 이상에 있어 우리가 설정한 일반 명령
모델과는 다른 법들을 포함하고 있는 것처럼 보인다. 이 장의 나머지
부분에서는 이러한 세 가지 유형의 반론을 각각 따로 고찰할 것이다. 그리고
다음 장에서는 이보다 더 근본적인 비판, \textbf{(p.~27)} 즉 내용, 기원
방식, 적용 범위와 관련된 반론들을 넘어서, 이 모델이 전제하는 바---즉
최고권위자이자 독립된 주권자가 일반적으로 복종받는다는 관념---가 실제
법체계에서는 거의 대응되는 것이 없다는 점에서, 이 전체 개념 자체가
오해를 불러일으킨다는 비판을 다루기로 하겠다.

\subsection{\texorpdfstring{\textbf{1. 법의
내용}}{1. 법의 내용}}\label{uxbc95uxc758-uxb0b4uxc6a9}

형법은 우리가 복종하거나 위반하는 대상이며, 그 규칙이 요구하는 바는
`의무'라고 불린다. 이를 위반하면 우리는 법을 `어긴(break)' 것이며, 그
행위는 법적으로 `잘못(wrong)', `의무 위반(breach of duty)', 혹은
`범죄(offence)'라 불린다. 형사 법령이 수행하는 사회적 기능은 특정 행위
유형을 피하거나 실행해야 하는 것으로 설정하고 정의하는 데 있으며, 이는
그 법이 적용되는 자의 의사와 무관하다. 형법이 위반되었을 때 부과되는
형벌 또는 `제재(sanction)'는 (형벌이 다른 목적을 지닐 수도 있지만)
이러한 행위를 하지 않도록 유인하는 하나의 동기를 제공하려는 것이다. 이
모든 점에서 형법과 그 제재는 우리 모델의 위협에 의해 뒷받침된 일반
명령과 강한 유사성을 갖는다. 이러한 일반 명령과 불법행위법(tort law)
사이에도 일정한 유사성이 존재하는데, 후자는 본질적으로 타인의 행위로
인해 피해를 입은 개인에게 보상을 제공하는 것을 주목적으로 한다. 여기서도
어떤 행위 유형이 법적으로 문제 삼을 수 있는 잘못(actionable wrong)이
되는지를 결정하는 규칙들은 개인에게, 그 의사와 무관하게, 해당 행위를
하지 말라는 `의무(duties)'(혹은 드물게는 `법적 의무(obligations)')를
부과하는 것으로 서술된다. 이러한 행위는 `의무 위반(breach of duty)'이라
불리며, 보상이나 기타 법적 구제수단은 `제재(sanction)'라 불린다. 그러나
법 중에는 이러한 위협에 의해 뒷받침된 명령과의 유사성이 전혀 성립하지
않는 중요한 부류가 존재하며, 이는 전혀 다른 사회적 기능을 수행한다.
유효한 계약이나 유언, 결혼이 어떻게 성립하는지를 정의하는 법적 규칙들은
사람들이 원하든 원하지 않든 특정 방식으로 행동할 것을 요구하지 않는다.
이러한 법들은 의무나 책임을 부과하지 않으며, 대신 개인이 법이 정한 특정
절차와 조건에 따라 법적 권리와 의무의 구조를 창출할 수 있는 법적
권한(legal powers)을 부여함으로써 \textbf{(p.~28)} 자신의 소망을 실현할
수 있는 \emph{편의들(facilities)}을 제공한다.

계약, 유언, 결혼 등을 통해 개인이 타인과의 법적 관계를 형성할 수 있는
이러한 권한의 부여는 법이 사회생활에 기여한 가장 위대한 성과 중
하나이며, 모든 법을 위협에 의한 명령으로 환원하여 설명하려는 관점에 의해
간과되어 왔다. 이러한 권한을 부여하는 법들과 형사 법령 사이의 기능적
차이는 우리가 이 부류의 법에 대해 일상적으로 사용하는 언어에서도 분명히
드러난다. 우리는 유언서를 작성할 때 1837년 유언법 제9조(s. 9 of the
Wills Act, 1837)에 명시된 증인 수 요건을 `준수(compliance)'할 수도 있고
하지 않을 수도 있다. 준수하지 않을 경우, 우리가 작성한 문서는 권리와
의무를 창출하는 `유효(valid)'한 유언이 아니라 법적 `효력(force)'이나
`효과(effect)'가 없는 `무효(nullity)'가 된다. 그러나 그것이 무효라고
하여, 해당 법 조항을 준수하지 않은 것이 어떤 `의무나 책임의 위반(breach
or violation of duty)'이거나 `범죄(offence)'가 되는 것은 아니며, 이를
그런 식으로 이해하는 것은 혼란을 초래할 수 있다.

사적 개인에게 법적 권한을 부여하는 다양한 법 규칙들을 살펴보면, 그들
자체도 서로 구분되는 유형들로 나뉜다는 것을 알 수 있다. 예컨대, 유언이나
계약을 할 수 있는 권한 뒤에는, 그 권한을 행사하는 자가 갖추어야 할
최소한의 개인적 자격---예를 들어 성인일 것, 정신적으로 온전할 것 등---에
관한 \emph{능력(capacity)} 규칙들이 존재한다. 또 다른 규칙들은 그 권한이
행사되어야 하는 방식과 형식을 구체화하고, 유언이나 계약이 구두로
가능한지, 서면으로 해야 하는지, 서면이라면 어떤 형식으로 작성되고 증인이
있어야 하는지 등을 규정한다. 또 다른 규칙들은 그러한 법적 행위를 통해
창출될 수 있는 권리·의무 구조의 유형, 최대·최소 지속기간 등을 제한한다.
이러한 예로는 계약에 관한 공서양속 규칙이나, 유언·신탁에서의 축적금지
규칙들이 있다.

이후 우리는 ``당신이 이것을 하고자 한다면, 이렇게 하라''고 말하는 권한
부여 법들을, ``당신이 원하든 원하지 않든, 이렇게 하라''고 말하는 형법과
같은 위협 기반 명령과 동일시하려는 시도들을 살펴볼 것이다. 하지만
여기에서는, 이러한 사적 권한 부여 법들과는 대조적으로
\emph{공적(public)} 혹은 \emph{공무상(official)} 성격의 법적 권한을
부여하는 법들, \textbf{(p.~29)} 즉 사법, 입법, 행정이라는 정부의 세
영역(전통적으로 다소 모호하게 구분된 영역들)에서 발견되는 법들에 대해
살펴보겠다.

우선, 법원이 작동하는 데 배경이 되는 법들을 생각해 보자. 법원의 경우,
어떤 규칙들은 판사가 관할권을 행사할 수 있는 사안과 범위를 구체화하고,
우리는 이를 흔히 그가 특정 유형의 사건을 `재판할 권한(power to try)'을
가진다고 표현한다. 또 다른 규칙들은 판사의 임명 방식, 자격 요건, 임기
등을 규정하고, 또 다른 규칙들은 판사의 바람직한 행태에 대한 기준을
제시하거나 법정에서 따라야 할 절차를 정한다. 이러한 규칙들은 일종의 사법
법전(judicial code)을 형성하며, 1959년 카운티 법원법(County Courts Act,
1959), 1907년 형사항소법(Court of Criminal Appeal Act, 1907), 미국
연방법전 제28편(Title 28 of the United States Code) 등에서 그 예를 찾을
수 있다. 이러한 법령들에서 법원이 구성되고 정상적으로 작동하기 위한
다양한 규정들을 살펴보는 것은 유익하다. 이들 규정 중 상당수는 언뜻
보기에는 판사에게 무언가를 하거나 하지 말라고 명령하는 조항처럼 보이지
않는다. 물론, 법이 별도의 규칙을 통해 판사가 자신의 관할권을 넘어서는
행위나 금전적 이해관계가 있는 사건을 맡는 것을 금지하고, 이를 위반할
경우 처벌하는 것을 금할 수는 있지만, 이러한 규칙은 그에게 사법 권한을
부여하고 그의 관할권을 정의하는 규칙에 추가되는 성격을 갖는다. 권한을
부여하는 규칙의 주된 목적은 판사의 부적절한 행위를 억제하려는 것이
아니라, 법원의 결정이 유효하기 위한 조건과 한계를 규정하는 것이다.

이어서, 법원의 관할권 범위를 규정하는 전형적인 조항 하나를 간략히
살펴보는 것이 유익할 것이다. 가장 단순한 예로, 개정된 1959년 카운티
법원법에 따라 토지 반환 소송에 대한 관할권이 카운티 법원에 부여된 조항을
들 수 있다. 이 조항의 언어는 `명령'의 언어와는 매우 동떨어져 있으며
다음과 같다:

\begin{quote}
카운티 법원은, 문제된 토지의 연간 순과세가액이 100파운드를 초과하지 않을
경우, 해당 토지 반환 소송을 심리하고 판결할 수 있는 관할권을
갖는다.\footnote{Section 48 (1).}
\end{quote}

만약 카운티 법원 판사가 연간 가액이 100파운드를 초과하는 토지에 대해
관할권을 초과하여 재판하고 \textbf{(p.~30)} 이에 대해 명령을 내렸다면,
그 판사나 당사자들 중 누구도 \emph{범죄(offence)}을 저지른 것은 아니다.
그러나 이 경우는 개인이 유언을 할 때 어떤 조건을 충족하지 않아
`무효(nullity)'가 되는 문서를 작성한 상황과는 다소 다르다. 예컨대
유언자가 서명하거나 두 명의 증인을 확보하지 않은 경우, 그가 작성한
문서는 어떠한 법적 지위나 효력도 갖지 못한다. 하지만 법원의 명령은, 설령
그것이 명백히 법원의 관할권 밖에 속하는 사안이라 할지라도, 이런 식으로
처리되지는 않는다. 공공 질서의 유지라는 관점에서, 상급 법원이 그 명령의
무효를 확정하기 전까지는 해당 결정이 법적 권위를 갖는 것으로 간주되는
것이 바람직하기 때문이다. 따라서 관할권 초과로 인한 명령이라 하더라도,
항소를 통해 무효화되기 전까지는 당사자들 간의 법적으로 유효한 명령으로
존속하며 집행된다. 그러나 그것은 법적 결함을 갖는다: 즉, 관할권 부족으로
인해 항소를 통해 무효화(quashed)될 가능성이 있다. 영국에서 통상
`파기(reversal)'와 `무효화(quashing)' 사이에는 중요한 차이가 있다는 점은
주목할 만하다. 하급심의 판결이 파기되는 경우, 그것은 해당 판사가 적용한
법률 해석이나 사실 판단이 잘못되었다고 간주되기 때문이다. 반면, 관할권
부족으로 무효화되는 판결은 이 두 측면에서 아무런 문제가 없을 수 있다.
잘못된 것은 판사가 \emph{무엇을(what)} 말했는가 또는 명령했는가가
아니라, 그러한 말이나 명령을 \emph{그가(his)} 했다는 사실 자체이다. 즉,
그는 자신에게 법적으로 부여되지 않은 일을 행한 것이며, 다른 법원이라면
그럴 수 있었을지도 모른다. 단지 공공 질서 유지의 복잡성 때문에, 관할권을
초과하여 내려진 결정도 상급 법원이 무효를 선언하기 전까지는 유효한
것으로 간주될 뿐이다. 이 점에서, 사적 개인이 법적 권한을 행사할 때의
유효성 조건과 마찬가지로, 관할권 규칙에의 부합 여부는
`복종(obedience)'이나 `불복종(disobedience)'이라는 표현으로는 잘
설명되지 않으며, 이러한 표현은 형법과 같은 명령 유사 규칙의 경우에 보다
적절하다.

법률이 하위 입법 기관에게 입법권을 부여하는 경우 역시, 그러한 법규는
일반명령으로 환원될 수 없으며, \textbf{(p.~31)} 억지로 그렇게 환원할
경우 왜곡을 피할 수 없다. 이 경우에도 사적 권한의 행사처럼, 입법권을
부여하는 규정이 명시한 조건에 부합하는 것은 체스와 같은 게임에서의
`수'(move)와 같은 단계에 해당하며, 이는 체계가 개인들이 달성할 수 있도록
설정한 규칙에 의해 정의 가능한 결과를 가진다. 입법은 법적 권한의
행사이며, 이는 법적 권리와 의무를 창설하는 데 작용적---즉 효과적으로
기능한다. 부여 규칙의 조건에 부합하지 못할 경우, 그러한 행위는 효력을
갖지 못하며, 해당 목적에 있어 무효(nullity)가 된다.

입법권 행사의 근거가 되는 규정들은 법원의 사법권한을 구성하는 규정보다도
훨씬 다양하다. 왜냐하면 입법에 관한 다양한 측면들이 이를 통해 규율되어야
하기 때문이다. 예컨대, 일부 규칙은 입법권이 행사될 수 있는 주제를
명시하고, 다른 규칙은 입법기관의 구성원 자격이나 정체성을 명시하며, 또
다른 규칙은 입법의 형식 및 절차를 규정한다. 이들은 단지 관련된 사항 중
일부에 불과하다. 하위 입법 기관 또는 규칙 제정 기관의 권한을 부여하고
정의하는 법률, 예컨대 1882년 지방자치법(Municipal Corporations Act,
1882)과 같은 법령을 살펴보면 훨씬 더 많은 사항들이 드러난다. 이러한
규정들을 준수하지 못했을 때의 결과는 항상 동일하진 않지만, 적어도 어떤
규칙에 대해서는 이를 준수하지 않으면 입법권 행사의 시도가 무효로
간주되거나, 하급법원의 판결처럼 무효 선언 대상이 될 수 있다. 때때로
요구된 절차가 준수되었음을 입증하는 증명이 절차상 사항에 관해서는 결정적
효력을 갖도록 법적으로 규정되기도 하며, 자격을 갖추지 못한 자가 입법
절차에 참여할 경우, 특별 형법 규정에 따라 형사 처벌 대상이 되기도 한다.
그러나 이러한 복잡성에 의해 다소 가려지더라도, 입법권 행사 방식을
부여하고 정의하는 규칙들과 형법 규칙들---즉 위협에 의해 뒷받침되는
명령과 유사한 규칙들---사이에는 근본적인 차이가 존재한다.

일부 경우에는 이 두 유형의 규칙을 동일시하는 것이 기괴한 일일 수 있다.
입법기관에서 어떤 법안(a measure)이 요구되는 다수의 투표를 얻었고 그에
따라 통과되었다면, 그 법안에 찬성한 투표자들이 다수 결정을 요구하는 법에
`복종한(obeyed)' 것이 아니고, \textbf{(p.~32)} 반대한 이들 역시 법에
`복종했'거나 `불복종한(disobeyed)'것이 아니다. 만일 그 법안이 필요한
다수에 도달하지 못하여 통과되지 못했다면, 이 역시 마찬가지이다. 이러한
규칙들의 기능적 차이는, 형법 규칙과 관련된 행위에 사용하는 용어들을
이곳에 적용하는 것을 원천적으로 불가능하게 만든다.

현대 법체계에 포함된 다양한 법규들을 포괄적으로 분류하고, 그것들을 모두
하나의 단순한 유형으로 환원됨에 \emph{틀림없다(must)}는 편견 없이 이를
체계화하는 작업은 아직 완수되지 않았다. 우리가 지금까지 수행한 것은,
위협에 의해 뒷받침되는 명령과 유사한 의무 부과적 법규들과, 권한을
부여하는 법규들을 매우 거칠게 구별한 것에 불과하다. 그러나 여기까지의
논의만으로도, 법체계의 특수한 특징 중 일부는 바로 이러한 유형의 규칙을
통해 공적 및 사적 법적 권한의 행사를 가능케 하는 데 있다는 점이
드러났다고 할 수 있다. 만일 이러한 독자적인 유형의 규칙들이 존재하지
않는다면, 우리의 사회생활에서 가장 익숙한 개념들 중 일부는 존재하지 못할
것이다. 왜냐하면 그러한 개념들은 논리적으로 이러한 규칙들의 존재를
전제하기 때문이다. 형벌법과 같이 강제력을 동반하는 명령형 규칙이
존재하지 않는다면 범죄나 위반, 살인이나 절도라는 개념 또한 존재할 수
없으며, 마찬가지로 권한 부여 규칙이 없다면 매매, 증여, 유언, 혼인 등의
개념도 존재할 수 없다. 이러한 것들은---법원의 명령이나 입법기관의 법
제정과 마찬가지로---법적 권한의 유효한 행사로 이루어진 것이기 때문이다.

그럼에도 불구하고, 법이론에서의 통일성에 대한 강박은 여전히 강력하다.
이는 결코 부끄러운 것이 아니므로, 우리는 위대한 법학자들에 의해 제시된
두 가지 대안을 고찰해보아야 한다. 이 두 주장은 우리가 강조한 법의 유형
간 구별이 피상적이거나 실제로는 존재하지 않는 것이며, 결국 모든 규칙은
위협에 의해 뒷받침되는 명령 개념으로 분석 가능하다는 점을 보여주고자
한다. 오래 지속되어온 법이론 대부분이 그렇듯, 이 주장들에도 일정한
진실이 담겨 있다. 우리가 구별한 두 유형의 법규 사이에는 실제로 유사성이
존재하기 때문이다. 양쪽 경우 모두, 특정 행위는 규칙에 비추어 `법적으로
올바른' 혹은 `잘못된' 행위로 \textbf{(p.~33)} 평가될 수 있다. 예컨대
유언장을 작성하는 행위에 관한 권한 부여 규칙이나, 폭행을 금지하고
처벌하는 형법 규칙 모두는, 특정 행위를 이러한 방식으로 비판적으로
평가하는 \emph{기준(standards)}을 구성한다. 이 점만 보더라도, 이들
모두를 `규칙'이라 부를 수 있는 이유가 암시되어 있다. 더 나아가, 권한
부여 규칙은 의무를 부과하는 규칙들과는 다르지만, 그러한 규칙들---즉
위협에 의해 뒷받침되는 명령에 유사한 규칙들---과 항상 관계를 맺고 있다는
점이 중요하다. 왜냐하면 권한 부여 규칙들이 부여하는 권한은, 결국 후자의
규칙들을 만들어내거나, 그러한 규칙이 적용되지 않던 특정 개인에게 새로운
의무를 부과할 수 있는 권한이기 때문이다. 이는 입법권이 부여된 경우 가장
분명히 드러나며, 뒤이어 논의하겠지만, 다른 법적 권한의 경우에도
마찬가지이다. 이러한 점에서, 어느 정도의 부정확성을 감수한다면, 형법
규칙은 의무를 부과하고, 권한 부여 규칙은 의무를 생성하는
조리법(recipes)이라고 말할 수도 있을 것이다.

\subsubsection{\texorpdfstring{\emph{제재로서의 무효(Nullity as a
sanction)}}{제재로서의 무효(Nullity as a sanction)}}\label{uxc81cuxc7acuxb85cuxc11cuxc758-uxbb34uxd6a8nullity-as-a-sanction}

첫 번째 논변은, 두 종류의 규칙이 본질적으로 동일하며 둘 다 강제
명령(coercive orders)의 성격을 갖는다는 점을 보이려는 시도로, 권한
행사에 필수적인 조건이 충족되지 않을 때 발생하는 `무효(nullity)'에
주목한다. 이 논변에 따르면, 무효란 형법상의 처벌처럼, 규칙 위반에 대해
법이 부과하는 위협된 해악(threatened evil) 또는 제재(sanction)와
마찬가지로 이해될 수 있다는 것이다. 물론 어떤 경우에는 그러한 제재가
단지 약간의 불편에 불과할 수도 있다는 점은 인식된다. 이러한 시각에서
우리는 다음과 같은 사례를 바라보도록 요청받는다. 즉, 어떤 사람이
자신에게 한 약속을 법적으로 강제 가능한 계약(contract)으로서 이행시키려
하지만, 그 약속이 인영(seal)이 없고, 자신이 대가를 제공하지 않았기
때문에, 그 서면 약속이 법적으로는 무효라는 사실을 뒤늦게 깨닫고 낙담하는
경우이다. 이와 유사하게, 두 명의 증인이 없는 유언장은 효력이 없다는
규칙은, 유언자(testator)로 하여금 Wills Act 제9조를 준수하도록
유도한다는 점에서, 우리가 형법의 처벌을 생각하며 그것을 준수하게 되는
것과 마찬가지라고 주장된다.

이러한 무효와 심리적 요인(예: 법적 효력 발생에 대한 기대의 좌절) 간의
연관성이 어떤 경우에 존재함을 부정할 수는 없다. 그럼에도 불구하고,
제재(sanction)의 개념을 무효(nullity)로까지 확장하는 것은
\textbf{(p.~34)} 혼란의 원인이자 징후라 할 수 있다. 이에 대한 사소한
반론들은 이미 잘 알려져 있다. 예컨대, 어떤 경우에는 무효가 그 조건을
충족시키지 못한 당사자에게 해악이 아닐 수도 있다. 판사는 자기의
명령(order)의 유효성에 대해 아무런 물질적 관심이 없거나 무관심할 수
있으며, 계약상 상대방이 미성년자이거나 특정 계약에 요구되는 서면 기재
요건을 충족하지 않았다는 이유로 계약이 구속력을 갖지 않는 경우, 피고는
이를 `위협된 해악'이나 `제재'로 인식하지 않을 수도 있다. 그러나 이러한
사소한 반례들은 약간의 기지로 극복할 수 있다고 치더라도, 보다 본질적인
이유에서, 무효는 규칙이 금지하는 행위를 억제하기 위한 유인으로서의
처벌과 동일시될 수 없다. 형법 규칙의 경우, 우리는 다음 두 요소를
식별하고 구분할 수 있다: (1) 규칙이 금지하는 특정 유형의 행위, (2)
그것을 억제하기 위한 제재. 그러나 이러한 관점에서, 법적 형식 요건을
충족하지 않은 약속을 주고받는 것과 같은 사회적으로 바람직한 행위를
어떻게 설명할 수 있을까? 계약서 형식을 요구하는 법 규칙은 그러한 행위를
금지하거나 억제하려는 것이 아니라, 단지 그것들에 법적 인식을 부여하지
않을 뿐이다. 더욱이, 입법 조치가 필요한 정족수를 얻지 못해 법으로서의
지위를 획득하지 못하는 사실을 제재라고 간주하는 것은 더욱 부조리하다.
형법상의 제재와 이러한 사실을 동일시하는 것은, 마치 득점(goal)이나
출루(run)를 제외한 모든 움직임을 제거하려는 것으로 점수 규칙을 이해하는
것과 같다. 이것이 성공한다면 게임은 존재할 수 없게 될 것이다. 그러나
권한 부여 규칙(power-conferring rules)을 사람들로 하여금 특정한 방식으로
행동하게 만들기 위한 도구로 보고, `무효'를 복종의 동기로 추가하는
방식으로 해석할 때에만, 우리는 이 규칙들을 위협에 기반한 명령으로
동일시할 수 있다.

무효(nullity)를 형법상의 제재(threatened evil or sanction)와 유사한
것으로 간주하는 데서 발생하는 혼란은 다음 방식으로도 드러낼 수 있다.
형법 규칙의 경우, 설령 어떠한 처벌이나 해악이 위협되지 않는다고
하더라도, 그런 규칙들이 존재할 수 있으며 또 존재하는 것이 바람직할 수
있다는 점에서, 논리적으로 이러한 규칙은 성립 가능하다. 물론 이러한 경우
그것이 \emph{법적(legal)} 규칙인지에 대해서는 논쟁이 있을 수 있지만,
우리는 \textbf{(p.~35)} 특정한 행위를 금지하는 규칙과, 그 규칙 위반 시
부과되는 처벌 규정을 명확히 구분할 수 있으며, 전자의 규칙이 후자의 처벌
없이도 존재할 수 있다고 상정할 수 있다. 즉, 우리는 제재를 제거한 후에도
여전히 유지될 수 있는 행위의 기준(standard of behaviour)을 상정할 수
있다. 그러나 유효한 유언을 위해 필요한 증인의 참여와 같은 조건을
요구하는 규칙과, 그것을 따르지 않았을 때의 `무효'라는 이른바 제재
사이에는 그러한 논리적 구분이 성립하지 않는다. 이 경우, 만약 이러한 필수
조건을 따르지 않아도 무효가 되지 않는다면, 그러한 규칙 자체는 비법적
규칙(non-legal rule)로서조차 존재한다고 말할 수 없게 된다. 무효 규정은
이러한 유형의 규칙에서 처벌이 의무 부과 규칙에 결합된 방식과는 달리, 그
규칙 자체의 \emph{부분(part)}이다. 만약 공 사이로 골대를 통과시키지
못하는 것이 `득점 실패(nullity of not scoring)'를 의미하지 않는다면,
득점 규칙 자체가 존재한다고 말할 수 없을 것이다.

우리가 비판한 이 논변은, 권한 부여 규칙(power-conferring rules)과 강제
명령(coercive orders)이 본질적으로 동일하다는 점을 입증하고자 하는
시도이며, 그 수단으로는 제재 또는 위협된 해악의 의미를
\emph{확장하여(widening)}, 그러한 규칙을 따르지 않을 경우 법적 행위가
무효가 된다는 점까지 포함시키려는 것이다. 반면, 다음에서 살펴볼 두 번째
논변은 전혀 다른, 오히려 정반대의 입장을 취한다. 이 논변은 이러한
규칙들이 강제 명령의 일종이라는 점을 보여주려 하기보다는, 아예 그것들을
`법(law)'의 범주에서 배제하려 한다. 이는 그러한 규칙을 제외하기 위해
단어 `법'의 의미를 \emph{좁히는 것이다(narrows)}. 이 논변은 다양한
법학자들 사이에서 다소 차이는 있지만 유사한 형태로 나타난다. 그 일반적
형식은 다음과 같다. 즉, 통상적인 언어 사용이나 일상적 표현에서 `완전한
법 규칙(complete rules of law)'이라고 간주되는 것들이, 실제로는 강제
규칙(coercive rules)의 단편(fragment)에 지나지 않으며, 오직 강제
규칙만이 `진정한(genuine)' 법 규칙이라는 주장을 내세우는 것이다.

\subsubsection{\texorpdfstring{\emph{법의 일부 조각으로서의 권한부여
규칙(Power-conferring
rules)}}{법의 일부 조각으로서의 권한부여 규칙(Power-conferring rules)}}\label{uxbc95uxc758-uxc77cuxbd80-uxc870uxac01uxc73cuxb85cuxc11cuxc758-uxad8cuxd55cuxbd80uxc5ec-uxaddcuxce59power-conferring-rules}

이 주장의 극단적인 형태에서는 형법(criminal law)의 규칙들조차, 그것들이
흔히 표현되는 문구로 보았을 때, 진정한 법(law)이 아니라고 부정하게 된다.
이와 같은 형태로 이 주장을 채택한 인물이 켈젠(Kelsen)이다. 그는 다음과
같이 말한다. ``법이란 제재(sanction)를 명하는 기본규범(primary
norm)이다.''\footnote{\emph{General Theory of Law and State}, p.~63. See
  above, p.~2.} 살인을 금지하는 법은 존재하지 않는다: 오직
\textbf{(p.~36)} 살인을 저지른 자에게 특정한 상황에서 특정한 제재를
가하도록 공무원(official)에게 명하는 법만이 있을 뿐이다. 이러한
관점에서, 일반 시민의 행위를 지도하기 위해 설계되었다고 여겨지는 법의
내용은, 실제로는 시민이 아닌 공무원에게 지시되는 규칙의 전건 또는
``if-절''에 불과하며, 그 규칙은 특정 조건이 충족되면 특정 제재를
가하도록 명령한다. 이 관점에 따르면, 모든 진정한 법은 공무원에게 제재를
적용하라는 조건적 명령이다. 이들은 모두 다음과 같은 형식을 갖는다.
``어떤 종류 X의 행위가 행해지거나 생략되거나 발생한 경우, 그러면 Y
종류의 제재를 가하라.''

이러한 전건 또는 if-절의 점차적인 정교화를 통해, 사적 또는 공적 권한의
부여 및 그 행사 방식에 관한 규칙들을 포함한 모든 유형의 법규칙은 이와
같은 조건적 형식으로 재진술될 수 있다. 예컨대, 유언장법(Wills Act)의 두
명의 증인을 요구하는 규정은 다음과 같이 표현될 수 있다. 유언의 규정을
위반하고 유산을 지급하지 않는 집행자(executor)에게 법원이 제재를
가하도록 명하는 다양한 지시들에 공통적으로 포함된 부분으로서 말이다. 즉,
``(if and only if) 이러한 규정을 담은 적법한 증인의 유언장이 있고 만약
\ldots{} 라면, 그에게 제재를 가해야 한다.'' 이와 유사하게, 법원의
관할권(jurisdiction)의 범위를 명시하는 규칙 또한 제재를 적용하기 전에
충족되어야 할 조건의 공통 요소로 나타날 수 있다. 또한, 입법 권한을
부여하고 입법의 방식과 형식을 정의하는 규칙들(헌법에서 최고의 입법기관에
대해 규정하는 조항 포함) 또한 조건의 충족 여부에 따라 법원이 법률에
명시된 제재를 적용해야 한다는 것을 명시하는 조건절로 재구성될 수 있다.
따라서 이 이론은 실질을 형식에서 분리하여, `의회 내 여왕이 제정한 것이
법이다'라는 영국 헌정 형식이나, 미국 헌법의 의회의 입법권에 대한 규정은
단지 법원이 제재를 적용해야 할 일반 조건을 명시하는 것일 뿐이라는 점을
보여준다. 이러한 형식들은 본질적으로 ``if-절''이며, 완전한 규칙이
아니다. 즉, ``\emph{만약(If)} 의회 내 여왕이 그렇게 제정하였다면
\ldots{}'' 또는 ``\emph{만약(if)} 의회가 헌법에 명시된 한도 내에서
그렇게 제정하였다면 \ldots{}''이라는 형식은 법원이 제재를 가하거나 특정
행위를 처벌하도록 지시하는 수많은 명령의 조건에 해당한다.

\textbf{(p.~37)} 이는 매우 정교하고 흥미로운 이론으로, 다양한 일반적
형식과 표현에 의해 가려져 있던 법의 진정하고 통일적인 본성을 드러내려는
시도이다. 이 이론의 결함들을 살펴보기 전에 먼저 주목할 점은, 이 극단적인
형태의 이론에서는 법을 단지 명령과 제재(threats of sanctions)로 구성된
것으로 보는 초기 개념에서 벗어나, 공무원에게 제재를 가하도록 명령하는
것으로 중심 개념이 이동한다는 점이다. 이 관점에서는 모든 법 위반에 대해
제재가 규정되어 있어야 할 필요는 없다. 다만, 모든 `진정한' 법이 어떤
제재의 적용을 명령하고 있으면 충분하다. 따라서 이러한 지시를 무시한
공무원이 처벌되지 않을 수도 있으며, 실제로도 많은 법체계에서 그러하다.

이 일반 이론은, 앞서 말했듯이, 두 가지 형태로 존재할 수 있는데, 하나는
다른 것보다 덜 극단적이다. 덜 극단적인 형태에서는, 많은 사람들이
직관적으로 더 수용 가능한 것으로 여기는 원래의 법 개념---즉, 일반 시민을
포함한 사람들에게 위협(threats)을 동반한 명령이 내려지는 구조---이
적어도 상식적으로 시민의 행위를 주로 대상으로 삼는 규칙들에 대해서는
유지된다. 이 보다 온건한 견해에 따르면, 형법의 규칙들은 그것 자체로
법이며, 다른 완전한 규칙의 일부 조각으로 재구성될 필요가 없다. 왜냐하면
그것들은 이미 위협을 동반한 명령이기 때문이다. 그러나 다른 경우에는
재구성이 필요하다. 사인(private individuals)에게 법적 권한을 부여하는
규칙들은, 이 온건한 이론에서도, 더 극단적인 이론에서와 마찬가지로,
진정한 완전한 법---위협을 동반한 명령---의 단편(fragment)에 불과하다.
이러한 진정한 법은 다음과 같은 질문을 통해 발견된다. 즉, 어떤 사람에게
법이 무언가를 하라고 명령하며, 이를 따르지 않을 경우 어떤 제재가
부과되는가? 이러한 질문에 따라, 1837년 유언장법(Wills Act)의 증인
규정이나, 사인에게 권한을 부여하고 그 유효한 행사의 조건을 정의하는 다른
규칙들은, 궁극적으로 그러한 법적 의무가 발생하는 조건 중 일부를 명시하는
것으로 재구성될 수 있다. 이들은 조건적 명령의 전건 또는 `if-절'의
일부로서, 위협을 동반하거나 의무를 부과하는 규칙의 일부로 나타난다.
예컨대, ``(If and only if) 유언자가 서명하고 두 명의 증인이 명시된
방식으로 입회하여 유언장을 작성하였고 \textbf{(p.~38)} 만약 \ldots{}
라면, 집행자(또는 다른 법적 대표자)는 유언의 조항을 이행해야 한다.''
계약의 성립과 관련된 규칙들도 마찬가지로, 특정 조건이 충족되었거나
특정한 말이나 행위가 있었을 경우(예: 당사자가 성년이며 인장을 통해
약속했거나 고려사항을 제공받았을 경우), 계약상 이행해야 할 행위를 하도록
명령하는 규칙의 단편으로 나타나게 된다.

입법 권한을 부여하는 규칙들(헌법에 있는 최고 입법기관 관련 조항 포함)을
`진정한' 규칙의 단편으로 재구성하는 작업도, 이 이론의 더 극단적인
버전에서 설명된 36쪽의 방식과 유사하게 수행될 수 있다. 단지 차이점은,
보다 온건한 견해에서는 권한부여 규칙들이 일반 시민에게 위협을 통해 어떤
행위를 하도록 명령하는 규칙의 전건 또는 if-절로 나타나는 반면, 더
극단적인 이론에서는 공무원에게 제재를 가하도록 지시하는 규칙의 if-절로만
나타난다는 것이다.

이 이론의 두 가지 버전은 모두 외견상 서로 다른 다양한 법 규칙들을 하나의
형식으로 환원하려 시도하며, 이 형식이 법의 정수를 전달한다고 주장한다.
양자는 서로 다른 방식으로 제재를 중심 요소로 삼으며, 만일 제재 없는 법이
충분히 개념 가능하다는 점이 입증된다면, 양자 모두 실패하게 된다. 이
일반적인 반론은 후에 다루기로 하고, 여기서는 이 두 이론 모두에 제기할 수
있는 특정한 비판을 전개할 것이다. 그것은 이 이론들이 모든 법을 통일된
패턴으로 환원하는 데서 오는 통일성의 쾌감을 지나치게 높은 대가로
구매한다는 점이다. 즉, 서로 다른 유형의 법 규칙들이 수행하는 사회적
기능의 차이를 왜곡하고 있다는 것이다. 이 점은 이 이론의 두 형태 모두에
해당하지만, 특히 극단적인 형태가 요구하는 형법의 재구성에서 가장
뚜렷하게 나타난다.

\subsubsection{\texorpdfstring{\emph{통일성의 대가로서의
왜곡}}{통일성의 대가로서의 왜곡}}\label{uxd1b5uxc77cuxc131uxc758-uxb300uxac00uxb85cuxc11cuxc758-uxc65cuxace1}

이러한 재구성(recasting)에 의해 발생하는 왜곡(distortion)은 주목할
가치가 있다. 이는 법(law)의 여러 측면을 비추어 주기 때문이다. 사회를
통제하는 방식은 다양하지만, 형법(criminal law)의 고유한 방식은
규칙(rules)을 통해 특정한 유형의 행위를 전 사회 구성원 전체 혹은 특정
집단에 대한 지침(standards)으로 지정하는 것이다: 이들은 공식적인
담당자(official)의 도움이나 개입 없이도 \textbf{(p.~39)} 규칙을
이해하고, 규칙이 자신에게 적용된다는 점을 인식하며, 이에 따르도록
기대된다. 오직 법이 위반되어 이러한 법의 일차적 기능이 실패할 때에만,
담당자들이 위반 사실을 식별하고 예정된 제재(sanction)를 부과하는 문제에
관여하게 된다. 이 방식이, 교통 경찰이 운전자에게 주는 개별적이고
대면적인 명령과 비교해 고유한 점은, 사회 구성원들이 스스로 규칙을
발견하고 그에 따라 자신의 행동을 조정해야 한다는 것이다. 이 의미에서
그들은 규칙을 스스로에게 `적용(apply)'하는 것이며, 규칙에 덧붙여진
제재가 순응의 동기를 제공한다. 만약 우리가 규칙의 기능을,
불복종(disobedience)의 경우 법원이 제재를 부과해야 한다는 규칙에 초점을
맞추거나 그것을 중심으로 삼아 이해한다면, 이러한 규칙이 작동하는 고유한
방식을 가리게 된다. 이러한 후자의 규칙들은 필수 불가결할 수 있으나,
어디까지나 부차적인 것이다.

형법의 실질 규칙(substantive rules)이 그 기능---그리고 넓은 의미에서 그
의미---을 단지 제재 체계를 운영하는 담당자들만이 아니라, 공적 직무를
수행하지 않는 일반 시민들이 일상생활 속에서 따를 수 있도록 인도하는 데에
둔다는 관점은, 핵심적인 구별을 포기하고 법이라는 사회 통제 수단의 고유한
성격을 흐리지 않고서는 제거될 수 없다. 예컨대 범죄에 대한 처벌로서의
벌금(fine)은 특정 행위 양식에 부과되는 세금(tax)과 동일하지 않다. 비록
둘 다 같은 금전적 손실을 초래하라는 지시를 담당자에게 내린다는 점에서는
유사하더라도 말이다. 이 둘을 구분짓는 핵심은, 첫 번째는 일반 시민의
행위를 인도하기 위해 설정된 규칙을 위반하는 의무 위반(breach of duty),
즉 범죄(offence)를 수반한다는 점이다. 물론 이러한 구분은 특정 상황에서는
모호해질 수 있다. 예컨대 세금이 세수(revenue) 목적이 아니라 해당 행위를
억제하려는 목적으로 부과되기도 한다. 그러나 이 경우, 법은 해당 행위를
범죄화할 때처럼 명시적으로 그 행위를 중단할 것을 지시하지는 않는다.
반대로 어떤 형사 범죄에 대한 벌금이 화폐 가치 하락으로 너무 작아질 경우,
사람들은 이를 기꺼이 지불하게 되고, `단순한 세금일 뿐'이라는 인식이
생기며 `위반'는 빈번해진다. 이러한 맥락에서는, 해당 규칙이 형법
대부분처럼 진지하게 따를 행위 기준이라는 감각 자체가 사라진 것이다.

\textbf{(p.~40)} 이와 같은 이론을 옹호하는 사람들은 종종, 법을 제재
부과의 지시라는 형식으로 재구성함으로써 명료성을 얻는다고 주장한다. 이는
`나쁜 사람(bad man)'이 법에 대해 알고 싶어하는 바를 분명히 해준다는
것이다. 그럴 수도 있으나, 이론에 대한 충분한 방어가 되지는 않는다. 법은
`무엇을 해야 하는지 알려주기만 하면 따를 의사가 있는 당황한 사람(puzzled
man)'이나 `무지한 사람(ignorant man)', 혹은 `자신의 일을 계획하고자 하는
사람'에게 더 중점을 두어야 하는 것 아닌가? 물론 법을 제대로 이해하기
위해서는 법원이 제재를 적용하는 방식도 중요하다. 하지만 이로 인해 법을
이해한다는 것이 법정에서 벌어지는 일만을 아는 것으로 협소화되어서는 안
된다. 법이 사회 통제 수단으로서 수행하는 주요 기능은 사적 소송(private
litigation)이나 형사 기소(prosecution)에서가 아니라, 법이 법정 밖의 삶을
어떻게 통제하고, 인도하며, 계획하도록 작용하는 다양한 방식에서 나타난다.

이러한 이론의 극단적 형태가 일으키는 `주된 것과 부차적인 것의
전도(inversion of ancillary and principal)'는 다음과 같은 사례에 비유할
수 있다. 어떤 이론가가 크리켓이나 야구의 규칙을 살피며, 규칙들이 단지
용어와 관례적 표현 때문에 구별되어 왔을 뿐 사실상 모두 동일한 유형의
규칙이라고 주장한다고 하자. 이 이론가는 ``모든 규칙은
\emph{실제로는(really)} 특정 조건 하에서 담당자가 특정한 일을 하도록
지시하는 규칙이다''라고 주장할 수 있다. 예를 들어 ``타격 후의 특정
동작은 `출루(run)'가 된다''거나 ``포구되면 아웃(out)이다''는 규칙은,
실상은 각각 득점 기록원(scorer)에게 `출루'를 기록하라고,
심판(umpire)에게 퇴장을 명하라고 지시하는 복합 규칙에 불과하다는 것이다.
그러나 이에 대한 자연스러운 반론은, 이러한 식의 규칙 일원화는 규칙들이
실제로 어떻게 작동하는지, 선수들이 목적 지향적 활동을 이끌기 위해 어떻게
사용하는지를 숨기며, 규칙이 본래 협력적이면서 경쟁적인 사회적 활동인
게임 속에서 수행하는 기능을 흐린다는 것이다.

이 이론의 덜 극단적인 형태는, 의무를 부과하는 형법 \textbf{(p.~41)} 및
다른 모든 법은 건드리지 않는다. 이들은 이미 강제 명령의 단순한 모델에
부합하기 때문이다. 그러나 이론은 권한을 부여하거나 행사 방식 정의에 관한
모든 규칙을 단일한 형태로 축소하려 한다. 그러나 여기서도 동일한 비판이
제기될 수 있다. 만약 우리가 법을, 단지 의무가 부과되는 사람들의
관점에서만 바라보고, 법의 다른 모든 측면을 그들에게 의무가 발생하는
조건으로 환원한다면, 우리는 의무만큼이나 법의 고유하고 사회적으로 가치
있는 요소들을 단순히 부차적인 것으로 취급하게 된다. 사적 권한(private
powers)을 부여하는 규칙은, 이를 제대로 이해하려면, 그것을 행사하는
사람들의 관점에서 살펴보아야 한다. 그럴 때, 이러한 규칙은 단순한 강제
통제를 넘어서 사회생활에 도입된 추가적 요소로 드러난다. 이러한 권한은
일반 시민으로 하여금 단순한 의무 수납자가 아니라, 일종의 사적
입법자(private legislator)가 되게 한다. 시민은 자신의 계약,
신탁(trusts), 유언(wills), 기타 권리와 의무 구조 내에서 법의 경로를 정할
수 있는 권한을 가진 존재가 되는 것이다. 이처럼 특별한 방식으로 사용되며
고유한 혜택을 제공하는 규칙이, 단순히 의무를 부과하는 규칙과 동일하게
취급되어야 할 이유는 무엇인가? 이 권한 부여 규칙들은 사회생활 속에서
의무 규칙과는 전혀 다르게 인식되고, 언급되며, 사용되고, 다른 이유로 가치
있게 여겨진다. 그렇다면 이러한 특성을 구별하는 기준은 무엇이 있을 수
있는가?

입법 및 사법 권한을 부여하고 정의하는 규칙들을, 단지 의무가 발생하는
조건으로 환원하는 접근은 공적 영역에서도 유사한 왜곡을 낳는다. 이러한
권한을 행사하여 법적 효력을 가진 법령이나 명령을 만드는 사람들은, 단순한
의무 수행이나 강제 통제에 복종하는 것과는 전혀 다른 목적 지향적 활동을
수행하고 있다. 이러한 규칙들을 단지 의무 규칙의 파편이나 측면으로
나타내는 것은, 사적 영역에서보다도 더 심하게, 법의 고유한 특성과 그 틀
내에서 가능한 활동의 다양성을 흐리게 만든다. 사회에 입법자가 새로운 의무
규칙을 도입하거나 \textbf{(p.~42)} 기존 규칙을 변경하도록 허용하는 규칙,
그리고 판사가 어떤 규칙이 위반되었는지를 판단할 수 있도록 하는 규칙을
도입한 것은, 바퀴의 발명에 비견될 만큼 중요한 진보였다. 이는 단지 중요한
발전일 뿐 아니라, 우리가 제5장에서 논의하듯, 전(前)법적(pre-legal)
세계에서 법적 세계(legal world)로 넘어가는 결정적 전환으로 간주될 수
있는 발전이기도 하다.

\subsection{\texorpdfstring{\textbf{2. 적용
범위}}{2. 적용 범위}}\label{uxc801uxc6a9-uxbc94uxc704}

형벌법(penal statute)은 다양한 법의 형태 중에서도 단순한 강제
명령(coercive orders)의 모델에 가장 가까운 형태임이 분명하다. 하지만
이러한 법조차도, 이 절에서 살펴보듯, 이 모델이 우리에게서 가리게 만드는
특유의 성질을 지니고 있다. 그리고 우리는 이 모델의 영향력을 떨쳐내기
전까지 그것들을 제대로 이해하지 못할 것이다. 위협(threats)을 동반한
명령(order)은 본질적으로 \emph{타인(others)}이 어떤 행동을 하거나 하지
않기를 바라는 소망의 표현이다. 물론, 입법이 이러한 오직 타인
지향적인(other-regarding) 형태를 취하는 것도 가능하다. 입법권을 가진
절대군주(absolute monarch)는 일부 체계에서 자신이 만든 법의 적용대상에서
항상 제외되는 존재로 간주될 수 있다. 민주주의 체계에서도, 법을 만든
이들에게는 적용되지 않고 법에서 명시한 특정 계층에만 적용되는 법이
제정될 수 있다. 그러나 어떤 법의 적용 범위(range of application)는
언제나 그 법의 해석(interpretation)에 달려 있다. 해석 결과, 입법자가 법
적용 대상에서 제외될 수도 있고 그렇지 않을 수도 있다. 물론, 현대에는
법을 만든 사람들에게도 법적 의무(legal obligations)를 부과하는 입법이
자주 이루어진다. 단순히 \emph{타인(others)}에게 위협을 통해 무언가를
하게 하는 명령과 구분되는 입법은, 얼마든지 스스로에게 구속력을
부여하는(self-binding) 힘을 가질 수 있다. 입법은
\emph{본질적으로(essentially)} 타인 지향적일 필요가 없다. 이러한 법적
현상은, 우리가 법을 항상 법 위에 서 있는 어떤 사람이 법 아래에 있는 다른
사람들을 위해 정한 것이라고 보는 모델의 영향력 하에 있을 때에만 난해한
것으로 여겨진다.

이러한 수직적(vertical) 또는 `상층에서 하층으로(top-to-bottom)'의 법제정
이미지---그 단순성으로 인해 매우 매력적인 이
이미지는---입법자(legislator)를 그의 공적 지위(capacity)에서는 한
사람으로, 사적 지위에서는 또 다른 사람으로 구별함으로써만 현실과
조화시킬 수 있다. 입법자는 첫 번째 지위에서 행동함으로써, 두 번째
지위(자신의 사적 지위)를 포함하여 타인들에게 의무를 부과하는 법을
제정하게 된다. 이러한 표현 방식에는 문제가 없지만, \textbf{(p.~43)}
우리가 제4장에서 보게 될 바와 같이, 이러한 지위(capacities)의 개념은
강제 명령으로 환원될 수 없는 권한 부여 규칙(power-conferring rules of
law)을 전제로 해야만 비로소 이해 가능하다. 한편으로는, 이러한 복잡한
장치 자체가 사실상 전혀 필요 없다는 점도 주목할 필요가 있다. 우리는
입법의 자기구속적 성격(self-binding quality)을 이 장치를 동원하지 않고도
설명할 수 있다. 우리의 일상생활과 법제도 속에는 이를 훨씬 더 잘 이해할
수 있게 해주는 것이 존재하기 때문이다. 그것은 바로
\emph{약속(promise)}의 작용이다. 강제 명령보다도 오히려 약속은 법의 여러
특성을 이해하는 데 있어 훨씬 더 나은 모델이 될 수 있다.

약속하기란, 약속하는 사람(promisor)에게 의무를 발생시키는 말을 하는
것이다. 이러한 말이 그런 효과를 가지려면, 적절한 인물이 적절한
상황에서(즉, 자신의 위치를 이해하고 정신이 온전하며 다양한 외적 압력에서
자유로운 경우) 특정한 말을 사용하면, 그 말을 사용한 사람은 그 내용대로
행동해야 한다는 규칙이 존재해야 한다. 그러므로 우리가 약속을 할 때,
우리는 규칙에 의해 부여된 `권한(a power)'을 행사하여, 자신에게 의무를
부과하고 타인에게 권리를 부여함으로써 자신의 도덕적 상황을 변화시키는
특정 절차를 사용하는 것이다. 이 과정은 다음과 같은 비유로도 설명할 수
있다. 약속자 내부에 두 인물이 있다고 상정하고, 한 인물은 의무를 만드는
자의 지위에서, 다른 인물은 그 의무에 속박되는 자의 지위에서 행동한다고
하며, 전자가 후자에게 무언가를 명령한다고 생각하는 것이다. 그러나 이는
가능한 상정일 뿐, 도움이 되는 설명 방식은 아니다.

입법의 자기구속력을 이해하는 데 있어서도, 우리는 이와 같은 장치를 생략할
수 있다. 법을 만드는 행위는 약속을 만드는 행위와 마찬가지로, 특정한
과정을 지배하는 규칙의 존재를 전제로 한다. 이러한 규칙에 의해 정당하게
자격을 부여받은 사람이, 그 절차에 따라 말하거나 글로 쓴 내용은,
명시적이든 묵시적이든 해당 범위에 포함되는 모든 사람에게 법적 의무를
발생시킨다. 이에는 입법 과정에 참여한 사람들도 포함될 수 있다.

물론, 입법과 약속 사이에는 자기구속력을 설명해주는 유비(analogy)가
존재하긴 하지만, 둘 사이에는 많은 차이점도 있다. 입법을 규율하는 규칙은
훨씬 더 복잡하며, 약속이 지니는 쌍방성(bilateral character)은 입법에는
없다. 일반적으로 약속에는 \emph{그것이 향하는(to whom)} 특별한 지위를
가진 상대방---즉 약속을 받은 사람(promisee)---이 있으며, 이 사람은
\textbf{(p.~44)} 그 이행에 대한 특별한, 때로는 유일한 권리를 가진다.
이러한 점에서, 입법의 자기구속적 측면과 더 밀접한 유비를 제공하는 것은,
영국법에서 인식되는 다른 형태의 자기의무 부과 방식일 수 있다. 예컨대
어떤 사람이 자신이 특정 재산의 신탁자(trustee)임을 선언하는 경우가
그러하다. 그럼에도 일반적으로 말해, 법 제정을 통한 법의 생성은 특정한
법적 의무를 창출하는 이러한 사적 방식들을 통해 가장 잘 이해될 수 있다.

강제 명령 또는 규칙이라는 모델에 대한 가장 필요한 교정적 관점은, 입법을
사회 전체가 따를 일반적 행위 기준(general standards of behaviour)의 도입
혹은 수정으로 새롭게 개념화하는 것이다. 입법자는 반드시 타인에게 명령을
내리는 자---즉 자신이 하는 일의 적용 범위 밖에 있는 자---처럼 행동하는
것은 아니다. 그는 약속을 하는 사람(the giver of a promise)처럼, 규칙에
의해 부여된 권한을 행사하는 자이다. 그리고 종종, 약속자(promisor)가
\emph{반드시 그러해야(must)} 하듯이, 그 자신도 그 권한의 적용 범위 안에
포함될 수 있다.

\subsection{\texorpdfstring{\textbf{3. 기원의
방식}}{3. 기원의 방식}}\label{uxae30uxc6d0uxc758-uxbc29uxc2dd}

지금까지 우리는 법의 다양한 형태 중에서도, 강제 명령(coercive orders)과
유사한 점이 두드러지는 제정법(statute)에 논의를 국한해 왔다. 앞서 강조한
차이점들에도 불구하고, 제정법은 강제 명령과 하나의 뚜렷한 유사점을
지닌다. 즉, 법의 제정(enactment)은, 명령을 내리는 행위와 마찬가지로,
의도적이고 특정한 시점에 이루어지는 행위\emph{이다}(\emph{is} a
deliberate datable act). 입법에 참여하는 자들은 의식적으로 법을 만드는
절차를 작동시키며, 이는 마치 명령을 내리는 자가 자신의 의도를 인식시키고
복종을 확보하기 위해 특정한 표현을 의도적으로 사용하는 것과 유사하다.
이러한 이유로, 법을 강제 명령의 모델로 분석하는 이론들은 모든 법이 결국
이 점에서 입법과 유사한 형태를 가지며, 그 법적 지위를 의도적인 법 창조
행위(law-creating act)에 기인한다고 주장한다. 이 주장을 가장 명백하게
거스르는 법 유형은 관습(custom) 이다. 그러나 관습이 `진정한 법(real
law)'인가에 관한 논의는, 다음의 두 가지 문제를 분리하지 못했기 때문에
종종 혼란스럽게 진행되어 왔다. 첫 번째 문제는, `관습 그 자체로서(custom
as such)' 법이 될 수 있는가이다. 관습 그 자체가 법이 아니라는 부정에는,
어떤 사회든 다수의 관습이 존재하지만 그 모두가 법의 일부는 아니라는
단순한 진리가 담겨 있다. 예컨대 여성에게 모자를 벗지 않는 것은 법 규칙
위반이 아니며, 법적으로는 단지 허용(permitted)된 행동일 뿐이다. 이것은,
관습이 법이 되기 위해서는 반드시 \textbf{(p.~45)} 특정 법체계 내에서
`법으로서 인식(legal recognition)'되어야 하는 관습의 하나여야 함을
보여준다. 두 번째 문제는, `법적 인식(legal recognition)'이라는 표현의
의미이다. 하나의 관습이 법으로 인식된다는 것은 무엇을 의미하는가? 강제
명령의 모델이 요구하듯, 이는 누군가---예컨대 `주권자(the sovereign)'
혹은 그의 대리인(agent)---가 그 관습을 따르라고 명령하였기 때문인가?
그렇다면 그 관습의 법적 지위는, 이러한 점에서 입법 행위와 유사한 어떤
것에 근거한다는 뜻인가?

현대 사회에서 관습은 그다지 중요한 법의 `원천(source)'이 아니다. 대개는
종속적(subordinate) 지위에 있으며, 이는 입법부가 법률로 관습 규칙의 법적
지위를 박탈할 수 있다는 의미이다. 또한 많은 법체계에서 법원이 어떤
관습이 법적 인식을 받을 수 있는지를 판단할 때 적용하는 기준은
`합리성(reasonableness)'과 같은 유동적인 개념을 포함한다. 이러한 점
때문에, 법원이 관습을 수용하거나 배제할 때 거의 무제한적
재량(discretion)을 행사한다고 보는 견해에 일정한 근거가 있다. 그럼에도
불구하고, 관습이 법적 지위를 갖는 이유가 법원이나 입법부 또는 주권자가
그러한 `명령'을 내렸기 때문이라고 말하는 것은, 지나치게 확장된 의미의
`명령' 개념을 전제하지 않고서는 유지될 수 없는 이론을 받아들이는 것이다.

이러한 법적 인식(legal recognition)이라는 교리를 제시하기 위해, 우리는
강제 명령 모델에서 주권자(the sovereign)가 수행하는 역할을 상기해야
한다. 이 이론에 따르면, 법은 주권자 혹은 그가 대신하여 명령을 내리도록
지명한 하급자의 명령이다. 첫 번째 경우, 법은 `명령(order)'이라는 가장
문자적인 의미로 주권자의 명령에 의해 만들어진다. 두 번째 경우, 하급자가
내린 명령은, 그것이 다시 주권자의 명령에 근거한 것일 때만 법으로 간주될
수 있다. 이 하급자는 주권자로부터 명령을 내릴 권한을 위임받아야 한다.
때때로 이러한 권한은, 예컨대 장관에게 특정 주제에 대해 `명령을
내리도록(make orders)' 지시하는 명시적 명령(express direction)을 통해
부여된다. 그러나 이 이론이 여기서 멈춘다면 현실을 설명하지 못하게
되므로, 이 이론은 확장되어 주권자가 자신의 의사를 보다 간접적으로
표현하는 경우도 있다고 주장하게 된다. 그의 명령은 `묵시적(tacit)'일 수
있다. 즉, 주권자가 명시적인 명령을 내리지 않더라도, 그의 하급자들이
국민에게 명령을 내리고 불복종에 대해 처벌할 때 그에 개입하지 않음으로써,
주권자는 국민이 특정한 일을 하기를 바란다는 의사를 나타내는 것이다.

\textbf{(p.~46)} `묵시적 명령(tacit order)'의 개념을 가능한 명확하게
설명하기 위해 군사적 예시를 들어보자. 어느 하사가 자신보다 상급자의
명령에는 늘 복종하면서, 부하 병사들에게 특정한 잡무(fatigues)를
지시하고, 명령 불복종 시 벌을 준다. 그 장군은 이 사실을 알고도 이를
허용한다. 만약 장군이 하사에게 잡무를 중지하라고 명령했다면, 그는 따랐을
것이다. 이러한 상황에서는, 장군이 병사들이 잡무를 수행하길 원한다는
의사를 묵시적으로 표현했다고 볼 수 있다. 개입할 수 있었지만 하지
않았다는 점에서, 그 침묵은 마치 그가 잡무를 명령했을 때 사용했을지도
모를 말을 대체하는 것이다.

이러한 시각에서 우리는 법체계 내에서 법적 지위를 가진 관습 규칙들을
바라보도록 요청받는다. 법원이 구체적 사건에 그것을 적용하기 전까지,
이러한 규칙들은 \emph{단지(mere)} 관습일 뿐이며, 어떤 의미에서도 법은
아니다. 법원이 그 규칙들을 사용하고, 그것에 따라 명령을 내리며, 그것이
집행될 때 비로소 이 규칙들은 법적 인식(legal recognition)을 받는다. 이때
개입할 수도 있었지만 개입하지 않은 주권자는, 법관들이 기존 관습에 기초해
내린 명령을 국민이 따르도록 묵시적으로 명령한 셈이 되는 것이다.

이러한 관습(custom)의 법적 지위에 대한 설명은 두 가지 상이한 비판에
직면한다. 첫 번째 비판은, 관습 규칙이 소송(litigation)에서 사용되기
전까지는 법적 지위를 전혀 갖지 못한다는 것이 \emph{반드시(necessarily)}
참은 아니라는 점이다. 이러한 주장이 `반드시' 그렇다고 말하는 경우, 이는
단순한 독단(dogma)이거나, 법체계에 따라 그럴 수도 있는 경우와 반드시
그런 경우를 구분하지 못하는 데서 비롯된다. 만일 특정 방식으로 제정된
제정법(statutes)이, 법원이 구체적 사건에서 이를 적용하기 이전에도 법으로
간주된다면, 일정한 기준을 만족하는 관습 역시 마찬가지로 법이 될 수
없는가? 입법자의 제정은 법이라는 일반 원칙을 법원이 구속력 있게
받아들이듯이, 일정한 유형의 관습 역시 법이라는 또 다른 일반 원칙을
법원이 구속력 있게 인식한다고 보는 것이 왜 불합리한가? 어떤 사건이
발생했을 때, 법원이 관습을 마치 제정법처럼, 이미 법이기 때문에 적용하는
것이라고 주장하는 것이 어째서 말이 되지 않는가? 물론 어떤 법체계에서는,
관습 규칙이 법으로서의 지위를 가지기 위해서는 반드시 법원이 그것을
그렇게 선언해야 한다고 규정할 \emph{수도 있다(possible)}. 그러나 이것은
단지 \emph{하나의(one)} 가능성일 뿐이며, 법원이 그와 같은
재량권(discretion)을 갖지 않는 체계의 가능성을 배제할 수는 없다.
그렇다면 \textbf{(p.~47)} 관습 규칙은 반드시 법원에 의해 적용되기 전에는
법이 \emph{될 수 없다(cannot)}는 일반적 주장을 어찌 확립할 수 있겠는가?

이러한 반론들에 대한 답변은 종종, 단지 ``무언가가 법이 되기 위해서는
반드시 어떤 사람이 그것이 법이라고 \emph{명령(order)} 해야 한다''는
독단을 되풀이하는 데 그친다. 이와 같은 입장에서, 법원이 제정법과 관습에
대해 취하는 관계가 유사하다는 주장은 거부된다. 이유는, 제정법은 법원이
적용하기 전에 이미 `명령'되었지만, 관습은 그렇지 않다는 것이다. 좀 더
비독단적인 주장들도, 특정 법체계의 구체적인 규정을 지나치게 일반화하기
때문에 설득력을 갖지 못한다. 예컨대, 영국법에서는 어떤 관습이
`합리성(reasonableness)' 기준을 통과하지 못하면 법원에 의해 거부될 수
있다는 사실을 들어, 관습이 법원이 적용하기 전까지는 법이 아니라고
말한다. 하지만 이것은 기껏해야 영국법에서의 관습에 관해 무언가를 말해줄
뿐이다. 그것조차도 확실하지 않다. 왜냐하면, 일부 주장과 달리, 법원이
`합리적인' 관습만을 적용하도록 구속되어 있는 체계와, 법원이 전적인
재량으로 관습을 수용 또는 배제하는 체계를 구별하는 것이 무의미하다고
단정할 수는 없기 때문이다.

두 번째, 보다 근본적인 비판은 다음과 같다. 관습이 법이 될 경우, 그것이
주권자의 묵시적 명령(tacit order)에 의해 법적 지위를 갖게 된다는 주장은
설득력이 없다. 비록 개별 사건에서 법원이 그것을 집행하기 전까지는 관습이
법이 아니라고 인식한다 하더라도, 주권자가 개입하지 않았다는 사실만으로
그가 해당 규칙의 준수를 원했다는 의사 표현으로 해석할 수 있는가? 심지어
앞서 46쪽에서 언급된 매우 단순한 군사적 예시에서도, 장군이 하사의 명령에
개입하지 않았다는 사실로부터 그 명령이 이행되기를 바랐다고 반드시 추론할
수는 없다. 그는 단지 유능한 부하를 달래고자 했거나, 병사들이 어떻게든
잡무를 피하길 바랐을 수도 있다. 물론 경우에 따라, 장군이 해당 잡무가
수행되기를 바랐다고 추론할 수도 있을 것이다. 하지만 그렇다 하더라도, 그
추론의 중요한 근거는 장군이 명령의 존재를 알고 있었으며, 그것을 숙고할
시간도 있었고, 그럼에도 불구하고 아무 행동도 하지 않기로 결정했다는
사실이다. 그러나 현대 국가에서 관습의 법적 지위를 설명하기 위해 주권자의
의사가 묵시적으로 표현되었다는 개념을 사용하는 것은 매우 부적절하다.
현대 국가에서는, 우리가 주권자를 최고 입법기관(supreme legislature)이나
유권자(electorate)로 식별한다 하더라도, 그 `주권자'에게 그러한 지식,
숙고, 비개입하기로 하는 결정을 \textbf{(p.~48)} 귀속시키는 것이 거의
불가능하다. 대부분의 법체계에서 관습은 제정법보다 하위의 법 원천이라는
점은 사실이다. 이는 입법부가 언제든지 그 법적 지위를 박탈\emph{할 수
있다(could)}는 뜻이다. 그러나 입법부가 그렇게 하지 않았다는 사실이 곧
입법부의 의사(wish)를 반영한다고 말할 수는 없다. 입법부가 법원이
적용하는 관습 규칙에 관심을 기울이는 경우는 드물며, 유권자의 관심은
그보다 훨씬 더 드물다. 그러므로 이들의 `비개입'은, 장군이 하사의 명령에
개입하지 않은 경우와 비교될 수 없다. 하사의 사례에서조차, 우리가 장군의
의사를 그렇게 추론하려면 상당한 근거가 필요한 것이다.

그렇다면, 관습의 법적 인식(legal recognition)은 도대체 무엇을
의미하는가? 만약 그것이 특정 사건에 관습을 적용한 법원의 명령이나, 최고
입법기관의 묵시적 명령에 의한 것이 아니라면, 관습 규칙은 어떻게 법적
지위를 가지는가? 법원이 그것을 적용하기 전에, 제정법과 마찬가지로 법일
수 있는 이유는 무엇인가? 이러한 질문들은, 우리가 다음 장에서 상세히
검토할 ``법이 존재하려면, 반드시 어떤 주권적 인물의 일반
명령---명시적이든 묵시적이든---이 존재해야 한다''는 교리를 분석하기
전까지는 완전히 답변될 수 없다. 그에 앞서, 이 장의 결론을 다음과 같이
요약할 수 있다:

법을 강제 명령의 체계로 보는 이론은 처음부터 다음과 같은 세 가지 점에서
이 모델에 들어맞지 않는 법의 다양성에 직면한다.첫째, 형벌법조차도
타인에게 내리는 명령과는 적용 범위가 다른 경우가 많다. 왜냐하면 이러한
법은 다른 사람뿐 아니라 그 법을 제정한 사람에게도 의무를 부과할 수 있기
때문이다. 둘째, 다른 제정법들은 명령과는 달리 특정 행위를 요구하지 않고,
오히려 권한(power)을 부여하는 기능을 수행한다. 이러한 법들은 의무를
부과하기보다는, 법의 강제적 틀 내에서 법적 권리와 의무를 자율적으로
창출할 수 있는 수단을 제공한다. 셋째, 비록 제정법의 제정은 여러 면에서
명령과 유사할 수 있으나, 일부 법 규칙은 관습에서 유래하며, 의도적인 법
창조 행위에 그 지위를 의존하지 않는다.

이러한 비판들에 맞서기 위해 다양한 우회적 장치들이 제안되어 왔다.
악(惡)의 위협 또는 `제재(sanction)'라는 단순한 개념은 법적 행위의
무효(nullity)까지 포함하도록 확장되었고, \textbf{(p.~49)} 법적 규칙의
개념은 권한을 부여하는 규칙을 법의 단편(fragment)으로 치부하여
제외하도록 좁혀졌으며, 입법자의 자기구속 문제는 한 사람 안에 두 인물이
있다고 가정하는 방식으로 해결되었다. 나아가, 명령의 개념 자체도 언어적
표현에서 하급자의 명령에 대한 비개입이라는 `묵시적 표현'으로 확장되었다.
이러한 장치들의 교묘함에도 불구하고, 위협에 의해 뒷받침되는 명령의
모델은 오히려 법의 본질을 가리는 경향이 있다. 법의 다양성을 이러한
단일하고 단순한 형태로 환원하려는 노력은, 허위의 통일성(spurious
uniformity)을 법에 강요하게 된다. 사실, 제5장에서 주장하겠지만, 법의
가장 핵심적인 특성---혹은 적어도 결정적인 특성 중 하나---은 바로 다양한
유형의 규칙이 융합되어 있다는 데에 있을지도 모른다.

\newpage

\subsection{CHAPTER III 주석}\label{chapter-iii-uxc8fcuxc11d}

\emph{26쪽. 법의 다양한 형태(The varieties of law).} 법에 대한 일반적
정의를 추구하는 과정에서, 서로 다른 유형의 법 규칙들 사이에 존재하는
형식과 기능상의 차이가 가려지게 되었다. 이 책의 주된 주장은,
의무(duty)를 부과하는 규칙과 권한(power)을 부여하는 규칙 사이의 차이가
법철학(jurisprudence)에서 결정적으로 중요하다는 것이다. 법은 이 두 가지
상이한 유형의 규칙이 결합된 것(union)으로 이해될 때 가장 잘 설명된다.
따라서 본 장에서는 법 규칙 유형들 사이의 주요 구분으로 이 점을 강조하고
있지만, 다른 여러 구분 역시 가능하며, 어떤 목적에서는 반드시 이루어져야
한다. 보다 다양한 사회적 기능을 반영하는 법의 분류는, 그 언어적 형식에도
자주 드러나며, 이에 관해서는 Daube의 \emph{Forms of Roman Legislation}
(1956)을 참조할 수 있다.

\emph{27쪽. 형사법과 민사법에서의 의무(Duties in criminal and civil
law).} 의무 부과 규칙과 권한 부여 규칙 사이의 구분에 주의를 집중하기
위해, 우리는 형법상의 의무와 불법행위법(tort) 및 계약법(contract)상의
의무 간의 많은 차이들을 생략하였다. 이러한 차이에 주목한 일부
이론가들은, 계약이나 불법행위의 경우에 특정 행위를 이행하거나 금지할
`1차(primary)' 또는 `선행(antecedent)' 의무는 실체가 없으며, 진정한
의무는 특정한 사정(예: 이른바 1차 의무 불이행) 하에서 손해배상을 하도록
하는 구제(remedial) 또는 제재적(sanctioning) 의무뿐이라고 주장하기도
한다. 이에 대해서는 Holmes의 \emph{The Common Law} 제8장을 참조하되,
Buckland의 \emph{Some Reflections on Jurisprudence} 96쪽과 \emph{The
Nature of Contractual Obligation}, \emph{Cambridge Law Journal}
제8권(1944년)에서의 비판을 함께 참고하라. 또한 Jenks의 \emph{The New
Jurisprudence} 179쪽도 참조할 것.

\emph{27쪽. 의무(obligation)와 의무(duty).} 영미법(Anglo-American
Law)에서는 이 두 용어가 현재 대체로 동의어로 사용되지만, 형법이
`의무(obligations)'를 부과한다고 말하는 것은 (법의 요구 사항에 대한
추상적 논의, 예컨대 도덕적 의무와 법적 의무의 분석 같은 경우를 제외하면)
일반적이지 않다. 변호사들은 여전히 계약상의 의무나 불법행위 후 손해배상
의무 등처럼 특정 개인이 다른 특정 개인에 대해 가지는 권리(right \emph{in
personam})가 존재하는 경우에 `obligation'이라는 단어를 가장 흔히
사용한다. 그 외의 경우에는 `duty'라는 표현이 보다 일반적이다. 이는 현대
영어법에서, 로마법의 \emph{obligatio}가 특정한 개인들 사이를 묶는
\emph{법적 유대(vinculum juris)}라는 본래 의미에서 살아남은 유일한
흔적이다. (Salmond, \emph{Jurisprudence}, 제11판, 제10장 260쪽 및 제21장
참조; 또한 본서 제5장 제2절 참조).

\emph{28쪽. 권한 부여 규칙(Power-conferring rules).} 대륙법계의
법이론에서는 법적 권한을 부여하는 규칙들을 종종 `권능 규범(norms of
competence)'이라 부른다. (Kelsen, \emph{General Theory}, 90쪽, A. Ross,
\emph{On Law and Justice} (1958), 34, 50--59, 203--225쪽 참조). Ross는
사적 권능(private competence)과 공적 권능(social competence)을 구분하며,
이에 따라 계약과 같은 사적 행위와 공적 법행위를 구별한다. 그는 또한 권능
규범은 의무를 명령하지 않는다고 지적한다. ``권능 규범은 그 자체로는
직접적인 지시(directive)가 아니며, 의무로서 절차를 명령하지 않는다.
\ldots{} 권능 규범은 권능을 가진 자에게 그 권한을 행사해야 한다고 말하지
않는다.'' (위 책, 207쪽). 그러나 이러한 구분에도 불구하고, Ross는 본
장에서 비판된 관점을 받아들인다. 즉, 그는 권능 규범도 결국 `행위
규범(norms of conduct)'으로 환원될 수 있다고 주장하며, 두 유형의 규범은
모두 ``법원을 향한 지시(directives to the Courts)''로 해석되어야 한다고
본다. (같은 책, 33쪽).

이 장의 본문에서 다양한 시도를 비판하며 제시한 논점은, 이러한 두 규칙
유형 간의 구분을 없애거나 단지 피상적인 차이에 불과하다고 보는 입장들이
갖는 문제점에 있다. 이러한 구분이 중요한 의미를 갖는 다른 사회적 삶의
양태들을 고려하는 것도 유익하다. 도덕(morals)에서, 어떤 사람이 구속력
있는 약속을 했는지를 판단하는 모호한 규칙들은, 개인에게 제한된 도덕적
입법권을 부여하는 기능을 하며, 강제적 의무 규칙과는 구분될 필요가 있다
(\emph{Melden, `On Promising',} \emph{Mind} 제65권 (1956); Austin,
`Other Minds', \emph{PAS} 보충권 20권 (1946), \emph{Logic and Language}
2집에 재수록; Hart, `Legal and Moral Obligation', Melden 편,
\emph{Essays on Moral Philosophy} 참조). 복잡한 게임의 규칙들도 이
관점에서 유익하게 분석될 수 있다. 일부 규칙은 형법처럼 특정 행위를
금지하고 위반 시 제재를 가한다 (예: 반칙 행위나 심판에 대한 무례). 또
다른 규칙들은 심판, 기록원, 주심 등의 경기 담당자들의
관할권(jurisdiction)을 정의한다. 다시 다른 규칙들은 득점 조건을 규정한다
(예: 득점(goal)이나 출루(run)). 출루나 득점을 위한 조건을 충족하는 것은
경기에서 승리를 향한 결정적 단계이며, 이를 충족하지 못한 것은 득점
실패이며, 이 관점에서 `무효(nullity)'로 간주된다. 여기에는 표면적으로
서로 다른 유형의 규칙들이 존재하며, 이들은 게임 내에서 다양한 기능을
수행한다. 그러나 어떤 이론가는 이러한 규칙들이 모두 하나의 유형으로
환원될 수 있으며, 환원되어야 한다고 주장할 수도 있다. 예컨대 득점
실패라는 무효는 금지된 행위에 대한 제재(sanction)로 간주될 수도 있고,
혹은 모든 규칙은 특정 조건 하에서 경기 담당자들에게 특정 행동을 하도록
지시하는 것으로 해석될 수 있다는 것이다 (예: 점수를 기록하거나 선수를
퇴장시키는 것). 그러나 이러한 방식으로 두 유형의 규칙을 하나로 환원하는
것은, 규칙들의 고유한 성격을 흐리게 만들며, 게임에서 중심적인 요소를
단지 부차적인 요소에 종속시키는 결과를 낳는다. 따라서 본 장에서 비판한
환원주의적 법이론들이, 사회 활동의 한 체계로서 법이 구성하는 법 규칙들의
다양한 기능들을 유사하게 은폐하고 있는지, 다시금 숙고할 필요가 있다.

\emph{29쪽. 사법권을 부여하는 규칙과 판사에게 의무를 부과하는 추가 규칙}
동일한 행위가 관할권을 초과한 행위(excess of jurisdiction)로 간주되어
사법 결정이 \emph{무효(nullity)}로 취소될 수 있는 동시에, 판사가 자신의
관할권을 넘지 않도록 요구하는 특별한 규칙의 위반으로 간주될 수 있다고
해도, 이 두 규칙 유형의 구분은 여전히 유지된다. 예컨대, 판사가 자신의
관할권을 넘어 사건을 심리하거나 그 결정이 무효화될 다른 방식으로
행동하는 것을 방지하기 위한 금지명령(injunction)이 인식된다면, 혹은
그러한 행위에 대해 제재(penalty)가 규정되어 있다면 이러한 경우에
해당한다. 마찬가지로, 법적으로 자격이 없는 인물이 공적 절차에 참여한
경우, 그 절차를 무효화시킬 뿐만 아니라 해당 인물에게 제재를 가할 수
있다. (이와 관련된 제재에 대해서는 \emph{Local Government Act 1933},
제76조 및 \emph{Rands v. Oldroyd} (1958), 3 AER 344 참조. 다만 이 법은
지방정부의 절차가 구성원 자격의 결함으로 인해 무효가 되지는 않는다고
규정함---같은 법, 제3부, 부속서 III, 제5항).

\emph{33쪽. 제재로서의 무효(nullity as a sanction)} Austin은 \emph{The
Lectures} 제23강에서 이 개념을 채택하였으나, 별도로 전개하지는 않았다.
이에 대한 비판은 Buckland, 앞서 인용한 책의 제10장에서 참조하라.

\emph{35쪽. 권한 부여 규칙을 의무 부과 규칙의 단편으로 보는 이론} 이
이론의 극단적인 형태는 Kelsen이 전개한 것으로, 법의 기본 규칙(primary
rules of law)은 법원이나 공적 기관이 일정 조건 하에서 제재를 부과해야
한다는 규칙이라는 견해와 함께 제시된다 (\emph{General Theory}, 58--63쪽;
헌법 관련 논의는 143--144쪽 참조). 그는 ``헌법 규범은 독립적이고 완결된
규범이 아니라, 법원이나 기타 기관이 적용해야 할 모든 법 규범의 내재적
부분(intrinsic parts)이다''라고 주장한다. 이러한 교리는 법을
\emph{정적(static)} 방식으로 설명할 때에만 성립하는 것으로 제한되며,
\emph{동적(dynamic)} 방식과는 구별된다 (같은 책, 144쪽). Kelsen의 논의는
또한 다음과 같이 복잡해진다. 즉, 그는 계약 체결 권한 등 사적 권한을
부여하는 규칙의 경우, 계약에 의해 생성되는 `이차 규범(secondary norm)'
또는 의무는 ``단순한 법이론상의 보조적 구성물이 아니다''라고 주장한다
(같은 책, 90쪽, 137쪽). 그럼에도 본질적으로 Kelsen의 이론은 본 장에서
비판된 이론에 속한다. 보다 단순한 형태로는 Ross의 주장을 참조할 수 있다:
``권능 규범은 간접적 형식의 행위 규범이다''(Ross, 같은 책, 50쪽). 모든
규칙을 의무 부과 규칙으로 환원하는 보다 온건한 이론으로는 Bentham의
\emph{Of Laws in General}, 제16장 및 부록 A--B 참조.

\emph{39쪽. 법적 의무를 예측으로, 제재를 행위에 대한 세금으로 보는 이론}
이 두 이론은 Holmes의 ``The Path of the Law'' (1897, \emph{Collected
Legal Papers})를 참조하라. Holmes는 의무 개념이 도덕적 의무와
혼동되었다고 보고, 그것을 ``냉소적 산(acid of cynicism)''에 담궈
씻어내야 한다고 주장하였다. ``우리는 `의무(duty)'라는 단어에 도덕에서
끌어온 모든 내용을 채워 넣는다'' (같은 책, 173쪽). 그러나 법 규칙을 행위
기준(standards of conduct)으로 간주하는 관점은, 그것들을 도덕 기준과
동일시할 필요는 없다 (제5장 제2절 참조). Holmes가 `나쁜 사람(Bad Man)'이
어떤 일을 할 경우 불쾌한 결과를 겪게 될 것이라는 `예언(prophecy)'으로서
의무를 정의한 데 대한 비판은 A. H. Campbell의 Frank 저서 \emph{Courts on
Trial}에 대한 서평, \emph{Modern Law Review} 제13권(1950), 그리고 본서
제5장 제2절, 제7장 제2절과 제3절에서 확인할 수 있다.

미국 연방대법원은 제1조 제8절(의회에 과세 권한을 부여하는 미국 헌법
조항) 적용에 있어서 벌금(penalty)과 세금(tax)을 구별하는 데 어려움을
겪어왔다. 참조: \emph{Charles C. Steward Machine Co.~v. Davis}, 301 U.S.
548 (1937).

\emph{41쪽. 개인은 의무 수납자인 동시에 사적 입법자임} Kelsen의 법적
능력(legal capacity)과 사적 자율성(private autonomy) 논의와 비교하라
(\emph{General Theory}, 90쪽, 136쪽).

\emph{42쪽. 입법자가 자신을 구속하는 입법(Legislation binding the
legislator)} 명령이나 사령은 언제나 타인에게 적용되는 것이라는 전제에
기반하여 명령 이론(imperative theories of law)을 비판하는 논의는 Baier의
\emph{The Moral Point of View} (1958), 136--139쪽을 참조하라. 그러나
일부 철학자들은 자기지향 명령(self-addressed command)의 개념을 수용하며,
이를 1인칭 도덕 판단(first-person moral judgments) 분석에도 적용한다.
(Hare, \emph{The Language of Morals}, 제11--12장에서 `Ought' 논의 참조).
본문에서 입법과 약속의 유비를 제시한 부분은 Kelsen, \emph{General
Theory}, 36쪽도 함께 보라.

\emph{45쪽. 관습과 묵시적 사령(Custom and tacit commands)} 본문에서
비판된 교리는 Austin의 것으로, \emph{The Province of Jurisprudence
Determined}, 제1강 30--33쪽 및 \emph{The Lectures}, 제30강 참조. 묵시적
사령(tacit command) 개념을 사용하여 명령 이론과 조화되게 다양한 법
형태의 인식을 설명하는 시도는 Bentham의 \emph{Of Laws in General},
21쪽에서 제시된 `수용(adoption)' 및 `수습(susception)' 이론을 참조하라.
또한 Morison, `Some Myth about Positivism', \emph{Yale Law Journal}
제68권 (1958) 및 본서 제4장 제2절도 참고. 묵시적 사령 개념에 대한 비판은
Gray, \emph{The Nature and Sources of the Law}, §§ 193--199 참조.

\emph{49쪽. 명령 이론과 법령 해석(Imperative theories and statutory
interpretation)} 법이 본질적으로 명령(order)이며, 따라서 입법자의
의지(will) 또는 의도의 표현이라는 교리는, 본장에서 제기한 것 외에도
다양한 비판에 직면한다. 어떤 비판자들은, 이러한 교리가 법령 해석의
과업을 `입법자의 의도(the intention of the legislator)'를 찾는 것으로
오도하였다고 본다. 그러나 입법자가 복합적이고 인위적인 집단(body)일
경우, 그 의도를 찾거나 증명하는 데 어려움이 있을 뿐 아니라, ``입법자의
의도''라는 표현 자체에 명확한 의미가 없을 수 있다. 관련 논의는
Hägerström, \emph{Inquiries into the Nature of Law and Morals}, 제3장
74--97쪽 참조. 입법 의도라는 개념에 포함된 허구에 대해서는 Payne, `The
Intention of the Legislature in the Interpretation of Statute',
\emph{Current Legal Problems} (1956) 참조. 입법자의 `의지(will)' 개념에
대한 논의는 Kelsen, \emph{General Theory}, 33쪽도 함께 보라.

\subsection{\texorpdfstring{\textbf{CHAPTER III 제3판
주석}}{CHAPTER III 제3판 주석}}\label{chapter-iii-uxc81c3uxd310-uxc8fcuxc11d}

\emph{27--32쪽. 권한 부여 규칙(Power conferring rules).} Hart는 『벤담에
대한 에세이: 법철학과 정치이론』(\emph{Essays on Bentham: Jurisprudence
and Political Theory}, 옥스퍼드대 출판부, 1982) 제8장 ``법적 권한(Legal
Powers)''에서 자신의 견해를 수정하였다. 권한 부여 규칙에 대한 보다
일반적인 논의로는 Joseph Raz의 『실천적 이유와 규범(\emph{Practical
Reason and Norms})』(제2판, 옥스퍼드대 출판부, 1990) 97--106쪽을
참조하라. 또한 Joseph Raz와 D. N. MacCormick 간의 「자발적 의무와 규범적
권한」(`Voluntary Obligation and Normative Powers')에 대한 심포지엄도
참고할 것 (1972년 \emph{아리스토텔레스 학회 회보 보충판}, 제46권, 59쪽).
공적 권한(public powers)에 대해서는 G. H. Von Wright, 『규범과
행위(\emph{Norm and Action})』(Humanities Press, 1963) 제10장을
참조하고, 또한 Eugenio Bulygin, 「권능 규범에 관하여」(\emph{On Norms of
Competence}) (1992) \emph{Law and Philosophy} 제11권 201쪽도 참고할 수
있다.

\emph{32쪽. 법의 분류법(Taxonomy of laws).} 보다 정교한 분류 체계에
대해서는 A. M. Honoré의 「실질적 법률(Real Laws)」을 참조하라. 이 글은
P. M. S. Hacker와 J. Raz가 편집한 『법, 도덕, 그리고 사회(\emph{Law,
Morality and Society})』에 수록되어 있다.

\emph{38--39쪽. 제재와 법의 사회적 기능(Sanctions and the social
functions of law).} Hans Oberdiek, 「법과 법체계를 이해하는 데 있어
제재와 강제의 역할」(\emph{The Role of Sanctions and Coercion in
Understanding Law and Legal Systems}) (1976) \emph{American Journal of
Jurisprudence} 제71권 21쪽 참조. Joseph Raz, 『실천적 이유와 규범』
157--162쪽, John Finnis, 『자연법과 자연권(\emph{Natural Law and Natural
Rights})』 266--270쪽, Grant Lamond, 「강제와 법의 본성」(\emph{Coercion
and the Nature of Law}) (2001) \emph{Legal Theory} 제7권 35쪽, Frederick
Schauer, 「Austin이 결국 옳았는가?---법 이론에서 제재의 역할」(\emph{Was
Austin Right After All? On the Role of Sanctions in a Theory of Law})
(2010) \emph{Ratio Juris} 제23권 1쪽도 참조할 것. 법이 강제를 `허용'하는
방식으로 기능한다고 보는 견해에 대해서는 Ronald Dworkin, 『법의
제국(\emph{Law's Empire})』 제3장을 참조하라. 특히 92--94쪽을 주목할 것.

\emph{38--42쪽. `나쁜 사람'의 관점(The `bad man's' point of view).}
William Twining, 「`나쁜 사람' 다시 보기」(\emph{The Bad Man Revisited})
Stephen Perry는 Hart의 이론이 `나쁜 사람'의 관점을 배제할 철학적 자원을
갖추고 있지 않다고 비판한다. 그의 글 「Holmes 대 Hart: 법이론에서의 나쁜
사람」(\emph{Holmes versus Hart: The Bad Man in Legal Theory})은 Steven
J. Burton 편, 『법의 길과 그 영향: 올리버 웬델 홈즈 주니어의
유산』(\emph{The Path of the Law and Its Influence: The Legacy of Oliver
Wendell Holmes Jr.}), 캠브리지대 출판부, 2000년에 수록되어 있다. 법을
전적으로 유인 수단(incentivizing device)으로 이해하려는 시도는 특히
경제학자들 사이에서 지속되고 있다. 관련 저작으로는 Richard Posner의
『법경제학(Economic Analysis of Law)』(제8판, Aspen Publishers, 2010)과
『법철학의 문제들(\emph{The Problems of Jurisprudence})』(하버드대
출판부, 1990)이 있다. 가장 정교한 논의는 Lewis A. Kornhauser, 「법의
규범성」(\emph{The Normativity of Law}) (1999) \emph{American Law and
Economics Review} 제1권 3쪽 참조.

\emph{44--49쪽. 관습법(custom)과 법의 원천으로서의 관습} John Gardner,
『법은 신념의 도약이다(\emph{Law as a Leap of Faith})』 제3장 「일부
법의 유형(Some Types of Law)」 참조. 또한 Amanda Perreau-Saussine과
James B. Murphy 편집, 『관습법의 본성: 법적, 역사적, 철학적
관점(\emph{The Nature of Customary Law: Legal, Historical and
Philosophical Perspectives})』(케임브리지대 출판부, 2007)의 수록
논문들도 참고할 것. Dworkin은 Hart가 그의 승인 규칙(rule of recognition)
이론에 비추어 관습을 일관되게 설명할 수 없다고 비판한다. 관련 논의는
\emph{Taking Rights Seriously}, 41--44쪽 참조.

\end{document}
