\documentclass[12pt, oneside]{book}  % 13pt는 직접 설정함

% --- Language and Font Settings ---
\usepackage[english]{babel}
\usepackage{fontspec}
\usepackage{kotex}  % Korean typesetting

% --- Font Configuration (Main + Korean) ---
\usepackage{libertinus}
\setmainhangulfont[
  Path = ./,
  UprightFont = *Batang Light.ttf,
  BoldFont    = *Batang Medium.ttf
]{KoPubWorld}

% --- Page Geometry: A4 with wide margins for print readability ---
\usepackage[a4paper, margin=1in]{geometry}

% --- Line Spacing ---
\usepackage{setspace}
\setstretch{1.45}

% --- Section Formatting: Disable numbering ---
\usepackage{titlesec}
\setcounter{secnumdepth}{0}

\titleformat{\chapter}[display]
  {\normalfont\Huge\bfseries}
  {}{0pt}{\Huge}\titleformat{\section}
  {\normalfont\Large\bfseries}
  {}{0pt}{\Large}
\titleformat{\subsection}
  {\normalfont\large\bfseries}
  {}{0pt}{\large}

% --- Math and Tables ---
\usepackage{amsmath, amssymb, amsthm, mathtools}
\usepackage{graphicx}
\usepackage{enumitem}
\usepackage{tabularx, booktabs}
\usepackage{footmisc}

% --- Quotes, Hyperlink, Header/Footer ---
\usepackage{csquotes}
\usepackage[hidelinks]{hyperref}
\usepackage{fancyhdr}
\pagestyle{fancy}
\fancyhf{}
\fancyfoot[C]{\thepage}
\setlength{\headheight}{15pt}

% --- Bibliography (biblatex) ---
\usepackage[style=verbose-note, backend=biber, maxbibnames=99]{biblatex}
\addbibresource{references.bib}
\AtEveryBibitem{\clearfield{pages}}

% --- Title Info ---
\title{\Huge\textsc{The Concept of Law} \\[2ex] \Large Third Edition}
\author{\Large H. L. A. Hart}
\date{}

\begin{document}

% --- Title Page ---
\begin{titlepage}
  \centering
  \vspace*{3cm}
  {\Huge\textsc{The Concept of Law}}\\[1.5ex]
  {\Large Third Edition}\\[4ex]
  \textsc{H. L. A. Hart}\\[6ex]
  {\small 2025 SUMMER 강의용 한국어판\\
  강좌 외의 사용을 불허함}
  \vfill
\end{titlepage}

\section{\texorpdfstring{\textbf{II. 법(Laws), 사령(Commands), 그리고
명령(Orders)}}{II. 법(Laws), 사령(Commands), 그리고 명령(Orders)}}\label{ii.-uxbc95laws-uxc0acuxb839commands-uxadf8uxb9acuxace0-uxba85uxb839orders}

\subsection{\texorpdfstring{\textbf{1. 명법의 다양성(Varieties of
Imperatives)}}{1. 명법의 다양성(Varieties of Imperatives)}}\label{uxba85uxbc95uxc758-uxb2e4uxc591uxc131varieties-of-imperatives}

법(law)의 개념을 사령(command)과 습관(habit)이라는 외견상 단순한
요소들로 분석하려는 가장 명확하고 철저한 시도는 오스틴(Austin)이
\emph{『법학의 범위는 어떻게 결정되는가(Province of Jurisprudence
Determined)』}에서 보여준 것이다. 이 장과 다음 두 장에서 우리는
본질적으로 오스틴의 이론과 동일하지만 몇몇 지점에서는 다소 차이를 보일
수도 있는 입장을 서술하고 비판할 것이다. 우리의 주요 관심은 오스틴
개인에 있지 않고, 결함에도 불구하고 지속적인 매력을 지니는 특정 유형의
이론이 가진 정당성에 있다. 따라서 오스틴의 의미가 불분명하거나 그의
견해가 일관되지 않아 보이는 지점에서는 이를 무시하고 명확하고 일관된
입장을 서술하는 데 주저하지 않았다. 또한 오스틴이 비판을 어떻게 반박할
수 있을지에 관해 암시만 제공한 경우, 우리는 이후 이론가들(Kelsen 등)이
취한 접근 방식 일부를 따라 그 이론을 가능한 강한 형태로 구성하고 비판할
수 있도록 발전시켰다.

사회생활의 여러 다양한 상황에서, 한 사람이 다른 사람에게 어떤 행위를
하거나 하지 않기를 바라는 바람을 표현하는 경우가 있다. 이 바람이 단순한
흥미로운 정보 제공이나 의도적 자기노출이 아니라, 상대방이 그 표현된
바람에 따라 행동하기를 의도하여 표명되는 경우, 영어를 포함한 많은
언어들에서는 (비록 필수는 아니지만) \emph{명령법(imperative mood)}이라
불리는 특별한 언어형식을 사용하는 것이 관례적이다. 예컨대 `집에
가라!'(Go home!), `이리 와!'(Come here!), `멈춰!'(Stop!), `그를 죽이지
마라!'(Do not kill him!) 등이 그러하다. 이러한 명령법 형태로 타인을
부르는 사회적 상황은 매우 다양하며, 몇 가지 익숙한 범주를 통해 주요
유형들이 반복되는 것이 확인된다. 예를 들어 `소금 좀 주세요'는 보통
단순한 \emph{요청(request)}이다. 일반적으로 화자가 어떤 서비스를 제공할
수 있는 상대방에게 말하며, 이행하지 않을 경우 어떤 긴급함이나 결과에
대한 암시도 포함되지 않기 때문이다. `나를 \textbf{(p.~19)} 죽이지
마시오'는 상대방의 자비에 의존하거나 상대방이 자신을 구할 수 있는 곤란한
상황에 처한 경우의 \emph{애원(plea)}으로 발화된다. 반면에 `움직이지
마!'는 경우에 따라 \emph{경고(warning)}가 된다. 예컨대 화자가 상대방에게
다가오는 위험(풀숲의 뱀 등)을 알고 있고, 가만히 있는 것이 그 위험을
피하는 데 도움이 되는 상황이 그렇다.

명령법 형식이 사용되는 사회적 상황의 다양성은 그 수가 많을 뿐 아니라,
서로 자연스럽게 경계를 이루며 이어지기도 한다. `애원', `요청', `경고'와
같은 용어들은 단지 거친 구별을 가능케 할 뿐이다. 이 가운데 가장 중요한
상황은 `명법(imperative)'이라는 용어가 특히 적절하게 적용되는 경우이다.
이는 다음과 같은 사례로 잘 드러난다. 권총강도가 은행 직원에게 ``돈을
내놓지 않으면 쏘겠다''고 말하는 경우이다. 여기서 우리가 `돈을
내놓으라'고 은행 직원에게 단순히 \emph{요청하거나(asking)}
\emph{애원하는(pleading with)} 것이 아니라 권총강도가
\emph{명령하고(ordering)} 있다고 말하게 되는 결정적인 특징은, 화자가
자신의 바람을 관철하기 위해 일반인이 해롭거나 불쾌하게 여길 만한 일을
하겠다고 위협함으로써, 돈을 가지고 있는 것이 해당 직원에게 훨씬 덜
매력적인 선택지가 되도록 만든다는 점이다. 만약 권총강도가 자신의 의도를
성공시킨다면, 우리는 그가 직원을 \emph{강압했다(coerced)}고 말하며,
직원이 어떤 의미에서 권총강도의 권한(power) 하에 있었다고 말하게 된다. 이와
같은 경우에는 언어적으로 미묘한 문제들이 발생할 수 있다. 우리는 `권총강도가
직원에게 돈을 내놓으라고 \emph{명령했다(ordered)}'고 말할 수 있고,
`직원이 복종했다(obeyed)'고도 할 수 있다. 그러나 `권총강도가 직원에게 돈을
내놓으라는 \emph{명령을 내렸다(gave an order)}'고 말하는 것은 다소
오해를 낳을 수 있다. 이 표현은 군대식 표현처럼 들리며, 일정한
명령(order)을 내릴 권리나 권위(authority)가 있는 것으로 암시하기
때문이다. 그러나 `권총강도가 자신의 부하에게 문을 지키라고 명령을
내렸다(gave an order)'는 식의 표현은 자연스럽다.

우리는 여기서 이러한 미묘한 차이들에 대해 깊이 들어갈 필요는 없다. 비록
`명령(order)'이나 `복종(obedience)'이라는 단어에는 종종
권위(authority)와 그에 대한 존중(respect)이 내포되어 있지만, 우리는
`위협에 의해 뒷받침되는 명령(orders backed by threats)' 또는 `강압적
명령(coercive orders)'이라는 표현을, 권총강도의 명령처럼 단지 위협에
의해서만 뒷받침되는 명령을 지칭하는 데 사용할 것이다. 그리고 우리는
`복종(obedience)' 또는 `복종하다(obey)'라는 표현을 그러한 명령에 따르는
경우까지 포함하는 의미로 사용할 것이다. 그러나 오스틴의 사령(command)
개념이 법학자들에게 끼친 큰 영향을 고려할 때, \textbf{(p.~20)} 단순히
해악의 위협만으로 복종을 강제하는 상황은 우리가 자연스럽게
`사령'이라고 부르는 상황이 \emph{아니라는(not)} 점은 분명히
짚어야 한다. 이 단어는 군사적 맥락 외에는 흔히 사용되지 않으며,
사령이라는 표현에는 일반적으로 다음과 같은 강한 함의가
따라붙는다. 즉, 군대나 제자 집단처럼 상대적으로 안정된 위계 조직이
존재하고, 그 속에서 사령자는 우월한 지위에 있다는 것이다. 일반적으로는
하사관이 아니라 장군(general)이 사령자(commander)로서 사령을
내리는 것이며, 이러한 특수한 우위의 다른 형태도 이와 같은 방식으로
표현된다. 예컨대 신약성경에서 예수가 제자들에게 사령(command)한 것으로
표현되는 경우가 그러하다. 더 중요한 점은---그리고 이는 다양한 형태의
명법(imperative) 사이에서 핵심적인 구분이다---사령이 존재하는
경우라고 해서 항상 불복종 시 해악이 가해질 수 있다는 위협(threat)이
내재해 있어야 하는 것은 아니라는 점이다. 사령은 본질적으로
사람에 대한 해악을 가할 권한(power)이 아니라, 권위(authority)를 행사하는
것이며, 비록 해악의 위협과 결합될 수는 있어도, 사령(command)은
본질적으로 공포가 아니라 권위에 대한 존중을 호소하는 행위이다.

명백하게도, 사령(command)이라는 개념은 권위(authority)와 매우 강하게
연관되어 있기 때문에, 단순히 위협에 의해 뒷받침되는 권총강도의
명령(order)보다는 법(law)의 개념에 훨씬 더 가깝다. 그럼에도 불구하고,
오스틴(Austin)은 앞 단락에서 지적된 구분을 무시하고 후자의 경우를
사령(command)이라고 오도하게 부르고 있다. 그러나 사령(command)은 우리의
목적에는 오히려 법에 너무 가까운 개념이다. 왜냐하면 법(law)에 내재된
권위(authority)의 요소는 법이 무엇인지 쉽게 설명하기 어렵게 만드는
장애물 가운데 하나이기 때문이다. 그러므로 우리는 법을 해명함에 있어,
동일하게 권위를 포함하고 있는 사령(command)의 개념을 유익하게 사용할 수
없다. 오히려 오스틴의 분석은 여러 결점에도 불구하고 하나의 장점을
지닌다. 즉, 권총강도(gunman)의 상황에 등장하는 요소들은 권위의 요소와 달리
그 자체로 모호하지 않으며 복잡한 설명을 요하지 않는다. 이러한 이유로
우리는 오스틴을 따라 법의 개념을 그로부터 구성해보려 할 것이다. 다만
우리는 오스틴처럼 그것이 성공하리라 기대하기보다는, 실패를 통해 배우기를
희망할 것이다.

\subsection{\texorpdfstring{\textbf{2. 법(law)과 강압적 명령(coercive
orders)}}{2. 법(law)과 강압적 명령(coercive orders)}}\label{uxbc95lawuxacfc-uxac15uxc555uxc801-uxba85uxb839coercive-orders}

현대 국가와 같은 복잡하고 대규모의 사회에서도, 공식적인 인물이 개인과
직접 마주하여 무언가를 하라고 명령(order)하는 상황은 종종 존재한다.
예컨대 경찰관이 \textbf{(p.~21)} 특정 운전자에게 정지하라고 하거나, 특정
거지에게 자리를 옮기라고 명령(order)하는 경우가 있다. 그러나 이러한
단순한 상황들은 법이 기능하는 표준적 방식이 아니며, 그럴 수도 없다. 단지
사회의 모든 구성원에게 그들이 해야 할 모든 행위가 각각 공식적이고
개별적으로 통지되어야 한다면, 이를 보장하기 위해 필요한 공무원의 수를
어느 사회도 감당할 수 없기 때문이다. 대신 이러한 개별화된 통제 방식은
예외적이거나, 특정인을 지명하지 않고 특정 행위도 직접 지시하지 않는
일반적인 지시 형식을 보완하거나 강화하는 부수적 수단에 불과하다. 따라서
(법의 모든 다양성들 중에서도 위협에 의해 뒷받침되는 명령(order)과 가장
유사한) 형사 법규(criminal statute)의 \emph{기준(standard)} 형식 조차도
다음의 두 가지 측면에서 일반적(general)이다; 즉, 그것은 일반적인 행위
유형을 지시하고, 그러한 지시가 자신에게 적용됨을 이해하고 이를 준수할
것으로 기대되는 일반적인 사람의 범주에 적용된다. 이때 공무원이 개별인을
대상으로 직접 대면하여 전달하는 지시는 부차적이다: 만약 일반 지시를 어떤
개인이 따르지 않을 경우, 공무원은 그의 주의를 환기시키고 준수를 요구할
수 있으며(세무 공무원이 그러하듯), 또는 불복종이 공식적으로 확인되고
기록되어, 위협된 처벌이 법원에 의해 부과될 수 있다.

따라서 법적 통제는 본질적으로, 비록 예외는 있지만, 이러한 이중의
의미에서 \emph{일반적(general)}인 지시에 의한 통제이다. 이것이 우리가
법의 특성을 재현하기 위해 권총강도의 단순한 모델에 추가해야 할 첫 번째
요소이다. 영향을 받는 사람의 범위와 그 범위가 표시되는 방식은 서로 다른
법체계, 혹은 심지어 서로 다른 법률에 따라 다양할 수 있다. 현대
국가에서는 특별한 확대 또는 축소의 지시가 없는 한, 일반 법률은 해당 영토
내의 모든 사람에게 적용된다고 이해된다. 교회법(canon law)에서도 이와
유사하게, 특정한 경우를 제외하면 교회의 모든 구성원이 법의 적용
대상이라고 일반적으로 이해된다. 모든 경우에서 특정 법률의 적용 범위는
해당 법률의 해석과 일반적인 이해에 의해 결정된다. 이 지점에서 주목할
만한 것은, 오스틴을 포함한 여러 법학자들이 법이 특정 집단의 사람들에게
\emph{향하여 있다(addressed)}고 말하지만,\footnote{`Addressed to the
  community at large', Austin, above, p.~1 n.~4 at p.~22.} 이러한 표현은
\textbf{(p.~22)} 실제로 존재하지 않는 대면 상황(face-to-face
situation)과의 유사성을 암시하기 때문에 오해를 불러일으킬 수 있다는
점이다. 사람들에게 무언가를 하도록 명령(order)하는 것은 일종의 의사소통
방식으로, 실제로 그들의 주의를 끌거나(attracting their attention) 그것을
끌기 위한 조치를 취하는 것을 포함한다. 그러나 사람들을 위해 법을
제정하는 것에는 그러한 절차가 따르지 않는다. 예컨대 권총강도는 `그 지폐를
내놔라'라고 발화함으로써, 동시에 직원이 무언가를 하길 바란다는 자신의
의사를 표현하고, 그것을 직원의 주의에 도달하게 만든다. 만약 권총강도가 빈
방에서 같은 말을 했을 뿐이라면, 그는 결코 직원을 \emph{지칭(address)}한
것이 아니며, 직원에게 어떤 행위를 하라고 \emph{명령(order)}한 것도
아니다. 우리는 단지 그가 `그 지폐를 내놔라'는 말을 했다고 서술할 수 있을
뿐이다. 이 점에서 법 제정은 타인에게 무언가를 하도록 명령하는 행위와
다르며, 법에 대한 모델로서 이러한 단순한 개념을 사용할 때는 이 차이를
반드시 고려해야 한다. 실제로 법은 제정된 직후 적용 대상자의 주의에
최대한 신속히 도달하는 것이 바람직하다. 입법자의 법 제정 목적은
일반적으로 이것이 이루어지지 않으면 달성될 수 없기 때문에, 법체계는 종종
공포(promulgation)에 관한 특별 규정을 통해 이를 보장한다. 그러나 이러한
절차가 이루어지기 이전에도 법은 법으로서 완결될 수 있으며, 아예 공포되지
않았더라도 여전히 법일 수 있다. 별도의 규정이 없다면, 법은 그 영향을
받는 사람들이 스스로 법이 무엇이며, 그 법이 자신에게 적용되는지를
파악해야 하는 상황에서도 유효하게 성립한다. 일반적으로 법이 특정한
사람들에게 `향해 있다(addressed)'고 말할 때 의미하는 바는, 그 사람들이
해당 법률이 적용되는 대상자들이라는 것이다. 즉, 법률이 일정한 방식으로
행동하라고 요구하는 대상이라는 뜻이다. 만약 우리가 여기서
`향하다(addressed)'는 단어를 사용할 경우, 법 제정과 대면 명령 간의
중요한 차이를 간과할 위험이 있으며, 동시에 두 개의 상이한 질문---`그
법은 누구에게 적용되는가?'와 `그 법은 누구에게 공표되었는가?'---를
혼동할 수 있다.

이러한 일반성(generality)의 특성을 도입하는 것 외에도, 우리가 법이
존재하는 상황을 설명하기 위한 타당한 모델을 얻고자 한다면, 권총강도 상황에
대해 보다 근본적인 변경이 필요하다. 물론 어떤 의미에서 권총강도는 은행
직원보다 우위를 점한다. 그것은 \textbf{(p.~23)} 권총강도가 단기간이지만
위협을 가할 수 있는 능력을 가지며, 이는 은행 직원이 그가 지시받은 특정
행위를 수행하게 만들 가능성이 있기 때문이다. 이 둘 사이에는 이러한
일시적 강압 관계 외에 다른 어떤 우위-열위 관계도 존재하지 않는다. 그러나
강도의 목적에는 그것으로 충분할 수 있다. `그 지폐를 내놓지 않으면
쏘겠다'는 단순한 대면 명령은 그 상황에서 끝나버린다. 권총강도는 은행
직원에게 (자신의 부하들에게는 가능할지 몰라도) 반복적으로 지켜야 할
\emph{지속적 명령(standing orders)}을 내리지는 않는다. 그러나 법은
본질적으로 이러한 `지속성(standing)' 또는 항구적 성격을 지닌다. 따라서
우리가 위협에 의해 뒷받침되는 명령(order)을 법이 무엇인가를 설명하는
개념으로 사용하고자 한다면, 반드시 법이 지니는 이러한 지속적 성격을
재현하려는 노력이 필요하다.

다음과 같은 가정이 필요하다. 즉, 일반적 명령(order)이 적용되는 사람들
사이에, 해당 명령에 불복종할 경우 그에 따른 위협이 실행될 가능성이
있다는 일반적인 믿음이 존재해야 하며, 이는 단지 그 명령이 처음 공포될
때에만이 아니라, 그 명령이 철회되거나 폐지될 때까지 지속적으로
유지되어야 한다. 이러한 불복종의 결과에 대한 지속적인 믿음은, 원래의
명령들이 계속해서 살아 있으며 `지속적(standing)'인 상태를 유지한다고
말할 수 있게 해준다. 다만, 우리가 뒤에서 보게 되듯이, 법의 지속적
성격(persistent quality)을 이러한 단순한 개념으로 분석하는 데는 어려움이
있다. 실제로, 많은 수의 사람들에게 영향을 미치는 이와 같은 지속적 명령이
실행될 가능성에 대한 일반적인 믿음이 존재하려면, 단순한 권총강도(gunman)
상황에서는 재현할 수 없는 많은 요소들이 결합되어야 할 수도 있다. 즉,
이와 같은 위협을 실제로 집행할 수 있는 권한이 존재한다고 믿기 위해서는,
상당수의 인구가 단지 위협을 두려워해서가 아니라 자발적으로 순종하고,
불복종자에 대한 위협의 집행에 협력할 의사가 있어야만 할 수도 있다.

이러한 위협이 실행될 가능성에 대한 일반적인 믿음의 근거가 무엇이든 간에,
법이 존재하는 안정된 상황과 유사한 모델을 구성하기 위해, 우리는 권총강도
상황에 또 하나의 필수적 요소를 추가해야 한다. 즉, 동기가 무엇이든 간에,
대부분의 명령들이 해당되는 사람들에 의해 불복종보다 더 자주 복종된다는
점을 가정해야 한다. 우리는 이를 오스틴(Austin)을 따라 \textbf{(p.~24)}
`일반적 복종 습관(general habit of obedience)'이라고 부를 것이며, 그와
함께, 이 개념이 법의 여러 다른 측면들과 마찬가지로 본질적으로 모호하거나
불분명한 개념이라는 점을 지적할 것이다. 얼마나 많은 사람들이 얼마나 많은
일반 명령에 얼마 동안 복종해야 법이 존재한다고 볼 수 있는가는, 대머리로
간주되기 위해 사람이 얼마나 적은 머리카락을 가져야 하는가 하는
질문만큼이나 명확한 답이 없는 문제이다. 그러나 이와 같은 일반 복종의
사실 안에야말로, 법과 권총강도의 단순한 명령(order) 사이의 결정적인 구별이
존재한다. 단기간 동안 한 사람이 다른 사람에 대해 가지는 일시적 우위는
본래 법의 상대적 지속성과 안정성에 반대되는 것으로 여겨지며, 실제로
대부분의 법체계에서는 권총강도가 가지는 이와 같은 단기적 강압력을 행사하는
것 자체가 형사범죄를 구성한다. 그렇다면, 위협에 의해 뒷받침되는 일반
명령에 대한 일반적이고 습관적인 복종이라는 이 단순하지만 인정컨대 모호한
개념이, 실제 법체계가 가지는 안정성과 연속성을 충분히 재현해낼 수 있는
것인지 살펴보아야 한다.

우리가 권총강도 상황이라는 단순한 모델에 점차 요소를 덧붙여 구성한, 위협에
의해 뒷받침되고 일반적으로 복종되는 일반 명령의 개념은, 분명히 근대
국가의 입법기관에 의해 제정된 형사 법규(penal statute)에 가장 가까운
법의 형태이다. 그러나 이러한 형사 법규와는 명백히 매우 다른 듯 보이는
다른 유형의 법도 존재하며, 우리는 이후에 이들 또한 실상은 단지 이러한
형식의 복잡하거나 위장된 변형일 뿐이라는 주장에 대해 살펴볼 것이다.
그러나 우리가 일반적으로 복종되는 일반 명령이라는 구성된 모델을 통해
형사 법규의 특성조차 재현하고자 한다면, 명령을 내리는 사람에 대해 좀 더
구체적인 설명이 필요하다. 근대 국가의 법체계는 그 영토 내에서의 일정한
\emph{최고성(supremacy)}과 다른 법체계로부터의
\emph{독립성(independence)}이라는 특징을 지닌다. 이러한 개념들은 단순해
보일 수 있지만 실제로는 그렇지 않으며, 우리가 일상적 상식(common-sense)
수준에서 이해하는 바를 표현하자면 다음과 같다. 영국법, 프랑스법, 또는
여타 근대 국가의 법은 비교적 명확한 지리적 경계를 가진 영토 내에
거주하는 인구의 행위를 규율한다. 각 국가의 영토 내에는 \textbf{(p.~25)}
위협에 의해 뒷받침되는 일반 명령을 내리고 이에 대해 습관적 복종을 받는
여러 개인 또는 집단이 있을 수 있다. 그러나 우리는 이들 가운데
일부(예컨대 런던시의회(LCC)나 위임입법(delegated legislation)을 행사하는
장관 등)를 \emph{하위(subordinate)} 입법자로서 영국 의회 내의 여왕(Queen
in Parliament)과 구분해야 하며, 여왕은 최고(supreme) 입법자이다. 이
관계는 습관의 용어로도 설명될 수 있다. 즉, 여왕은 법을 제정할 때
아무에게도 습관적으로 복종하지 않으며, 하위 입법자들은 법에 의해 규정된
범위 내에서 행동함으로써 여왕의 대리인(agent)으로 간주된다. 만약 이들이
그러한 범위를 넘어서 행동한다면, 우리는 더 이상 하나의 영국법 체계를
갖는 것이 아니라 복수의 법체계를 갖게 될 것이다. 그러나 실제로는 여왕이
이와 같은 의미에서 영토 내의 모든 사람에 대해 최고 권위를 가지며, 다른
기관들이 그렇지 않기 때문에, 우리는 영국 내에서 하나의 단일한 체계를
갖고 있으며, 그 안에서 최고와 하위 요소의 위계를 구별할 수 있는 것이다.

여왕이 다른 사람의 명령을 습관적으로 복종하지 \emph{않는다(not)}는 이와
같은 부정적 성격의 설명은, 서로 다른 국가들의 독립된 법체계에 대해
우리가 사용하는 \emph{독립성(independence)}의 개념을 대략적으로 정의해
준다. 예컨대 소비에트 연방의 최고 입법기관은 여왕의 명령에 복종하는
습관을 가지지 않으며, 여왕이 소비에트 연방에 관해 제정한 어떠한
법률도(비록 그것이 영국법의 일부가 될 수는 있겠지만) 소비에트 법의
일부가 되지는 않는다. 그러한 법률이 소비에트 연방의 법이 되기 위해서는
소비에트 입법기관이 여왕의 명령에 습관적으로 복종해야 할 것이다.

이러한 단순한 설명에 따르면(우리는 이후 이를 비판적으로 검토할 것이다),
법체계가 존재하는 곳에서는 어디에나 일반적으로 복종되는 위협에 의해
뒷받침된 일반 명령을 발령하는 개인 또는 집단이 있어야 하며, 그 위협이
불복종 시에 실제로 실행될 것이라는 일반적인 믿음이 존재해야 한다. 이러한
개인 또는 집단은 내부적으로는 최고(supreme)이며 외부적으로는
독립(independent)적이어야 한다. 오스틴을 따라 이러한 최고이자 독립적인
개인이나 집단을 주권자(sovereign)라고 부른다면, 어떤 국가의 법도 주권자
또는 그에게 복종하는 하위 입법자들이 발한, 위협에 의해 뒷받침된 일반
명령으로 구성된다고 할 수 있다.

\newpage

\subsection{CHAPTER II 주석}\label{chapter-ii-uxc8fcuxc11d}

\emph{18쪽. 명법(imperative)의 다양성.} 명법을 `명령(order)',
`간청(plea)', `논평(comment)' 등으로 분류하는 것은, 사회적 상황,
당사자들 사이의 관계, 힘의 사용에 대한 의도와 같은 여러 요소에 의존하며,
이는 사실상 아직 거의 탐구되지 않은 주제이다. 명법에 관한 대부분의
철학적 논의는 (1) 명법 언어와 서술(indicative) 혹은 기술(descriptive)
언어 간의 관계 및 전자를 후자로 환원할 수 있는 가능성에 관한 문제(예:
Bohnert, 「The Semiotic Status of Commands」, \emph{Philosophy of
Science} 제12권 (1945))나, (2) 명법들 사이에 어떤 연역적 관계가
존재하는지 여부에 관한 문제(예: Hare, 「Imperative Sentences」,
\emph{Mind} 제58권 (1949), 또한 \emph{The Language of Morals} (1952);
Hofstadter and McKinsey, 「The Logic of Imperatives」, \emph{Philosophy
of Science} 제6권 (1939); Hall, \emph{What is Value} (1952), 제6장;
Ross, 「Imperatives and Logic」, \emph{Philosophy of Science} 제11권
(1944))에 집중되어 있다. 이러한 논리적 문제에 대한 연구는 중요하지만,
사회적 맥락에 따라 명법의 다양한 유형을 식별하는 작업 역시 큰 필요성이
있다. 문법적 명령형(imperative mood) 문장이 어떤 표준적 상황에서
일반적으로 `명령(order)', `간청(plea)', `요청(request)',
`사령(command)', `지시(directions)', `설명(instructions)' 등으로
분류되는지를 묻는 것은, 단순히 언어에 관한 사실을 밝히는 것이 아니라,
언어 속에 드러나는 다양한 사회적 상황과 관계들 사이의 유사성과 차이를
파악하는 방법이기도 하다. 이러한 구별에 대한 인식은 법, 도덕, 사회학의
연구에 매우 중요하다.

\emph{18쪽. 명법은 타인이 어떤 행위를 하거나 하지 않기를 바라는 바를
표현한 것.} 언어에서 명령형의 표준적 사용을 이와 같이 설명할 때,
주의해야 할 점은 화자가 단순히 타인이 특정 방식으로 행동하길 바란다는
자신의 의사를 정보로서 드러내는 경우와, 그 발화를 통해 실제로 상대방이
그렇게 행동하도록 유도하려는 의도를 지닌 경우를 구분해야 한다는 것이다.
전자의 경우에는 통상 명령형이 아니라 평서형(indicative mood)이
적절하다(이 구분에 대해서는 Hägerström, \emph{Inquiries into the Nature
of Law and Morals}, 제3장, §4, 116--126쪽 참조). 그러나 화자의 발화
목적이 상대방이 자신의 의도대로 행동하는 것이라고 해서, 그것만으로
명령형의 표준적 사용을 설명하기에는 부족하다. 왜냐하면, 화자가 단순히
그와 같은 목적을 갖는 것뿐만 아니라, 상대방이 그러한 목적이 있음을
인식하고, 그에 따라 영향을 받아 행동하기를 바라는 의도도 함께 갖고
있어야 하기 때문이다. 이러한 복합적인 요소는 본문에서는 생략되었으나,
이에 대해서는 Grice, 「Meaning」, \emph{Philosophical Review} 제66권
(1957), Hart, 「Signs and Words」, \emph{The Philosophical Quarterly}
제2권 (1952)을 참조하라.

\emph{19쪽. 권총강도(gunman) 상황, 명령(order), 복종(obedience).}
`명법(imperative)'이라는 일반 개념을 분석함에 있어 직면하는 어려움 중
하나는, 명령(order), 사령(command), 요청(request) 등의 다양한 유형들에
공통된 요소, 즉 타인이 어떤 행동을 하거나 하지 않기를 바라는 의도의
표현을 가리키는 단어가 존재하지 않는다는 점이다. 이와 마찬가지로, 그러한
행동을 실제로 수행하거나 자제하는 것을 지칭하는 단일 단어도 없다.
`명령(order)', `요구(demand)', `복종(obedience)', `이행(compliance)'과
같은 자연스러운 표현들은, 각각이 통상 사용되는 특수한 상황의 성격에 의해
그 의미가 영향을 받는다. 가장 중립적으로 보이는 `지시하기(telling
to)'라는 표현조차, 어느 한쪽이 다른 쪽에 대해 일정한 우위를 점하고
있다는 뉘앙스를 담고 있다. 권총강도 상황을 설명하기 위해, 우리는
`명령(order)'과 `복종(obedience)'이라는 표현을 선택했는데, 이는 권총강도가
은행원에게 돈을 건네라고 \emph{명령하였다(ordered)}고 말하거나, 은행원이
이에 \emph{복종하였다(obeyed)}고 말하는 것이 자연스럽기 때문이다. 물론,
추상 명사인 `명령(order)'과 `복종(obedience)'이라는 표현은 이 상황을
묘사하기 위해 자연스럽게 쓰이지는 않을 수 있다. 왜냐하면 전자에는 일정한
권위(authority)의 뉘앙스가 내포되어 있고, 후자는 종종 미덕으로 간주되기
때문이다. 그러나 강압적 명령(coercive orders)으로서의 법 이론을 설명하고
비판하는 데 있어, 우리는 이러한 권위나 당위의 함축 없이 `명령(order)'과
`복종(obedience)'이라는 명사 및, `명령하다(order)'와
`복종하다(obey)'라는 동사를 사용하였다. 이는 단지 편의상 선택된 것이며,
어떠한 쟁점에 대해 결론을 미리 내리는 것은 아니다. 벤담(Bentham)은
(\emph{Fragment of Government}, 제1장, 제12절 주석에서),
오스틴(Austin)은 (\emph{The Province}, 14쪽에서) 이러한 방식으로
`복종(obedience)'이라는 용어를 사용하였다. 벤담은 여기서 언급된 모든
어려움을 인식하고 있었으며, 이에 대해서는 \emph{Of Laws in General},
298쪽 주석(a) 참조.

\emph{20쪽. 사령(command)으로서의 법, 그리고 오스틴(Austin) 이론과의
관계.} 본 장의 제2절에서 구성된 강제적 명령(coercive orders)으로서의
법에 관한 단순 모델은 오스틴의 『법철학의 원리(The Province of
Jurisprudence Determined)』에서의 이론과 다음과 같은 점에서 다르다.

(\textbf{a}) \emph{용어 사용.} 본문에서 설명된 이유로 인해,
`사령(command)'이라는 용어 대신 `위협이 뒷받침된 명령(order backed by
threats)' 혹은 `강제적 명령(coercive orders)'이라는 표현이 사용되었다.

(\textbf{b}) \emph{법의 일반성.} 오스틴은 (같은 책, 19쪽) `법(laws)'과
`개별 사령(particular commands)'을 구분하며, 사령(command)은 그것이
``일반적인 행위나 금지를 의무화한다면'' 법 또는 규칙이라 주장한다. 이
관점에 따르면, 사령이 비록 주권자(sovereign)에 의해 단 한 명의 개인에게
`주소(addressed)'되었더라도, 그것이 단일한 행위가 아닌 어떤 행위 유형
또는 종류에 대한 의무를 부과한다면, 법으로 간주될 수 있다. 본문의 법체계
모델에서는 명령들이 일반적이며, 이는 그것들이 개인들의 범주에 적용되고,
행위의 범주를 지시한다는 이중적 의미에서이다.

(\textbf{c}) \emph{공포(fear)와 의무(obligation).} 오스틴은 때때로 어떤
사람이 실제로 제재(sanction)를 두려워할 경우에만 의무를 지닌다고
암시한다(같은 책, 15쪽과 24쪽 및 『강의록』 제22강 (5판), 444쪽: ``그는
해악에 노출되어 있으며 그것을 두려워하기 때문에 \emph{구속되거나(bound)}
\emph{의무를 지닌다(obliged)}''). 그러나 그의 주요 주장은, 그 사람이
그것을 두려워하든 말든, ``아주 작은 해악에 처할 아주 작은
가능성''만으로도 충분하다는 것이다(『The Province』, 16쪽). 강제적
명령으로서의 법의 모델에서는, 단지 불복종이 위협된 해악으로 이어질
가능성이 있다는 \emph{일반적인 믿음}이 존재해야 한다는 조건만을
설정한다.

(\textbf{d}) \emph{권한(power)과 법적 의무.} 이와 유사하게,
사령(command)과 의무(obligation)를 분석하면서, 오스틴은 처음에는 사령의
발령자가 실제로 해악을 가할 \emph{능력}을 가져야 한다고 제안한다(즉 `할
수 있고, 하려는 willing' 자이어야 한다). 하지만 이후에는 그 요건을
``아주 작은 해악의 아주 작은 가능성''으로 약화시킨다(같은 책, 14쪽,
16쪽). 이러한 오스틴의 정의상의 모호성에 대해서는 하트(Hart),
「법적·도덕적 의무」, Melden 편 『도덕철학 에세이』(1958), 제5장 §2
참조.

(\textbf{e}) \emph{예외 사례들.} 오스틴은 선언적 법(declaratory laws),
허용적 법(permissive laws, 예: 폐지 규정), 불완전한 법(imperfect laws)을
그의 사령 중심의 일반적 법 정의의 예외로 간주한다(같은 책, 25--29쪽). 본
장 본문에서는 이 점을 고려하지 않았다.

(\textbf{f}) \emph{입법부와 주권자.} 오스틴은 민주주의에서
유권자(electorate)가 입법부에 있는 대표자들이 아니라 주권자 집단을
구성하거나 그 일부를 이룬다고 보았다. 비록 영국에서는 유권자가 자신의
주권을 행사하는 유일한 방식이 대표자들을 선출하고 나머지 주권적 권한을
그들에게 위임하는 것이지만 말이다. 그는 `정확하게 말하면' 이러한 입장이
옳다고 주장했으나, 모든 헌법학자들이 그러하듯이 의회(parliament)가
주권을 가진다고도 말했다(같은 책, 강의 VI, 228--235쪽). 본문의
논의에서는 의회와 같은 입법부를 주권자와 동일시하고 있으나, 이에 대한
정밀한 분석은 제4장 §4를 참조하라.

(\textbf{g}) \emph{오스틴 이론의 세부 보완과 수정.} 본서의 후속
장들에서는, 오스틴 이론을 방어하기 위해 사용된 몇몇 개념들이 자세히
검토되며, 이들은 본 장에서 구성된 모델에는 포함되어 있지 않다. 이들
개념은 오스틴이 직접 도입한 것이나, 경우에 따라서는 개략적 또는 미완의
형태로만 제시되었으며, 켈젠(Kelsen)과 같은 후속 이론가들의 주장을
예견하는 요소도 있다. 여기에는 다음 개념들이 포함된다: `묵시적
사령(tacit command)' (제3장 §3, 45쪽 및 제4장 §2, 64쪽 참조),
무효(nullity)를 하나의 제재(sanction)로 보는 견해 (제3장 §1), `진정한
법'은 제재를 적용하도록 공직자에게 요구하는 규칙이라는 이론 (제3장 §1),
유권자를 비상적 주권 입법기관으로 보는 관점 (제4장 §4), 주권자의
통일성과 지속성 개념 (제4장 §4, 76쪽). 오스틴에 대한 평가에 있어서는 W.
L. Morison, 「법실증주의에 대한 몇 가지 신화(Some Myth about
Positivism)」, \emph{Yale Law Journal} 제68권 (1958)을 참조하라. 이는
오스틴에 대한 초기 학자들의 심각한 오해들을 바로잡는다. 또한 A. Agnelli,
『존 오스틴: 법실증주의의 기원에 대하여(John Austin alle origini del
positivismo giuridico)』(1959), 제5장도 참고하라.

\subsection{CHAPTER II 제3판
주석}\label{chapter-ii-uxc81c3uxd310-uxc8fcuxc11d}

\emph{18쪽. 오스틴(Austin)의 이론.} 하트(Hart)는 여기서 오스틴 이론의
단순화된 재구성을 다루고 있음을 명확히 하고 있다(오스틴의 이론 자체도
벤담(Bentham)의 이론의 단순화 버전이다). 오스틴 자신의 견해에 대해서는
다음 문헌들을 참조하라: W. L. Morison, 『존 오스틴(John
Austin)』(스탠퍼드 대학 출판부, 1982); W. E. Rumble, 『존 오스틴의 사상:
법이론, 식민지 개혁, 영국 헌정(John Austin: Jurisprudence, Colonial
Reform, and the British Constitution)』(애슬론 출판사, 1985). 18세기
후반부터 19세기 중반까지의 영국 법사상 전반에 대해서는 마이클
롭번(Michael Lobban), 『보통법과 영국 법이론 1760--1850(The Common Law
and English Jurisprudence 1760--1850)』(옥스퍼드 대학 출판부, 1991)를
참고하라.

\emph{18--20쪽. 명법(imperatives)으로서의 법.} 이 주제에 대해서는 닐
매코믹(Neil MacCormick), 「법적 의무와 명법 오류(Legal Obligation and
the Imperative Fallacy)」, A. W. B. 심프슨 편 『옥스퍼드 법이론 에세이
제2집(Oxford Essays in Jurisprudence, 2nd series)』(옥스퍼드 대학
출판부, 1973)도 참조하라. 하트 자신의 이론 속에도 일종의 명법적
이론(imperatival theory)이 남아 있다고 지적한 것으로는 G. J. 포스테마(G.
J. Postema), 「사령으로서의 법: 현대 법이론 속 사령 모델(Law as Command:
The Model of Command in Modern Jurisprudence)」, (2001)
\emph{Philosophical Issues} 제11권 470쪽을 참고하라. 명법적 이론의
요소들을 방어하고자 한 시도들로는 매튜 H. 크레이머(Matthew H. Kramer),
『법실증주의의 옹호: 불필요한 장식을 걷어낸 법(In Defense of Legal
Positivism: Law Without Trimmings)』(옥스퍼드 대학 출판부, 1999),
83--87쪽; 로버트 레이든슨(Robert Ladenson), 「홉스적 법개념의 옹호(In
Defense of a Hobbesian Conception of Law)」, (1980) \emph{Philosophy and
Public Affairs} 제9권 134쪽 등을 참조하라.

\emph{20--22쪽. 법의 일반성(generality).} 일반성 개념에 대한
풍부한(thicker) 해석과 얇은(thinner) 해석을 비교하고자 한다면 다음을
참조하라: 프리드리히 하이에크(Friedrich Hayek), 『법, 입법, 자유(Law,
Legislation, and Liberty)』 제1권 (시카고 대학 출판부, 1973), 제2장; 론
L. 풀러(Lon L. Fuller), 『법의 도덕성(The Morality of Law)』 개정판
(예일 대학 출판부, 1969), 46쪽 이하; 티모시 엔디콧(Timothy Endicott),
「법의 일반성(The Generality of Law)」, 루이스 두아르트 달메이다(Luís
Duarte d'Almeida), 안드레아 돌체티(Andrea Dolcetti), 제임스
에드워즈(James Edwards) 편 『『법 개념』 읽기(Reading \emph{The Concept
of Law})』(하트 출판사, 2013).

\emph{24--25쪽. 최고성과 독립성(Supremacy and Independence).} 이 주제에
대해서는 조셉 라즈(Joseph Raz), 『법체계 개념(The Concept of a Legal
System)』 제2판 (옥스퍼드 대학 출판부, 1980), 제1장도 함께 참고하라.

\end{document}
