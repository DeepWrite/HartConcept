\documentclass[12pt, oneside]{book}  % 13pt는 직접 설정함

% --- Language and Font Settings ---
\usepackage[english]{babel}
\usepackage{fontspec}
\usepackage{kotex}  % Korean typesetting

% --- Font Configuration (Main + Korean) ---
\usepackage{libertinus}
\setmainhangulfont[
  Path = ./,
  UprightFont = *Batang Light.ttf,
  BoldFont    = *Batang Medium.ttf
]{KoPubWorld}

% --- Page Geometry: A4 with wide margins for print readability ---
\usepackage[a4paper, margin=1in]{geometry}

% --- Line Spacing ---
\usepackage{setspace}
\setstretch{1.45}

% --- Section Formatting: Disable numbering ---
\usepackage{titlesec}
\setcounter{secnumdepth}{0}

\titleformat{\chapter}[display]
  {\normalfont\Huge\bfseries}
  {}{0pt}{\Huge}\titleformat{\section}
  {\normalfont\Large\bfseries}
  {}{0pt}{\Large}
\titleformat{\subsection}
  {\normalfont\large\bfseries}
  {}{0pt}{\large}

% --- Math and Tables ---
\usepackage{amsmath, amssymb, amsthm, mathtools}
\usepackage{graphicx}
\usepackage{enumitem}
\usepackage{tabularx, booktabs}
\usepackage{footmisc}

% --- Quotes, Hyperlink, Header/Footer ---
\usepackage{csquotes}
\usepackage[hidelinks]{hyperref}
\usepackage{fancyhdr}
\pagestyle{fancy}
\fancyhf{}
\fancyfoot[C]{\thepage}
\setlength{\headheight}{15pt}

% --- Bibliography (biblatex) ---
\usepackage[style=verbose-note, backend=biber, maxbibnames=99]{biblatex}
\addbibresource{references.bib}
\AtEveryBibitem{\clearfield{pages}}

% --- Title Info ---
\title{\Huge\textsc{The Concept of Law} \\[2ex] \Large Third Edition}
\author{\Large H. L. A. Hart}
\date{}

\begin{document}

% --- Title Page ---
\begin{titlepage}
  \centering
  \vspace*{3cm}
  {\Huge\textsc{The Concept of Law}}\\[1.5ex]
  {\Large Third Edition}\\[4ex]
  \textsc{H. L. A. Hart}\\[6ex]
  {\small 2025 SUMMER 강의용 한국어판\\
  강좌 외의 사용을 불허함}
  \vfill
\end{titlepage}

\section{1961년 제1판의 하트의
머리말(PREFACE)}\label{uxb144-uxc81c1uxd310uxc758-uxd558uxd2b8uxc758-uxba38uxb9acuxb9d0preface}

이 책의 목적은 법(law), 강제(coercion), 도덕(morality)을 서로 다른
동시에 관련 있는 사회적 현상으로서 이해하는 데 기여하는 것이다. 이 책은
주로 법철학(jurisprudence)을 공부하는 학생을 위한 것이지만, 법보다는
도덕철학(moral philosophy)이나 정치철학(political philosophy), 또는
사회학(sociology)에 주된 관심을 둔 이들에게도 유익하기를 바란다.
법률가들은 이 책을 분석법학(analytical jurisprudence)에 대한 하나의
에세이로 여길 것이다. 왜냐하면 이 책은 법에 대한 비판이나 법정책(legal
policy)에 대한 비판보다는, 법적 사고의 일반적 틀을 명확히 하는 데 관심을
두고 있기 때문이다. 더구나 나는 여러 지점에서 단어의 의미에 관한 질문을
제기해 왔다. 예를 들어 `\textasciitilde 할 수밖에 없는 상태(being
obliged)'와 `의무(obligation)를 지닌 상태'는 어떻게 다른가, 어떤 규칙이
유효한 법규칙(valid rule of law)이라는 진술은 공직자의 행동을 예측하는
진술과 어떻게 다른가, 한 사회집단이 규칙을 준수한다는 주장에서는 무엇이
의미되는가, 그리고 그 주장은 구성원들이 습관적으로 어떤 행위를 한다는
주장과 어떻게 유사하며 또 어떻게 다른가 등을 고찰하였다. 사실 이 책의
중심 주제 가운데 하나는, 법뿐 아니라 모든 형태의 사회구조도 `내적
진술(internal statement)'과 `외적 진술(external statement)'이라는 두
가지 서로 다른 종류의 진술 사이의 중요한 구별을 이해하지 않고는 올바로
이해될 수 없다는 것이다. 이 두 가지 진술은 사회적 규칙이 준수될 때
언제나 모두 가능하다.

이 책은 분석에 주된 관심을 두고 있음에도 동시에 기술적
사회학(descriptive sociology)에 대한 하나의 에세이로 간주될 수도 있다.
왜냐하면 단어의 의미에 대한 탐구는 단어에 대해서만 조명을 비춘다는
생각은 거짓이기 때문이다. 사회적 상황이나 관계의 유형들 사이에 존재하는,
즉각적으로는 자명하지 않은 많은 중요한 구별은, 관련 표현들의 표준적인
사용방식과 그러한 사용이 의존하는 사회적 맥락(대개 명시되지 않은 채로
남겨진다)을 검토함으로써 가장 잘 드러날 수 있다. 이러한 연구 영역에서는,
J. L. 오스틴 교수의 말처럼, ``단어에 대한 예민한 감각을 사회현상에 대한
예리한 인식의 수단으로 삼는'' 것이 특히 유효하다.

나는 다른 많은 저자들에게, 그리고 그들의 저작에 명백하고도 깊은 빚을
지고 있다. 실제로 이 책의 많은 부분은 오스틴(Austin)의 명법(imperative)
이론을 토대로 구성된 단순한 법체계 모델의 한계점을 다루고 있다. 하지만
본문에서는 다른 저자들에 대한 직접적인 인용이나 주석이 거의 없다. 그
대신 책의 끝부분에 각 장을 읽은 뒤 참고할 수 있도록 광범위한 주석들을
마련해 두었다. 여기서 본문에서 제시된 견해들은 나의 선구자 및 동시대
이론들과 비교되며, 이 책의 논의를 그들의 저작에서 어떻게 더 확장해나갈
수 있을지도 제안하고 있다. 이러한 방식을 택한 것은 부분적으로 이 책의
논의가 하나의 연속적인 흐름을 지니고 있기 때문이며, 타 이론들과의 비교는
그 흐름을 방해할 수 있기 때문이다. 그러나 나는 또한 교육적인 목적을 함께
염두에 두었다: 나는 이러한 배열이, 법이론에 관한 책이란 기본적으로 다른
책들에 무엇이 들어 있는지를 배우기 위한 책이라는 믿음을 약화시키기를
바란다. 이 믿음을 저술하는 사람이 갖고 있다면, 그 학문은 거의 진전을
이루지 못할 것이며, 독자가 그 믿음을 갖고 있다면, 그 학문의 교육적
가치는 극히 제한된 수준에 머물 것이다.

나는 너무나 오랜 시간 동안, 너무나 많은 친구들에게 빚을 졌기 때문에 이제
와서는 나의 의무(obligations)들을 모두 밝히는 것이 어렵다. 하지만 나는
A. M. Honoré 씨에게는 특별한 빚을 고백해야 한다. 그의 상세한 비판은 내
사고의 혼란과 문체의 부적절함을 드러내 주었고, 나는 그것들을 제거하기
위해 노력하였다. 그럼에도 여전히 그가 못마땅해할 부분이 남아 있을까
두렵다. 이 책의 정치철학 부분과 자연법(natural law)에 대한 재해석 가운데
가치 있는 것이 있다면, 그것은 G. A. Paul 씨와의 대화 덕분이다. 또한 그는
교정쇄도 읽어 주었다. 나는 루퍼트 크로스 박사(Dr Rupert Cross)와 P. F.
스트로슨(Mr P. F. Strawson) 씨가 본문을 읽고 보내준 유익한 조언과
비판에도 깊이 감사하고 있다.

\vspace{1em}
\begin{flushright}
\textbf{H. L. A. 하트 (H. L. A. HART)}
\end{flushright}

\newpage

\section{1994년 제2판의 Bulloch와 Raz의 편집자 주(Editors'
Note)}\label{uxb144-uxc81c2uxd310uxc758-bullochuxc640-razuxc758-uxd3b8uxc9d1uxc790-uxc8fceditors-note}

\textbf{(제2판을 위해 쓰여짐)}

\emph{The Concept of Law}는 출간된 지 불과 몇 년 만에 영미권을 넘어 전
세계적으로 법철학(jurisprudence)의 이해 방식과 연구 방식을 근본적으로
바꾸어 놓았다. 이 책은 엄청난 영향을 미쳤으며, 법이론의 맥락에서뿐
아니라 정치철학(political philosophy)과 도덕철학(moral philosophy)의
맥락에서도 이 책과 그 이론들에 대해 논의하는 수많은 저작들이 출판되었다.

하트(Hart)는 오랜 시간 동안 \emph{The Concept of Law}에 한 장(chapter)을
추가할 계획을 염두에 두고 있었다. 그는 엄청난 영향력을 지닌 본문의
내용을 손보는 것을 원치 않았으며, 그의 뜻에 따라 본문은 소소한
정정(minor corrections) 외에는 변경 없이 그대로 출판되었다. 그러나 그는
그동안 이 책에 대한 다양한 논의들에 응답하고자 하였다. 그의 견해를
오해한 이들에 대해 자신의 입장을 옹호하고, 근거 없는 비판을 반박하며, 그
못지않게 중요하다고 그가 여겼던 바와 같이 정당한 비판에 대해서는 그
타당성을 인정하고, 그 지점을 보완하기 위해 책의 이론들을 조정하는 방식을
제안하고자 했던 것이다. 새로운 장(처음에는 머리말preface으로
구상되었다가 최종적으로 후서postscript로 계획된)은 하트의 생애가 끝났을
때 아직 미완성이었는데, 이는 그의 철저한 완벽주의(perfectionism)
때문만은 아니었다. 이 작업의 적절성에 대한 지속적인 의구심과, 원래
구상했던 책의 핵심 테제들(theses)의 생명력과 통찰력에 걸맞은 작업을
자신이 과연 해낼 수 있을지에 대한 끊이지 않는 불안감 또한 그 이유였다.
그럼에도 불구하고 그는 많은 중단 속에서도 이 후서 작업을 계속하였고,
사망 당시 두 부분으로 구성될 계획 중 첫 번째 부분은 거의 완성된
상태였다.

제니퍼 하트(Jennifer Hart)가 우리에게 초고들을 검토해 볼 것을 요청하며,
그 가운데 출판할 만한 것이 있는지를 판단해 달라고 했을 때, 우리에게 가장
중요한 생각은 하트 자신이 만족하지 않을 만한 것이 출판되어서는 안 된다는
것이었다. 따라서 우리는 후서의 첫 번째 부분이 대부분 완성된 상태였음을
발견하고 매우 기뻤다. 두 번째 부분에 해당하는 것은 손으로 쓴
메모들뿐이었고, 그것은 지나치게 단편적이고 미완성된 것이어서 출판할 수
없었다. 반면 첫 번째 부분은 여러 버전이 존재했으며, 그것들은 타이핑되고,
수정되고, 다시 타이핑된 뒤 다시 수정되었다. 가장 최근의 버전조차도 하트
자신이 그것을 최종본으로 여기지는 않았던 것이 분명하다. 연필과
바이로펜(Biro)으로 이루어진 수많은 수정 흔적들이 있었기 때문이다. 더구나
하트는 이전 버전들을 폐기하지 않았고, 어떤 버전이든 손에 닿는 대로 계속
작업을 이어갔다. 이러한 점은 편집 작업을 어렵게 만들었지만, 지난 2년간
이루어진 수정들은 대부분 문체상의 뉘앙스에 대한 것이었으며, 이는 그가
본문 내용 자체에는 대체로 만족하고 있었다는 것을 보여준다.

우리의 과제는 여러 버전들을 비교하고, 한 버전에만 존재하는 텍스트의
일부가 나머지 버전에서 누락된 이유가 그가 의도적으로 삭제했기 때문인지,
아니면 모든 수정을 포함한 완전한 버전을 그가 아직 만들지 못했기
때문인지를 판별하는 것이었다. 이번에 출판된 텍스트는 하트가 폐기하지
않았으며, 이후에도 계속 수정해 나간 버전에 나타나는 모든 수정을 포함하고
있다. 때때로 텍스트 자체가 비논리적이거나 일관되지 않은 경우도 있었는데,
이는 필시 타자 작업을 맡은 사람이 그의 원고를 잘못 읽었기 때문이며,
하트가 그러한 오류를 항상 인지하지 못했기 때문일 것이다. 또 어떤
경우에는 문장을 쓰는 과정에서 자연스럽게 발생한 혼란일 수도 있으며, 이는
마지막 원고 정리 단계에서 다듬어졌을 문장이었으나, 하트가 그 단계까지
다다르지 못했기 때문이기도 하다. 이런 경우 우리는 원래의 텍스트를
복원하거나 하트의 사유를 최소한의 개입으로 되살리는 방식으로 편집하려
했다. 특히 제6절(재량discretion에 관하여)은 특별한 문제를 제기하였다.
우리는 그 첫 단락의 두 가지 버전을 발견했는데, 하나는 그 단락에서 끝나는
버전에 있었고, 다른 하나는 나머지 부분을 포함한 사본에 있었다. 잘려 있는
버전은 그의 최근 수정을 많이 포함한 사본에 있었으며, 하트가 이를 폐기한
적이 없었고, 후서 전반의 논의와도 잘 부합하였기에, 우리는 두 버전을 모두
출판하기로 결정하였고, 이어지지 않는 버전은 미주(endnote)로 실었다.

하트는 주로 참고문헌들로 구성된 주석(notes)을 타이핑하지 않았다. 그는
자필로 주석을 작성하였고, 그 주석의 단서는 가장 이른 시기의 타자본에서
가장 잘 추적할 수 있었다. 이후 그는 간혹 여백에 참고문헌을 덧붙이기도
했지만, 대체로 불완전한 형태였으며, 경우에 따라서는 단지 참고문헌을
추적할 필요가 있다는 표시만 있을 뿐이었다. 티머시 엔디컷(Timothy
Endicott)은 모든 주석을 확인하고, 누락된 부분을 추적하여 보완했으며,
하트가 드워킨(Dworkin)을 인용하거나 그를 밀접하게
패러프레이즈(paraphrase)하면서도 출처를 명시하지 않은 경우에 대해 출처를
추가하였다. 엔디컷은 또한 인용이 정확하지 않았던 본문을 수정하였으며,
이러한 작업은 폭넓은 연구와 창의성을 필요로 했다. 그는 또한 본문의 여러
오류들을 앞서 밝힌 편집 방침에 따라 수정할 것을 제안하였고, 우리는 이를
감사히 반영하였다.

우리는 하트가 더 오랜 시간이 주어졌다면 본문을 더욱 다듬고 개선하였을
것이라는 점에 의심의 여지가 없다고 생각한다. 그러나 이번에 출판된
후기(postscript)는 드워킨의 여러 논변에 대한 하트의 신중한 응답을 담고
있다고 우리는 믿는다.

\vspace{1em}

\begin{flushright}
\textbf{페넬로페 A. 불로흐 (Penelope A. Bulloch) 조셉 라즈 (Joseph Raz)}
\\ 1994
\end{flushright}

\newpage

\section{2012년 제3판의 Leslie Green의 머리말(Preface to the Third
Edition)}\label{uxb144-uxc81c3uxd310uxc758-leslie-greenuxc758-uxba38uxb9acuxb9d0preface-to-the-third-edition}

\emph{The Concept of Law}는 허버트 하트(Herbert Hart)가 옥스퍼드
대학교(University of Oxford)에서 법학도들을 대상으로 진행한
법철학(jurisprudence) 입문 강의에 기반을 두고 있다. 이 책은 1961년
초판이 출간되자마자 영어로 저술된 법철학 분야에서 가장 영향력 있는
저작으로 자리매김하였다. 법학, 철학, 정치이론 분야의 학자들은
오늘날까지도 이 책의 논증을 발전시키고, 토대를 넓히며, 비판하고 있다.
동시에 이 책은 여전히 법철학 입문서로 널리 활용되고 있으며, 원어로든
다양한 번역본으로든 세계 곳곳의 학생들이 읽고 있다.

초판 출간 50주년이 가까워졌을 무렵, 옥스퍼드 대학출판부(Oxford
University Press)는 새로운 판을 준비하는 가능성에 관해 나에게 제안을
해왔다. 1994년에 출간된 하트 사후의 제2판(posthumous second edition)은
페넬로페 불로흐(Penelope Bulloch)와 조셉 라즈(Joseph Raz)의 편집으로
이루어졌으며, 하트가 생전에 발표하지 않았던 로널드 드워킨(Ronald
Dworkin)에 대한 반론을 바탕으로 한 후기(Postscript)를 포함하고 있었다.
그 판은 하트의 이론들 및 일반적인 법철학 논의에 대한 새로운 논쟁의
물결을 촉발시켰다. 이후 수차례 재쇄가 이루어진 뒤, 본문에 남아 있는 몇
가지 오류를 바로잡고 책의 디자인을 새롭게 구성할 시점이 되었다. 이로
인해 새로운 내용을 일부 포함시킬 수 있는 가능성도 열리게 되었다.

\emph{The Concept of Law}는 변명할 필요 없는 저작이지만, 반세기가 지난
지금 더 이상 소개(introduction)가 필요하지 않다고는 말할 수 없다. 이번에
추가된 서론에서는 이 책의 주요 주제들을 개관하고, 몇 가지 비판적 논점을
간략히 스케치하며, 가장 중요한 것으로서 이 책의 기획 자체에 대한 몇몇
오해들을 미연에 방지하고자 한다. 하트는 본문 곳곳에 참고문헌, 설명, 추가
독서를 제시하는 주석(notes)을 덧붙였다. 이 주석들은 원형 그대로
유지되었다. 그러나 그 주석에서 인용된 많은 문헌들은 이미 시대에
뒤처졌고, 이후 수많은 저작들이 그의 논의를 계승하고 있다. 이에 따라 학생
독자들이 주요 논쟁 지점을 따라갈 수 있도록 새로운 주석을 추가하였다.
마지막으로, 과거의 연구들은 초판의 쪽수(pagination)에 따라 인용하고
있으나, 이제 초판 사본은 거의 유통되지 않고 있으며, 그 쪽수에 익숙한
사람들도 점차 줄어들고 있다. 따라서 나는 제2판의 쪽수를 따르기로
결정하였다.

서론(Introduction)은 내가 이전에 발표한 논문 「The Concept of Law
Revisited」 (\emph{Michigan Law Review}, 1997년, 제94권, 1687쪽)에서
발췌한 자료를 활용하고 있다. 이 프로젝트를 처음 제안하고 여러 중요한
시점마다 귀중한 조언을 아끼지 않았던 옥스퍼드 대학출판부의 알렉스
플래치(Alex Flach)에게 깊이 감사드린다. 내 동료 존 피니스(John Finnis)는
하트의 본문에 대한 오류 수정에 도움을 주었고, 톰 애덤스(Tom Adams)는
주석 작업을 위한 자료 조사에 협력해 주었다. 두 사람 모두에게 따뜻한
감사를 전한다. 그리고 서론을 읽고 논평해 준 드니즈 레옴(Denise
Réaume)에게 특히 깊이 감사드린다.

\vspace{1em}

\begin{flushright}
\textbf{레슬리 그린(Leslie Green)} 
\\ 옥스퍼드, 베일리얼 칼리지(Balliol College) 
\\ 2012년 트리니티(Trinity 2012)
\end{flushright}

\newpage

\section{2012년 제3판의 Leslie Green의
서론(INTRODUCTION)}\label{uxb144-uxc81c3uxd310uxc758-leslie-greenuxc758-uxc11cuxb860introduction}

\subsection{\texorpdfstring{\textbf{1. 하트의 메시지(HART'S
MESSAGE)}}{1. 하트의 메시지(HART'S MESSAGE)}}\label{uxd558uxd2b8uxc758-uxba54uxc2dcuxc9c0harts-message}

법(law)은 사회적 구성물(social construction)이다. 그것은 역사적으로
우연히 특정 사회들에 나타난 특징이며, 제도를 통해 운영되는 체계적인
사회적 통제(social control)의 등장을 통해 그 모습을 드러낸다. 어떤
면에서 법은 관습(custom)을 대체하지만, 또 다른 면에서는 관습에 의존한다.
법은 인간의 행위를 지시하고 평가하는 일차적 규칙(primary rules)들과,
이러한 일차적 규칙을 식별하고(identify), 시행하며(enforce),
변경(change)하는 방법에 관한 이차적 사회규칙(secondary social rules)으로
이루어진 하나의 체계다. 이러한 구조는 특정한 맥락에서 유익할 수 있지만,
항상 일정한 대가를 수반한다. 왜냐하면 그것은 부정의(injustice)의 위험과,
그 사회 구성원들을 그들의 삶을 지배하는 가장 중요한 규범들로부터
소외시키는 위험을 동반하기 때문이다. 따라서 법에 대해 우리가 취해야 할
적절한 태도는 경계(caution)이지, 찬양(celebration)이 아니다. 더욱이 법은
때로 자신에게 없는 객관성(objectivity)을 가장하기도 한다. 왜냐하면
판사들이 무엇이라고 말하든 실제로는 그들이 법을 창출하는 강력한 권한을
행사하고 있기 때문이다. 따라서 법과 재판(adjudication)은
정치적이다(political). 그리고 다른 방식으로, 법이론(legal theory) 역시
정치적이다. `순수한 법이론(pure theory of law)'이란 존재할 수 없다. 법
자체의 개념들만으로 구성된 법철학(jurisprudence)은 법의 본성을
이해하기에 충분하지 않다. 법철학은 사회이론(social theory)과 철학적
탐구(philosophic inquiry)에서 도움을 받아야 한다. 이런 점에서 법철학은
법률가나 법학 교수들의 고유 영역도 아니며, 심지어 그들에게 자연스러운
활동이라 보기도 어렵다. 법철학은 보다 일반적인 정치이론(political
theory)의 일부일 뿐이다. 그 가치는 고객에게 법률 자문을 제공하거나
사안들의 결정에 있는 것이 아니라, 우리 문화와 제도들을 이해하고,
그것들에 대한 도덕적 평가(moral assessment)를 정초하는 데 있다. 이러한
평가는 법의 본성뿐 아니라 도덕성(morality)의 본성에도 민감해야 한다.
도덕성은 복수적이고 상충하는 가치들로 구성되어 있기 때문이다.

이상은 현대 법철학에서 가장 영향력 있는 저작 중 하나인 H. L. A. 하트의
\emph{법의 개념(The Concept of Law)}의 핵심 사상들이다. 그러나 다른
중요한 책들과 마찬가지로, 하트의 책도 실제로 읽은 것보다는
풍문(rumour)을 통해 알려진 경우가 많다. 그 책의 존재는 알고 있지만 그
내용을 진지하게 읽어보지 않은 이들에게는, 방금 소개한 요약이 낯설게
들릴지도 모른다. 그들이 들은 바에 따르면, 하트는 다음과 같은 생각을 가진
사람처럼 보인다: 법은 폐쇄적인 논리 체계인가? 법은 다른 사회질서가 가진
결함을 치유하는 하나의 사회적 성취인가? 법은 대체로 명확하며, 법원은
도덕적 가치와 무관하게 이를 적용해야 하는가? 법과 도덕은 개념적으로
구분되며, 반드시 분리되어야 하는가? 법철학은 가치로부터 자유롭고,
`법(law)'과 같은 단어들의 진정한 의미를 탐구함으로써 그 진리를 밝힐 수
있는 학문인가?

이 모든 질문에 대한 간단한 대답은 ``아니다.'' 하트는 그런 생각을 갖고
있지 않다. 이러한 왜곡된 해석들은 크게 세 가지 원인에서 비롯된다. 첫째는
철학 전반에 걸쳐 자주 등장하는 난점이다. 하트가 다루는 문제들은
복잡하며, 참과 거짓 사이의 간극은 종종 미묘하거나 쉽게 간과될 수 있는
구분에 달려 있다. (예: 법과 도덕이 분리될 수 있다고 주장하는 것과, 법과
도덕이 실제로 분리되어 있다고 주장하는 것은 전혀 다른 이야기다.) 둘째는
역사적 요인이다. 반세기가 지난 지금, 책의 언어와 사례들은 사회적으로도,
철학적으로도 낯설게 느껴질 수 있다. 오늘날 `원시적(primitive)'
사회질서라는 표현을 쓰는 이는 드물고, 어떤 것의 본질에 대한 설명을
그것의 개념에 대한 `해명(elucidation)'이라 부르지도 않는다. 셋째는
독자의 기대와 관련된다. 모든 책은 이른바 `암묵적 독자(implied reader)'를
전제로 한다. 하트의 책이 전제하는 독자는 우리가 가진 가장 중요한 정치
제도 중 하나인 `법'의 본성과, 그것이 도덕 및 강제력(coercive force)과
맺는 관계에 철학적으로 호기심을 가진 사람이다. 그러나 실제 독자는 반드시
그렇지는 않다. 일부 독자들은 법철학에서 실천적 조언을 기대한다. 예컨대
그들은 헌법을 어떻게 해석해야 하는지, 또는 어떤 사람이 판사가 되어야
하는지를 알고 싶어 한다. 그들은 법 이론에 관한 책이 요리 이론에 관한
책이 요리에 대해 해줄 수 있는 역할---즉 다양한 상황에 적용 가능한
일반적인 `방법론(how-to)'---을 수행할 것이라 기대하는 것이다.

하트의 책은 충분히 명료하므로 요약이 필요하지는 않지만, 그 주제들을
간략히 탐색해보는 일은 위와 같은 오해들을 방지하는 데 도움이 될 수 있다.
나는 그의 법과 사회규칙(social rules), 강제력(coercion),
도덕성(morality)에 관한 견해를 살펴본 뒤, 몇 가지 방법론적 논점을 간단히
언급할 것이다. 나는 중립적 태도를 취할 생각은 없다. 하트의 법이론은
일부는 옳고, 일부는 잘못되었으며, 여기저기서 약간은 난해하기도 하다.
그러나 다음에 이어지는 내용은 평가(appraisal)가 아니다. 나는 사람들이
흔히 빠지거나 오도되는 지점들을 강조하고, 몇몇 쟁점에 대해 비판적 논평을
덧붙이겠지만, 전체적인 평가는 독자의 몫으로 남겨두겠다.

\subsection{\texorpdfstring{\textbf{2. 사회적 구성물로서의 법(LAW AS A
SOCIAL
CONSTRUCTION)}}{2. 사회적 구성물로서의 법(LAW AS A SOCIAL CONSTRUCTION)}}\label{uxc0acuxd68cuxc801-uxad6cuxc131uxbb3cuxb85cuxc11cuxc758-uxbc95law-as-a-social-construction}

법(law)과 법체계(legal systems)는 자연(nature)의 산물이 아니라 인위적인
것(artifice)이다. 다시 말해, 그것들은 사회적 구성물(social
constructions)이라고 할 수 있다. 그렇다고 해서 이 말이 특별히 중요하거나
주목할 만한 대비를 이룬다고 할 수 있을까? 어떤 이들은 모든 것이
사회적으로 구성된 것이기 때문에, 법도 사회적 구성물이라고 생각한다.
데리다(Derrida)는 ``\emph{il n'y a pas de hors-texte}''(텍스트 바깥은
없다)라는 표현으로 이를 비꼬았다. 만약 이 말이 이해 가능하다 하더라도,
법에 관한 논의에서는 무관한 말이다. 예를 들어, 누군가가 ``인종(race)은
사회적 구성물이다''라고 말한 뒤, 그것을 ``경찰봉(truncheons)이나
감옥(prisons)처럼 말이죠''라고 설명한다면 어떻겠는가? 이는 마치 누군가가
신(God)은 존재하지 않는다고 말했는데, 알고 보니 그 사람은 개(dogs)조차
존재하지 않는다고 믿는다는 사실을 알게 되는 것과 같다. 내가 법이 사회적
구성물이라고 말할 때, 그것은 어떤 것들이 그렇지 \emph{않은} 방식에서
그렇게 구성된 것이라는 뜻이다. 법은 사령(command)이나 규칙(rule)과 같은
제도적 사실(institutional facts)로 구성되어 있으며, 그것들은 인간들이
사고(thinking)하고 행위(acting)함으로써 만들어낸 것이다.\footnote{See
  eg. John Searle, \emph{The Construction of Social Reality} (Allen
  Lane, 1995); and Neil MacCormick, \emph{Institutions of Law: An Essay
  in Legal Theory} (Oxford University Press, 2007).} 그러나 법은
사회적으로 구성되지 않은 물리적 우주 속에 존재하며, 또한 법은 사회적으로
구성되지 않은 인간들에 의해, 그리고 인간들을 위해 만들어진 것이다.
어쩌면 이 점은 너무 뻔한 말처럼 들릴 수 있다. 유행에 발맞추려는 듯이
`예절의 사회적 구성(social construction of etiquette)'에 대해 말할 수도
있겠지만, 예절(manners)이 관습(conventional)이라는 것은 누구나 알고 있는
일이기에 굳이 강조할 필요가 없다. \footnote{Cf. Ian Hacking, \emph{The
  Social Construction of What?} (Harvard University Press, 1999).}
예절은 공동의 관행(common practice)에 의존하고, 나름의 역사(history)를
가지며, 지역마다 달라진다. 그렇다면 법도 이와 같다는 것은 자명한 진실
아닌가? 하지만 이 점을 다시 생각하게 만드는 유명한 스토아 학파(Stoic)의
자연법(natural law) 요약문을 보자:

\begin{quote}
진정한 법은 자연(Nature)과 일치하는 올바른 이성(right reason)이다.
그것은 보편적으로 적용되며, 변하지 않고 영원하다\ldots{} 로마와
아테네에서 다른 법이 있을 수 없고, 현재와 미래에 서로 다른 법이 존재할
수 없으며, 오직 하나의 영원하고 불변하는 법만이 모든 민족과 모든 시대에
타당할 것이다\ldots{}\footnote{Cicero, \emph{De Re Republica} III. xii.
  33, tr. C. W. Keyes (Harvard University Press, Loeb Classical Library,
  1943) 211.}
\end{quote}

이 영원하고 보편적인 법은 누군가가 창조한 것도 아니고, 누군가가 바꿀 수
있는 것도 아니라고 말한다. 자연법(natural law)은 의지(will)의 문제가
아니라 이성(reason)의 문제이다. 이러한 전체 주장에 동의하는 법이론가는
드물지만, 그 중 일부에 동의하는 이들은 여전히 많다.\footnote{Perhaps
  John Finnis comes closest, in his \emph{Natural Law and Natural
  Rights} (2nd edn., Oxford University Press, 2011).} 예컨대 로널드
드워킨(Ronald Dworkin)은 다음과 같이 주장한다. 우리 법은 조약(treaties),
관습(customs), 헌법(constitutions), 성문법(statutes), 사례(cases)에서
발견되는 규범들뿐 아니라, 그 규범들에 대한 최선의 정당화(best
justification)를 제공하는 도덕적 원칙들(moral principles)도 포함한다는
것이다.\footnote{Ronald Dworkin, \emph{Taking Rights Seriously} (Harvard
  University Press, 1978), chap.~4; Ronald Dworkin, \emph{Law's Empire}
  (Harvard University Press, 1986), chaps. 2--3.} 드워킨의 설명에
따르면, 도덕적 원칙들에 의해 \emph{정당화되는 대상}은 사회적으로 구성된
것이지만, 그 정당화들 자체는 사회적으로 구성된 것이 아니다. 여기서
중요한 점은, 정당화(justification)는 사건(event)이 아니라는 것이다.
정당화를 믿거나(believing), 수용하거나(accepting), 주장(asserting)하는
행위는 사건이다. 하지만 드워킨은 법이 구성되는 것은, 단지 구성된
내용들과 사람들이 그것을 정당화라고 \emph{믿었거나}, \emph{수용했거나},
\emph{주장했기 때문}이라고 말하지 않는다. 그는 그 내용들에 \emph{실제로
정당화가 되는} 도덕적 원칙들이 더해져야 법이 구성된다고 주장한다. 만약
어떤 것이 법이 되기 위한 조건이 그것이 다른 어떤 법적 대상에 대한 최선의
도덕적 정당화(best moral justification)라는 사실이라고 믿는다면, 당신은
키케로(Cicero)와 마찬가지로, `올바른 이성(right reason)'의 요청이기
때문에 법의 지위를 갖는 법이 있다고 믿는 셈이다. 우리가 하는 어떤 일도
타당한 정당화(sound justification)를 부당한 것으로 만들 수 없다면,
우리는 바꿀 수 없는 법의 존재를 인정하게 된다. 또한 어떤 도덕 원칙이
어떤 제도를 정당화하는지는 누가 그것을 알고 있거나 믿는지 여부와
무관하다면, 아무도 들어본 적 없는 법---아주 많은 법---이 존재할 수 있게
된다. 도덕적 인식(moral knowledge)의 가능성에 따라, 아예 알 수조차 없는
법이 존재할 수도 있다.

하트(Hart)의 접근은 이러한 입장을 전면적으로 거부한다. 법 안에 존재하는
모든 것은 누군가, 혹은 어떤 집단이 그것을 고의적으로든 우연히든 그곳에
두었기 때문에 존재하는 것이다. 법은 모두 역사를 가지며, 모두 변화
가능하며, 모두 알려져 있거나 알려질 수 있는 것이다. 우리의 법 중 일부는
좋은 정당화(good justification)를 갖고, 일부는 그렇지 않으며,
정당화만으로는 결코 법을 성립시키기에 충분하지 않다. 법을 성립시키기
위해서는 실제 인간의 개입(actual human intervention)이 필요하다. 즉,
사령(command)이 내려지고, 규칙(rule)이 적용되고, 결정(decision)이
내려지고, 관습(custom)이 형성되고, 혹은 정당화(justification)가
승인되거나 주장되어야 한다.

법철학자들은 인간의 개입(human interventions)으로 설정된 것들을 지칭하기
위해 오래된 용어를 사용하는 경우가 있다. 그들은 그것들을
`정립되었다(posited)'고 말한다. 모든 법이 정립된 것이라고 여기는 사람을
\emph{법실증주의자}(legal positivist)라고 부르며, 사회구성주의자(social
constructivist)는 그러한 법실증주의자의 일종이다. 그러나 모든
법실증주의자가 사회구성주의자인 것은 아니다. 한스 켈젠(Hans Kelsen)은
사회구성주의자가 아니었다. 그는 모든 법이 정립된 것이라고 생각했지만,
동시에 모든 법체계에는 정립된 것이 아니라 단지 `전제(presupposed)'된
규범(norm)이 적어도 하나는 포함된다고 보았다.\footnote{Hans Kelsen,
  \emph{Pure Theory of Law} (Max Knight tr., 2nd edn., University of
  California Press, 1967) 193--205.} 켈젠에 따르면, 법규범(legal norm)은
그것이 유효(valid)할 때에만 존재한다. 여기서 `유효하다(valid)'는 것은 그
규범의 대상자(subjects)가 그것에 따라야 한다(ought to conform)는 것을
의미한다. 그는 흄(Hume)과 칸트(Kant)의 뒤를 이어, `존재한다(is)'는
사실만으로는 `마땅하다(ought)'는 결론을 도출할 수 없다고 주장했다.
따라서 어떤 사회적 구성물(social construction)도, 혹은 그것들의 총합이라
할지라도, 규범(norm)이 될 수 없다. 규범을 만들어내기 위해서는, 사회
내에서의 근본적인 입법 과정(fundamental law-making processes)이
유효하다는 전제가 필요하다. 원초적 헌법(original constitution)이 진정한
권위를 가져야만 그 아래의 어떤 것도 권위를 가질 수 있다. 따라서 우리는
원초적 헌법 아래에서 만들어진 자료들을 법으로 간주하려면, 그 헌법이
구속력을 갖는다는 것을 전제(presuppose)해야 한다. 이제,
전제(presupposition)라는 것도 정당화(justification)와 마찬가지로 어떤
사건(event)이 아니다. 켈젠은, 법이 무엇을 요구하는지를 알기 위해서는
사람들이 실제로 무엇을 정립(posited)했는지를 알아야 한다는 점은 부정하지
않았다. 하지만 그는, 그들이 만들어낸 산물을 \emph{법으로서(as law)}
이해하기 위해서는, 사회적이거나 역사적인 것이 아닌 어떤 요소가 더해져야
한다고 주장했다. 이 때문에 켈젠은 법실증주의자임에도 사회구성주의자는
아니다. 그는 우리가 사회적으로 구성된 규범들을 연구하는
방식---사회학적(sociological), 심리학적(psychological),
역사적(historical) 탐구---을 법철학(jurisprudence)에서는 `외부
요소(alien elements)'로 간주해야 한다고 보았다.\footnote{Ibid. 1.}

하트(Hart)는 켈젠의 이러한 견해 역시 거부한다.\footnote{See below,
  292--3, and Hart's essays `Kelsen Visited' and `Kelsen's Doctrine of
  the Unity of Law' in H. L. A. Hart, \emph{Essays in Jurisprudence and
  Philosophy} (Oxford University Press, 1983).} 법의 궁극적 기초는
정당화(justification)도 아니고 전제(presupposition)도 아니다. 그것은
사람들이 특정한 사고(thinking)와 행위(doing)를 함으로써 생겨나는 사회적
구성물(social construction)이다. 법철학(jurisprudence)의 역할은 이
구성물이 무엇인지, 그리고 그것이 일상의 사회적 사실들로부터 어떻게
구축(built up)되는지를 설명하는 것이다. 하트는 자신의 이 설명을 `기술적
사회학에 관한 하나의 에세이(an essay in descriptive sociology)'라고
부르기까지 한다(vi).\footnote{Parenthetical page references are all to
  this volume.} 이것은 다소 과장된 표현일 수 있다. 이는 분석적
법철학(analytic legal philosophy)의 하나의 시도이지만, 이 시도는
이론적으로 예리한 법사회학(theoretically astute sociology of law)이
유용하게 사용할 수 있는 개념들을 활용한다. 그중 가장 중요한 것이 바로
사회적 규칙(social rule)이라는 개념이다.

\subsubsection{\texorpdfstring{(i) \emph{법, 규칙, 관행} (Law, Rules,
and
Conventions)}{(i) 법, 규칙, 관행 (Law, Rules, and Conventions)}}\label{i-uxbc95-uxaddcuxce59-uxad00uxd589-law-rules-and-conventions}

하트는 원래 홉스(Hobbes), 벤담(Bentham), 오스틴(Austin)에게서 발견되는
초기의 법실증주의 설명을 거부한 뒤, 규칙(rules)이 법의 가장 중요한 구성
요소라고 생각하게 되었다. 이들은 법을 명령(command), 위협(threat),
복종(obedience)으로 구성된 것이라고 보았다. 주권자(sovereign)란 대부분의
사람들의 습관적인 복종(habitual obedience)을 받지만, 자신은 아무에게도
습관적으로 복종하지 않는 개인이나 집단을 말한다. 법이란 주권자의 일반적
명령(general command)이며, 힘(force)의 위협에 의해 뒷받침된다.

1977년, 미셸 푸코(Michel Foucault)는 다음과 같이 말했다. ``우리가 필요로
하는 것은 주권의 문제를 중심으로 세워진 정치철학이 아니다\ldots{} 우리는
왕의 목을 쳐야 한다. 정치이론에서는 그것이 아직 끝나지
않았다.''\footnote{Michel Foucault, `Truth and Power' in his
  \emph{Power/Knowledge: Selected Interviews and Other Writings,
  1972--1977} (Colin Gordon ed., Vintage, 1980) 121.} 그러나 이 `왕
살해'(regicide)의 소식은 이미 영국 해협을 건넜는지도 모른다. 하트는 이미
오래 전에 그 작업을 끝마쳤기 때문이다. 하트는 『법의 개념』 제3장과
제4장에서 다음을 보여준다. 법은 모두 명령으로 구성되어 있지 않으며,
법체계는 반드시 어떤 개인이나 집단이 주권자의 속성을 가져야만 성립되는
것이 아니다. 법은 그것을 창조한 자들이 사라진 이후에도 계속 존재하며,
위협(threats)은 사람들에게 어떤 행위를 하도록 \emph{강제(oblige)}할 수는
있지만, \emph{의무(obligation)}를 부여할 수는 없다. 결국 주권 이론에서
누락된 핵심은 사회적 규칙(social rule)의 개념이다. 우리가 규칙을
이해하게 되면, 우리는 그것이 주권, 권한(powers), 관할(jurisdiction),
유효성(validity), 권위(authority), 법원(courts), 법률(laws),
법체계(legal systems)---심지어 하트에 따르면 정의(justice)의 한
유형---을 설명하는 열쇠라는 것을 알게 될 것이다. 법(law) 자체는 사회적
규칙들의 \emph{결합(union)}이다. 즉, 인간의 행위를 지도하기 위해
의무(duty)를 부과하거나 권한(power)을 부여하는 \emph{1차 규칙(primary
rules)}과, 1차 규칙을 식별하고(identification), 변경(alteration),
집행(enforcement)하기 위한 \emph{2차 규칙(secondary rules)}의 결합이다.
이 2차 규칙들 중에서도, 궁극적인 \emph{승인 규칙(rule of recognition)}은
특별한 중요성을 갖는다. 승인 규칙은 1차 규칙을 적용하는 역할을 맡은
사람들의 관습적 관행(customary practice)으로, 어떤 행위가 법을
형성하는지를 결정함으로써 법적 유효성(legal validity)의 기준(criteria)을
제공한다. 따라서 법체계의 근본 헌법(fundamental constitution)은 도덕적
정당화(moral justifications)나 논리적 전제(logical presuppositions)에
근거하는 것이 아니라, `법원(courts), 공무원(officials), 일반인(private
persons)의 복합적 관행(complex practice)'에 의해 생성된 이 관습적
사회규칙(customary social rule)에 기반한다(107쪽). 하트는 영국의 인식
규칙(rule of recognition)이 다음과 같은 것이라고 제안한다. ``여왕과
의회가 제정한 것은 곧 법이다(Whatever the Queen in Parliament enacts is
law).'' 의회의 제정물은, 그것이 도덕적으로 정당하다거나, 어떤 논리적
전제에 의해 정당화되었기 때문에 법인 것이 아니라, 실제로 시행되고 있는
관습적 규칙(customary rule)이 그것들을 법으로 인정하고 있기 때문에 법인
것이다.

그러므로 법은 사회적 규칙(social rules)으로 구성되어 있다. 그렇다면
규칙(rules) 자체는 어떤가? 그것들 역시 사회적 구성물(social
constructions)이며, 하트(Hart)는 그것들이 실천(practice)으로 이루어져
있다고 말한다. (이러한 입장을 흔히 `규칙에 대한 실천 이론(practice
theory of rules)'이라 부른다.) 관습적 규칙(customary rules)은 `외적
측면(external aspect)'을 가지며, 이는 행동의 규칙성(behavioural
regularity)으로 나타난다. 즉, 사람들은 공통된 방식으로 행동한다. (규칙의
성격에 따라, 이는 그 규칙의 요구에 부응하거나 또는 타인에게 그것을
적용하는 행위를 포함할 수 있다.) 규칙은 또한 `내적 측면(internal
aspect)'을 갖는데, 이는 하트가 `수용(acceptance)'이라 부른 복합적인
태도를 포함한다. 수용이란, 그 규칙적 행태를 행동을 인도하고 평가하는
기준(standard)으로 삼으려는 의지로서, 특히 일치(conformity)를 칭찬하고
위반(breach)을 비판하며, 그러한 칭찬과 비판이 적절하다고 여기는
태도이다. 수용은 찬성(approval)을 요구하지 않는다. 그것은 사람들이 그
규칙에 대해 어떻게 느끼는가의 문제가 아니라, 그 규칙을 사용할 의사가
있는가의 문제이다. 사람들은 하트가 말하는 의미에서 규칙을 수용할 수
있다. 그 규칙이 좋다고 생각해서 수용할 수도 있고, 다른 사람을 기쁘게
하려고, 혹은 두려움이나 동조심(conformism) 때문에 수용할 수도 있다
(56--7, 115, 257쪽). 밀턴(Milton)의 말을 믿는다면, 사탄(Satan)조차도 그
규칙이 나쁘다고 생각하는 이유로 규칙을 수용할 수 있다. ``악이여, 나의
선이 되어라(Evil be thou my good).'' 중요한 것은 사람들이 특정
기준(standard)에 수렴(converge)하고, 그것을 행위의 지침으로 삼으며,
그렇게 함으로써 그것을 규범적인 것(normative)으로 대우한다는 점이다.

규칙에 대한 실천 이론(practice theory of rules)은 논쟁적이다. 몇 가지
난점을 살펴보고, 그런 뒤에 하트가 제기된 반론을 어떻게 회피하려 하는지를
살펴보자. 하트는 다음 두 가지를 만족시키는 규칙의 존재 테스트를 원했다.
첫째, 단순한 우연적 또는 습관적 행동 패턴과 진정한 규칙 준수를 구별할 수
있을 것. 둘째, 관습 규칙이 의무(obligatory) 또는 구속력을 가진(binding)
것으로 되는 것이 무엇을 의미하는지 설명할 수 있을 것. 그러나 실천 이론은
이 요구를 충족시키지 못한다.\footnote{See Ronald Dworkin, \emph{Taking
  Rights Seriously} (rev. edn., Harvard University Press, 1978) 48--58;
  Joseph Raz, \emph{Practical Reason and Norms} (2nd edn., Oxford
  University Press, 1999) 49--58.} 예컨대 규칙이지만 사회적 실천이 아닌
것들이 있다 (예: 개인의 규칙). 또한 사회적으로 받아들여지고 수용된
실천이지만 규칙은 아닌 것도 있다 (예: 권총강도에게 지갑을 넘겨주는 것이
일반적이고 수용된 관행일 수 있지만, 그것이 규칙인 것은 아니다). 그리고
어떤 규칙을 인용하는 것이, 단지 어떤 정당화가 존재한다고 여긴다는 표시가
아니라, 자기 행위의 \emph{정당화로서(as a justification)} 제시될 수도
있다. 이는 실천 이론과 맞지 않는다. 더 나아가, 우리가
의무(obligation)라는 개념을 이해하는 데 사회적 규칙(social rule)이라는
개념이 반드시 필요한지도 명확하지 않다. 어떤 사람은 항공 여행에 대해
탄소 상쇄(carbon offsets)를 구매해야 할 의무가 있다고 믿을 수 있지만,
그런 실천이 보편적으로 존재한다거나 관행적으로 수용된다고 생각하지는
않을 수 있다.

하트는 이 책의 \emph{후기(Postscript)}에서 이러한 비판들에 대응하기 위해
자신의 설명을 한정한다. 그는 이제 모든 규칙이 실천 규칙(practice
rules)은 아니라고 인정하지만, \emph{관습적 규칙(conventional rules)}은
그렇고, 법은 그러한 규칙들에 기반한다고 말한다. 하트는 어떤 규칙이
관습적(conventional)이라고 하려면 ``그 집단 구성원 각각이 그 규칙을
수용하는 이유 중에, 집단 전체의 일반적 일치(conformity)가 포함되어
있어야 한다''고 말한다 (255쪽). 예컨대, ``우측 통행''의 규칙은 사람들이
대다수가 그렇게 하지 않는다면 따르지 않을 것이므로 관습이다. 반면,
``졸릴 때 운전하지 말라''는 규칙은 해당되지 않는다. 왜냐하면 도로에 졸린
운전자가 많을수록 오히려 더욱 깨어 있어야 할 이유가 생기기 때문이다.
하트는 궁극적 승인 규칙(rule of recognition)이 하나의 관습이라고 본다.
``영국 판사가 의회의 입법을 법의 원천으로 삼아 다른 원천들보다 우위에
둔다면, (또는 미국 판사가 헌법을 그런 원천으로 삼는다면) 그 이유는
그들의 사법적 동료들 역시 그렇게 해왔으며, 선배들도 그렇게 해왔기
때문이다''(267쪽). 법은 ``판사들과 법률가들이 수용한 단지 하나의 관습적
승인 규칙(a mere conventional rule of recognition)''에 기반한다 (267쪽).
그러나 여기서 \emph{mere}(단지)라는 표현은 삭제해야 할 것이다.
공직자들이 승인 규칙(rule of recognition)이나 다른 근본 규칙들을 따르는
유일한 이유가 다른 사람들도 그렇게 하기 때문이라고 보기는 어렵다. 예컨대
영국에서 의회 제정 법률(parliamentary statutes)이 최고 법원(supreme
source of law)이라는 지위는, 그것이 최고로 간주되는 관행뿐 아니라, 이
관행이 민주적이라거나 우리 문화의 핵심이라는 신념에 의존하고 있을 수
있다. 미국의 경우, 헌법의 최고성은 단지 공동의 관행뿐 아니라, 그것이
정당한 정부형태를 구성한다는 믿음이나, 지혜로운 사람들이 제정한 것이므로
그것에 충실해야 한다는 믿음에 기초할 수 있다. 이러한 신념들이 반드시
올바를 필요는 없으며, 반드시 모든 사람에게 공유될 필요도 없다. 그러나
이러한 신념들은 일반적으로 공동 관행 기반의 이유들과 함께 존재한다.
하트가 말하는 의미에서 승인 규칙이 관습적이라고 하려면, 공직자들이
그것을 적용하는 다른 이유들이 무엇이든 간에, 그러한 공동 관행이 없었다면
그것을 적용하지 않았을 것이라는 조건이 충족되어야 한다.

이러한 수정(관습 이론의 도입)을 감안하더라도, 또 다른 문제가
남는다.\footnote{For other doubts about Hart's argument on this point
  see Leslie Green, `Positivism and Conventionalism' (1999) 12
  \emph{Canadian Journal of Law and Jurisprudence} 35; and Julie
  Dickson, `Is the Rule of Recognition Really a Conventional Rule?'
  (2007) 27 \emph{Oxford Journal of Legal Studies} 373.} 승인 규칙(rule
of recognition)은 의무나 책무를 부과하는 규칙이다: 그것은 법의 원천을
식별할 뿐 아니라, 식별된 법을 판사들과 다른 이들이 적용하도록 지시한다.
실천 이론에 따르면, 사회적 규칙이 의무를 부과하려면 다음 세 가지 조건이
모두 충족되어야 한다(if and only if): (a) 그 규칙이 사회적으로 필요하다고 믿어져야 하며,
(b) 강한 사회적 압력(social pressure)에 의해 강화되어야 하고, (c) 그
규칙이 규범 대상(norm-subject)의 당면한 자기이익(self-interest)과 충돌할
수 있어야 한다 (86--8쪽). 이러한 조건들이 이 경우 충족되는가? 이
조건들은 사실적(factual)이므로, 구체적인 사례마다 실제 조사가 필요하다.
일반적으로 보자면, 법원의 입장에서는 법의 기준(test)을 고정시키는 것이
필요하다고 여겨질 가능성이 높고, 이를 벗어난 행동은 강한 압력에 의해
교정될 것이다. (예를 들어, 미국 지방법원이 대법원의 모든 판결을 무시하고
샤리아법(Sharia)을 구속력 있는 법의 원천으로 적용한다면, 어떤 반응이
나타날지 상상해 보라.) 그러나 관습 규칙의 경우, 조건 (c)가 왜 충족되는지
확인하는 것은 더 어렵다. 특정한 관습 기준이 중요하다고 믿어질수록,
그것을 어길 유인은 오히려 줄어든다. 우리는 모두 좌측 통행이든 우측
통행이든 선택할 수 있지만, 공통된 관행이 존재하는 상황에서는 그
반대편으로 운전할 유혹은 별로 없다. 일탈의 유혹은 일반적으로
\emph{공공재(public goods)}에 관한 규칙, 즉 무임승차(free-riding)가
가능한 규칙에서 더 자주 발생한다. 하지만 규칙이 관습적인 경우, 의무와
욕망이 같은 방향으로 작용한다. 이때는 `의무 또는 책무와 이익 사이에 항상
존재하는 충돌 가능성(the standing possibility of conflict between
obligation or duty and interest)'이 존재하지 않는다 (87쪽). 위에서 말한
것처럼, 판사들은 서로 다른 승인 규칙들 사이에 선호(preferences)를 가질
수 있으며, 그 선호는 정당성(legitimacy) 등에 대한 견해를 반영할 수 있다.
그러나 모든 판사들의 기본적인 태도가 `군중과 함께 가고자 하는
욕망'이라면, 우리가 일반적으로 `의무'라고 부르는 규범적 긴장(normative
push and pull)은 드물어질 것이다. 이 점에서 하트는 처음의 설명이 더
진실에 가까웠다: 판사들조차 일탈의 유혹을 받을 수 있지만, 그것은 또한
비판과 강한 복종 압력을 불러일으키는 계기가 된다. 안타깝게도, 이와 같은
비판과 압력은 규칙이 전혀 없는 경우, 단지 보편 적용 가능한 이유(reason
of general application)가 존재하는 경우에도 마찬가지로 발생한다. 사회적
규칙(social rule)의 정확한 성격에 대한 논쟁은 여전히 계속되고 있으며,
하트의 일반적인 설명 틀 안에서 수용 가능한 다른 설명들도
있다.\footnote{For example: Joseph Raz, \emph{Practical Reason and
  Norms}; Frederick F. Schauer, \emph{Playing by the Rules: A
  Philosophical Examination of Rule-based Decision-making in Law and in
  Life} (Oxford University Press, 1993); Andrei Marmor, \emph{Social
  Conventions}: \emph{From Language to Law} (Princeton University Press,
  2009); Scott J. Shapiro, \emph{Legality} (Harvard University Press,
  2011). A distinction between rules and standards is drawn in Henry M.
  Hart and Albert Sacks, \emph{The Legal Process: Basic Problems in the
  Making and Application of Law} (W. N. Eskridge, Jr.~and P. P. Frickey
  eds., Foundation Press, 1994) 139--41.} 그러나 단순한 실천 이론도,
하트의 관습주의적 수정안도, 그것들만으로는 충분하지 않다.

\subsubsection{\texorpdfstring{(ii) \emph{규칙의 적용 범위(The Reach of
Rules)}}{(ii) 규칙의 적용 범위(The Reach of Rules)}}\label{ii-uxaddcuxce59uxc758-uxc801uxc6a9-uxbc94uxc704the-reach-of-rules}

법 이론에서 규칙 기반 이론(rule-based theory)에 대한 독립적인 회의
근거는 그 범위(scope)에 관련된다. 사회적 규칙(social rules)이 법적
현상(legal phenomena)을 이해하는 데 필수적이라는 점이 입증된다고
하더라도, 그것만으로 충분한가? 그렇지 않다. 그 이유는 여러 가지다.

첫 번째 이유는 하트(Hart) 자신이 강조한 것이다. 모든 1차 및 2차
규칙(primary and secondary rules)의 체계가 법체계(legal system)인 것은
아니다. 예컨대, 북미 하키 리그(National Hockey League)에는 규칙 체계가
존재한다. 선수, 심판, 커미셔너의 행동을 지시하는 1차 규칙과, 공식 규칙에
대해 인정(recognition), 변경(change), 판정(adjudication)의 기능을
수행하는 2차 규칙들이 있다. 하지만 하키 규칙들은 법체계는 아니다. (물론
그것들이 법체계와 \emph{유사하다}는 점은 누구도 부정하지 않는다.) 그러면
무엇이 빠져 있는가? 하키 규칙은 특수 목적(special-purpose)의 규칙이다.
하나의 게임만을 규율하는 반면, 법은 삶의 많은 부분을 규율할 수 있다.
그리고 법체계는 하키 규칙을 포함하여 하키 전반을 규율하지만, 하키 규칙은
법을 규율하지 않는다. 하트는 나아가, 법은 단지 포괄적으로 규율할
\emph{수 있는} 것뿐 아니라, 실제로 재산(property), 계약(agreements),
폭력의 사용(use of force) 등을 포함하여 광범위한 사안들을 규율해야만
비로소 법체계라고 할 수 있다고 주장한다 (193--200쪽). 하트는 이를
법체계의 `최소한의 내용(minimum content)'이라 부르며, 이러한 내용을
규율하는 것이 인간 생존(human survival)을 증진시키고, 인간 생존이 (그는
당연하다는 듯) 도덕적으로 선(good)하다고 가정하기 때문에, 모든 법체계는
어떤 형태로든 선(good)을 지향한다고 본다. 따라서 법체계에 대해
`형식적(formal)' 기준을 적용할 수 있는가의 문제는 무의미하다. 이것이
바로 서두에서 언급한 것처럼, 하트의 이론이 법체계를 논리나 수학의 형식
체계(formal system)처럼 간주한다고 생각하는 것이 잘못된 이유 중 하나다.
그리고 이 논점은 더 큰 함의를 갖는다. 어떤 것도 법이 되려면 어떤
법체계에 속해야 하기 때문에, 법 자체에 대해서도 순수하게 형식적인
기준이라는 것은 있을 수 없다. 법은 특정한 \emph{유형(kind)}의 규범
체계(normative system) 속에서 일정한 역할을 수행하는 규칙(rules)인
것이다. 그리고 이 규범 체계는 부분적으로 그 \emph{내용(content)}에 의해
구별된다.

두 번째 논점은 하나를 명확하게 해 준다. 어떤 학자들은 하트의 설명이
부정확하거나 불완전하다고 생각한다. 그 이유는 법체계에는 규칙 외의 다른
규범(norms)들이 존재한다는 것이다. 예컨대 `기준(standards)'이나
`원칙(principles)'이 그러한 것이라 한다.\footnote{A distinction between
  rules and principles is drawn in Ronald Dworkin, \emph{Taking Rights
  Seriously} 22--8.} 위에서 본 것처럼(xviii쪽 참조), 이러한 것이 법률의
도덕적 정당화(moral justifications)를 가리킨다고 본다면, 하트의 관점에서
그것들은 공식적으로 채택(adopted)되거나 승인(endorsed)되지 않는 한 법의
일부가 아니다. 그러나 `기준'과 `원칙'이라는 용어는 또한 융통성 있거나
반박 가능한(defeasible) 일반적 법 규범을 지칭하기 위해 흔히 사용된다.
이런 의미에서라면, 그것들은 하트의 이론에 쉽게 들어맞는다. 법이 어떤
쟁점에 미치는 영향을 알기 위해서는, 서로 교차하거나 충돌할 수 있는 여러
규칙들의 총합적 효과(net effect)를 이해해야 하며, 그 충돌을 해결할 수
있는 허용 가능한 방법(permissible way)은 하나 이상 존재할 수 있다.
이것이 반박 가능성(defeasibility)의 한 원천이다. 또 다른 원천은 하트가
제7장에서 설명했듯이, 모든 규칙은 일정 정도 모호하며(open-textured),
불명확한 부분이 존재한다는 사실이다. 어떤 경우에는 규칙의 적용이
명확하고, 어떤 경우에는 명백히 적용되지 않지만, 어떤 경우에는 적용
가능성은 있지만 명확하지 않다. 항소심 법원(appellate courts)의 업무 중
많은 부분이 바로 이러한 명확하지 않지만 법적으로는 논의 가능한(arguable)
사건들에 대한 판단을 포함한다. 이러한 의미에서 법의 불확정성(legal
indeterminacy)은 규칙의 경계(margins)에서 발생하긴 하지만, 그 자체가
\emph{주변적} 현상은 아니다. 이는 모든 법체계와 그 안의 모든 규칙의
특징이며, 그 결과로 ``법원과 다른 공직자들의 재량(discretion) 영역이
넓고 중요하다''(136쪽)고 할 수 있다. 법원은 법을 권위적으로 적용하는
특별한 임무를 가지지만, 동시에 입법기관과 마찬가지로 새로운 법을
창출하는 임무도 수행한다. 이 법 창출적 기능을 언제, 어떻게 사용할지는
어떤 종류의 규칙도 단순히 적용하는 문제가 아니라 실천적 판단(practical
judgement)을 요한다. 불확정성 해결에 있어 법원의 역할을 고려할 때,
오로지 또는 주로 항소심 법원이나 그 어떤 법원의 역할만을 통해 일반적인
법 이론을 구성하려는 것은 오도될 수 있다. `선택 효과(selection
effect)'로 인해 법적 불확실성이 과도하게 강조될 위험이 있기 때문이다.
법이라는 개념이 너무 고정적(cut-and-dried)이라고 보고 그것이 실제 법의
유동적이며 논쟁적인 성격을 포착하지 못한다고 생각하는 이들은 이러한
함정에 빠지기 쉽다. 그들은 법원을 구성하는 비교적 확립된 규칙들이나,
사람들이 법원에 가지 않고도 자신의 행동을 결정할 때 사용하는 평범한 법
규칙들을 보지 못한다. 예컨대 도로 표지판의 ``정지(Stop)''는 ``깜빡이는
것을 멈추라''는 뜻이 아니라 ``차를 멈추라''는 의미임을 운전자들은 알고
있다. 이를 위한 사법적 판결은 필요 없다. 이는 실제로 작동하는 법(law in
action)의 전형적인 사례다.

세 번째 논점은 다음 사실을 상기시킨다. 즉, 법적 규칙 체계 안의 모든
요소가 어떤 종류의 규칙인 것은 아니라는 점이다. 예컨대 『1998년 영국
인권법(UK Human Rights Act)』 제6조는 다음과 같이 규정한다. ``이
조항에서 `공공기관(public authority)'이란 다음을 포함한다: (a) 법원 또는
재판소, 그리고 (b) 그 기능 중 일부가 공적 성격(public nature)의 기능인
사람\ldots.'' 이는 규칙인가? 그것은 정의(definition)이다. 하지만 정의가
언어 사용을 위한 규칙이라고 본다면? 그렇다면 이는 행위를 요구하거나
부여하거나 허용하지 않기 때문에 규범(norm)은 아닌 규칙(rule)일 것이다.
정의의 법적 역할은, 그것이 섹션 1에 나오는 규범(norm)인 규칙과
\emph{함께(along with)} 어떻게 작동하는지를 보여줌으로써 설명된다.
예컨대 제1조는 이렇게 말한다. ``공공기관은 협약 권리(Convention right)와
양립할 수 없는 방식으로 행동하는 것이 불법(unlawful)이다.'' 그리고 불법
행위에 대한 구제는 다른 조항에 의해 제공된다. 이는 (함축적으로)
사람들에게 무엇을 하라고 지시하는 것이다. ('불법'이라는 표현이 이
문맥에서 무엇을 의미하는지를 아는 것이 필요하다.) 이는 규칙이 아닌
규범인 법적 자료들을 다루는 방법이기도 하다. 예컨대 법원 판결은 종종
특정한 명령(order)으로 끝난다. 즉, 누군가에게 무엇을 하라, 지불하라,
또는 어떤 불이익을 감수하라는 지시(directive)다. 이러한 일회적
명령(one-off order)은 규칙이 아니라 개별 규범(individual norm)이다. 개별
규범만으로는 통치(govern)가 사실상 불가능하지만, 그것 없이는 통치는
논리적으로 불가능하다. 법원이 사람들의 법적 지위를 권위 있게 결정하려면,
특정한 사람들에게 구속력을 갖는 판단(ruling)을 내려야 한다. 따라서
규칙은 법과 법체계를 이해하는 데 필수적이지만, 그것만으로는 충분하지
않다. 우리는 규칙이 어떤 내용을 다루는지, 무엇을 하도록 기대되는지,
규칙과 함께 작동하는 다른 자료들이 무엇인지, 그리고 규칙에 의해 지배되지
않는 법적 의사결정이 무엇인지도 알아야 한다. 하트는 이러한 주제들 가운데
일부에 다른 것보다 더 많은 시간을 할애하긴 했지만, 이 모든 주제는 그의
이론 안에 수용 가능하다. 널리 퍼진 오해와 달리, 하트는 법이
\emph{단순히} 규칙의 문제라고 말한 적이 없다. 또한 규칙이 \emph{모든}
법적 현상을 설명한다고도 말하지 않았다. 사실 그는 그러한 오류를
경계한다. ``기본 및 2차 규칙의 결합은 법의 많은 측면을 설명해 주기
때문에 중심적인 위치를 차지할 자격이 있다. 그러나 이것만으로 모든 문제를
밝힐 수는 없다.\ldots{} {[}이 결합은{]} 법체계의 중심에 있긴 하지만,
그것이 전부는 아니다\ldots.'' (99쪽) 법철학(jurisprudence)은 이 밖의
여러 관심사들을 함께 고려해야 한다.

\subsection{\texorpdfstring{\textbf{3. LAW AND POWER (법과
권한)}}{3. LAW AND POWER (법과 권한)}}\label{law-and-power-uxbc95uxacfc-uxad8cuxd55c}

지금까지의 논의는 다음과 같다. 법(law)은 사회적 규칙(social rules)의
산물이며, 이러한 규칙들 역시 실천(practice)으로부터 구성된 것이다. 이는
결국 사회적 통제(social control)의 수단인 제도를 다소 안이하게 이해한
것처럼 들릴 수도 있다. 그렇다면 갈등(conflict), 강제(coercion),
권한(power)은 어떻게 되는가?

\subsubsection{\texorpdfstring{(i) \emph{법에서의 분업(The Division of
Labour in
Law)}}{(i) 법에서의 분업(The Division of Labour in Law)}}\label{i-uxbc95uxc5d0uxc11cuxc758-uxbd84uxc5c5the-division-of-labour-in-law}

하트(Hart)는 법을 주권자(sovereign)의 명령이 위협(threats)을 수반하는
것으로 본 오스틴(Austin)의 상향적 피라미드식 관점(top-down, pyramidal
view)에 반대한다. 이러한 관점이 조잡하다는 점은 널리 인정되지만, 일부는
그것이 일종의 경각심을 주었다고 보고, 하트가 법실증주의(legal
positivism)를 보다 정교하게 만들었지만 그 강력함을 잃었다고 느낄 수도
있다. 법은 단지 합의(consensus)나 일치(agreement)에 관한 것이 아니라,
갈등(conflict)과 불일치(disagreement)에도 관한 것이다.\footnote{On other
  aspects of this theme see Jeremy Waldron, \emph{Law and Disagreement}
  (Oxford University Press, 1999).} 물론, 만약 이러한 주장이 항소심
법원(appellate courts)에서의 활동이 고도로 정치화되어 있고, 기존의 법을
적용한다기보다 논쟁 가능한 사건을 결정(settle)하는 것이 주된 활동이라는
익숙한 주장이라면, 이를 반대할 이유는 없다. 방금 전 우리는 이것이 하트의
이론과 어떻게 들어맞는지를 살펴보았다. 다만 하트는 최소한 어떤 수준의
합의가 제도가 작동을 시작하려면 필요하다고 본다. 적어도 `승인 규칙(rule
of recognition)'은 무엇이 법을 구성하는 행위인지에 대한 합의 위에
성립되어야 한다. 하지만 누구의 합의란 말인가? 다음은 드워킨(Dworkin)이
하트의 이론을 재현한 방식이다:

\begin{quote}
법의 진정한 근거(grounds of law)는 전체 공동체가 하나의 근본적이고
포괄적인 규칙(master rule)을 수용(acceptance)하는 데 있다. (그는 이것을
`승인 규칙(rule of recognition)'이라 부른다)\ldots{} 오스틴에게는
캘리포니아의 제한 속도가 55마일이라는 명제가 참인 이유가 단순히 그
규칙을 제정한 입법자들이 그곳을 지배하고 있기 때문이다. 반면 하트에게
그것이 참인 이유는, 캘리포니아 주민들이 주 및 연방 헌법의 권위
체계(authority scheme)를 수용해 왔고 계속 수용하고 있기
때문이다.\footnote{Ronald Dworkin, \emph{Law's Empire} 34.}
\end{quote}

이 설명이 `법의 근거(grounds)'를 설명하는 방식으로서 문제가 있다는 점은
분명하다. 캘리포니아의 많은 사람들은 주 및 연방 헌법의 권위 체계가
무엇인지 전혀 알지 못하며, 심지어 `주 헌법(state constitution)'이
존재한다는 사실조차 모르는 사람들도 있다. 또한 이 설명은 하트의 이론에
대한 해석으로도 문제가 있다. 하트에 따르면, 법 이전의(pre-legal)
사회에서 사회 규범(social norms)은 광범위한 지지 없이는 존재할 수 없다.
``보다 단순한 구조(법이 형성되기 이전)에서는, 공직자들이 존재하지 않기
때문에, 규칙은 집단의 행동에 대한 결정적 기준을 설정하는 것으로서 널리
수용되어야 한다. 그곳에서는 `내면적 관점(internal point of view)'이 널리
퍼져 있지 않다면 논리적으로 규칙이 존재할 수 없다''(117쪽). 관습적
규칙(customary rules)은 일반적인 동의(general buy-in)를 필요로 한다.
그러나 ``\ldots 일차 및 이차 규칙이 결합된 사회에서는\ldots{} 규칙을
집단의 공통된 기준(common standards)으로 수용하는 것이, 단지 개인이
자발적으로 규칙을 따르는 것과는 분리될 수 있다. 극단적인 경우에는,
`이것은 유효한 규칙이다'라는 규범적 언어를 사용하는 내면적 관점이 오로지
공직자 세계에만 제한될 수 있다. 이러한 더 복잡한 체계에서는,
공직자들만이 유효성 기준(criteria of validity)을 수용하고 사용할 수도
있다. 이러한 사회는 한없이 순종적인 양 떼(sheeplike) 같을 수 있고, 결국
도살장으로 끌려갈 수도 있다. 그러나 그것이 존재할 수 없거나 법체계로
인정받을 수 없다고 생각할 근거는 거의 없다''(117쪽).

이 인용문을 길게 제시한 이유는, 그것이 법의 본질(nature of law)을
이해하는 데 핵심적인 점을 제시하며, 동시에 법이 그러한 본질을 지닌다는
사실의 정치적 의미를 보여주기 때문이다. 하트는 관습(custom)과 사회적
도덕성(social morality)은 의도적인 변화에 대해 면역(immunity)을 지니며,
점진적으로만 진화한다고 지적한다. 작은 규모의 안정된 공동체에서는
그것들이 제도를 운영하는 꽤 괜찮은 방식이다. 우리가 일상생활의 많은
부분을 그렇게 운영하는 것이 일반적이다. 그러나 규모가 크고 복잡한 사회는
규범(customs and other norms)을 공적으로 확인할 수 있고 즉시 변경할 수
있는, 보다 의도적인 사회 통제 메커니즘(deliberate mechanisms of social
control)을 필요로 한다. 예컨대 통치자의 선언, 다수결 투표 등의 방식으로
말이다.

이러한 메커니즘은 제도화(institutionalization)를 통해 가능해진다. 즉,
규칙을 식별하고 변경하며 집행할 권한을 가진 특화된 기관(specialized
organs)의 등장이다. 그 결과 나타나는 규범적 분업(division of normative
labour)은 일종의 양면성을 지닌다. 그것은 이득과 비용을 모두 수반한다.
``그 이득은 변화에 대한 적응력, 확실성(certainty), 효율성(efficiency)
등이고\ldots{} 그 비용은, 중앙집중화된 권력이 집단적 지지 없이도
다수에게 억압(oppression)을 가할 수 있는 위험이다. 이는 보다 단순한 일차
규칙 체계에서는 발생할 수 없는 일이다''(202쪽). 따라서 법은
\emph{보편적으로 좋은 것}도 아니고, \emph{무조건 좋은 것}도 아니다.
제도적 성격(institutional character)이 특정한 이득을 가능하게 하지만,
동시에 특정한 비용도 가능하게 한다. 이러한 비용은 법 없는 사회에서는
발생하지 않을 수 있는 것들이다. 하트가 위에서 논한 극단적인 사례까지
가지 않더라도, 법이 작동하는 일반적인 사회에서는 광범위한 사회적
합의보다는 좁은 공직자 집단의 합의에 더 의존한다.\footnote{이는
  구체적으로 어떤 공직자(official)가 중요한지, 그리고 `공직자'의 역할이
  어떻게 규정되어야 하는지를 정확히 다루지 않는다. 일반적으로 말해,
  하트(Hart)는 적어도 판사(judges)와 입법자(legislators)를 포함할 의도를
  지니며, 이때 `공직자'는 법적 정의가 아니라 사회정치적(socio-political)
  정의를 따른다.} 법이 존재하기 위해 일반 대중에게 요구되는 것은, 해당
체계의 강제 규범(mandatory norms)에 대해 수동적으로 따르는
것(acquiescence) 이상은 거의 없다.

그렇다면, 법이 의존하는 합의는 절대 편안하거나 공동체적인 것이 아니다.
그것은 가치에 대한 일치(value agreement)를 전제로 하지 않으며, 법의 작동
속에서의 중대한 반대(dissent)를 배제하지도 않는다. 이것은 왜 ``모든
법체계는 필연적으로 그 공동체의 가치를 표현한다''는 낭만적 신념이 잘못된
것인지를 보여준다. 정의롭고 소중한 법체계조차도, 결국에는 난해하고,
기술적이며, 그 지배 대상자들의 삶에서 멀어질 수 있다. 규범적 분업 때문에
법은 항상 `법률주의적(legalistic)'이 될 위험에 노출되어 있다. 모든
법철학자들은 법이 도덕적으로 오류 가능(morally fallible)하다는 사실에
동의한다. 하트의 특별한 기여는, 법이 실패하는 몇 가지 방식이 바로 그것이
사회 제도(social institution)라는 본질과 밀접하게 연결되어 있다는 점을
보여준 데 있다.

\subsubsection{\texorpdfstring{(ii) \emph{강제와 권한(Coercion and
Power)}}{(ii) 강제와 권한(Coercion and Power)}}\label{ii-uxac15uxc81cuxc640-uxad8cuxd55ccoercion-and-power}

법은 본질적으로 강제적 장치(coercive apparatus)라는 생각은 일반인의
직관과 잘 맞아떨어지며, 법철학(jurisprudence)에서도 오랫동안 인기를
끌어온 견해이다. 그러나 하트(Hart)는 이 견해가 잘못되었다고 본다. 모든
법체계(legal system)는 강제로 집행되지 않는 규범(norms)을 일부 포함하며,
심지어 전적으로 그러한 규범들로 구성된 법체계조차도 상상 가능하다고
한다(199--200쪽). 그렇다면 제재(sanction)가 없는 법은 무슨 의미가
있는가? 제재가 있는 법과 마찬가지로, 사람들의 행위를 지시(direct)하는 데
그 목적이 있다. 제재는 법의 \emph{플랜 B}다. \emph{플랜 A}는 법의
대상자(subjects)들이 추가적인 감독 없이도 법에 자발적으로 따르도록 하는
것이다. 지시가 필요하지만 강화 동기(reinforcing motivation)는 필요하지
않은 경우에는 제재 없는 법이 그리 드문 것이 아니다. 예컨대 \emph{미국
연방법전}(United States Code)에는 국기에 대한 예우를 지시하는 규범이
포함되어 있다. (``국기는 어떤 것도 받거나, 보관하거나, 운반하거나
전달하는 용기로 사용되어서는 안 된다.''\footnote{4 U.S.C. § 8 (h).})
하지만 이 규범을 위반했을 경우 처벌 조항은 존재하지 않는다. 만약 인간
본성이 지금과 달랐다면, 모든 법 규범이 이와 같을 수도 있었을 것이다.

하지만 인간 본성이 지금과 같음에도 불구하고, 많은 법 규범은 제재로
뒷받침되지 않는다. 중요한 범주의 하나는 권한 부여 규범(power-conferring
norms)이다. 즉, 법 규범 중에서 다른 법적 규범이나 지위를 변경할 수 있는
능력을 창출하는 규칙들이다. 예컨대, 입법, 법인 설립, 계약 체결, 혼인
등을 가능하게 하는 규칙들이 그러하다. 이러한 권한이 자발적인
것이라면(일반적으로 그렇다), 사람들은 그것을 행사할지 여부를 자유롭게
선택할 수 있다. 입법하거나, 법인을 설립하거나, 계약하거나, 혼인하려는
사람은 법이 정한 절차를 따르지 않으면 그 행위는 무효(null and void)가
된다. 예컨대, 법적 요건을 갖추지 못한 `혼인'은 무효다. 그러나 이를
따르지 않았다는 이유로 처벌을 받지는 않는다. 그렇다면 무효 자체가 일종의
처벌이므로, 이 역시 강제적인 법이라고 보아야 할까? 하트는 그렇게
보아서는 안 되는 이유를 설명한다. 여기에는 두 가지 별개의
요소---무언가를 하라는 명령(order)과 그것을 위반했을 때의
제재(sanction)---가 존재하지 않기 때문이다. 아예 명령이 존재하지 않으며,
그 `제재'는 실제로는 권한 부여 규범 그 자체일 뿐이다. 켈젠(Kelsen)은
강제 이론(coercion theory)을 유지하기 위한 우회로(work-around)를
제안했다. 그는 권한 부여 규칙은 실제로는 법률의 조각(fragment)에
불과하므로, 그 안에 제재가 없다는 것은 놀라운 일이 아니라고 주장했다.
제재를 담고 있는 규범은, 예컨대 배우자를 부양할 의무와 같은 규범이며,
혼인에 관한 권한 부여 규칙은 누가 배우자인지 여부를 결정하는 것이라고
본다. 이리저리 따져보면 결국 모든 것은 강제로 환원된다(coercion). 하트는
이에 대해 흥미로운 방식으로 응답한다. 그는 켈젠의 재구성이
불가능하다거나 비논리적이라고 말하지 않는다. 다만 그것이 동기를
결여(unmotivated)하고, 법철학의 방법론적 제약(methodological
constraint)에 어긋난다고 주장한다.

\begin{quote}
``법이 사회 통제의 수단으로서 수행하는 주요 기능(principal functions)은
사적 소송(private litigation)이나 형사 기소(prosecutions)에서 나타나지
않는다. 그것들은 제도의 실패에 대비한 필수적이지만 보조적인 장치일
뿐이다. 법의 주요 기능은, 법이 법정 바깥에서 인생을 통제하고, 안내하고,
설계하는 다양한 방식에서 나타난다''(40쪽).
\end{quote}

개별 법률(laws)로 법적 자료(legal material)를 나누는 데 있어서
본질주의적(essentialist), 즉 `형이상학적(metaphysical)' 해답은 존재하지
않는다. 가장 좋은 접근 방식은, 법을 실제로 사용하는 사람들(대부분은 법정
밖에서 살아가는 사람들)에게 법이 어떻게 작용하는지를 이해할 수 있게
해주는 방식이다. 권한 부여 규칙은 의무를 부과하는 규칙과는 다르게
인식되고, 언급되고, 사용되며, 전혀 다른 이유로 가치가 부여된다. ``그
특성이 다르다는 것을 보여줄 다른 기준이 무엇이 있겠는가?''(41쪽) 이 점은
하트의 방법론(method)의 정수를 보여준다. 홈스(Holmes)의 `나쁜 사람(bad
man)'에게 법은 피해야 할 비용(costs)에 관한 것이며, 법률가에게 법은
가능한 법정 사건들과 거기에서 얻을 수 있는 수임료(legal costs)에 관한
것이다. 법이론들은 이러한 외눈박이(cyclopic) 관점에서 파생되어 왔다.
그들은 현실이긴 하지만 주변적인 요소(marginal)를 중심적인 것으로
간주하며, 법의 중요성이 드러나는 여러 차원을 납작하고 환원적인(flat and
reductive) 그림 속에 지워버린다.

여기까지는 모두 타당한 주장이다. 그러나 만약 우리가 법이 사회 통제의
수단으로서 수행하는 \emph{모든} 주요 기능을 고려하고자 한다면, 여기서
하트가 제시한 수준 이상으로 나아가야 한다. 권한 부여 규칙을 의무 부과
규칙으로 환원하거나, 무효(nullity)를 일종의 제재로 해석하려는 시도는
오류이지만, 권한 부여 규칙이 사회적 권한(social power)과 밀접하게 얽혀
있다는 점을 주목하는 것은 옳다. 그렇다면 애초에 우리는 왜
강제(coercion)에 관심을 갖는가? 그 하나의 이유는 책임(responsibility)과
관련되어 있다. 위협(threat)에 의해 어떤 일을 하게 된 사람은 일반적으로
그 행위에 대해 책임을 지지 않는다. 그 의지가 억압(overborne)되었기
때문이다. 물론 대부분의 법적 제재는 그렇게 극단적으로 가혹하지
않다(지속적인 불이행을 제외하면). 그럼에도 여전히 그것은 사람들의
유인구조(incentives)에 영향을 미치며, 이는 권한 부여 규칙에도 해당된다.
강제(coercion)는 법 권한의 날카로운 측면(hard edge)이고,
유인(incentivizing)과 표현(expressive) 기능은 법 규범의 부드러운
측면(soft edge)에 속한다.

다시 한 번, 혼인의 권한을 부여하는 규칙들을 떠올려보자. 이 규칙들은
일정한 조건 하에서만 해당 권한을 부여한다. 예전에는(그리고 어떤
지역에서는 여전히) 누구와 결혼할 수 있는지에 대해 인종이나 성별에 따른
제한이 존재했다. 인종이 다른 사람 사이의 결혼이나 동성 간의 결혼은
법적으로 무효였다. 하트가 제시한 이유들에 비추어 볼 때, 이것을 이성애적
또는 동인종적 관계를 `강요'한 강제라고 보는 것은 잘못일 것이다. 아무도
결혼을 반드시 해야 하는 것은 아니기 때문이다. 따라서 이런 법은 동성애
행위에 대한 형벌이나 도망 노예 법(fugitive slave laws)과는 다르다.
그러나 이러한 권한 부여 규칙(또는 그것들과 해석 규칙의 결합)에 의해
그러한 혼인이 무효가 된 것은 결코 우연한 결과나 의도하지 않은 부산물은
아니었다. 그것이 바로 해당 법의 목적이었다. 이러한 법은 명령이나 제재와
같은 조악한 수단에 의존하지 않고도 개인의 삶과 공동체의 문화를
형성하고자 했다. 그리고 실제로 어느 정도 성공했다. 따라서 하트가
호의적으로 `편의(facilities)'를 제공한다고 표현한 규칙들의 기능을 생각할
때, 우리는 이 점을 잊지 말아야 한다. 모든 법이 강제적인 것은 아니지만,
비강제적인 법 역시 강제적인 법과 동일한 기능을 수행할 수 있다. 즉, 사회
권한을 표현하고 그것을 유도하는 것이다. 이러한 기능은 법 규범의
내용(content)을 통해서도, 보다 일반적인 구조적 특징을 통해서도 작용한다.
예를 들어 자발적 권한(voluntary powers)은 자신의 의지를 행사할 수 있는
자에게 법적 통제권을 분배하고(individual powers), 개인적 권한은 그것을
개별 개인에게 분배한다. 이는 누구에게 무언가를 `강제'하지는 않지만,
그러한 법을 만들고 적용하는 이들이 의도했으며 예측 가능한 방식으로 사회
세계(social world)를 형성한다.

\subsection{\texorpdfstring{\textbf{4. 법과 도덕 (LAW AND
MORALITY)}}{4. 법과 도덕 (LAW AND MORALITY)}}\label{uxbc95uxacfc-uxb3c4uxb355-law-and-morality}

이 책의 중심 문제 중 하나는 법과 도덕 간의 다형적(pluriform) 관계---즉
관습적 또는 `사회적 도덕'(social morality)과 이상적 또는 `비판적
도덕'(critical morality)---에 관한 것이다. 하트는 법과 도덕 사이에
일정한 \emph{분리(disjunction)}가 존재한다고 강하게 주장한 것으로
유명하다. 그의 이론에 대해 아는 바가 거의 없는 사람조차도, 그가 역사적인
홀름스 강연에서 다음과 같이 말했음을 알고 있다: ``법과 도덕 사이에는
필연적 연결이 없다(no necessary connection between law and
morals).''\footnote{H. L. A. Hart, `Positivism and the Separation of Law
  and Morals' (1958) 71 \emph{Harvard Law Review} 593, at 601 n.~25.}
우리는 이미 위의 3 (i)에서, 법이 반드시 지배받는 인구가 실제로 승인하는
도덕적 가치들을 반영할 필요는 없다는 이유를 살펴보았다. 하지만 그들이
\emph{마땅히} 따라야 할 도덕적 가치들과는 어떠한가? 하트는 이 경우에도
필연적 연결이 없다고 말하고자 하는가? 『법의 개념(The Concept of
Law)』에서 그는 때때로 이 생각을 다른 방식으로 서술한다. 한 지점에서
그는 실증주의의 핵심 명제를 다음과 같이 묘사한다: ``법이 도덕의 특정
요구들을 재현하거나 충족시킨다는 것은 결코 필연적 진리가
아니다''(185--6쪽). 이는 더 협소한 주장처럼 보인다: 모든 법이 건전한
도덕 기준을 `재현하거나 충족시켜야 한다'는 것을 요구하지 않으면서도,
법과 도덕 사이에 필연적 관계가 존재할 수 있는 것이다.

하트의 첫 번째이자 보다 광범위한 서술은, 그의 견해(즉, 법이 사회적
구성물이라는 견해)를 공유하는 이들 사이에서도 거의 지지를 얻지
못했다.\footnote{See John Gardner, `Legal Positivism: 5½ Myths', chap.~2
  of his \emph{Law as a Leap of Faith} (Oxford University Press, 2012);
  and Leslie Green, `Positivism and the Inseparability of Law and
  Morals' (2008) 83 \emph{New York University Law Review} 1035.} 법과
도덕이 모두 인간 행위를 규율한다는 것이 단순한 우연일 뿐이라고 보기는
어렵다. 인간이 어떻게 살아야 하는가에 대해 전혀 언급하지 않는 규범
체계는 법적 규범도 아니고, 도덕적 규범도 아니다. 이는 법과 도덕 사이의
필연적 연결 하나를 암시한다. 그리고 또 다른 연결들도 있다. 실제로 하트의
성숙한 이론은 법과 도덕 사이의 두 가지 흥미로운 필연적 연결을 더
승인한다. 하나는 법의 목적을 경유하며, 다른 하나는 법과 정의 간의 추정적
연결을 경유한다. 또한 하트의 이론은, 많은 다른 실증주의자들이 인정하기를
꺼리는, 법과 도덕 간의 우연적 연결도 허용한다. 이 세 가지 주장은 분리
명제의 어떤 버전 못지않게 하트 이론에서 중요하다. 그러나 이 중 명확하게
타당한 것은 첫 번째 주장뿐이다.

\subsubsection{\texorpdfstring{(i) \emph{법의 목적 (Law's
Purpose)}}{(i) 법의 목적 (Law's Purpose)}}\label{i-uxbc95uxc758-uxbaa9uxc801-laws-purpose}

법은 단순한 규칙의 체계가 아니라, 다양한 목적을 수행하는 체계이다.
토마스 아퀴나스는 법이 \emph{총체적} 목적(\emph{overall} purpose)을
가진다고 보았는데, 그는 이를 ``공공선을 위한 이성적 명령(an ordinance of
reason made for the common good)''이라고 정의하였다.\footnote{\emph{Summa
  Theologica} II-I, q. 90 a. 4.} 현대적 접근으로는, 법이 행위를 지도하기
위해, 공공선을 위한 활동을 조정하기 위해, 정의를 실현하기 위해, 혹은
강제력을 허용하기 위해 제정된다는 견해들이 있다.\footnote{Guiding
  conduct: Lon L. Fuller, \emph{The Morality of Law} (rev. edn., Yale
  University Press, 1969); coordinating activity: John Finnis,
  \emph{Natural Law And Natural Rights} (Oxford University Press, 1980);
  doing justice: Michael Moore, `Law as a Functional Kind,' in R. P.
  George ed., \emph{Natural Law Theory: Contemporary Essays} (Oxford
  University Press, 1992) 221; licensing coercion: Ronald Dworkin,
  \emph{Law's Empire} 93.} 이러한 주장은 법에 대한 이상적 제안이 아니라,
\emph{구성적 목적(constitutive aims)}에 대한 주장으로 이해되어야 한다.
핵심 아이디어는 다음과 같다: 이러한 목적을 지니지 않은 사회적 통제
체계는 법체계라고 할 수 없으며, 이는 지식 추구를 목적으로 하지 않는
기관이 대학이 아닌 것과 같다. 구성적 목적을 갖는다는 것만으로는
도덕성과의 연결을 확립하지 않는다. 이는 그 목적이 무엇이냐에 달려 있다.
식기세척기가 설거지를 위해 존재하지 않는다면 식기세척기가 아니며, 설거지
능력은 식기세척기의 품질을 판단하는 주요 기준 중 하나다. 그러나 설거지는
일반적으로 도덕적으로 중요한 활동이 아니므로, 좋은 식기세척기가 곧
도덕적으로 좋은 식기세척기인 것은 아니다. 법의 구성적 목적에 대한 위의
제안들은 도덕적으로 중요한 것에서 도덕적으로 중립적인 것까지 다양하다.
정의 실현은 도덕적으로 선한 것이고, 행위 지도는 도덕적으로 중립적이며,
강제력 허용은 도덕적으로 모호하다.\footnote{강제(coercion)를 허용한다는
  것은, `현재 진행 중인 강제 행위에 대해 정당화(justification)를
  제공한다'는 의미인가? 아니면 `정당화될 수 없는 강제가 발생하지 않도록
  보장한다'는 의미인가? 혹은 `정당화될 경우에만 사람들에게 강제를
  가한다'는 의미인가?}

도덕성과의 연결은 또한 구성적 목적이 어느 정도 달성되는가에 달려 있다.
제9장에서 하트는, 인간 생존이 도덕적으로 선한 것이며, 생존을 지향하지
않는 규범 체계는 법체계가 될 수 없다고 전제한다. 또한 어떤 체계가
법체계로 존재하려면, 반드시 전 인구에게 항상은 아니더라도, 상당수에게
상당 시간 성과를 제공해야 한다(deliver the goods)고 본다. 일반적으로,
구성적 목적을 지닌 것들은 그 종류에 속할 자격을 박탈당하기 전까지 상당한
여지를 가진다. 고장이 났거나 결함이 있는 식기세척기도, 수정되거나 수리될
경우 설거지를 수행할 수 있는 능력이 있다면 여전히 식기세척기이다.
법체계도 마찬가지다. 법이 수행해야 할 기능을 제대로 수행하지 못하는
통합된 법률 체계라도 여전히 법체계로 간주될 수 있다. 이는 \emph{어떤
것을 지향한다(aim)}는 것이 반드시 그것을 \emph{성취한다(succeeding)}는
것을 요구하지 않는다는 사실에서 비롯된다.

이 문제에 대한 하트의 마지막 성찰에서는, 법이 생존을 지향할 필요조차
없다는 입장으로 선회한 듯하다. 그는 법이 어떤 흥미로운 구성적 목적도
지니지 않는다고 본 막스 베버(Max Weber)와 한스 켈젠(Hans Kelsen)의
입장에 동참한다. (켈젠은 ``법은 목적이 아니라, 특정한 사회적
수단이다(Law is a means, a specific social means, not an end)''라고
말했다.\footnote{Hans Kelsen, \emph{General Theory of Law and State} (A.
  Wedberg tr., Harvard University Press, 1949) 20.}) 하트는 이렇게 쓴다:
``나는 법이 인간 행위를 지도하고, 그 행위에 대한 비판 기준을 제공하는 것
외에, 그 자체로 수행하는 보다 구체적인 목적을 찾는 것은 매우 헛된
일이라고 본다''(249쪽). 여기에는 생존에 대한 언급이 없다. 그러나 하트가
이전의 `법은 생존을 촉진하는 목적을 가진다'는 주장을 철회한 것일 수도
있고, 혹은 법이 그러한 목적들에 의해 \emph{식별될 수 있다}(identified)는
주장 자체를 부정하는 것일 수도 있다. 즉, 법체계들 사이에서 보편적이며
동시에 고유한 목적은 없다는 것이다. 법은 생존을 촉진하려는 목적을 가질
수 있고, 행위를 지도하거나 평가하려는 목적을 가질 수 있다. 그러나 이들
모두는 관습, 종교, 도덕 등과 법을 구분해 주지는 못한다. 오히려 이들은
상호 중첩되는 지점이다. 법과 도덕은 유사한 과업에 주목하며, 유사한
이유로 그러한 과업에 접근하고, 일부 유사한 기법을 사용한다.

\subsubsection{\texorpdfstring{(ii) \emph{법과 정의 (Law and
Justice)}}{(ii) 법과 정의 (Law and Justice)}}\label{ii-uxbc95uxacfc-uxc815uxc758-law-and-justice}

제8장에서 하트는 법과 도덕 사이의 놀라운 연결을 옹호한다. 그의 논증은
규칙 준수(rule-following)를 정의(justice)와 연결시키는데, 이때 둘 다
유사한 사례는 유사하게 대우되어야 한다는 생각을 공유한다. 실천
이론(practice theory)에 따르면, 일반 규칙(general rules)은 일정한
방식으로 그것에 부합하거나(conform) 반복적으로 적용되지 않으면 존재할 수
없다. 그런데 하트는 이 일관성(constancy) \emph{자체가(itself)} 일종의
정의라고 말한다. 그는 이렇게 쓴다: ``{[}가장 혐오스러운 법조차도
정의롭게 적용될 수 있지만{]}, 일반 법규칙을 적용한다는 개념 자체 안에는
적어도 정의의 씨앗(germ of justice)이 담겨 있다''(206쪽, cf.~160쪽).
이로부터, 모든 존재하는 법체계는 어떤 식으로든 정의를 실현한다고
결론지을 수 있다. 물론 이는 `실질적 정의(substantive justice)'가 아니다.
혐오스러운 법을 일관되게 적용하는 것이 그것의 혐오스러움을 보완하거나
완화하지는 못한다. 하지만 그러한 적용은 여전히 법
\emph{적용(application)}에 있어서의 정의---즉 어떤 이들이 말하듯 `형식적
정의(formal justice)'---를 산출한다.\footnote{David Lyons criticizes it
  under that label in `On Formal Justice' (1973) 58 \emph{Cornell Law
  Review} 833. Matthew Kramer defends it under the better label of
  `constancy': Matthew Kramer, `Justice as Constancy' (1997) 16
  \emph{Law and Philosophy} 561.} 이는 하나의 법이, 그 법이 스스로(그
판단이 옳든 그르든) 관련성 있다고 간주하는 방식으로 유사한 자들에게만,
그리고 전적으로 적용되어야 함을 요구한다. 그리고 이 요건은 `극도로
억압적인(hideously oppressive)' 법이나, `권리를 박탈당한 노예(rightless
slaves)'에게 어떤 작동 가능한 법체계의 최소한의 혜택조차도 부정하는 법에
대해서도 동일하게 적용된다.\footnote{H. L. A. Hart, `Positivism and the
  Separation of Law and Morals' 593, at 626.}

나는 `혐오스러운(odious)', `극도로 억압적인(hideously oppressive)' 등의
하트의 표현을 인용한 바 있는데, 이는 그의 `정의의 씨앗(germ of
justice)'이라는 논제가 얼마나 대담한지를 분명히 하기 위함이다. 적용에
있어서의 일관성은, 그 법이 단지 약간 부정의한 수준일 때는 납득 가능하게
느껴질 수도 있다. 예컨대, 정당화 목적에 비추어 과포괄적이거나
과소포괄적인 법---17세 미만 모두에게 운전을 금지하는 규범이 몇몇에게는
과도하게 금지하고, 또 다른 이들에게는 과도하게 허용하는 경우 등---이
그러하다. 완벽한 법은 없으며, 작지만 명백한 결함이 있는 법을 일관되게
적용할 이유는 여럿 있을 수 있다. 하지만 이것만으로는 하트의 주장을
정당화할 수 없다. 그는 이 논지가 혐오스러운 법에도 적용된다고 말하며,
우리가 실질적 정의, 형평성(equity), 자비(mercy), 혹은 제정신(sanity)을
허용하는 경우조차도, 우리는 그 과정에서 무언가 귀중한 것을 상실했음을
인식한다는 것이다: 우리는 적어도 하나의 측면에서는 부정의(injustice)를
저지른 셈이 된다.

그러나 이 `형식적(formal)' 정의 개념에는 어딘가 이상한 구석이 있다.
결국, 정의\emph{의 형식}(\emph{the form of} justice)을 가진 모든 것이
반드시 정의의 \emph{한} 형식(\emph{a} form of justice)인 것은 아니다.
이는 낙타의 형식(the form of a camel)을 가진 것이 반드시 한 낙타(a
camel)인 것이 아닌 것과 마찬가지다. 더 나아가, 정의 규범(norms of
justice)과 부정의 규범(norms of injustice)은 그 형식에서 반드시 다르지
않다.\footnote{Following John Gardner, `The Virtue of Justice and the
  Character of Law', chap.~10 of his \emph{Law as a Leap of Faith}.}
예를 들어, ``남성과 여성은 동등한 가치의 노동에 대해 동일한 보수를
받아야 한다''는 규범은 정의 규범이다. 반면, ``남성과 여성은 동등한
가치의 노동에 대해 서로 다른 보수를 받아야 한다''는 규범은 \emph{부정의
규범}이다. 두 규범은 동일한 형식을 가진다. 그렇다면 ``모든 규칙을 그것의
적용대상에게만, 그리고 전원에게 적용하라''는 규범은 어떠한가? 이것은
``이미 친구로 삼기로 한 사람들에게만 친구로 대해라''는 규범과 동일한
형식을 갖는다. 후자의 규범은 정의의 규범인가? 이 경우, 한 규범이
정의인지, 부정의인지, 또는 둘 다 아닌지 여부를 \emph{형식}만으로는
판단할 수 없다는 점이 분명해진다.

간통죄에 대해 유죄 판결을 받은 여성을 돌로 쳐죽이는 법이 존재하는
법체계에서 일하는 판사를 상상해 보자. 이 법을 그 적용대상 모두에게
적용해야 할 이유가 있는가? 특별한 경우에는 가능할지도 모른다: 그가
그러지 않으면 생명의 위협을 받을 수 있고, 거부할 경우 폭동이나 더 많은
여성들의 희생이 야기될 수 있으며, 그는 이 법을 한 번 적용함으로써 신뢰를
확보해 이후 더 효과적으로 그것을 공격할 수 있을지도 모른다. 그러나 이런
법을 \emph{모든(all)} 경우들에 일관되게 적용해야 할 이유가 있는가?
여기서는 \emph{법률 없이는 형벌 없다(nulla poena sine lege)}라는 원칙
이상의 것이 필요하다. 그 원칙은 법을 위반\emph{하지 않은(have not)}
사람을 처벌하지 말라는 것이지, 법을 위반한 모든 사람을 반드시 처벌하라는
것이 \emph{아니다(not)}. 이 조건은 우리의 가설적 사례에서 충족된다.
하지만 그 원칙은 법을 위반\emph{한(has)} 모든 사람을 처벌하라고 말하지
않는다. 그렇게 하지 않는 것이 \emph{부정의}인가? 만약 그렇다면, 누구에게
부정의란 말인가? 다른 유죄 여성들(돌로 맞아 죽게 될), 또는 이미 그렇게
처형당한 여성들의 가족들이, 모든 다른 유죄 여성들도 똑같이 혐오스럽게
대우받아야 한다고 요구할 자격이 있다는 주장은 납득하기 어렵다.

하트는 아마도 `형식적(formal)' 정의를 두 가지 건전하지만 서로 관련 없는
아이디어와 혼동하고 있을 수 있다. 하나는, 정의와 부정의는 결과뿐 아니라
\emph{절차(procedures)}에서도 나타날 수 있다는 것이다. 예컨대, 법적
분쟁에서 양 당사자의 의견을 청취해야 한다는 것은 자연법상 정의(natural
justice)의 요건이다. 이러한 절차를 허용하지 않는 절차는 부정의하다.
하지만 부정의한 규칙 \emph{자체(itself)}가 자연법상 정의를 침해할 수도
있다: 예컨대 법이 부자에게 가난한 사람보다 두 배의 변론 시간을
허용한다면, 그 법을 엄격히 적용하는 것은 자연법적 정의를 진전시키는 것이
아니라 후퇴시키는 것이다. 이와 관련된 또 다른 아이디어는, 규칙의 적용에
있어 우리는 \emph{불편부당(impartial)}해야 하며, 판단자는 `편견,
이해관계, 또는 변덕(prejudice, interest or caprice)'에 따라 행동해서는
안 된다는 것이다(161쪽). 이 주장 또한 옳지만, 부정의한 법을 해당 규정에
따라 적용하지 않으려는 사람에게 반드시 그러한 동기가 있다고 보기는
어렵다. 실제로 혐오스러운 법 자체가 편향적이거나 변덕스러울 수 있으며,
그러한 법의 선택적 비적용은 오히려 가장 선한 동기에서 비롯된 것일 수
있다.

그렇다면 하트의 `정의의 씨앗(germ of justice)' 논제로부터 우리가 구제할
수 있는 것은 있는가? 어쩌면 다음과 같은 점일 수 있다: 우리가 규칙
적용(rule-application)에 주목하게 될 때, 우리는 필연적으로 \emph{특정
사례들에 있어(in particular cases)} 규칙이 어떻게 적용되어야 하는지를
고민하게 된다---즉, 사람들이 그러한 규칙 아래에서 실제로 어떻게 대우받고
있는지, 또는 대우받아야 하는지를 성찰하게 된다. 이는 단순한 총량적
물음(aggregative questions)이 아니라 분배적(distributive) 물음에
주목하게 만든다. 이는 오늘날 얼마나 많은 처벌이 이루어지고 있는지를 묻는
데 그치지 않고, 누가 어떤 잘못에 대해 어떻게 처벌받고 있는지를 묻도록
요구한다. 사람들 사이의 혜택과 부담이 어떻게 배분되어야 하는가를
고민한다는 것은 정의에 대한 질문을 던지는 것이다. A가 마땅히 받아야 할
대우를 받았는지, 혹은 A와 B 사이의 대우 차이가 정당화될 수 있는지를 주의
깊게 고민하는 것은 정의에 대한 관심이다. 우리가 그러한 질문을 고려하고
결정할 수 있는 권한을 가진 법원과 같은 제도를 갖추고 있다면, 우리는
정의를 수행할 수 있는 제도를 갖춘 것이다. (물론, 부정의를 수행할 수도
있다.) 어쩌면, 크고 복잡한 사회에서는 정의를 실현하기 위해 그런 제도가
필수적일지도 모른다.

\subsubsection{\texorpdfstring{(iii) \emph{법적 유효성과 도덕 원칙
(Legal Validity and Moral
Principles)}}{(iii) 법적 유효성과 도덕 원칙 (Legal Validity and Moral Principles)}}\label{iii-uxbc95uxc801-uxc720uxd6a8uxc131uxacfc-uxb3c4uxb355-uxc6d0uxce59-legal-validity-and-moral-principles}

법과 도덕 사이의 세 번째 접점은 다른 성격을 가진다. 하트는 도덕
원칙(moral principles)이 반드시 법의 근원이 되는 것은 아니지만, 법의
근원이 \emph{되는(are)} 요소들에 의해 승인(authorized)된다면 법의 근원이
\emph{될 수 있다(could)}는 점을 인정한다. 이를 다르게 표현하면, 하트는
승인 규칙(rule of recognition)이 필연적으로 사회적 구성물(social
construction)이라는 점은 인정하지만, 그 승인 규칙이 사용하는
기준(criteria) 자체가 사회적 구성물일 필요는 없다고 본다는 것이다. 그는
도덕 원칙들이 가치 있는 원칙이기 때문에 혹은 기존의 법을 정당화하기
때문에 법이 된다고 생각하지는 않는다. 하지만 어떤 방식으로든 법 안에
포함된다면, 그것들은 법이 될 수 있다.

이 지점에서 하트는 구성주의 논제(constructivist thesis)의 두 해석 가운데
하나에 입장을 명확히 하는데, 그것은 흔히 `포괄적 법실증주의(inclusive
legal positivism)'로 불리는 입장이다. 이 입장에 따르면 법의 근원(source
of law)은 이상적 도덕(ideal morality)의 원칙을 포함할 수 있다. 하트는
반대로, 이를 허용하지 않는 `배타적 법실증주의(exclusive legal
positivism)'에는 반대한다.\footnote{Hart calls his position `soft'
  positivism, but `inclusive' positivism is more perspicuous since it
  holds that law \emph{includes} anything to which law refers.} 그가 이
결론에 이르는 경로는 명확하게 되짚기 어렵다. 이는 그가 본문을 쓸
당시에는 이러한 대안들이 아직 명확히 구분되어 있지 않았기 때문이다.
하지만 후기(Postscript)에서 하트는 다음과 같은 결론에 도달한다: 법적
유효성(legal validity)은 어떤 규범(norm)의 `족보(pedigree)'---즉, `법이
법제 기관(legal institutions)에 의해 채택되거나 창출되는 방식에만 관련된
특성으로, 그 내용(content)과는 무관한 특성'---만으로 결정될 필요는
없다(247쪽).\footnote{The metaphor is first used by Ronald Dworkin,
  \emph{Taking Rights Seriously} 17.} 대신, 법적 유효성은 도덕적
적절성(moral propriety)에 의해 결정될 수 있다.

이 자리에서는 포괄적 법실증주의 논제 자체를 평가할 수 없다.\footnote{The
  theory is defended by W. J. Waluchow, \emph{Inclusive Legal
  Positivism} (Oxford University Press, 1994), and by Jules Coleman,
  \emph{The Practice of Principle}: \emph{In Defence of a Pragmatist
  Approach to Legal Theory} (Oxford University Press, 2001). It is
  criticized by Joseph Raz, \emph{Ethics in the Public Domain: Essays in
  the Morality of Law and Politics} (Oxford University Press, 1994),
  chaps. 9, 10; and Scott Shapiro, `On Hart's Way Out' (1998) 4
  \emph{Legal Theory} 469.} 그러나 하트가 이 논의에 끌고 들어온 혼동
하나를 제거하는 것은 의미 있다. 승인 규칙의 기준(criteria)이 도덕 원칙을
포함할 수 있는가라는 질문은, 이 기준들이 족보 기준(pedigree
test)인가라는 질문과 동일하지 않다. `족보(pedigree)'와 대비되는 두
가지는 다음과 같다. 하나는 \emph{내용(substance)}과의 대비이고, 다른
하나는 \emph{도덕성(morality)}과의 대비다. 이 혼동은 `실질적
정의(substantive justice)' 개념에서 보았듯이,
`실질적(substantive)'이라는 표현이 `정당화된(justified)' 혹은
`도덕적인(moral)'이라는 뜻으로 자주 사용되는 데서 비롯된다. 하트는
이러한 문제들을, 오스틴(Austin)의 견해---모든 법체계는 법적으로 무제한의
권력을 가진다는 주장---를 비판하면서 본문에서 제기한다. 오스틴주의자는,
입법자가 법을 제정하는 형식과 방식(manner and form)에 한해서는 제약을
받을 수 있다고 인정할 수 있을 것이다. 예컨대 고지와 토론의 요건, 또는
특별다수제 요건 등이 그것이다. 그러나 많은 성문헌법들은 그런 식으로는
설명할 수 없는 제약을 포함한다. 이들은 ``입법권의 범위에서 특정 사안을
완전히 제외함으로써 실질적 제약(substantive limitations)을
부과한다''(68쪽). 이러한 제약은 `단지 도덕적(moral) 또는
관습적(conventional)' 제약이 아니라 `법적' 무능력(legal
disabilities)이다(69쪽).\footnote{여기서 하트는 `도덕적 또는
  관습적(moral or conventional)'이라는 표현으로, 오스틴(Austin)이 법적
  권한에 대한 `실정적 도덕(positive morality)'의 비법적(non-legal)
  제약이라 부른 것과, 다이시(Dicey)가 말한 헌법적 `관례(conventions)'를
  모두 포괄하고자 한다.} 하트는 그 예로 미국 헌법 제16차 수정조항을
드는데, 이는 일부 내용으로 다음을 요구한다: ``개인 인두세(capitation
tax) 또는 기타 직접세(direct tax)는, 이 조항에 명시된 인구 조사 결과에
비례하지 않는 한, 부과될 수 없다.'' 이는 분명히 형식 요건이 아니므로, 한
측면에서 보면 실질적이다(substantive). 하지만 하트는 또한 다음과 같이
말한다: ``일부 법체계, 예컨대 미국에서는, 유효성의 궁극적 기준이
족보(pedigree)뿐 아니라 정의의 원칙이나 실질적 도덕 가치(substantive
moral values)를 명시적으로 포함할 수 있으며, 이러한 것들이 법적 헌법
제약의 내용을 구성할 수 있다''(247쪽). 이것은 앞서 언급한 `실질적'
개념과는 다소 다른 의미의 것이다. 다음은 가능한 승인 규칙의 조항들을
비교한 것이다:

(S1) 의회는 국교를 설립하는 어떤 법률도 제정할 수 없다.

(S2) 의회는 불공정한 법률을 제정할 수 없다.

(P1) 의회는 은밀하게 법률을 제정할 수 없다.

(P2) 의회는 불공정한 방식으로 법률을 제정할 수 없다.

규칙 (S1)과 (S2)는 유효성의 실질적 기준(substantive criteria of
validity)을 설정하고, (P1)과 (P2)는 절차적 기준(procedural criteria)을
설정한다. 하지만 다른 방식으로도 짝지을 수 있다: (S2)와 (P2)는 도덕적
기준(moral criteria)을 설정하고, (S1)과 (P1)은 사실적 기준(factual
criteria)을 설정한다. (이것이 반드시 명확히 결정가능한 기준을 설정한다는
뜻은 아니지만, 법률이 종교를 설립하는지 혹은 비밀리에 제정되었는지는
각각의 행위가 도덕적으로 타당한지 여부를 판단하지 않고도 확인할 수
있다.) 하지만 여기서 물어야 한다: (P2)는 또한 \emph{족보}의 테스트(a
test of \emph{pedigree})를 진술하는가? 그것은, 해당 기준을 적용하려면
제정된 법의 내용이 아니라, 제정 방식을 살펴보아야 한다는 의미에서는 족보
기준이다. 하지만 그것(P2)이 충족되었는지 \emph{여부(whether)}를 알기
위해서는 제정 과정의 \emph{도덕적 적정성(moral propriety)}에 대한 판단이
필요하다는 점에서는 족보 기준이 아니다. 따라서 `족보(pedigree)'라는
은유는 오해를 불러올 가능성이 크며, 우리는 이 개념을 버리고 사회적
사실(social fact)과 도덕 판단이 필요한 사안(moral judgement) 사이의
구분을 택해야 한다.

이 문제는 다음의 세 가지로 인해 더 복잡해진다. 첫 번째는
동음이의(homonymy) 문제다. 헌법이나 기타 문서에서 도덕적으로 들리는
\emph{용어들(terms)}이 있다고 해서, 그것이 곧 법에 대한 도덕적 기준이
존재한다는 것을 시사하지는 않는다. 어떤 용어들은 법적 맥락에서, 관습적
도덕(customary morality)과 사법적 결정들(judicial decisions), 그리고
법적 해석의 전통에 의해 특수한 의미를 지니게 되거나 이미 그러한 의미를
가진다.\footnote{언제나 그렇듯, `통제됨(controlled)'이란 `완전히
  통제됨'을 의미하지는 않는다.} 예를 들어 ``모든 개인은 법 앞에
평등하다''라는 헌법 조항이 도덕적 평등이라는 이상을 지향하고 있는지
여부는 법원이 그것을 어떻게 해석하고 적용하는지를 보아야 알 수 있다.
그리고 헌법 조항이 애초에 도덕적 이상을 표방했다 하더라도, 이후의 사법적
결정들은 그것에 사실적 불평등의 전형(factual paradigms)이나 복합적
기준(multi-prong doctrinal tests)을 덧붙이면서 점점 그것과 멀어질 수
있다. 이는 통상적인 법적 근거(source-based)의 유효성 판단이며, 그 원래의
추상적 도덕 이상이 여전히 작동하고 있는지조차 불분명할 수 있다.

두 번째 복잡성은 속존(supervenience)의 문제다. 예컨대 어떤 것이
`불공정하다(unfair)'고 말할 수 있다면, 그것이 불공정한 \emph{근거가
되는(in virtue of which)} 사실들이 있을 것이며, 두 사건이 하나는
불공정하고 다른 하나는 공정할 수 있으려면, 양자 사이에 사실적 차이가
반드시 존재해야 한다는 것은 타당한 주장이다. 이제 어떤 헌법이 차별적인
법률을 금지한다고 하자. 여기서 `차별(discrimination)'이 도덕적으로
잘못된 것으로 이해된다면, 그것이 잘못된 이유가 일반적인 사회적
사실들---예컨대 결정을 내릴 당시의 의도, 결정이 다양한 사람들에게 미치는
상대적 영향 등---이라면, 이러한 사실들은 어떤 입장에서도 법 판단 기준의
일부가 될 수 있다. 배타적 법실증주의는 법 판단의 최종 기준에, 어떤 것이
공정하거나 불공정하다고 판단하는 데 사용되는 사실들이 포함될 수 없다고
주장하지 않는다. 그것은 오히려, 해당 사실들을 판단하기 위해 도덕적
속성에 대한 견해에 의존하지 않고도 해당 사실을 인지할 수 있어야 한다는
것만을 요구한다.

세 번째 복잡성은 불확실성(uncertainty)이다. 하트는 입법자가 어떤 산업에
대해 `공정한 요금(fair rate)'만을 부과하도록 요구하는 사례를
고려한다(131--2쪽). 물론, 입법자가 명확히 고려한 극단적인 불공정
사례---예컨대 ``공공에게 필수 서비스를 인질로 잡고 과도한 요금을
부과하는 경우''(131쪽)---는 분명히 포함될 수 있다. 하지만 사전에
구체적으로 명시하기 어렵고, 명시하는 것이 바람직하지도 않은 수많은
사례들이 존재한다:

\begin{quote}
이 경우들에서, 규칙을 제정하는 권한은 반드시 재량(discretion)을 행사해야
하며, 다양한 사례에서 제기되는 문제를 단 하나의 고유한 정답이 존재하는
것처럼 다룰 수는 없다. 오히려 그 해답은 많은 상충하는 이해관계들
사이에서의 합리적인 타협일 것이다(131--2쪽).
\end{quote}

적어도 `공정성(fairness)'이 불확실한 한, 해당 용어가 법률에 언급될 경우
법원은 재량(discretion)을 행사해야 한다. `공정한 요금(fair rate)'을
요구하는 입법자는 이러한 점을 인지하고 있을 것이며, 따라서 법원에
무제한적 권한이 아니라, 무엇이 불공정한지를 판단할 권한, 그리고 그
판단이 구속력(binding)을 가지도록 하는 권한을 부여한 것으로 이해되어야
한다. 그러나 Postscript에서는 상황이 다르게 묘사된다. 이곳에서
`적법절차(due process)'나 `평등(equality)' 등의 헌법 조항은 도덕
원칙(moral principles)이 법에 편입된 사례로 제시된다. 그런데 이러한
조항들이 재량을 부여하는지 여부는, 이제는 그 불확정성의 정도가 아니라,
해당 조항과 관련된 분쟁을 해결하는 데 필요한 도덕적 판단이 `객관적
지위를 갖는지(objective standing)' 여부에 따라 달려 있다고
설명된다(253--4쪽). 도덕적 판단이 객관적이라면, 불확실성을 해소하는
판결은 도덕 기준에 근거한 기존의 법(pre-existing law)을 적용하는 것에
불과하다. 반면 도덕적 판단이 객관적이지 않다면, 그러한 조항은 ``법원에게
도덕에 따라 법을 \emph{창출하라}(\emph{make} law)는 지시를 내리는 것일
뿐이다''(254쪽). 하트는 법철학이 논쟁적인 형이상학적
윤리이론(meta-ethical theories)에 대한 입장을 가지는 것을 꺼려하므로, 이
문제는 열어둔 채로 남겨둔다.

만약 이 설명이 헌법 내 도덕적 언어가 일반 법률 내 도덕적 언어와 다르게
기능한다고 암시한다면, 이는 설득력이 떨어진다. 성문헌법(written
constitution)은 결국 특수한 형태의 법률일 뿐이다. 왜 일반 법률에서는
`공정성(fairness)'이라는 용어가 요구사항이 불확실할 경우 적어도 재량권을
부여한다고 보면서, 헌법에서는 예컨대 `근본적 정의(fundamental
justice)'라는 용어가 재량권을 부여하는 것은 단지 그 요구사항이 불확실할
뿐 아니라, \emph{또한(and)} 그 불확실성에 대한 도덕적 판단이
`객관적이지(objective)' 않을 때에만 해당된다고 보는가? 해당 문제에 대한
해명이 필요하다는 사실만으로 충분하지 않은가? 해당 판단의 기준이
무엇이든 간에 말이다. 헌법적 기본 원칙에 관한 문제에서는 단순 법률의
경우보다 재량이 더 문제적으로 보일 수는 있겠지만, 그렇다고 재량이
부재한다(absent)는 것을 의미하지는 않는다. 게다가 재량은 이분법적 개념이
아니다. 사실 기반 기준(fact-based criteria)은 재량적 판단에 대해
부분적으로(partial) 영향을 미칠 수 있다. 이는 켈젠(Kelsen)의 은유처럼,
단지 판단이 적합해야 하는 `틀(frame)'을 제공하는 데 그치지 않고, 특정한
종류의 결정에 대해 비결정적(non-conclusive) 이유를 제공하는 방식으로도
작용할 수 있다.\footnote{Hans Kelsen, \emph{Pure Theory of Law} 350--1.}

하트가 자신의 입장을 옹호하는 가장 근접한 논변은 다음과 같다. 그는
`도덕적 정당성(moral propriety)'을 법의 판단 기준으로 삼는 가상의 헌법을
상정한다. 즉, 어떤 것이 잘못되었거나 부정의하거나 불공정한 경우, 그것은
결코 법으로 간주될 수 없다는 것이다. 그는 이러한 헌법 구조에 대해
`논리적으로 불가능할 것은 없다'고 말한다. ``이러한 비범한 장치에 대한
반대는 `논리(logic)' 때문이 아니라, 그러한 법적 유효성(validity)
기준들의 지나친 불명확성(gross indeterminacy) 때문일 것이다. 실제 헌법은
이런 형태로 문제를 자초하지는 않는다.''\footnote{H. L. A. Hart,
  \emph{Essays in Jurisprudence and Philosophy} 361.} 그러나
`논리(logic)'에 개념적 논변(conceptual argument)이 포함된다면, 이러한
구조에 대해 반대할 논변이 존재할 수 있다. 조셉 라즈(Joseph Raz)는, 이는
단순히 문제를 자초하는 정도가 아니라, 법이 그 스스로 주장하는 유형의
권위(authority)를 가질 수 없게 된다고 주장한다.\footnote{Joseph Raz,
  \emph{The Morality of Freedom} (Oxford University Press, 1986),
  chap.~2; \emph{Ethics in the Public Domain}, chap.~10; and
  \emph{Between Authority and Interpretation: On the Theory of Law and
  Practical Reason} (Oxford University Press, 2009), chap.~5.} 모든 법은
정당한 권위를 주장하고 있으며, 그것이 정당한 권위를 가질 수 있는 종류의
것일 때에만 그 주장은 일관성을 가질 수 있다. 법의 주장은
공허하거나(hollow), 불성실하거나, 부정의하거나, 지혜롭지 못할 수는
있지만, 이해가능한(intelligible) 것이어야 한다. 실천적 권위(practical
authorities)---법체계 역시 포함된다---의 역할은 사람들이 그들이 마땅히
해야 할 일을 따르도록 돕는 것이다.\footnote{이는 단지 그들이
  \emph{이기적(selfish)} 이익에 부합하는 행동이거나, 자신이
  \emph{이유(reason)} 있다고 \emph{생각(think)} 하는 행동이 아니라,
  그들이 `객관적으로(objective)' 이유가 있는 행위를 의미한다.} 권위는
오직 그들의 명령(directives)이 그 이유(reason)에 기초할 때, 그리고
사람들이 그 명령을 따르려는 시도를 통해 그들에게 적용되는 이유(reason)에
더 잘 부합할 가능성이 있을 때만 그런 기능을 수행할 수 있다. 그리고 이는
오직 해당 명령이 그 이유 자체를 참조하지 않고도 식별될 수 있어야 가능한
일이다. 즉, 어떤 사람이 법이 무엇을 요구하는지를 알기 위해 먼저 자기가
무엇을 해야 하는지를 \emph{파악(figure out)}해야만 한다면, 법은 그에게
어떤 도움도 줄 수 없다. 법률이 ``서로 공정하게 대하라''고 명령하는
것만으로는 더 공정한 결과를 확보할 수 없고, 헌법이 ``모든 사람은
평등하다''고 선언하는 것만으로 더 평등한 사회를 실현할 수 없다. 권위
있는 지침(authoritative direction)을 제공하려면, 법률과 헌법은 실제로
해당 개념들이 무엇을 요구하는지를 사람들에게 명시해야 한다. 이것이
라즈의 논변이다. 이는 분석법철학(analytical jurisprudence)에서 가장 많이
논의된 주장 중 하나이다. 이 논증에는 여러 단계가 있으며, 그 중 일부는
논쟁적이다. 그러나 이것은 하트가 `존재하지 않는다'고 선언한 유형의
논증이며, 우리가 하트의 견해를 받아들이기 전에 반드시 평가되어야 할
것이다.\footnote{게다가 이후의 한 논문에서 하트 자신도 이 입장의 몇 가지
  핵심 개념을 수용한다. 참고: H. L. A. Hart, 「Commands and
  Authoritative Legal Reasons」, \emph{Essays on Bentham} (Oxford
  University Press, 1982).}

포괄적 법실증주의(inclusive legal positivism)는 도덕성(morality)이 법적
논증에서 적절한 역할을 하려면, 초대받아야 한다(invited)는 점을 시사하는
것으로 보인다. 하트는 유효성의 궁극적 기준(ultimate criteria of
validity)을 통해 그러한 초대(invitation)를 발행한다.\footnote{토니
  오노레(Tony Honoré)는 법이 제기하는 도덕적 주장(moral claims)을 통해
  그러한 요청(invitation)이 이루어진다고 본다. 참고: 「The Necessary
  Connection between Law and Morality」, (2002) \emph{Oxford Journal of
  Legal Studies} 22:489.} 그러나 과연 초대가 정말 필요한가?\footnote{See
  Joseph Raz, \emph{Between Authority and Interpretation}, chap.~7.}
누구도 법률이 논리적 추론(logical inference)이나 산술(arithmetic)의
원리에 의존할 수 있는 것은 법이 그러한 원리를 특별히 인정했기 때문이라고
생각하지 않는다. 누구도 영어 문법의 규칙이 영국 법의 승인 규칙(rule of
recognition)에 포함되어야 한다고 주장하지 않는다. 그렇다면 어쩌면 초대는
필요하지 않은지도 모른다. 그리고 도덕 원칙 역시 이러한 다른 기준들처럼
이미 법원에 자연스럽게 자리 잡고 있을 수 있다. 만약 그렇다면, 하트는
존재하지 않는 문제에 대한 해법을 제안하고 있는 것일지도 모른다.

\subsection{\texorpdfstring{\textbf{5. 사실(fact), 가치(value),
방법(method)}}{5. 사실(fact), 가치(value), 방법(method)}}\label{uxc0acuxc2e4fact-uxac00uxce58value-uxbc29uxbc95method}

하트(Hart)는 (Preface)에서 이 책의 목적이 ``법(law), 강제(coercion),
도덕(morality)을 서로 다른 것이지만 관련된 사회적 현상(social
phenomena)으로 이해하는 데 기여''하는 것이라 말한다. 그는 법학자들이 이
책을 ``법철학의 분석적 접근(analytical jurisprudence)''으로 간주할
것이라 예측하면서, 이는 ``법이나 법정책에 대한 비판이 아니라 법사고의
일반적 틀을 명료화(clarification)하는 데 관심이 있기 때문''이라고 한다.
그는 또한 이 책을 ``기술적 사회학(descriptive sociology)에 대한
에세이로도 생각할 수 있다''고 말한다 (vi). 우리는 이 발언들을 어떻게
해석해야 할까?

\subsubsection{\texorpdfstring{(i) \emph{법철학(jurisprudence)과
사회학(sociology)}}{(i) 법철학(jurisprudence)과 사회학(sociology)}}\label{i-uxbc95uxcca0uxd559jurisprudenceuxacfc-uxc0acuxd68cuxd559sociology}

현장조사도, 통계모형도, 심지어 법 사례조차 거의 제시하지 않는 이 책을
사회학이라 부르기는 어렵다. 하트는 훗날 되돌아보며 이 책이 사회학의 한
\emph{종류(kind)} 라기보다는, 사회학을 위한 일종의
준비작업(preparatory)이라 말했어야 했다고 한다\footnote{David Sugarman,
  `Hart Interviewed: H. L. A. Hart in Conversation with David Sugarman'
  (2005) 32 \emph{Journal of Law and Society} 267, 291.}. 이 다섯 단어는
불필요한 논란을 일으켜왔다. 그러나 그의 요지는 단순하다. 하트의 방법론은
기술적 사회학의 방법들과 \emph{마찬가지로(like)}, 사실(facts)에 대해
책임을 지되, 그것들에 대해 도덕적이거나 정치적인 입장을 취하지 않는다는
점을 강조하는 것이다. 그는 법철학의 주장을 검토할 때, 법 세계에 대한
지식 있는 사람의 이해에 의거해 살펴보라고 독자에게 거듭 요청한다. 정말
법규칙(legal rules)이라는 것이 존재하는가? 직접 보라(136). 모든 법체계가
무제한 권력을 가진 주권자(sovereign)를 전제하는가? 직접 보라.
\emph{선험적(a priori)} 이론과 법에 대한 보통의 지식 사이에 충돌이
있다면, 후자를 따라야 한다. 그것은 수정이 필요할 수도 있고, 일정한
규율이 필요할 수도 있지만, 출발점으로 삼아야 한다. 이 책의 경험적
기반(empirical basis)은 그보다 정교하지 않지만, 기술적 사회학처럼 경험적
기반을 갖는다. 정의(definitions)나 공리(axioms)로부터 시작하여 법에 대한
필연적 진리들을 도출하려 하지 않는다. 법이 어떻게 되어야 하는가에 대한
도덕적 명제에서 출발하여 법이 실제로 어떠한가에 대한 결론을 이끌어내려
하지도 않는다.

기술적 사회학은 양적, 질적 측면에서 일상적 지식을 넘어선
관찰(observations)을 시도한다. 그것은 종종 이 관찰들을 일반화하거나 더
야심차게는 예측으로 통합하려 한다. 하지만 하트는 그러지 않는다. 그는
우리가 이미 알고 있는 기본적인 것들만을 다룬다. 그는 법의 본성을
설명하는 이론을 제시하는데, 그것은 기존의 영향력 있는 이론들과 달리
이러한 사실들과 일관된 방식으로 작동한다고 주장한다. 그의 이론은
법사회학적 데이터를 대규모로 수집하여 얻어진 일반화 집합이 아니며,
어떠한 예측도 시도하지 않는다. 그렇다면 그것이 우리가 이미 알고 있는
것에 어떤 가치를 더하는가? 그것을 더 깊이 이해하게 함으로써이다. 그것은
보통의 사실들 사이의 놀라운 관계들, 그 사실들이 전제하는 미처 인식되지
않은 조건들, 그리고 특히 특정 사실들이 지니는 더 넓은 함의를 보여준다.
물론 설명(explanation)과 이해(understanding) 사이에는 명확한 경계가
없다. 법사회학과 법철학의 풍부한 연구들은 둘 다를 수행한다. 이해란 법에
대한 설명적 혹은 예측적 탐구의 경쟁물이 아니다. 또한 그것은 누가 무엇을
어떻게 연구해야 하는지를 규율하는 허가제도도 아니다. 그것은 그 자체의
고유한 활동이다.

이것은 분석적 법철학(analytic jurisprudence)과 법사회학이 대체로
병행적으로 운영될 수 있음을 시사한다. 그렇다면 하트가 언급한, 전자가
때로는 후자에 대한 준비작업이 될 수 있다는 주장은 어떤 근거를 가질 수
있을까? 켈젠(Kelsen)은 ``사회학적 법학(sociological jurisprudence)은 법
개념을 전제로 한다. 그 법 개념은 규범적 법학(normative jurisprudence)에
의해 정의된다''고 말했다\footnote{Hans Kelsen, \emph{General Theory of
  Law and State} 178. `Normative' here means `having to do with norms',
  \emph{not} `morally appraisive'.}. 그의 기본 생각은, 사회학이 법
제도나 실천을 연구하고 일반화하려면 \emph{그것들을 확보하고(with them)}
시작해야 한다는 것이다. 권한 부여 규칙(power-conferring rules)이
사회적으로 어떤 영향을 미치는지에 관심이 있다면, 무엇이 그러한 규칙인지
먼저 판단할 수 있어야 한다. 이를 위해서는 `규범적 법철학'이 필요하다. 또
다른 예로, 사회학자들이 법의 지배(rule of law)의 실증적 지표(empirical
measures)를 개발하고자 한다면, 그것들이 유효한(valid) 측정치인지
판단하려면 법의 지배란 무엇인지부터 알아야 한다. 내가 보기엔, 주요
지표들 중 어느 것도 특정 관할권의 법이 소급적이지 않거나, 모호하지
않거나, 상호 충돌하지 않는 정도를 측정하려 하지 않는다---그러나 이것들은
모두 법의 지배의 핵심 요건들이다\footnote{For a review some attempts at
  an index, and the problems in constructing one, see Tom Ginsburg,
  `Pitfalls of Measuring the Rule of Law' (2011) 3 \emph{Hague Journal
  on the Rule of Law} 269.}. 그럼에도 몇몇 지표들은 다양한 국가에서 사유
재산의 보장이나 계약 자유의 정도를 측정하고, 그것이 클수록 법의 지배
점수를 높게 준다. 이는 명백한 정치적 편향을 드러내며, 동시에 개념적
혼란을 의미한다. 법의 지배 개념에는 논쟁의 여지가 있지만, 그것을
`자본주의의 전제 조건'으로 여긴다면 해당 개념에 대한 이해는 매우 빈약한
것이다. 이에 대해 적어도 이러한 실증 지표들이 신뢰할 수 있으며, 관련
변수들과 상관관계가 있고, 법과 사회에 대한 다른 지식들과 연속선상에
있다는 반론이 있을 수 있다\footnote{For a spirited defence of a
  social-scientific approach to jurisprudence see Brian Leiter,
  \emph{Naturalizing Jurisprudence: Essays on American Legal Realism and
  Naturalism in Legal Philosophy} (Oxford University Press, 2007).}.
그럼에도 법사회학이 관련 개념이 가리키는 핵심에서 너무 멀어진다면,
그것은 주제를 바꾼 셈이다. 어떤 경우에는 주제를 바꿔야 할 수도
있다---자연과학의 진보는 때때로 존재론을 폐기하고 새로 시작하는 방식으로
이루어졌다. 그러나 사회과학의 진보가 우리가 익숙한 법의 존재론(규칙,
의무, 권한, 법원 등)을 폐기해야 한다면, 그것은 더 이상 `법'의 사회학이
아닐 것이다. 그렇다면 법철학(jurisprudence)은 여전히 우선성을 가진다.

우리가 `법의 법리학적 개념(juristic concept of law)'과 그 관련 개념들을
계속 유지하고자 한다면, 하트가 약속한 깊은 이해는 어떻게 도달할 수
있는가? 어디서 시작해야 하는가? 하트는 종종 우리에게 특정 사례를 어떻게
판단하거나 분류할 것인지 질문함으로써 우리 자신의 일상적 지식을
이끌어내려 하고, 때때로 그는 우리가 무엇이라고 \emph{말할(say)} 것인지를
묻는 방식으로 \emph{그 작업을(that)} 수행한다. 이것이 하트의 법철학을
의미론(semantics)의 한 분야로 만든다고 볼 수 있을까? 하트는 철학의
`언어적 전환(linguistic turn)'에 영향을 받았고, 스스로 그 옹호자라
여겼다\footnote{See Richard Rorty ed., \emph{The Linguistic Turn: Recent
  Essays in Philosophical Method} (University of Chicago Press, 1967).}.
특히 그는 옥스퍼드에서 J. L. 오스틴(J. L. Austin)과 길버트 라일(Gilbert
Ryle)이 전개한 일상 언어철학에 영향을 받았다\footnote{그리고 후기
  비트겐슈타인(later Wittgenstein), 특히 그의 제자 프리드리히
  바이스만(Friedrich Waismann)을 통해 수용된 견해도 참고할 수 있다.
  하트와 그 당시 옥스퍼드 철학자들과의 개인적·지적 관계에 대한 통찰력
  있는 논의는 니콜라 레이시(Nicola Lacey), \emph{A Life of H. L. A.
  Hart: The Nightmare and the Noble Dream} (Oxford University Press,
  2004)을 참조하라.}. 그럼에도 이 책의 연대기적 위치와 철학적 계보를
고려할 때 가장 눈에 띄는 점은 『법의 개념(The Concept of Law)』에
언어분석이 \emph{거의 없다(little)}는 것이다. 언어는 다양한 기능을
가지며, 문장은 맥락을 가지며, 일부 이론들은 개념의 사용기준(criteria)을
제시한다고 말할 수 있다는 점이 언급된다. 몇몇 지점에서 언어적 구별이
강조되긴 한다. (하트는 `강요당하다(be obliged)'와 `의무를 지다(be
obligated)' 사이에는 차이가 있으며, `규칙에 따라 행위한다'는 것과
`규칙을 가지고 있다'는 것 사이에도 차이가 있다고 주장한다.) 하지만
그것뿐이다. 언어철학자들의 이론 구성에 대한 적대감은 보이지 않는다.
`법체계(legal system)'라는 개념이 가족 유사성(family resemblance)
개념이라는 식의 주장도 없다. 오히려 하트는 법체계가 되기 위한
필요충분조건(necessary and sufficient conditions)까지 제시한다! 그는 이
문제나 법철학의 다른 핵심 문제를 단어의 의미에 호소하는 방식으로
접근하지 않는다. 하트는 반복적으로 언어적 접근 방식의 무용함을 경고한다.
```법이란 무엇인가?'라는 질문에 대해 `법'이나 `법체계'라는 단어의 사용에
관한 기존 관습을 상기시키는 것으로 답하는 것은,'' 그는 말한다,
``\,`쓸모없다'\,'' (5). 더 넓은 개념과 좁은 개념 중 어느 것을 택할
것인지를 판단하려면 의미론(semantics) 이상의 것이 필요하다. ``이 문제를
단지 언어 사용의 적절성 문제로 본다면 우리는 이 문제를 충분히 다룰 수
없다''(209). 만약 누군가가 법철학의 방법론(methodology)에 관한 책을
쓰고자 한다면, 이 경고들과 동시에 책 전체에 드러나는 언어철학에 대한
호의적 태도를 조화롭게 해석할 필요가 있을 것이다. 그러나 하트는 어떤
방법론에 관한 책도 쓰고 있지 않다.

이 책의 제목을 포함하여 책 전반에 드러나는 수사적 표현은 언어철학의
색채를 띠고 있음은 부인할 수 없다. 그러나 좋은 사상사가는 표현 형식보다
내용에 주목해야 한다. 철학자가 자신이 무엇을 하고 있다고 생각했는지,
말했는지, 실제로 무엇을 했는지는 서로 다를 수 있다. (데이비드 흄(David
Hume)은 정치학이 과학이 될 수 있다고 말했지만, \emph{논고(Treatise)}나
\emph{탐구(Enquiry)} 어느 곳에서도 단 하나의 실험이나 증명을 제시하지
않았다.) 현대 철학의 수사적 스타일은 하트의 것과 매우 다르며, 하트의
스타일은 다시 벤담(Bentham)의 것과 매우 달랐다. 이러한 차이가 실제
기법의 차이를 얼마나 반영하는지는 분명하지 않다.

\subsubsection{\texorpdfstring{(ii) \emph{명료화(clarification)와
비판(criticism)}}{(ii) 명료화(clarification)와 비판(criticism)}}\label{ii-uxba85uxb8ccuxd654clarificationuxc640-uxbe44uxd310criticism}

이 책은 하트(Hart)가 `일반적이고 기술적인(general and descriptive)'
법이론(legal theory)이라 부르는 유형을 옹호한다 (239--240). 그는 이것이
영국법(English law), 관습법(common law), 자본주의적 법(capitalist law)에
대한 이론이 아니라, \emph{법 그 자체(as such)} 에 대한 이론이라고
말한다. 이 이론은 ``도덕적으로 중립(morally neutral)이며 정당화
목적(justificatory aims)을 지니지 않는다. 즉, 내가 법에 대해 일반적으로
설명하는 형태나 구조에 대해 도덕적 혹은 기타의 근거에서 이를
정당화하거나 칭찬하려는 목적은 없다''고 밝히며, 낙관적으로 다음과 같이
덧붙인다. ``다만 나는 이러한 구조들에 대한 명확한 이해가 법에 대한
유익한 도덕적 비판의 중요한 전제라고 생각한다''(240).

이 책이 정당화의 목적을 결여한다고 해서, 그것이 곧 도덕적으로
중립적이라는 의미는 아니다. 어쩌면 도덕적 편향이 무의식적으로 스며들 수
있다. 일부 사람들은 묘사(description)라는 것이 단순히 해당 대상에 관한
사실(facts)의 목록 이상일 수밖에 없다고 보고, 이런 점에서 이 책에도
도덕적 요소가 개입되었다고 본다. 관찰(observation)은 이론을 전제하며,
관찰에 기반한 묘사는 가치를 전제한다; 따라서 기술적 법철학(descriptive
legal philosophy)은 불가능하다는 것이다. 그러나 이는 성급한 일반화이다.
사실의 명제(statement of fact)는 참(true) 혹은 거짓(false)으로 평가될 수
있다. 반면 묘사는 일반적으로 참이나 거짓으로 평가되지 않고, 유익하거나
무익한(helpful or unhelpful), 통찰력 있는지 여부(illuminating or
unilluminating) 등으로 판단된다. 어떤 대상이나 사태(state of affairs)에
대해서도 가능한 묘사는 무한하다. 왜냐하면 각각에 관한 사실이 무한하기
때문이다. 실제의 묘사는 그에 관한 \emph{모든} 사실의 목록이 아니라, 어떤
이유에서 중요하거나, 두드러지거나, 관련 있거나, 흥미롭다고 여겨지는
사실들의 선별(selection)과 배열(arrangement)이다. 모든 묘사는 특정한
가치들에 의해 전제되거나, 그 가치들의 관점에서 이루어진다. 그러나
이로부터 묘사를 제시하는 사람이 그러한 가치들을 승인하거나, 그것들이
도덕적 가치(moral values)라는 결론이 반드시 도출되지는 않는다. 예컨대
사건관리시스템(case-management system)을 `비효율적(inefficient)'이라
묘사하는 것은 가치중립적이지 않다. 이는 해당 시스템의 어떤 특징에
주목하게 만든다. 그러나 이러한 선별(selection)의 이유가 반드시 화자가
효율성(efficiency)을 가치 있다고 보거나, 더 나아가 가장 중요한 가치라고
보거나, 도덕적 가치라고 생각해서일 필요는 없다. 그는 청중이 그것을
중요한 것으로 본다고 생각하거나, 일반 사람들이 그러하다고 보거나, 해당
시스템을 사용하는 판사들이 그렇게 본다고 추측해서 그렇게 말할 수도 있다.
가능한 경우는 다양하다.

비판자들이 하트의 기술적 기획에 도덕적 요소의 스며듦을 감지한다고 여기는
지점을 하나 살펴보는 것도 유익할 수 있다. 그는 자신의 주된 논증을 허구의
사회발전사(fictional history of social development)를 통해 전개하는
것으로 잘 알려져 있다(91--99). 우리는 `원시적(primitive)' 사회에서
출발한다. 이 사회에서는 사회질서(social order)가 무엇을 해야 하는지에
대한 합의(consensus)를 통해 유지된다. 그러나 사회 변화와 함께 이
합의들은 불확실하게 되고(static), 정태적이며, 비효율적으로 된다. 이
결함들(defects)은 법이 등장하면서 보완되며, 법은 인정규칙(rule of
recognition), 변경규칙(rule of change), 재판규칙(rule of adjudication)과
같은 이차적 규칙들(secondary rules)을 통해 확실성(certainty),
역동성(dynamism), 효율성(efficiency)을 부여한다. 이렇게
`발전된(developed)' 혹은 `복잡한(complex)' 사회에서 우리는 더 나은
조건을 누리게 된다. 이 지점에서 비판자들이 반박에 나선다\footnote{Among
  many pouncers, see Stephen Guest, `Two Strands in Hart's Concept of
  Law' in Stephen Guest ed., \emph{Positivism Today} (Dartmouth, 1996)
  29.}: 하트는 어떻게 법이 `결함'을 치유한다고 말하면서도, 법의 정당성에
대해서는 중립적이라고 할 수 있는가? 그는 어떻게 전(前)법적(pre-legal)
사회를 `원시적'이라고 부르면서도, 그것을 암묵적으로 비난하지 않는다고
말할 수 있는가?

이 문맥에서 `원시적(primitive)'이라는 말은 명백히 단순(simple)하다는
의미이며, 어리석거나 야만적(barbaric)이라는 뜻은 아니다. 그리고 단순한
사회들이 법을 가지지 않는 이유는 그들이 문명화되지 않아서가 아니라, 법이
\emph{필요하지(need)} 않기 때문이다. 인간 본성이나 사회라는 것의 구조
속에 법이 필수라는 요소는 없다; 실제로 많은 사회가 법 없이도 잘
작동해왔다(91). 국제 영역에서도 볼 수 있듯, 고도로 체계화된 규칙 체계는
``필수품이 아니라 사치품''이다(235). 단순한 형태의 사회질서는 실제로 잘
작동한다. 예컨대 사회가 작고 안정적이며, 사회적·이념적으로 통합되어 있을
때 그러하다(91). 그러나 ``이 외의 조건에서는 그러한 단순한 사회통제의
형태는 결함을 드러내며, 다양한 방식으로 보완될 필요가 있다''(92). 여기서
주목할 점은, 첫째로, 문제 삼는 `결함(defect)'은 \emph{도덕적}
악(\emph{moral} vice)이 아니라, 사회 통제 메커니즘의 기능적
결함(functional deficit)이라는 것이다. 이러한 결함이 도덕적으로
유감스러운 것인지 여부는, 그 통제 메커니즘이 선(good)을 향하는가,
악(evil)을 향하는가에 달려 있다. 더 효율적인 사회 통제가 도덕적으로 나쁜
것일 수도 있다. 둘째로, 결함이 있는 것은 단순한 사회 자체라기보다는,
통치 방식과 사회 복잡성 사이의 불일치이다. 결함은 낯선 사람들로 구성된
큰 사회를, 친구들로 이루어진 작은 사회처럼 다스리려 할 때
나타난다---일부 현대 공동체주의(communitarian) 정치이론의 오류처럼.
여기에는 단순한 사회(법이 없는)와 복잡한 사회(법이 있는) 사이에서 후자를
반드시 선호해야 한다는 어떠한 제안도 존재하지 않는다.

그러므로 이 논변은 하트가 추구한 중립성을 포기한 것으로 비난될 수 없다.
물론 하트는 이차적 규칙들의 등장에 주목하지만, 그것은 그가 그것이
중요하고 핵심적인 사실이라고 보기 때문이다. 그러나 그것이 바람직하다고
생각하기 때문은 아니다. 오히려 그는 위에서 본 것처럼, 그것이 도덕적으로
위험하다고 생각한다. 아이러니하게도, 편향은 오히려 하트의 논변을 일종의
근대주의적 승리론(modernist triumphalism)으로 해석하는 사람들 쪽에서
나타난다. 그들은 어떤 사회가 법체계를 결여하고 있다면, 그것은 근대의
성취 중 하나를 결여한 것이라고 본다. 따라서 `원시적' 사회가 문명화되지
않았다고 보는 것은 편협하고 모욕적인 생각이므로, \emph{후건부정법(modus
tollens)}에 따라, 모든 사회는 법체계를 가져야 한다. 그리고 어떠한
법철학이든 이러한 사회들을 법체계들\emph{로써(as)} 간주하지 않는다면,
그것은 근대적 또는 서구적 법에 편향된 것이다. 이러한 논변의 구조는
놀랍다. 그것은 모든 사회가 법을 가져야 한다는 명제로부터 시작하여, 모든
사회가 실제로 법을 가지고 있다는 결론을 도출한다. 이 추론은 명백히
타당하지 않다(invalid). 또한 이는 하트가 거부한 전제를 바탕으로 한다.
그는 법에 대해 특별한 열정을 공유하지 않기 때문에, 그것이 어디에나
존재한다고 믿어야 할 필요를 느끼지 않는다.

이와 유사한 논점은 제10장에서 하트가 국제법(international law)을
체계(system)가 아니라 규칙들의 \emph{집합}(\emph{set} of rules)으로
규정하는 데에도 적용된다. 그는 국제법이 단순한 사회질서와 유사한 점이
있다고 본다. 국제법학자들은 분개한다. 하트는 국제법에 대해 `낮은 점수'를
주며, 그것을 특별한 것으로 대우하기보다는 `이단(outcast)'처럼 취급한다고
본다\footnote{Ian Brownlie, `The Reality and Efficacy of International
  Law' (1981) 52 \emph{British Yearbook of International Law} 1, 7--8.}.
그들은 자신들의 학문분야의 지위에 대해 걱정하는 나머지, 이 장의 핵심이
오히려 국제법을 \emph{옹호하고(defend)} 있다는 점은 거의 눈치채지
못한다---즉, 중앙 주권자나 강제적 관할권이 없다는 이유로 국제법이 법이
아니라고 보는 잘못된 주장으로부터 국제법을 방어하고자 한 것이다. 하트는
국제법이 매우 체계적이라고는 보기 어렵다고 본다. 그는 국제법에서
전체적인 인정규칙(rule of recognition) 같은 것을 발견하지 못한다. 그러나
그가 다른 이차적 규칙들이 존재할 가능성을 배제하는 것은 아니다. 예컨대
어떤 실체(entity)가 국가(state)인지 여부나 조약(treaty)을 채택하기 위한
요건 등에 관한 규칙들이 그것이다. 흥미롭게도, 국제법 자체에 관해 기술할
때(즉, 국제법 이론에 대해 논하지 않고 실제 국제질서에 대해 기술할 때),
국제법학자들 스스로도 그 핵심 규칙들 중 많은 것이 적용
유효성(effectiveness)이 부분적이며, 중요한 규범들에 대한 기준이
불안정하고, 재판제도들이 목적별로 분산되어 있으며, 국제법의 여러 영역이
여전히 이상적 수준에 머물러 있다는 것을 기꺼이 인정한다. 그러나 법의
체계성(systematization)은 본질적으로 정도의 문제(matter of degree)이며,
오늘날 국제법이 1961년보다 더 체계화되어 있다고 보는 것이 타당할 수
있다\footnote{For a review of some of the issues see the editors'
  introduction to Samantha Besson and John Tasioulas eds., \emph{The
  Philosophy of International Law} (Oxford University Press, 2010)
  1--13, and sources therein cited.}. 그렇다고 해서 그것이 국내
법체계(domestic legal systems)와 동등하다고 보기는 어렵다. 하트가 이
점을 지적한다고 해서, 그가 국제법을 비판하거나 베스트팔렌식
국가주의(Westphalian statism)를 찬양하는 것은 아니다. 그는 국제법이
국내법과 유사한 점, 그리고 그렇지 않은 점을 이해하려는 것이다.

이 모든 것이 『법의 개념(The Concept of Law)』이라는 책이 정치적
도덕성(political morality)에 대한 일정한 관점을 전제로 구성되어 있다는
사실을 부인하는 것은 아니다. 이 책에는 하트의 밀적 자유주의(Millian
liberalism)나 민주적 사회주의(democratic socialism)에 대한 직접적인
논의는 많지 않다. 그러나 그의 또 다른 도덕적 신념, 즉 가치
다원주의(value pluralism 의 뚜렷한 흔적이 있다. 그의 친구 아이제이아
벌린(Isaiah Berlin)처럼\footnote{Isaiah Berlin, \emph{Four Essays on
  Liberty} (Oxford University Press, 1969), and also his `On Value
  Pluralism' (1998) \emph{New York Review of Books}, vol.~XLV No.~8.},
하트는 선(good)은 본질적으로 다원적(plural)이며, 실제 세계는 정당한
가치들 사이에서도 충돌이 자주 발생한다고 본다. 하나의 정당한 이익이 또
다른 이익의 희생을 통해 얻어질 수 있다는 것이다. 하트는 따라서
공리주의(utilitarianism)나 복지경제학(welfare economics)에서 나타나는
축소적 일원론(reductive monism), 즉 궁극적으로 추구해야 할 가치는 오직
하나뿐이라는 입장에 반대한다. 동시에, 다양한 가치를 인정하되 이들이
`적절히 해석되면' 결코 충돌하지 않는다고 주장하는 비축소적
일원론(nonreductive monism)에도 반대한다\footnote{이 입장에 대한 가장
  지속적인 옹호는 로널드 드워킨(Ronald Dworkin), \emph{Justice for
  Hedgehogs} (Harvard University Press, 2011)을 보라.}. 그는 ``법과 법의
집행에는 다양한 종류의 탁월성(excellences)이 있을 수도, 없을 수도
있다''고 하며(157), 정의(justice)조차도 그러한 탁월성들 중 하나일
뿐이라고 말한다. 그의 가치 다원주의는 법적 추론에 대한 언급에서도 분명히
드러난다. ``특히 중요한 헌법적 문제에 관한 사법적 판단(judicial
decision)은 종종 도덕적 가치들 간의 선택을 포함하며, 단 하나의 뛰어난
도덕 원칙의 적용만으로 설명되지 않는다''(204). 이 점은 사법적
추론(judicial reasoning)에 깊은 함의를 지닌다. ``이러한 원칙들이 다수
존재할 수 있기 때문에, 특정한 {[}사법적{]} 판단이 유일하게 정당하다는
것을 \emph{입증(demonstrate)}할 수는 없다\ldots{}'' (205).

\subsection{\texorpdfstring{\textbf{6. 요점(THE
POINT)}}{6. 요점(THE POINT)}}\label{uxc694uxc810the-point}

하트(Hart)의 구상에 따라 하나의 일반 법이론(general jurisprudence)이
성공적으로 전개될 수 있다고 가정해 보자. 그렇다면 왜 그런 수고를 들여야
하는가? 이 책의 사상들과 씨름하는 일은 현대 법철학의 방대한 문헌에
입장하기 위한 일종의 입장료가 되어버렸다. 이 책은 과거 철학 저작들에
대한 유익한 논의를 포함하고 있다. 하지만 우리는
법이론서(jurisprudence)를 \emph{다른(other)} 법이론서에 대해 알기 위해
주로 읽어야 한다는 생각에 대해, 그 누구보다 강력히 반대한 인물이 바로
하트 자신이다. 『법의 개념(The Concept of Law)』에는 학술 주석(scholarly
notes)이 거의 없으며, 그나마도 책의 맨 뒤로 밀려 있으며, 그것들은 차라리
나중에 읽으라고 권고한다:

\begin{quote}
나는 이러한 배열이, 법이론에 관한 책이란 기본적으로 다른 책들에 무엇이
들어 있는지를 배우기 위한 책이라는 믿음을 약화시키기를 바란다. 이 믿음을
저술하는 사람이 갖고 있다면, 그 학문은 거의 진전을 이루지 못할 것이며,
독자가 그 믿음을 갖고 있다면, 그 학문의 교육적 가치는 극히 제한된 수준에
머물 것이다 (vii쪽).
\end{quote}

그러나 그러한 장애물을 제거하는 것이 의미 있으려면, 그 학문 영역 자체의
진보가 그 자체로 의미가 있을 경우에만 그렇다. 하트식의 분석적
법철학(analytic legal philosophy)은 특정 목적에 국한되지 않는 일반적
가치(non-specific value)를 지닌다: 그것은 면밀한 독해(close reading),
명료한 사고(clear thinking), 정밀한 논증(careful argument)을 중시하고
증진시킨다. 이러한 능력들은 유용한 기술이며, 교육적으로도 가치가 있다.
그러나 이를 습득할 방법은 많고, 굳이 법이론 연구가 그 가장 빠르거나 쉬운
길이라고 볼 이유는 없다.

일반 법이론에 대한 진정한 변론(apology)은 다음과 같은 주장이다: 법은
중요한 사회 제도(social institution)이며, 이와 같은 제도에 대한 심층적
이해는 그 자체로 가치가 있다. 우리는 법의 본성을 이해하고자 한다. 그것은
시장(market)이나 가족(family)의 본성을 이해하고자 하는 이유와 같다.
그렇다고 해서 우리가 시장이나 가족보다 법에 집중해야 한다는 주장이
성립하는 것은 아니며, 더 나아가 법을 더 깊이 이해하는 일이 법 관련
행동을 예측하거나 법과 제도를 개혁하는 일보다 더 중요하다는 것을
입증하는 것도 아니다. 그러나 이러한 대안들 중 어떤 것도 법이론만의
고유한 가치를 약화시키지는 않는다.

어떤 사람들은 법이론서를 읽으며 삶과 법에 대한 지침을 기대한다. 어떤
사람들은 그 지침을 제공하고자 법이론서를 쓴다. (마키아벨리처럼, 군주의
귀를 사로잡기 위해 쓰는 경우도 포함해서.) 만약 이것이 독자의 목적이라면,
일반 법이론은 실망을 안겨줄 수도 있다. 그것이 일종의 고급 `법정
조언서(high-brow \emph{amicus curiae} brief)'로서 영향력을 행사한 예는
거의 없다. 물론 하나의 법정 판결(court case)이 법이론적
명제(jurisprudential theses)를 제기하거나 의존하는 경우는 있을 수 있다.
실제로 어떤 판결에서는 판사들이 법의 유효성(validity)과 실효성(efficacy)
사이의 관계, 벌금(penalty)과 세금(tax)의 차이, 혹은 법과 관습(custom)의
차이 등에 대해 의견을 제시한다. 그러나 그런 경우에도 법원은 지역적
교의적 제약(local doctrinal constraints) 내에서 해답을 모색하려 하며, 이
제약은 법철학이 반드시 따라야 할 것은 아니다. 우리는 사회
관습질서(customary social order)와 법체계(legal system)를 구별할 수 있는
충분한 이론적 근거를 가지고 있다. 그러나 특정한 법체계가 원주민
관습(aboriginal custom)을 하나의 독립된 법체계로 간주할 수도 있다.
그렇다면 그것이 하트의 이론을 반박하는가? 전혀 그렇지 않다.
「어업법(Fisheries Act)」이 포경(whaling)을 규제한다고 해서
고래(whale)가 물고기(fish)가 아니라는 확신이 흔들릴 이유는 없다.
(그렇다고 해서 어업법에 잘못이 있다는 뜻도 아니다.)

역사적으로 볼 때, 『법의 개념』은 영미권에서 정치철학(political
philosophy)이 부흥하던 전환기에 위치한 저작이다. 초판 출간 이후, 학계의
관심은 비교적 빠르게 가치 문제(value questions)로 옮겨갔다. 다음 세대의
주요 저작들은 정의(justice), 자유(liberty), 평등(equality)과 같은
개념들에 초점을 맞추었고, 법 제도나 구조에 대한 관심은
줄어들었다\footnote{이 주제가 응용 도덕철학(applied moral philosophy)의
  방향으로 지나치게 나아갔다는 제안은 제레미 월드론(Jeremy Waldron),
  「\emph{Political} Political Theory: An Oxford Inaugural Lecture」,
  \url{http://ssrn.com/abstract=2060344}에서 제시된다.}. 하트의 책은
평가적 문제들(evaluative issues)을 곁눈질하듯 살펴보긴 하지만---실제로
정의에 관해서도 일정한 논의가 있다---전반적으로 그것은 제도와 구조에
대한 책이다. 그리고 이러한 이유 때문에, 또 기술적 접근방식을 일관되게
유지하려는 태도 때문에, 이 책은 흔히 논변가들(advocates)의 단골 비난을
받는다: ``불모적이다(sterile)'' 혹은 ``지루하다(boring)''는 것이다.
이러한 비난은 어떻게 이해해야 할까? 그것은 지적 편협성(intellectual
narrowness)을 드러낼 뿐 아니라, 자기 성찰의 부족도 나타낸다. 법이
\emph{의심의 여지 없이(unquestionably)} 흥미롭다고 생각하고, 다른
사안들도 어떤 소송의 가능성과 관련된 방식으로 설명되어야만 흥미롭게
여겨진다는 것은, 일종의 법률가적 자만이다---(이는 어떤 질문이
\emph{진정(really)} 흥미로운 질문이 되기 위해서는 고객이 그 해답에 돈을
지불할 의사가 있어야 한다는 생각과 크게 다르지 않다.)

로널드 드워킨(Ronald Dworkin)은 ``법철학(jurisprudence)은 중요하다.
왜냐하면 판사들이 어떻게 판결을 내리는지가 중요하기 때문이다''라고
썼다\footnote{Ronald Dworkin, \emph{Law's Empire} 1.}. 그는 훨씬
이전에도 다음과 같이 썼다: ``법원이 내리는 결정에 대해 일반적으로 좋은
이유란 무엇인가? 이것이야말로 법철학의 \emph{그(the)}
질문이다\ldots{}''\footnote{Ronald Dworkin, `Does Law Have a Function? A
  Comment on the Two-Level Theory of Decision' (1965) 74 \emph{Yale Law
  Journal} 640.}. 하트는 단지 가치 다원주의자(value pluralist)일 뿐
아니라, 지적 다원주의자(intellectual pluralist)이기도 하다. 그는 결론을
향해 직행하는 경쟁에 참여하지 않는다. 이 책의 제1장은 `지속되는
질문들(Persistent Questions)'이라는 제목이며, 복수형이다. 그 질문들은
다음과 같다: 법은 강제적 위협(coercive threats)과 어떤 관계에 있는가?
법의 의무적 강제력(obligatory force)은 어떤 의미를 가지며, 그것은 도덕적
의무(moral obligation)와 어떤 관계를 가지는가? 사회 규칙(social
rules)이란 무엇이며, 법은 어떤 방식으로 규칙에 관한 것인가? 그러나 이들
중 어느 것도 법철학의 \emph{그(the)} 질문이 아니다. 이들은 모두 법철학의
\emph{질문들(questions)}일 뿐이다. 그렇다면 이 질문들은 흥미로운가? 물론
흥미롭다. 그러나 단순한 진실은, 사람들은 서로 다른 것에 흥미를 느낀다는
것이다. 데이비드 흄(David Hume)은 자신의 삶을 회고하며 이렇게 적었다:
``나의 학구적 성향, 절제된 생활, 근면한 태도 때문에 가족은 법률이 내게
적합한 직업이라고 생각했다. 그러나 나는 철학과 일반 학문 외에는 도저히
흥미를 느끼지 못했다\ldots{}''\footnote{David Hume, `My Own Life' in his
  \emph{Essays}, \emph{Moral, Political and Literary}, E. F. Miller,
  ed.~(Liberty Press, 1987).}. 흄은 법을 도저히 견딜 수 없을 만큼
지루하게 여겼다. 그는 혼자가 아니다. 분명 그 반대의 기호를 가진 이들도
있다. 그러나 지적 취향이 어떻든 간에, 누구나 깨달을 수 있는 점은 다음과
같다---그리고 흄도 분명히 깨달았던 점이다: `흥미롭다'는 것은 대상 그
자체의 성질이 아니라, 대상과 한 개인 사이의 관계 속에서 발생하는
성질이다. 하트의 법철학은 소송과의 관련성은 제한적일 수 있지만, 그 문제
자체의 본질적 중요성을 인식하는 사람들, 혹은 더 넓은 정치적 관심을 지닌
사람들에게는 분명 흥미를 유발할 수 있다. 법정에서는 `\emph{당해 사안의
법(the law)}이 무엇인가'를 아는 것이 `\emph{법(law)}이 무엇인가'를 아는
것보다 중요하다. 그러나 법정 밖에서는, 하트가 상기시키듯이 실제의 법적
실천(law-in-action)의 대부분이 이루어지는 곳이며, 이때 법의 본질에 대한
질문이 제자리를 찾는다. 우리는 법을 통치 수단(governance)으로 삼아
무엇을 기대할 수 있는가? 그것의 이점은 무엇이며 위험은 무엇인가? 법의
지배(rule of law)를 왜 가치 있게 여겨야 하는가? 우리는 법에 복종해야
하는가? 만약 그렇다면, 어디까지 복종해야 하는가? 그리고 물론, 우리는 왜
법을 가져야 하는가? 이런 질문을 던지는 사람, 혹은 더 이상 이런 질문을
피할 수 없다고 느끼는 사람이라면 누구든, 그 해답을 찾기 위해 일반
법이론의 도움을 필요로 하게 된다. 하트가 그 작업을 이렇게 훌륭하게
시작해 두었다는 것은 우리에게 정말로 행운이다.  % For Pandoc

\end{document}
