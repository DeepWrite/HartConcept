\documentclass[12pt, oneside]{book}  % 13pt는 직접 설정함

% --- Language and Font Settings ---
\usepackage[english]{babel}
\usepackage{fontspec}
\usepackage{kotex}  % Korean typesetting

% --- Font Configuration (Main + Korean) ---
\usepackage{libertinus}
\setmainhangulfont[
  Path = ./,
  UprightFont = *Batang Light.ttf,
  BoldFont    = *Batang Medium.ttf
]{KoPubWorld}

% --- Page Geometry: A4 with wide margins for print readability ---
\usepackage[a4paper, margin=1in]{geometry}

% --- Line Spacing ---
\usepackage{setspace}
\setstretch{1.45}

% --- Section Formatting: Disable numbering ---
\usepackage{titlesec}
\setcounter{secnumdepth}{0}

\titleformat{\chapter}[display]
  {\normalfont\Huge\bfseries}
  {}{0pt}{\Huge}\titleformat{\section}
  {\normalfont\Large\bfseries}
  {}{0pt}{\Large}
\titleformat{\subsection}
  {\normalfont\large\bfseries}
  {}{0pt}{\large}

% --- Math and Tables ---
\usepackage{amsmath, amssymb, amsthm, mathtools}
\usepackage{graphicx}
\usepackage{enumitem}
\usepackage{tabularx, booktabs}
\usepackage{footmisc}

% --- Quotes, Hyperlink, Header/Footer ---
\usepackage{csquotes}
\usepackage[hidelinks]{hyperref}
\usepackage{fancyhdr}
\pagestyle{fancy}
\fancyhf{}
\fancyfoot[C]{\thepage}
\setlength{\headheight}{15pt}

% --- Bibliography (biblatex) ---
\usepackage[style=verbose-note, backend=biber, maxbibnames=99]{biblatex}
\addbibresource{references.bib}
\AtEveryBibitem{\clearfield{pages}}

% --- Title Info ---
\title{\Huge\textsc{The Concept of Law} \\[2ex] \Large Third Edition}
\author{\Large H. L. A. Hart}
\date{}

\begin{document}

% --- Title Page ---
\begin{titlepage}
  \centering
  \vspace*{3cm}
  {\Huge\textsc{The Concept of Law}}\\[1.5ex]
  {\Large Third Edition}\\[4ex]
  \textsc{H. L. A. Hart}\\[6ex]
  {\small Oxford University Press\\
  2025 (Reprint Typeset Edition)}
  \vfill
\end{titlepage}

\section{\texorpdfstring{\textbf{I. 지속되는
질문들}}{I. 지속되는 질문들}}\label{i.-uxc9c0uxc18duxb418uxb294-uxc9c8uxbb38uxb4e4}

\subsection{\texorpdfstring{\textbf{1. 법이론의
난제들}}{1. 법이론의 난제들}}\label{uxbc95uxc774uxb860uxc758-uxb09cuxc81cuxb4e4}

`법(law)이란 무엇인가?'라는 질문만큼 오랫동안 제기되어 왔고, 진지한
사상가들에 의해 이렇게 다양하고 기묘하며 때로는 역설적인 방식으로
답변되어 온 인간 사회에 관한 질문은 거의 없다. 고전 및 중세의 법
'본성(nature)'에 관한 사유를 제쳐두고, 지난 150년 동안의 법이론만을
살펴보더라도, 우리는 다른 어떤 독립된 학문 분야에서도 찾아볼 수 없는
독특한 상황을 발견하게 된다. '화학이란 무엇인가?' 또는 '의학이란
무엇인가?'라는 질문에 대해 이처럼 방대한 문헌이 헌신된 예는 없다. 이러한
과학들을 배우는 학생은 교과서의 첫 페이지에 나오는 몇 줄의 설명 정도만
접할 뿐이며, 거기서 제시되는 답변은 법학도에게 주어지는 답변과는 매우
다르다. 그 누구도 의학이란 '의사가 병에 대해 하는 일'이라거나 '의사가
무엇을 할 것인지를 예측하는 것'이라고 굳이 주장하지 않는다. 또한
일반적으로 화학의 핵심적이고 특징적인 일부로 간주되는, 예컨대 산(acid)에
대한 연구가 사실은 화학이 아니라고 말하지도 않는다. 그러나 법의 경우에는
이처럼 기묘하게 보이는 주장들이 종종 제기되어 왔으며, 단순히 제기되었을
뿐만 아니라, 그것이 법의 본질적 성격에 대한 오랜 오해를 벗겨낸 진리를
계시하는 것인 양 열정적으로, 또 때로는 웅변적으로 주장되기도 했다.

`공직자들이 분쟁에 대해 하는 일이 곧 \ldots{} 법이다';\footnote{Llewellyn,
  \emph{The Bramble Bush} (2nd edn., 1951), p.~9.} `법이란 법원이 무엇을
할 것인지에 대한 예언이다 \ldots{} 그것이 내가 말하는
법이다';\footnote{O. W. Holmes, `The Path of the Law' in \emph{Collected
  Papers} (1920), p.~173.} 제정법(statutes)은 `법의 부분이 아니라
\ldots{} 법의 원천(sources of Law)이다';\footnote{J. C. Gray, \emph{The
  Nature and Sources of the Law} (1902), s. 276.} `헌법법(constitutional
law)은 단지 실정적 도덕(positive morality)일 뿐이다';\footnote{Austin,
  \emph{The Province of Jurisprudence Determined} (1832), Lecture VI
  (1954 edn., p.~259).} `도둑질하지 말라; 누군가가 도둑질을 하면 그는
처벌받을 것이다. \textbf{(p.~2)} \ldots{} 만약 어떤 규범이 존재한다면,
첫 번째 규범은 두 번째 규범에 포함되어 있으며, 두 번째 규범이야말로
유일한 진정한 규범이다 \ldots{} 법이란 제재(sanction)를 규정하는 1차
규범(primary norm)이다.'\footnote{Kelsen, \emph{General Theory of Law
  and State} (1949), p.~61.}

이들은 법의 본성에 관해 제기된 수많은 주장(assertions)과 부정(denials)
중 일부일 뿐이며, 적어도 처음 보았을 때는 낯설고 역설적으로 보인다. 이들
중 일부는 우리의 깊은 신념과 충돌하는 듯하며 쉽게 반박될 수 있는 것처럼
보인다. 그래서 우리는 다음과 같이 말하고 싶은 유혹을 느낀다. `제정법은
분명히 법\emph{이다}. 적어도 법의 한 종류이며, 다른 종류들도 있을 수
있다': `공직자나 법원이 무엇을 하는지가 곧 법이라는 것은 말이 안 된다.
왜냐하면 공직자나 법원을 만들기 위해서는 법이 먼저 존재해야 하기
때문이다'.

그러나 이러한 역설적으로 보이는 발언들은 상식의 명백한 사실을 의심하는
데 종사하는 몽상가나 전문 철학자들에 의해 제기된 것이 아니다. 그것은
오랜 기간 법에 대해 성찰해온 이들, 즉 주로 법을 가르치거나 실무에서
다루거나, 때로는 판사로서 법을 집행하는 것을 직업으로 삼은
법률가들(lawyers)의 성찰의 산물이다. 더 나아가, 그들이 말한 것들은
실제로 그들의 시대와 맥락에서 법에 대한 우리의 이해를 심화시켰다. 이와
같은 주장들은 그 문맥 안에서 이해될 때, 계몽적이면서도 \emph{동시에}
당혹스럽다: 그것들은 차가운 정의(definition)라기보다는, 법에 대한 어떤
진리를 지나치게 과장한 표현이며, 그 진리는 오랫동안 간과되어 왔던
것이다. 이러한 주장은 우리가 이전에는 보지 못했던 법의 측면을 보게 하는
빛을 비추지만, 그 빛이 너무 강렬한 나머지 나머지 부분을 가려버리고, 결국
전체를 명확히 파악하지 못하게 만든다.

책 속에서 끝없이 이어지는 이러한 이론적 논쟁과는 대조적으로, 대부분의
사람들은 '법'의 사례를 쉽게, 자신 있게 들 수 있는 능력을 지니고 있다.
대부분의 영국인은 살인을 금지하거나, 소득세 납부를 요구하거나, 유효한
유언장을 작성하기 위해 필요한 절차를 규정하는 법이 있다는 사실을 알고
있다. 영어 단어 'law'를 처음 접하는 어린이 또는 외국인을 제외하면, 거의
모든 사람들은 이와 같은 예시들을 얼마든지 더 많이 들 수 있다. 그리고
대부분의 사람들은 더 나아가, 어떤 것이 영국법인지 알아내는 방법에
대해서도 적어도 개요 정도는 설명할 수 있다. 사람들은 전문가에게 자문할
수 있으며, 이와 관련된 모든 질문에 대해 최종적인 권위를 가진 법원이
있다는 사실을 알고 있다. \textbf{(p.~3)} 이보다 더 많은 사실들이
일반적으로 널리 알려져 있다. 대부분의 교육받은 사람들은 영국의 법들이
일종의 체계(system)를 이루고 있다는 인식을 가지고 있다. 또 프랑스, 미국,
소련 등 --- 심지어 독립된 '국가(country)'로 간주되는 세계의 거의 모든
지역에는, 중요한 차이가 있음에도 불구하고 대체로 구조가 유사한
법체계(legal systems)가 존재한다는 사실도 알고 있다. 만약 이러한
사실들을 모른 채 교육을 마쳤다면, 그 교육은 심각하게 실패한 것이며, 다른
법체계들 간의 유사점을 말할 수 있다고 해서 그것이 대단한 교양의 표시라고
보기도 어렵다. 어느 정도 교육을 받은 사람이라면 누구나 다음과 같이
기본적 골격을 통해 그 핵심 요소들을 식별할 수 있어야 한다. 그것들은
다음과 같은 요소들로 구성된다: (i) 특정한 유형의 행위를 금지하거나
명령하고 이를 위반할 경우 제재를 가하는 규칙들; (ii) 특정한 방식으로
타인에게 피해를 준 경우 배상을 요구하는 규칙들; (iii) 유언, 계약 또는
권리를 부여하고 의무를 생성하는 기타 약정을 유효하게 만들기 위해 필요한
절차를 규정하는 규칙들; (iv) 이러한 규칙이 무엇인지, 언제 위반되었는지를
판단하고, 제재나 배상의 액수를 결정하는 법원(courts); (v) 새로운 규칙을
제정하고 기존 규칙을 폐지하는 입법기관(legislature).

만약 이러한 사실들이 일반적으로 잘 알려져 있다면, '법이란 무엇인가?'라는
질문이 왜 그토록 지속되어 왔으며, 그에 대해 이처럼 다양하고도 비범한
답변들이 제시되어 온 이유는 무엇일까? 이는 아마도 현대 국가들의
법체계(legal systems)와 같이, 누구도 이성적으로는 의심하지 않는 분명한
표준 사례들 외에도, 그 '법적 성격(legal quality)'에 대해 교육받은
일반인들뿐만 아니라 심지어 법률가들조차도 주저하는 의심스러운 사례들이
존재하기 때문일 것이다. 원시법(primitive law)과 국제법(international
law)은 그러한 의심스러운 사례들 중 대표적이며, 이들에 대해 일반적으로
통용되는 '법(law)'이라는 단어 사용의 적절성에 의문을 제기할 만한
이유들이 존재한다고 많은 이들이 느껴 왔다. 비록 그 이유들이 결정적인
경우는 드물지만 말이다. 이와 같은 의문스러운 또는 이견이 가능한 사례들의
존재는 실제로 오랜 기간 이어져 온, 다소 불모의 논쟁을 야기해 왔으나,
그렇다고 해서 그것이 '법이란 무엇인가?'라는 지속적인 질문에서 나타나는
법의 일반적 성격에 대한 난해함 전체를 설명할 수는 없다. 이러한 사례들이
그 난제의 뿌리가 아니라는 점은 두 가지 이유로 명확하다.

첫째, 이러한 사례들에서 주저함이 느껴지는 이유는 자명하다. 국제법은
입법기관(legislature)이 없고, 국가들은 사전 동의 없이는 국제법정에
회부될 수 없으며, 중앙집중적이고 효과적인 제재 체계도 존재하지 않는다.
일부 원시법의 유형들, 그리고 이들로부터 현대 법체계가 점차 진화했을지도
모를 경우들도 이와 유사하게 이러한 요소들을 결여하고 있다. 그리고 이러한
사례들이 표준 사례로부터 이 점에서 벗어난다는 사실이 그것들을 의심스러운
것으로 보이게 만든다는 점은 누구에게나 명확하다. 이 점에는 아무런
신비로움이 없다.

둘째, '법(law)'이나 '법체계(legal system)'처럼 복잡한 용어들만이 명백한
표준 사례와 논란이 되는 경계 사례를 모두 포함하게 된다는 점은 특이한
일이 아니다. 이제는 잘 알려진 사실이지만(한때는 충분히 강조되지
않았지만), 우리가 인간 삶이나 세계의 특징들을 분류할 때 사용하는 거의
모든 일반 용어들에 있어서도 이와 같은 구분은 필수적이다. 때로는, 어떤
표현의 사용에 있어 명백하고 표준적인 사례(또는 전형적 사례)와 의심스러운
사례들 간의 차이는 단지 정도의 차이일 뿐이다. 머리가 반짝이는 매끈한
두피를 가진 사람은 명백히 대머리(bald)이고, 무성한 머리숱을 가진 사람은
명백히 그렇지 않지만, 여기저기 머리칼이 드문드문 있는 제3의 사람을
대머리라고 할 수 있는지에 대해서는 실용적인 쟁점이 걸려 있다면 끝없이
논쟁될 수 있을 것이다.

때로는 표준 사례로부터의 벗어남이 단순한 정도 차이가 아니라, 실제로는
여러 가지 별개의 요소들이 보통 함께 나타나는 복합적 사례에서 일부 요소가
결여된 경우에 발생한다. 예를 들어, 수상비행기(flying boat)는
'선박(vessel)'인가? 퀸(queen) 없이 진행되는 체스 게임은 여전히
'체스(chess)'인가? 이러한 질문들은 표준 사례의 구성에 대해 성찰하고 이를
명시화하게끔 유도하기 때문에 교훈적일 수 있다. 그러나 이른바
경계적(borderline) 특성이 너무도 일반적인 현상이기 때문에, 법에 대한
오랜 논쟁을 그것만으로 설명할 수는 없다는 점도 분명하다. 더구나, 가장
유명하고 논쟁적인 법이론들 중 오직 소수만이 '원시법'이나 '국제법'이라는
표현을 해당 사례에 적용하는 것이 적절한지에 초점을 맞추고 있을 뿐이다.

사람들이 법의 사례들을 인식하고 인용할 수 있는 일반적 능력과, 법체계의
표준 사례에 대해 일반적으로 알려진 사실들을 고려해보면, `법이란
무엇인가?'라는 지속적인 질문을 그저 이미 익숙한 것들을 상기시키는
방식으로 간단히 끝낼 수 있을 것처럼 보이기도 한다. 우리가 (앞서 3쪽에서)
한 교육받은 사람의 입을 빌려 낙관적으로 제시한 지방법체계(municipal
legal system)의 핵심 요소에 대한 개요를 반복하는 방식으로 말이다. 우리는
이렇게 말할 수 있을 것이다. '이것이 ``법(law)''과 ``법체계(legal
system)''라는 말로 의미되는 표준 사례이다. 이 표준 사례들 외에도, 사회적
삶 속에는 이러한 주요 특징 중 일부를 공유하면서도 다른 일부는 결여한
제도적 구조들이 존재한다. 이것들은 법으로 분류하는 것이 적절한지에 대해
결정적 논거가 존재하지 않는 논쟁적 사례들이다.'

이러한 방식은 간결하고 명료하다는 장점은 있겠지만, 그 외에는 특별히
추천할 만한 점이 없다. 무엇보다도, `법이란 무엇인가?'라는 질문에 대해
가장 크게 당혹감을 느끼는 이들은 이 요약적 답변이 제시하는 익숙한
사실들을 잊었거나, 그것을 상기시켜야 할 만큼 몰랐던 이들이 아니다. 이
질문을 지속시켜 온 깊은 당혹감은 무지나 망각, 혹은 '법(law)'이라는
단어가 통상 지시하는 현상을 인식할 능력 부족 때문이 아니다. 더 나아가,
우리가 제시한 법체계에 대한 요약적 설명을 검토해 보면, 그것은 다양한
종류의 법률들이 일반적으로 함께 작동한다는 사실 외에 거의 아무것도
주장하지 않는다. 이는 요약 설명 속에서 법체계의 전형적 요소로 등장하는
'법원(court)'과 '입법기관(legislature)' 자체가 법의 산물이기 때문이다.
어떤 사람이 사건을 재판할 수 있는 관할권(jurisdiction)이나 법을 제정할
수 있는 권한(authority)을 부여하는 일정한 유형의 법률이 있을 때에만,
그것은 법원이나 입법기관이 된다.

따라서 이러한 질문에 대한 간단한 접근법은, 결국 '법'과 '법체계'라는 단어
사용을 지배하는 기존의 관습들을 상기시키는 것 이상은 하지 못하므로
무의미하다. 분명히 가장 바람직한 접근은, '법이란 무엇인가?'라는 질문에
섣불리 답변하기보다는, 그 질문을 제기하거나 답하려 했던 사람들이, 비록
법에 대한 친숙함이나 사례를 식별하는 능력은 의심할 여지 없이 충분함에도
불구하고, 실제로 \emph{무엇}에 대해 당혹감을 느껴왔는지를 먼저 파악하는
것이다. 그들은 무엇을 더 알고 싶어 하며, 왜 그것을 알고 싶어 하는가?
\emph{이 질문}에 대해서는 일반적 수준에서의 답변이 가능하다. 왜냐하면
법의 본성에 관해 논박과 반론의 지속적 중심이 되어 온 몇 가지 반복되는
주요 주제들이 존재하며, 이것들이 앞서 언급한 바와 같은 과장되고 역설적인
주장들을 촉발시켜 왔기 때문이다. 법의 본성에 대한 사유는 길고 복잡한
역사를 지니고 있지만, 회고해 보면 그것은 거의 끊임없이 소수의 핵심적인
쟁점들에 집중되어 있었음을 알 수 있다. 이러한 쟁점들은 단순히 학술적
논의의 즐거움을 위해 자의적으로 선택되거나 창작된 것이 아니다. 그것들은
모든 시대에 걸쳐 법에 대해 자연스럽게 오해를 낳기 쉬운 측면들에 관련된
것이며, 따라서 법에 대해 견고한 이해와 전문성을 갖춘 사려 깊은 사람의
마음속에서도 혼란과 그에 따른 명확성에 대한 요구가 공존할 수밖에 없는
지점들이기 때문이다.

\subsection{\texorpdfstring{\textbf{2. 세 가지 반복되는
쟁점들}}{2. 세 가지 반복되는 쟁점들}}\label{uxc138-uxac00uxc9c0-uxbc18uxbcf5uxb418uxb294-uxc7c1uxc810uxb4e4}

여기에서는 그러한 주요 반복 쟁점들 중 세 가지를 구분하고, 왜 그것들이
결국 \emph{법의 정의(definition)}에 대한 요구나 '법이란 무엇인가?'라는
질문의 형태로, 혹은 더 모호한 방식의 질문인 '법의 본성(nature) 또는
본질(essence)이 무엇인가?'라는 형태로 통합되어 나타나는지를 이후에
설명할 것이다.

이 세 가지 쟁점 중 두 가지는 다음과 같은 방식으로 제기된다. 언제,
어디서나 법의 가장 두드러진 일반적 특징은, 법이 존재한다는 것은 어떤
종류의 인간 행위가 더 이상 선택 사항이 아니라, \emph{어떤} 의미에서는
의무적(obligatory)이라는 것이다. 그러나 이처럼 겉보기에는 단순한 법의
특징은 실제로는 단순하지 않다. 왜냐하면, 선택 불가능하고 의무적인 행위의
영역 내에서도 우리는 서로 다른 형식을 구분할 수 있기 때문이다. 가장
단순한 의미에서 어떤 행위가 더 이상 선택사항이 아니게 되는 경우는, 한
사람이 다른 사람에게 어떤 행동을 하도록 강요하는 상황이다. 이때 신체가
실제로 밀리거나 당겨지는 식의 물리적 강제가 있는 것이 아니라, 거부할
경우 불쾌한 결과가 따를 것이라는 협박(threat)에 의해 강요되는 경우이다.
예컨대, 총을 든 강도가 희생자에게 지갑을 건네라고 명령하고, 거부할 경우
총을 쏘겠다고 위협하는 경우, 희생자가 순응한다면 우리는 그가
\emph{강요(obliged)} 되었다고 말한다. 어떤 이들에게는, 이러한 상황 ---
한 사람이 위협을 수반하는 명령(order)을 내리고, 이 의미에서 타인을
\emph{강요}하여 따르게 하는 상황 --- 에서야말로 법의 본질(essence), 또는
최소한 '법학(jurisprudence)의 열쇠'가 있다고 보이기도 한다. \footnote{Austin,
  op. cit., Lecture I, p.~13. He adds `and morals'.} 이는 영국의
법철학에 지대한 영향을 끼친 오스틴(Austin)의 분석이 출발한 지점이다.

물론, 법체계가 이러한 측면을 포함하는 경우가 자주 있다는 점에는 의심의
여지가 없다. 특정한 행위를 범죄로 선언하고, 그 행위에 대해 어떤 처벌이
부과되는지를 명시하는 형벌 제정법(penal statute)은 일종의 확대된 `총 든
강도' 상황처럼 보일 수도 있으며, 유일한 차이점은 그 명령이 특정한
개인에게가 아니라, 그러한 명령을 관습적으로 따르는 집단 전체를 향하고
있다는 점뿐일 수도 있다. 그러나 법의 복잡한 현상을 이처럼 단순한 요소로
환원하려는 시도는 매력적으로 보일 수 있음에도, 면밀히 살펴보면, 그러한
환원은 오히려 왜곡이며 혼란의 원천이 되어 왔다. 형벌 제정법의
경우조차도, 이 단순한 용어로 분석하는 것이 가장 설득력 있어 보일 때조차
그러하다. 그렇다면 법과 법적 의무(legal obligation)는 협박에 뒷받침된
명령과 어떻게 다르며, 어떠한 관련을 갖는가? 이것이야말로 항상 '법이란
무엇인가?'라는 질문 속에 잠재해 있는 핵심 쟁점이었다.

두 번째 쟁점은 또 다른 방식으로 어떤 행위가 선택 사항이 아니라 의무적일
수 있다는 점에서 제기된다. 도덕 규칙(moral rules)은 개인의 자율적 선택
영역에서 특정한 행위를 제외시키고, 그에게 의무(obligation)를 부과한다.
마치 법체계가 위협을 수반한 명령의 단순한 사례들과 밀접하게 연결된
요소들을 포함하고 있는 것처럼, 그것은 또한 도덕성과 관련된 특정한
측면들과도 밀접하게 연관된 요소들을 명백히 포함하고 있다. 두 경우 모두,
그 관계를 정확히 규정하는 데 어려움이 있으며, 그 명백한 유사성을 실제
`동일성(identity)'으로 착각하려는 유혹이 존재한다. 법과 도덕은 공통된
어휘를 공유할 뿐만 아니라, 법적·도덕적 의무, 권리, 책임이라는 표현들이
양쪽 모두에 존재한다. 그리고 모든 지방법체계는 특정한 근본적 도덕
명령들의 실질을 재현한다. 살인이나 무분별한 폭력 사용은 법과 도덕의
금지가 일치하는 가장 명백한 예에 불과하다. 게다가, '정의(justice)'라는
하나의 개념은 두 영역을 연결해주는 것으로 보인다. 정의는 법에 특히
적합한 덕목일 뿐만 아니라, 가장 '법적인' 덕목이기도 하다. 우리는 '법에
따른 정의(justice \emph{according to} law)'를 생각하고 말하면서도, 법
자체의 정의로움 혹은 부정의(justice or injustice \emph{of} the laws)
역시 논한다.

이러한 사실들은, 법은 도덕(morality)이나 정의(justice)의 한
`분야(branch)'로 이해하는 것이 가장 적절하며, 명령과 위협의
통합이라기보다는 도덕 또는 정의 원칙과의 일치(congruence)가 법의
'본질(essence)'이라는 견해를 지지하는 듯하다. 이러한 견해는 중세 스콜라
자연법 이론(scholastic theories of natural law)뿐 아니라 오스틴으로부터
계승된 법실증주의(legal positivism)를 비판하는 일부 현대 법이론에도
특징적으로 나타난다. 그러나 이와 같이 법을 도덕과 밀접하게 동일시하려는
이론들 역시, 결국에는 서로 다른 유형의 의무적 행위를 혼동하는 경향이
있으며, 법 규칙과 도덕 규칙 간의 종류 차이와 그 요구사항의 차이를 충분히
설명하지 못하는 경향이 있다. 이러한 차이들은 유사성과 수렴만큼이나
중요하다. 따라서 '부정의한 법은 법이 아니다(an unjust law is not a
law)'\footnote{`Nam mihi lex esse non videtur quae justa non fuerit':
  St.~Augustine I, \emph{De Libero Arbitrio}; Aquinas, \emph{Summa
  Theologica}, 1--11, Qu. 95 Art. 2; Qu. 96 Art. 4.}라는 주장은,
`제정법은 법이 아니다' 또는 `헌법법은 법이 아니다'라는 주장 못지않게
과장되고 역설적인 울림을 가진다. 법이론의 역사는 이러한 극단 사이의
진동을 보여주는 특징을 갖는다. 도덕과 법이 동일한 권리와 의무의 어휘를
공유한다는 사실로부터 잘못된 결론을 끌어낸 것에 불과하다고 본 사람들은,
그것에 대해 똑같이 과장되고 역설적인 방식으로 반박하곤 했다. '법이란
법원이 실제로 무엇을 할 것인지에 대한 예언이며, 그것 이상으로 허세를
부릴 것은 아무것도 아니다.'\footnote{Holmes, loc. cit.}

세 번째 주요 쟁점은 '법이란 무엇인가?'라는 질문을 끊임없이 자극하는 보다
일반적인 문제이다. 언뜻 보기에는, 법체계(legal system)가
\emph{규칙들(rules)}로 구성되어 있다는 주장은 거의 의심받을 일도,
이해하기 어려운 일도 없어 보인다. 위협을 수반한 명령의 개념에서 법
이해의 열쇠를 찾는 이들이든, 혹은 도덕(morality)이나 정의(justice)와의
관계에서 그것을 찾는 이들이든 모두, 법이 규칙을 포함하거나, 아니면 상당
부분 규칙으로 구성된다고 말한다. 그러나 이처럼 별 문제가 없어 보이는
개념에 대한 불만, 혼란, 불확실성이야말로 법의 본질(nature)에 대한
당혹감의 핵심을 이루고 있다. 규칙이란 \emph{무엇}인가? 규칙이
\emph{존재한다}고 말하는 것은 무슨 의미인가? 법원은 정말로 규칙을
\emph{적용}하는가, 아니면 단지 그런 척을 하는 것인가? 이 개념에 대해
질문이 제기되기 시작하면---특히 20세기 법철학에서 그러하였듯이---의견의
심각한 분열이 나타난다. 우리는 여기서 그것들을 간략히 개요만 제시할
것이다.

우선, 규칙에는 여러 다른 유형이 존재한다는 점은 분명하다. 이것은 단지
법적 규칙 외에도 예절의 규칙, 언어의 규칙, 게임과 동호회의 규칙 등이
존재한다는 명백한 의미에서뿐 아니라, 더 미묘한 의미에서도 그러하다. 즉,
한 영역 내에서도 규칙이라 불리는 것들은 서로 다른 방식으로 발생할 수
있으며, 관련된 행위와의 관계에서도 매우 다를 수 있다. 예컨대 법 내에서도
어떤 규칙은 입법(legislation)을 통해 만들어지지만, 다른 규칙은 그러한
의도된 행위에 의해 만들어지지 않는다. 더 중요한 것은, 어떤 규칙은
강제적(mandatory)으로, 사람들에게 그들의 의사와 무관하게 폭력을
자제하거나 세금을 납부하는 등의 행동을 요구한다는 점이다. 반면, 결혼,
유언, 계약을 성립시키기 위한 절차나 형식, 요건을 규정하는 규칙은
사람들이 자신이 원하는 바를 실현하기 위해 무엇을 해야 하는지를 지시한다.
이 두 가지 유형의 규칙 간의 대조는 게임의 규칙에서도 발견된다. 예컨대,
반칙(foul play)이나 심판에 대한 욕설을 금지하는 규칙과, 점수를 얻거나
승리하기 위해 무엇을 해야 하는지를 명시하는 규칙이 그것이다. 그러나
이러한 복잡성을 잠시 제쳐두고, 형사법(criminal law)의 전형적 사례인 첫
번째 유형의 규칙만 고려하더라도, 이 단순한 강제 규칙이 \emph{존재한다}는
진술의 의미에 대해서는 현대의 많은 학자들 사이에서도 견해가 크게 갈린다.
어떤 이들은 이 개념을 극히 불가해한 것으로 간주한다. 처음에 우리는 이
단순한 강제 규칙 개념에 대해 아마도 자연스럽게 다음과 같은 설명을 하고
싶어질 것이다: 규칙이 존재한다는 것은 단지 어떤 집단의 사람들, 혹은 그들
대부분이 특정한 상황에서 일반적으로(\emph{as a rule}) 유사한 방식으로
행동한다는 것을 의미한다. 예컨대, 영국에 교회 안에서 모자가 금지되거나,
`God Save the Queen'이 연주될 때 일어서는 규칙이 존재한다고 말하는 것은,
이 설명에 따르면, 단지 대부분의 사람들이 그러한 행동을 \emph{일반적으로}
수행한다는 의미일 뿐이다. 하지만 분명 이것만으로는 부족하다. 이런 설명은
부분적으로는 의미를 담고 있지만, 사회 집단 내의 단순한 행동 수렴만으로는
'규칙'이 존재한다고 할 수 없다. 예컨대 모든 이들이 아침마다 차를
마시거나 매주 영화관에 간다고 해도, \emph{그렇게 해야만 한다}는 규칙이
있는 것은 아니다. 단순한 행동 수렴과 사회 규칙의 존재는 언어적으로도
차이를 드러낸다. 후자를 묘사할 때 우리는 (꼭 사용해야 하는 것은
아니지만) 'must', `should', `ought to' 같은 표현을 사용하게 되며, 이는
전자의 경우 사용하면 오해를 불러일으킬 수 있다. 예컨대 영국에서는 매주
영화관에 \emph{가야 한다}거나 \emph{가야만 마땅하다}는 규칙은 없다. 단지
사람들이 정기적으로 영화관을 찾는 것뿐이다. 그러나 교회에서는 남자가
모자를 벗어야 한다는 규칙은 \emph{존재}한다.

그렇다면 단순한 행동 수렴과, `must', `should', `ought to' 같은 단어로
표현되는 규칙의 존재 사이의 핵심적 차이는 무엇인가? 이 점에 대해
법이론가들은 특히 오늘날 들어 의견이 크게 나뉘고 있다. 이 쟁점이
전면화된 데에는 여러 요인이 작용했다. 법적 규칙의 경우, 이 핵심적
차이---즉 `must'나 'ought' 요소---는 특정한 행위로부터의 일탈이 적대적
반응을 초래할 \emph{가능성}, 그리고 법적 규칙의 경우에는 공직자에 의해
\emph{처벌}될 가능성에 있다고 종종 여겨진다. 반면 매주 영화관에 가는
것과 같은 단순한 집단 습관(group habits)의 경우, 이러한 일탈은 처벌은
물론 질책조차 유발하지 않는다. 그러나 특정한 행동을 요구하는 규칙이
존재하는 경우, 그것이 법적이지 않더라도(예: 교회에서 모자를 벗어야 하는
규칙), 그러한 일탈은 대개 일정한 반발을 유발하게 된다. 법적 규칙의 경우,
이와 같은 결과는 예측 가능하고 제도적으로 조직되어 있으나, 비법적 규칙의
경우에는 비슷한 반응이 예상되더라도 조직화되어 있지는 않다.

예측 가능한 처벌은 법적 규칙의 중요한 측면임이 분명하지만, 그것만으로는
사회 규칙이 \emph{존재}한다는 진술이나 규칙에 포함된 `must' 또는 `ought'
요소를 모두 설명할 수 없다. 이러한 예측 이론에 대해 많은 비판이 있으나,
특히 스칸디나비아 법이론 학파 전체를 특징짓는 하나의 주장은 주의 깊게
검토할 만하다. 그 주장은 다음과 같다. 우리가 법적 규칙을 위반한 자를
처벌하는 판사나 공직자의 활동(또는 비법적 규칙의 위반에 대해 비판하거나
질책하는 일반인들)의 행위를 면밀히 살펴보면, 규칙은 이들의 행위 안에서
단순한 예측의 대상이 아니라 \emph{지침(guide)}이자 \emph{처벌의
이유(reason)}이자 \emph{정당화(justification)}로 작동하고 있다는 것이다.
판사는 규칙을 단지 자신과 타인이 위반을 처벌할 \emph{가능성}이 있다는
진술로 받아들이지 않으며, 그렇게 보더라도 그것은 제3자의 시각이다.
판사에게 중요한 것은 규칙이 \emph{지침}이자 \emph{정당화 근거}라는
점이며, 예측 가능성이라는 측면은 그에게는 중요하지 않다. 비법적 규칙의
위반에 대한 비공식적 질책의 경우도 마찬가지이다. 이러한 반응 역시 단순히
예측 가능한 반응이 아니라, 규칙의 \emph{존재}가 그러한 반응을
\emph{유도}하고 \emph{정당화}한다고 여겨지는 것이다. 그래서 우리는 어떤
사람이 규칙을 어겼기 때문에 그를 질책하거나 처벌한다고 말하며, 단순히
그럴 \emph{가능성}이 높았다는 말로는 충분치 않다.

그러나 이러한 예측 이론에 반대하는 비평가들 중 일부는 이 지점에서 여전히
어떤 불투명함이 존재한다고 인식한다. 곧, 명확하고 사실적으로 분석할 수
없는 무언가가 있다는 것이다. 일반적인 행위 패턴에서 벗어난 사람을
처벌하거나 질책하는 \emph{예측 가능한 반응} 외에, 규칙 안에 단순한 집단
습관과 구별되는 어떤 요소가 \emph{정말로} 존재하는가? 판사가 처벌을
정당화할 때 실제로 그를 이끄는 어떤 \emph{추가적인 요소}가 존재하는가?
이 추가 요소가 정확히 무엇인지 말하기 어려운 점 때문에, 예측 이론에
비판적인 이들은 이 시점에서 아예 모든 규칙 담론과, `must', `should',
`ought' 등의 사용이 혼란을 내포한 것이라고 주장하기도 한다. 이러한
혼란은 사람들 눈에 규칙의 중요성을 부풀리는 효과는 있을 수 있으나,
실제로는 아무런 이성적 근거가 없다는 것이다. 우리는 단지, 규칙 안에 어떤
구속력이 있어 우리를 특정한 행위로 이끌고 정당화한다고 \emph{생각}할
뿐이며, 실제로는 그저 우리가 그러한 규칙에 따라 행동해야 한다는 강한
\emph{감정}을 경험하고 있을 뿐이다. 우리는 이 감정이 우리 안에서 발생한
것임을 인식하지 못하고, 그것이 마치 우리 외부에 있는 어떤 '보이지 않는
질서'에서 비롯된 것처럼 믿는다. 이 지점에서 우리는 법이 늘 연루되어
왔다는 '허구(fiction)'의 세계에 들어서게 된다. 바로 이런 허구를 채택하기
때문에 우리는 '인간이 아닌 법에 의한 지배(government of laws, not
men)'를 진지하게 말할 수 있게 되는 것이다. 이러한 유형의 비판은, 그
실증적 주장에 대한 평가 여부와는 별개로, 사회 규칙(social rules)과
단순한 행동 수렴 사이의 구별을 더 명확히 설명할 필요성을 제기한다. 이
구별은 법을 이해하는 데 결정적으로 중요하며, 본서 초반 장들 대부분이
이에 초점을 두고 있다.

그러나 법적 규칙의 성격에 대한 회의론(scepticism)이 항상 구속적 규칙
개념 자체를 혼란스럽거나 허구적이라고 정죄하는 극단으로까지 나아가는
것은 아니다. 오히려 영국과 미국에서 지배적인 회의론 형태는, 법체계가
\emph{전적으로}, 혹은 \emph{주로} 규칙으로 구성된다는 기존 견해 자체를
재고할 것을 요청한다. 법원이 판결문을 구성하는 방식은, 그들의 판결이
의미가 명확하고 고정된 선행 규칙으로부터 필연적으로 도출된 것처럼 보이게
만든다. 단순한 사건에서는 그것이 사실일 수도 있다. 그러나 법원을
곤란하게 하는 사건들의 대부분에서는, 그 규칙이 담겨 있다고 여겨지는
제정법(statutes)이나 선례(precedents) 어느 쪽도 단 하나의 결과만을
허용하지 않는다. 대부분의 중요한 사건에서는 항상 선택의 여지가 있다.
판사는 제정법의 문구에 어떤 해석을 부여할지, 선례의 '의미'를 어떻게
해석할지 사이에서 선택해야 한다. 판사는 법을 \emph{창조하는} 것이 아니라
\emph{발견한다}는 전통이, 마치 판결이 명확한 기존 규칙에서 아무런 선택의
여지 없이 자연스럽게 도출된 것처럼 보이게 만든다. 법적 규칙들은 종종
논란의 여지 없이 받아들여지는 핵심(core)을 가질 수 있으며, 경우에
따라서는 그러한 규칙의 의미에 대한 논쟁이 발생하는 것이 상상조차 어려울
수 있다. 예컨대 《유언장법(Wills Act), 1837》 제9조의 '유언장에는 두
명의 증인이 있어야 한다'는 규정은 해석상의 문제가 발생할 가능성이 낮아
보인다. 그러나 모든 규칙에는 해석의 불확실성(penumbra of uncertainty)이
존재하며, 그 안에서 판사는 반드시 선택해야 한다. 그 유언장이
'서명(sign)'되어야 한다는 단순해 보이는 조항조차도, 다음과 같은
상황에서는 의미가 의심스러워질 수 있다. 유언자가 가명을 사용했다면? 다른
사람이 손을 잡고 서명했다면? 이니셜만 썼다면? 정확한 전체 이름을 직접
썼지만, 마지막 페이지 아래가 아닌 첫 페이지 위에 썼다면? 이러한 경우
모두가 과연 법적 규칙의 의미 안에서 '서명'에 해당하는가?

이렇듯 사적 법(private law)의 일상적인 분야에서조차 불확실성이 나타날 수
있다면, 미국 헌법의 제5조 및 제14조 수정조항과 같은 웅대한 문구들에서는
얼마나 더 많은 불확실성이 존재하겠는가? 예컨대 ``정당한 법 절차(due
process of law) 없이 생명, 자유, 재산을 박탈당하지 아니한다''는 조항에
대해 한 평론가는 \footnote{J. D. March, `Sociological Jurisprudence
  Revisited', 8 \emph{Stanford Law Review} (1956), p.~518.} 이렇게 말한
바 있다. 이 문구의 진짜 의미는 매우 명확하다. 곧 ``어떤 \emph{w}도
\emph{x} 또는 \emph{y} 없이 \emph{z} 되어서는 안 된다''는 것이며, 여기서
\emph{w}, \emph{x}, \emph{y}, \emph{z}는 광범위한 범위 내에서 임의의
값을 취할 수 있다는 것이다. 이야기를 마무리하듯 회의론자들은 우리에게
상기시킨다. 규칙은 불확실할 뿐 아니라, 법원의 해석은 권위적일 뿐만
아니라 \emph{최종적}이기도 하다. 이 모든 점을 고려할 때, 법이란
본질적으로 규칙의 문제라는 개념은 심각한 과장이거나 심지어 오류가
아닌가? 이러한 사고는 우리가 앞서 인용한 역설적인 부정으로 나아가게
만든다. 곧, `제정법은 법의 일부가 아니라 법의 원천일 뿐이다.'\footnote{Gray,
  loc. cit.}

\subsection{\texorpdfstring{\textbf{3.
정의(Definition)}}{3. 정의(Definition)}}\label{uxc815uxc758definition}

여기 이제 세 가지 반복되는 쟁점이 있다. 첫째, 법은 위협을 수반한 명령과
어떻게 구별되며 어떤 관계에 있는가? 둘째, 법적 의무(legal obligation)는
도덕적 의무(moral obligation)와 어떻게 구별되며 어떤 관계에 있는가?
셋째, 규칙(rules)이란 무엇이며, 법은 어느 정도까지 규칙의 문제인가? 이
세 가지 쟁점에 대한 의심과 당혹을 해소하려는 것이 '법의 본성(nature)'에
대한 대부분의 사유의 주된 목적이었다. 이제 왜 이러한 사유들이 대개 '법의
정의(definition)'에 대한 탐구로 여겨져 왔는지, 또 익숙한 정의의 형식들이
왜 지속되는 난점과 의문들을 거의 해결하지 못했는지도 알 수 있다.
정의라는 말이 암시하듯, 정의는 본질적으로 구분선을 긋는 일, 즉 언어가
독립된 단어로 구별해 온 서로 다른 사물의 종류를 나누는 일이다. 이러한
구분선의 필요는, 일상에서 그 단어를 아무 문제 없이 사용하는 사람이라
할지라도, 자신이 감지하는 구별이 어떤 것인지 명확히 말하거나 설명하지
못할 때 자주 나타난다. 우리 모두는 때때로 이러한 곤경에 빠진다.
'코끼리는 보면 알겠는데 설명은 못하겠다'는 사람의 상황이 그것이다. 성
아우구스티누스(St.~Augustine)의 시간 개념에 대한 유명한 말\footnote{\emph{Confessiones},
  xiv. 17.}도 같은 곤경을 표현한 것이다. ``그렇다면 시간은 무엇인가?
아무도 묻지 않으면 나는 안다. 그러나 누군가 그것을 설명하라 하면 나는
모른다.'' 많은 숙련된 법률가들도 이와 같은 방식으로 느껴 왔다. 즉,
그들은 법을 알고 있지만, 법과 다른 사물과의 관계에 대해 설명할 수 없는
많은 것들이 있으며, 그것을 완전히 이해하지는 못한다고 느낀다. 마치
익숙한 도심에서 목적지까지 길은 잘 찾아가지만, 다른 사람에게 그 길을
설명해줄 수는 없는 사람과 같다. 정의를 요구하는 사람은 자신이 어렴풋이
감지하고 있는 법과 다른 사물들 사이의 관계를 명확히 보여주는 지도가
필요한 것이다.

이러한 경우, 단어의 정의는 그런 지도를 제공할 수 있다. 그것은 한편으로
우리가 어떤 단어를 사용할 때 암묵적으로 따르는 원리를 명시적으로
드러내며, 동시에 우리가 그 단어를 적용하는 현상들과 다른 현상들 간의
관계를 드러낸다. 정의가 `단지 말뿐'이거나 `단어에 관한 것일 뿐'이라는
말도 있지만, 그것이 일상적으로 사용되는 표현일 때는 특히 오해를 불러올
수 있다. 삼각형을 '세 개의 직선으로 이루어진 도형'이라거나, 코끼리를
'두꺼운 피부, 상아, 코를 가진 네발짐승'으로 정의하는 경우조차, 비록
겸손한 수준일지라도, 그 단어의 표준적 사용법과 그 단어가 가리키는 대상을
모두 가르쳐준다. 이러한 익숙한 유형의 정의는 두 가지 기능을 동시에
수행한다. 하나는 그 단어를 다른 잘 알려진 용어들로 번역하는 코드나
공식(formula)을 제공하고, 다른 하나는 그 단어가 어떤 종류의 대상에
사용되는지를 알려주며, 그것이 더 넓은 범주의 것들과 공유하는 특징들과,
같은 범주 내 다른 것들과 구별되는 특징들을 함께 제시한다. 이러한 정의를
탐색하고 찾아낼 때, 우리는 '단지 단어들을 들여다보는 것'이 아니라,
'우리가 단어로 지칭하고자 하는 실재를 함께 바라보게 된다. 단어에 대한
감각을 날카롭게 함으로써, 우리가 말하는 현상에 대한 인식 역시 날카롭게
되는 것이다.'\footnote{J. L. Austin, `A Plea for Excuses',
  \emph{Proceedings of the Aristotelian Society}, vol.~57 (1956--7),
  p.~8.}

이러한 정의 형식은 (\emph{per genus et differentiam}, 즉 속(genus)과
종차(differentia)를 통한 정의) 삼각형이나 코끼리 같은 단순한 사례에서 볼
수 있으며, 가장 단순하면서도 어떤 이들에게는 가장 만족스러운 정의
방식이다. 왜냐하면 그것은 정의된 단어를 언제든 대체할 수 있는 표현을
제공해 주기 때문이다. 하지만 이 정의 형식은 항상 사용 가능한 것도
아니며, 사용 가능하더라도 항상 유익한 것도 아니다. 그것의 성공은 특정한
조건들이 충족될 때에만 가능하다. 그 중에서도 가장 중요한 조건은, 우리가
그 특성을 분명히 이해하고 있는 더 넓은 범주, 즉 속(genus)이 존재해야
한다는 것이다. 정의는 우리가 알고 있는 일반적인 것들 속에서 어떤 특수한
하위 범주를 위치시키는 것이기 때문이다. 그런데 그 속 자체에 대해
모호하거나 혼란스러운 개념만 갖고 있다면, 그 속의 일원이 무엇인지를
알려주는 정의는 아무런 도움이 되지 못한다. 이 조건은 법의 경우에 이러한
정의 형식을 쓸모없게 만든다. 법이 속하는 친숙하고 잘 이해된 일반
범주라는 것이 존재하지 않기 때문이다. 법의 정의에 있어 가장 뚜렷한
후보는 '행동 규칙(rules of behaviour)'이라는 일반 범주일 것이다. 그러나
우리가 이미 본 바와 같이, 규칙의 개념 자체가 법의 개념만큼이나 난해하기
때문에, 법을 규칙의 일종으로 파악하는 정의는 법에 대한 이해를 더 이상
나아가게 하지 못한다. 법을 익숙하고 잘 이해된 일반 항목 속의 하위
항목으로 위치시키는 데 성공하는 정의 형식보다 더 근본적인 무언가가
요구된다.

그러나 법의 경우, 이러한 단순 정의 방식의 유용성을 방해하는 더 큰
장애물들이 있다. 일반 표현을 이러한 방식으로 정의할 수 있다는 가정은,
정의하고자 하는 것의 모든 사례가 그 표현이 지시하는 공통된 특성을
지닌다는 암묵적 전제 위에 놓여 있다. 물론, 비교적 초급 단계에서조차 경계
사례(borderline cases)의 존재는 우리의 주의를 끌게 되며, 이는 일반
용어의 여러 사례들이 동일한 특성을 가진다는 전제가 하나의 독단적인
명제일 수 있음을 보여준다. 흔히 일상어, 심지어 기술적 용어의 사용조차도
상당히 `열려(open)' 있어서, 통상 함께 수반되는 특성들 중 일부만 가진
사례에도 용어가 확장 적용되는 것을 금지하지 않는다. 우리가 이미 본 바와
같이, 국제법(international law)이나 특정 형태의 원시법(primitive law)에
대해 그러한 확장에 찬성하거나 반대하는 주장 모두 일리가 있다. 더 중요한
점은, 그러한 경계 사례 외에도 일반 용어의 여러 사례들은 단순 정의 방식이
전제하는 것과는 전혀 다른 방식으로 서로 연결되어 있다는 사실이다. 예컨대
'발(foot)'이라는 말이 사람에게도, 산의 밑부분에도 사용되는 경우처럼,
유추(analogy)를 통해 연결될 수 있다. 또는 각기 다른 방식으로 중심 요소에
관계된 경우도 있다. 예컨대 '건강하다(healthy)'는 말은 사람뿐 아니라
피부, 아침 운동에도 쓰이는데, 이 중 피부는 건강함의
\emph{징후(sign)}이고, 운동은 \emph{원인(cause)}이다. 다시 말해, 이들은
중심 개념을 기준으로 서로 다른 방식으로 관련되어 있다. 또는---아마도
법체계를 구성하는 다양한 규칙 유형을 통일시키는 원리와 유사한
방식으로---각 사례가 하나의 복합 활동의 서로 다른 구성 요소들일 수 있다.
'철도(railway)'라는 형용사는 기차뿐 아니라 선로, 역, 짐꾼, 철도회사 등에
쓰이며, 이들은 이와 같은 통일 원리에 따라 연결되어 있다.

물론, 여기서 논의한 단순하고 전통적인 정의 형식 이외에도 다양한 정의
방식이 존재한다. 그러나 우리가 앞서 확인한 세 가지 핵심 쟁점의 성격을
떠올려보면, `법이란 무엇인가?'라는 반복되는 질문에 대해 만족스러운
답변을 줄 수 있을 만큼 간결한 정의는 존재할 수 없다는 것이 분명하다. 이
쟁점들은 서로 너무 다르며, 그 각각이 너무 근본적이기 때문에, 그러한
방식으로 해결될 수 없다. 간결한 정의를 제시하려는 시도들의 역사 자체가
이를 보여준다. 그럼에도 이 세 가지 쟁점을 하나의 질문, 또는 정의
요청으로 묶으려는 본능은 잘못된 것이 아니었다. 왜냐하면 우리는 이 책의
진행 과정에서, 이 세 가지에 대한 공통된 답변의 일부를 이루는 중심
요소들을 분리해내고 그 특징을 규명하는 것이 가능하다는 것을 보여줄
것이기 때문이다. 이 요소들이 무엇이며, 왜 이 책에서 그토록 중요한 위치를
부여받는지는, 먼저 오스틴(Austin)이 제시한 단순한 '위협을 수반한
사령(command)' 개념에서 법 이해의 열쇠를 찾으려는 이론의 결함을 자세히
고찰해 봄으로써 가장 잘 드러날 것이다. 이 단순 명령 이론의 문제점을
비판적으로 분석하는 것이 다음 세 장(chapters)의 주된 과제이다. 우리가 이
이론을 먼저 비판하고, 그 주요 경쟁자인 '법은 도덕과 필연적 연결을
갖는다'는 전통적 이론을 이후 장에서 다루기로 한 것은, 현대 법이론이
발전해온 역사적 순서를 의도적으로 무시한 것이다. 왜냐하면 도덕과의
연결을 강조하는 이 경쟁 이론은 오스틴보다도 앞선 벤담(Bentham) 시대부터
존재해 왔으며, 오스틴 자신이 주요 비판 대상으로 삼은 바 있기 때문이다.
만일 이러한 비역사적 접근에 대해 변명이 필요하다면, 그것은 이 단순 명령
이론의 오류들이 오히려 더 복잡한 경쟁 이론들의 오류보다 진리에 더 가까운
방향을 가리키기 때문이라고 하겠다.

이 책의 여러 부분에서는 `법' 또는 '법체계'라는 표현의 적용 가능성에 대해
법이론가들이 의문을 품어온 경계 사례들에 대한 논의가 등장할 것이다.
하지만 이러한 의문들에 대한 해답은 이 책의 부차적 관심일 뿐이다. 이 책의
목적은 '법'이라는 단어의 적절한 사용 여부를 판단하기 위한 규칙을
제시하는 정의를 제공하는 데 있는 것이 아니다. 그것은 오히려
지방법체계(municipal legal system)의 고유한 구조에 대한 분석을 개선하고,
법, 강제(coercion), 도덕이라는 사회 현상의 유형들 사이의 유사성과 차이를
더 잘 이해하도록 함으로써 법이론을 진전시키는 데 있다. 이 책의 다음 세
장에서의 비판적 논의 과정에서 확인되고, 제5장과 제6장에서 자세히 설명될
일련의 요소들은 바로 이러한 목적을 위해 기능하며, 책의 나머지 부분에서
그 역할이 입증된다. 이러한 이유로, 이 요소들은 법 개념의 핵심
요소들이자, 법 개념의 명료화를 위한 가장 중요한 요소들로 간주된다.

\newpage

\section{주석(Notes)}\label{uxc8fcuxc11dnotes}

이 책의 본문은 자족적인(self-contained) 구조를 가지고 있으므로, 독자는
각 장을 처음부터 끝까지 읽은 뒤에 이 주석 부분을 참고하는 것이 가장 좋을
수 있다. 본문의 각주는 인용문의 출처, 그리고 인용된 판례(case)나
법령(statute)에 대한 간단한 정보만을 제공한다. 이에 반해, 다음의
주석들은 다음과 같은 세 가지 유형의 내용을 독자의 주의에 환기시키기 위해
마련되었다: (i) 본문에 제시된 일반적 진술에 대한 추가적인 예시나 사례;
(ii) 본문에서 채택하거나 언급한 견해를 더 확장하거나 비판한 문헌; (iii)
본문에서 제기된 물음들에 대한 추가적인 탐구를 위한 제안. 이 책 자체에
대한 모든 인용은 장 번호와 절 번호만으로 간단히 표기한다 (예: 제1장
제1절). 다음은 본문에 사용된 주요 약어(abbreviation)들이다:

\vspace{1em}

\begin{tabular}{@{}l p{10cm}@{}}
\textbf{Austin, The Province}   & \textit{오스틴, 『법철학의 관할 범위(The Province of Jurisprudence Determined)』} (하트 편, 런던, 1954) \\
\textbf{Austin, The Lectures}   & \textit{오스틴, 『실정법 철학 강의(Lectures on the Philosophy of Positive Law)』} \\
\textbf{Kelsen, General Theory} & \textit{켈젠, 『법과 국가의 일반이론(General Theory of Law and State)』} \\
\textbf{BYBIL}                  & \textit{영국 국제법 연보(British Year Book of International Law)} \\
\textbf{HLR}                    & \textit{하버드 로 리뷰(Harvard Law Review)} \\
\textbf{LQR}                    & \textit{법 분기별 리뷰(Law Quarterly Review)} \\
\textbf{MLR}                    & \textit{현대 법 리뷰(Modern Law Review)} \\
\textbf{PAS}                    & \textit{아리스토텔레스 학회 발표집(Proceedings of the Aristotelian Society)} \\
\end{tabular}

\subsection{\texorpdfstring{\textbf{CHAPTER I
주석}}{CHAPTER I 주석}}\label{chapter-i-uxc8fcuxc11d}

\textbf{1--2쪽.} Llewellyn, Holmes, Gray, Austin, Kelsen의 인용문들은
모두, 저자의 견해에 따르면, 일반적인 법률 용어에 의해 가려졌거나 이전의
이론가들에 의해 과도하게 간과된 법의 어떤 측면을 강조하기 위해 사용된
역설적이거나 과장된 표현이다. 중요한 법학자에 관해서는, 그가 한 법에
대한 진술이 문자 그대로 참인지 거짓인지를 판단하기에 앞서, (1) 그가
자신의 진술을 뒷받침하기 위해 제시한 구체적 이유와, (2) 그 진술이
대체하고자 한 법 개념 또는 법 이론을 먼저 검토하는 것이 종종 유익하다.

간과된 진리를 강조하는 수단으로서, 이와 유사한 역설적 또는 과장된 주장
방식은 철학에서도 흔히 볼 수 있다. J. Wisdom, `Metaphysics and
Verification' (\emph{Philosophy and Psychoanalysis}, 1953); Frank,
\emph{Law and the Modern Mind} (런던, 1949), 부록 VII (`Notes on
Fictions') 참조.

이 페이지들에 인용된 다섯 개의 주장 속에 담긴 이론들은 다음에서
분석된다: 제7장 제2·3절 (Holmes, Gray, Llewellyn), 제4장 제3·4절
(Austin), 제3장 제1절 35--42쪽 (Kelsen).

\textbf{4쪽.} \emph{표준 사례와 경계 사례.} 여기서 언급된 언어적 특징은
일반적으로 제7장 제1절 `법의 개방적 구조(The Open Texture of Law)'에서
논의된다. 이 특징은 '법', `국가', `범죄' 등의 일반 용어에 대한 정의가
명시적으로 요구될 때뿐만 아니라, 일반적 용어로 구성된 규칙을 특정 사례에
적용할 때 사용되는 추론의 성격을 규정하려는 시도에서도 염두에 두어야
한다. 이 언어적 특징의 중요성을 강조한 법학자들로는 다음이 있다: Austin,
\emph{The Province}, Lecture VI, 202--207쪽, 및 \emph{Lectures in
Jurisprudence} (5판, 1885), 997쪽 (`Note on Interpretation'); Glanville
Williams, `International Law and the Controversy Concerning the Word
``Law''\,', \emph{22 BYBIL} (1945); `Language in the Law' (5편의 논문),
\emph{61} 및 \emph{62 LQR} (1945--6). 단, 후자의 경우 J. Wisdom이
\emph{Philosophy and Psycho-Analysis} (1953) 수록 논문 `Gods' 및
'Philosophy, Metaphysics and Psycho-Analysis'에서 제시한 비평 참조.

\textbf{6쪽.} \emph{Austin의 의무 개념.} \emph{The Province}, Lecture I,
14--18쪽; \emph{The Lectures}, Lecture 22, 23 참조. 의무 개념과, `의무를
가진 것(having an obligation)'과 `강제에 의해 강요받는 것(being
obliged)' 사이의 차이는 제5장 제2절에서 자세히 분석된다. Austin의 분석에
대한 주석은 제2장 아래, 282쪽 참조.

\textbf{8쪽.} \emph{법적 의무와 도덕적 의무.} 법이 도덕과의 연결을 통해
가장 잘 이해된다는 주장은 제8·9장에서 분석된다. 이 주장은 매우 다양한
형태를 띤다. 고전 및 중세 스콜라 자연법 이론에서처럼, 이 주장은 근본적인
도덕적 구별이 인간 이성에 의해 발견 가능한 '객관적 진리'라는 주장을
수반하기도 하지만, 법과 도덕의 상호의존성을 강조하면서도 도덕의 본성에
대해 이런 관점을 채택하지 않는 법학자들도 많다. 제9장 주석, 아래 302쪽
참조.

\textbf{10쪽.} \emph{스칸디나비아 법이론과 구속 규칙 개념.} 영어
독자에게 가장 중요한 이 학파의 저작은 Hägerström (1868--1939),
\emph{Inquiries into the Nature of Law and Morals} (Broad 번역, 1953),
그리고 Olivecrona, \emph{Law as Fact} (1939)이다. Olivecrona는 법 규칙의
성격에 대한 그들의 견해를 가장 명확히 서술하고 있다. 그는 미국
법학자들이 선호한 법 규칙에 대한 예측적 분석을 비판하는데 (op. cit.
85--88쪽, 213--215쪽), 이 비판은 Kelsen, \emph{General Theory} (165쪽
이하, `The Prediction of the Legal Function')에 나타난 유사한 비판과
비교할 가치가 있다. 이 두 법학자가 많은 점에서 견해를 공유하면서도 왜
규칙의 성격에 대해 전혀 다른 결론에 도달하는지를 묻는 것도 흥미롭다.
스칸디나비아 학파에 대한 비판으로는 다음을 참조: Hart, Hägerström 서평,
\emph{Philosophy} 제30권 (1955); `Scandinavian Realism', \emph{Cambridge
Law Journal} (1959); Marshall, `Law in a Cold Climate', \emph{Juridical
Review} (1956).

\textbf{12쪽.} \emph{미국 법이론에서의 규칙 회의론(rule-scepticism).}
제7장 제1·2절 `형식주의와 규칙 회의론(Formalism and Rule-scepticism)'
참조. 여기서는 이른바 '법 현실주의(Legal Realism)'로 알려진 주요
이론들이 검토된다.

\textbf{12--13쪽.} \emph{일상어 의미에 대한 의심.} `sign(표지)' 또는
'signature(서명)'의 의미에 관한 판례로는 \emph{Halsbury, Laws of
England} (2판) 제34권, 제165--169절 및 \emph{In the Estate of Cook}
(1960), 1 AER 689 및 그에 인용된 판례 참조.

\textbf{13쪽.} \emph{정의.} 정의의 형식과 기능에 대한 현대 일반 이론은
Robinson, \emph{Definition} (옥스퍼드, 1952) 참조. 전통적인 \emph{per
genus et differentiam} 정의 방식의 한계는 Bentham, \emph{Fragment on
Government} (제5장 제6절 주석), Ogden, \emph{Bentham's Theory of
Fictions} (75--104쪽) 참조. Hart, `Definition and Theory in
Jurisprudence', \emph{70 LQR} (1954); Cohen and Hart, `Theory and
Definition in Jurisprudence,' \emph{PAS} 보충권 제29권 (1955)도 참조.

`법(law)' 개념에 대한 정의는 다음을 참조: Glanville Williams, 앞서
인용한 저작; R. Wollheim, `The Nature of Law', \emph{2 Political
Studies} (1954); Kantorowicz, \emph{The Definition of Law} (1958), 특히
제1장. 일상적으로 의심 없이 사용되는 용어에 대해서도 정의가 왜 필요한지,
그리고 그것이 어떤 명료화 기능을 수행할 수 있는지를 보여주는 일반
논의로는 다음을 참조: Ryle, \emph{Philosophical Arguments} (1945);
Austin, `A Plea for Excuses', \emph{57 PAS} (1956--7), 15쪽 이하.

\textbf{15쪽.} \emph{일반 용어와 공통 속성.} 어떤 일반 용어(예: `법',
`국가', `국민', `범죄', `선', `정의')가 정확히 사용되기 위해서는 그
용어가 적용되는 사례들이 모두 '공통된 속성'을 공유해야 한다는 무비판적인
믿음은 많은 혼란을 초래해 왔다. 법철학에서는, 그러한 용어를 다양한
사물에 적용하는 유일하게 정당한 이유는 공통된 속성의 존재라는 가정 아래,
정의를 위해 공통 속성을 찾아내려는 헛된 시도에 많은 시간과 창의가
낭비되어 왔다(예: Glanville Williams, 앞서 인용). 다만, 일반 용어의
성격에 대한 이러한 잘못된 이해가 항상, 이 저자가 말하는 바와 같은
'언어적 질문'과 사실 문제 간의 혼동까지 수반하는 것은 아니라는 점에
주의해야 한다.

법적, 도덕적, 정치적 용어들에 있어, 일반 용어의 여러 사례들이 어떻게
상이하게 관련될 수 있는지를 이해하는 것은 특히 중요하다. 유추적 관계에
대해서는 Aristotle, \emph{Nicomachean Ethics} 제1권 제6장(`good'의
다양한 사례들이 이와 같은 관계로 연결되어 있다고 암시) 참조; Austin,
\emph{The Province}, Lecture V, 119--124쪽. 중심 사례와의 다른 방식의
관계(예: 'healthy')에 대해서는 Aristotle, \emph{Categories} 제1장 및
\emph{Topics} 제1권 제15장, 제2권 제9장의 `파생어(paronyms)' 사례들
참조. `가족 유사성(family resemblance)' 개념에 대해서는 Wittgenstein,
\emph{Philosophical Investigations}, 제1부, 제66--76절 참조. 제8장 제1절
`정의(just)' 개념 구조에 대한 논의도 비교 참조. Wittgenstein의 조언(op.
cit., 제66절)은 법적·정치적 용어 분석에서 특히 적절하다. 그는
'게임(game)'이라는 단어의 정의를 논하면서 이렇게 말했다. ``'게임'이라
불리는 것들 사이에 \emph{반드시} 공통점이 있어야 한다고 말하지 말고,
\emph{보라} 그리고 \emph{살펴보라}. 그러면 그 모든 것들에 공통된
무언가가 있는 것이 아니라, 유사성, 관계, 일련의 연속체들이 있을 뿐임을
보게 될 것이다.''

\subsection{\texorpdfstring{\textbf{CHAPTER I 3판
주석}}{CHAPTER I 3판 주석}}\label{chapter-i-3uxd310-uxc8fcuxc11d}

Hart 자신의 주석은 이 판에서도 변경 없이 유지되었다. 그것들은 여전히
그의 논증의 미묘한 차이, 본문에서 충분히 다루어지지 않은 사유, 다른
이들과의 입장 비교 등을 이해하는 데 유용하다. 그러나 학술적 자료로서 이
주석들은 종종 후속 연구들에 의해 대체된다. 아래는 Hart의 논증을
확장하거나 비판하는 최근 영어권 연구들에 대한 안내이며, 문헌이 방대하기
때문에 포괄성을 목표로 하지 않고, 학생들에게 특히 유용할 수 있는 일부
항목만을 제시한다. 가능하다면 Hart와 지속적으로 논쟁을 벌였던 인물들,
또는 그의 저작을 직접 분석·계승한 학자들의 저작을 중심으로 선정했다.
Hart의 법철학에 대한 일반 연구는 매우 많으며, 권위 있는 단행본 두 권은
다음과 같다: Neil MacCormick, \emph{H. L. A. Hart} (2판, Stanford
University Press, 2008; 이하의 모든 쪽수 표기는 1판, 1981 기준), Michael
D. Bayles, \emph{Hart's Legal Philosophy: An Examination} (Kluwer
Academic, 1992). 간결한 개요로는 Joseph Raz의 부고문 `H. L. A. Hart
(1907--1992)' (1993) \emph{Utilitas} 제5권 145쪽이 있다. Hart의 생애와
지적 배경은 Nicola Lacey, \emph{A Life of H. L. A. Hart: The Nightmare
and the Noble Dream} (Oxford University Press, 2004) 참조.

\textbf{3--4쪽.} \emph{법에 관한 일반적 이해(common knowledge)}.
일반인의 이해와 자기 이해(self-understanding)가 법이론에 가지는 중요성에
대해서는 Joseph Raz, \emph{Between Authority and Interpretation} (Oxford
University Press, 2009), 제2장 참조.

\textbf{4쪽.} \emph{법체계의 경계 사례(borderline cases)}. Hart의
요지는, 법에 대한 철학적 논쟁은 일반적으로 경계 사례의 존재에서 비롯되는
것이 아니라는 것이다. Ronald Dworkin도 이 점에 동의한다: \emph{Law's
Empire} (Harvard University Press, 1986), 40--43쪽 참조. 국내법체계가
법의 중심 사례인지 여부에 의문을 제기하는 견해로는 다음을 참조: John
Griffiths, `What is Legal Pluralism?' (1986) \emph{Journal of Legal
Pluralism \& Unofficial Law} 제24권 1쪽; William Twining, \emph{General
Jurisprudence: Understanding Law from a Global Perspective} (Cambridge
University Press, 2009), 제4장; Keith Culver and Michael Giudice,
\emph{Legality's Borders} (Oxford University Press, 2010).

\textbf{6--13쪽.} \emph{반복되는 쟁점들(recurrent issues)}. Hart는 이후
법철학의 과제를 약간 다르게 보게 되었다. 그는 1967년에 본서에서 다룬
쟁점들에 추가하여 법적 추론(legal reasoning)의 문제와 법 비판(criticism
of law)의 문제들---법을 평가할 적절한 기준, 법의 도덕적 권위의 근거
등---을 법철학의 주요 과제로 포함시켰다. 관련 논의는 \emph{Essays in
Jurisprudence and Philosophy} (Oxford University Press, 1983), 제3장
`Problems of the Philosophy of Law' 참조.

법이론의 과제를 상반된 방식으로 제시하는 견해들은 다음에서 찾아볼 수
있다: Ronald Dworkin, \emph{Taking Rights Seriously} (개정판, Harvard
University Press, 1978), 14--16쪽; Ronald Dworkin, \emph{Law's Empire},
1--6쪽; John Finnis, \emph{Natural Law and Natural Rights} (2판, Oxford
University Press, 2011), 제1장; Hugh Collins, \emph{Marxism and Law}
(Oxford University Press, 1984), 제1장. Hart의 과제 목록에 대한 평가로는
Leslie Green, `General Jurisprudence: A 25th Anniversary Essay' (2005)
\emph{Oxford Journal of Legal Studies} 제25권 565쪽 참조.

\textbf{14쪽.} \emph{단어와 대상 간의 관계.} P. M. S. Hacker, `Hart's
Philosophy of Law', in P. M. S. Hacker and J. Raz eds., \emph{Law,
Morality, and Society: Essays in Honour of H. L. A. Hart} (Oxford
University Press, 1977), 특히 2--12쪽 참조. Neil MacCormick, \emph{H. L.
A. Hart}, 12--19쪽도 참조. Dworkin은 Hart가 ``법률가들이 모두 법적
명제를 판단할 때 특정한 언어적 기준을 따르고 있다''고 본다고 해석했으며,
이는 \emph{Law's Empire} 45--46쪽에서 `의미의 독침(semantic sting)'
논변의 기초를 이룬다. 이 측면에 대한 비판은 다음에서도 볼 수 있다: Nicos
Stavropoulos, `Hart's Semantics', in Jules Coleman ed., \emph{Hart's
Postscript} (Oxford University Press, 2001); Brian Leiter, `Beyond the
Hart/Dworkin Debate: The Methodology Problem in Jurisprudence', (2003)
\emph{American Journal of Jurisprudence} 제48권 17쪽, 특히 43--51쪽.

`법(law)'이라는 단어의 의미론적 분석은 Jules Coleman and Ori Simchen,
`Law' (2003) \emph{Legal Theory} 제9권 1쪽 참조. 법철학이 과연
의미론(semantics)과 관계가 있는지에 대한 의문은 Joseph Raz, \emph{Ethics
in the Public Domain} (개정판, Oxford University Press, 1995), 제9장,
특히 195--198쪽 및 \emph{Between Authority and Interpretation}, 49--59쪽
참조. 정치이론의 언어학적 접근에 대한 일반적 회의론은 David Miller,
`Linguistic Philosophy and Political Theory', in David Miller and Larry
Siedentop eds., \emph{The Nature of Political Theory} (Oxford University
Press, 1983) 참조.

\textbf{15--16쪽.} \emph{속·종차에 따른 정의(per genus et
differentiam)}. Hart의 입장에 대한 비판은 P. M. S. Hacker, `Definition
in Jurisprudence' (1969) \emph{Philosophical Quarterly} 제19권 343쪽
참조.

\textbf{16쪽.} \emph{공통된 요소 집합.} 이 요소들은 91--99쪽에서
제시된다. 이것이 `법'이나 `법체계'에 대한 개념 분석(conceptual
analysis)을 이루는지 여부는, 그러한 분석이 무엇을 요구한다고 보는가에
달려 있다. 비교문헌으로는 Frank Jackson, \emph{From Metaphysics to
Ethics: A Defense of Conceptual Analysis} (Oxford University Press,
1998), Colin McGinn, \emph{Truth by Analysis: Games, Names, and
Philosophy} (Oxford University Press, 2012), 특히 제2장 참조. Hart류의
이론과 사회학적 이론 간의 관계에 대해서는 다음을 참조: H. L. A. Hart,
`Analytical Jurisprudence in Mid-Twentieth Century: A Reply to Professor
Bodenheimer' (1956) \emph{University of Pennsylvania Law Review} 제105권
953쪽; M. Krygier, `\,``The Concept of Law'' and Social Theory' (1982)
\emph{Oxford Journal of Legal Studies} 제2권 155쪽; B. Z. Tamanaha,
`Socio-Legal Positivism and a General Jurisprudence' (2001) \emph{Oxford
Journal of Legal Studies} 제21권 1쪽; Denis Galligan, `Legal Theory and
Empirical Research', in Peter Cane and Herbert Kritzer eds.,
\emph{Oxford Handbook of Empirical Legal Research} (Oxford University
Press, 2010).

\subsection{FOOTNOTES CHAPTER
I}\label{footnotes-chapter-i}  % For Pandoc

\end{document}
